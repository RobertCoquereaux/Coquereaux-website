
\chapter{Espaces Fibr\'es}
\section{G\'en\'eralit\'es}
\subsection{Le langage des fibr\'es}
Avant d'introduire la notion d'espace fibr\'e, nous allons montrer comment 
``le langage des fibr\'es'' permet d'analyser de mani\`ere originale un aspect 
bien connu de la th\'eorie des ensembles, \`a savoir la notion d'application 
d'un ensemble dans un autre, tout en y apportant un nouvel 
\'eclairage.

Soit $\pi$ une application quelconque de $P$ (espace de d\'epart) dans $M$ 
(espace d'arriv\'ee). L'application $\pi$ n'ayant aucune raison d'\^etre 
injective, notons $F_x  =  \pi^{-1}(x)$ l'ensemble des ant\'ec\'edents de 
$x\in M$ par $\pi$.
Lorsque $x$ d\'ecrit $M$, les ensembles $F_x$ sont, bien entendu, 
distincts, puisque $\pi$ est une application (il n'en serait pas 
n\'ecessairement ainsi dans le cas o\`u  $\pi$ serait une simple correspondance).
En g\'en\'eral aussi, la cardinalit\'e de $F_x$, ou m\^eme sa structure 
topologique peut varier avec $x$. 

Utiliser ``le langage des fibr\'es'' consiste \`a effectuer le changement de 
vocabulaire suivant~:
 
\begin{tabular}{lcl}
Espace de d\'epart $P$ & $\longrightarrow$  & Espace total $P$ \\
Espace d'arriv\'ee $M$ & $\longrightarrow$ & Base $M$ \\
Application $\pi$ & $\longrightarrow$ &  Projection $\pi$ \\
Ensemble $F_x$ des ant\'ec\'edents de $x$ & $\longrightarrow$ &  Fibre  $F_x$ au dessus de $x$
\end{tabular}

et sch\'ematiser la situation par le dessin \ref{fig:espace fibre}.

\begin{figure}[htbp]
\epsfxsize=8cm
$$
\epsfbox{espace-fibre.eps}
$$
\caption{Un espace fibr\'e}
\label{fig:espace fibre}
\end{figure}

On peut toujours supposer $\pi$ surjective, au besoin en d\'egonflant 
l'espace d'arriv\'ee $M$, mais $\pi$ ne sera pas, en g\'en\'eral, injective (en 
d'autres termes, $\mbox{card}(F_x)$ est, en g\'en\'eral, plus grand que $1$).
Choisissons donc, pour chaque $x$ de $M$, un certain ant\'ec\'edent, c'est \`a 
dire un \'el\'ement de $F_x$. Nous noterons $\sigma(x)$ l'ant\'ec\'edent choisi. 
Par construction, $\sigma$ est une application de $M$ dans $P$ telle que 
$\pi \circ \sigma = \id$, o\`u  le $\id$ du membre de droite d\'esigne 
l'application identique de $M$. Une telle application $\sigma$ est, par 
d\'efinition, une {\sl section\/} \index{section} de l'application $P 
\stackrel{\pi}{\mapsto} M$. Le 
mot ``section'' vient du fait qu'intuitivement, on d\'efinit $\sigma$ en 
d\'ecoupant avec des ciseaux la figure~\ref{fig:espace fibre} le long du pointill\'e.
Le fait que $\pi$ ne soit pas injective montre qu'il existe g\'en\'eralement 
de nombreuses sections $\sigma$ diff\'erentes (notons que nous n'avons, 
pour le moment, rien impos\'e sur l'application $\sigma$). Chacune de ces 
sections d\'efinit donc un inverse (\`a droite) pour l'application $\pi$~; 
encore une fois, il nous faut insister sur le fait qu'il existe en 
g\'en\'eral plusieurs sections et que le choix d'une telle section r\'esulte 
$\ldots$ d'un choix! Lorsque $M$ est muni d'une topologie, le choix d'une 
section au dessus d'un ouvert $U \subset M$ est d\'esign\'e sous le nom de 
``choix d'une {\sl section locale\/}  \index{section locale}  pour l'application $\pi$''.

Le vocabulaire pr\'ec\'edent est tellement g\'en\'eral qu'il est utilisable pour 
des applications entre ensembles quelconques. Le lecteur pourra donc 
s'amuser \`a ``repenser'', en ces termes, certaines de ses applications 
favorites. 

Consid\'erons, \`a titre d'exemple, l'application suivante qui, \`a 
tout point de la sph\`ere unit\'e, associe sa projection sur l'axe $\mbox{Oz}$:
\begin{description}
	\item[Espace de d\'epart]  (espace total) $P$~: la sph\`ere $S^2$ de centre 
	$O$ et de rayon $1$ incluse dans $\RR^2$.
	
	\item[Espace d'arriv\'ee]  (base) $M$~: le segment $\{(0,0,z)\,
 \tq z \, \in 
	[-1,+1]\}$
	
	\item[Application]  $\pi$~: $(x,y,z)\in S^2 \mapsto (0,0,z) \in M$
\end{description}

La fibre au dessus d'un point quelconque int\'erieur au segment $M$ est un 
cercle de $S^2$ parall\`ele \`a l'\'equateur $\mbox{xOy}$. Lorsque le point de 
$M$ choisi co\"incide avec l'un des deux p\^oles $\{0,0,\pm 1\}$, la fibre 
correspondante se r\'eduit \`a un point (cercle de rayon nul).

De la m\^eme fa\c con, l'analyse de la fonction exponentielle $i\theta \in P 
 =  i \RR \mapsto e^{i\theta} \in M  =  S^1$ en utilisant le 
langage des fibr\'es revient \`a ``regarder'' la figure \ref{fig:exponentielle}

\begin{figure}[htbp]
\epsfxsize=8cm
\epsfxsize=6cm
$$
    \epsfbox{exponentielle.eps}
$$
\caption{La fonction exponentielle $e^{i\theta}$}
\label{fig:exponentielle}
\end{figure}

Dans le cas pr\'esent, la fibre au dessus de n'importe quel point $x_\theta 
 =  e^{i\theta}$ du cercle $S^1$ est un ensemble $F_\theta  =  
\{\theta + 2 k \pi {\hskip 0.5cm} \tq  {\hskip 0.5cm} k \in \ZZ \}$ qu'on peut mettre en bijection avec 
l'ensemble des entiers $\ZZ$. Toute section d\'efinit un inverse pour 
l'exponentielle, c'est \`a dire une d\'etermination du logarithme~: 
$\exp \, Log \, e^{i\theta} = e^{i\theta}$; en pratique, on veut que ces d\'eterminations 
soient des fonctions continues, ce qui impose des conditions 
suppl\'ementaires sur les sections correspondantes.

\subsection{Fibrations. Fibr\'es diff\'erentiables localement triviaux}
\subsubsection{Fibration}

Comme nous venons de le voir, toute application peut \^etre d\'ecrite dans 
le ``langage des fibr\'es'', mais toute application n'est pas une fibration 
: dans le cadre de cet ouvrage, nous travaillons dans la cat\'egorie des 
vari\'et\'es diff\'erentiables et nous dirons qu'une application $\pi: P 
\rightarrow M$ est une {\sl fibration\/}\index{fibration} si toutes les fibres $\pi^{-1}(x)$, $x \in M$ 
sont diff\'eomorphes (on suppose ici que $P$ et $M$ sont des vari\'et\'es et que 
$\pi$ est diff\'erentiable). Plus g\'en\'eralement, m\^eme dans le cas o\`u  les 
fibres sont discr\`etes, on dira qu'on a affaire \`a une fibration si 
toutes les fibres ont m\^eme cardinalit\'e.

Puisque toutes les fibres sont diff\'eomorphes, on dira que {\sl la fibre 
type\/} \index{fibre type} est $F$ lorsque toutes les fibres sont ``de 
type $F$'', ce qui signifie~: diff\'eomorphes \`a $F$.

\subsubsection{Fibr\'e localement trivial}

Soit $P \rightarrow M$ une fibration et $U \subset M$ un ouvert de $M$
appartenant \`a l'atlas d\'efinissant la structure diff\'erentiable de $M$. 
On a envie, intuitivement, de se repr\'esenter l'ensemble des ant\'ec\'edents 
$\pi^{-1}(U)$ comme un cylindre ``au dessus de $U$'' (voir figure 
\ref{fig:cylindre}).

\begin{figure}[htbp]
\epsfxsize=8cm
$$
    \epsfbox{cylindre.eps}
$$

\caption{Les points au dessus de l'ouvert $U$}
\label{fig:cylindre}
\end{figure}
 
 On appelle {\sl espace fibr\'e localement trivial\/} \index{espace fibr\'e 
 localement trivial} la donn\'ee d'une fibration $(P,M,\pi)$ telle que, 
 pour tout ouvert $U$ de $M$, l'ensemble $\pi^{-1}(U)$ soit diff\'eomorphe 
 \`a $U \times F$ o\`u  $F$ est la fibre type de la fibration. 
 La condition de trivialit\'e locale est souvent sous-entendue, et on parle alors simplement d'{\sl espace fibr\'e}.
 
 \subsubsection{Remarques}
\begin{itemize}
	\item  En anglais, les deux expressions ``{\it fibered space\/}'' et ``{\it fiber 
	bundle\/}'' correspondent respectivement aux termes ``fibrations'' et ``espace 
	fibr\'e localement trivial'', tout au moins, en g\'en\'eral. 	
	\item  Dans la suite de cet ouvrage, nous ne consid\'ererons que des 
	espaces fibr\'es localement triviaux, sauf mention explicite du contraire. 
	Pour cette raison, nous utiliserons souvent les deux expressions 
	``fibration'' et ``fibr\'e localement trivial'' de fa\c con \'equivalente et 
	donc sans faire explicitement mention de la propri\'et\'e de trivialit\'e locale.
	\item  Comme on l'a \'ecrit plus haut, un espace fibr\'e est d\'efini par un 
	triplet $(P,M,\pi)$ car il peut, en effet, exister plusieurs fibrations 
	diff\'erentes $(P,{M'},{\pi')}$ et $(P,M,\pi)$ du m\^eme espace 
	total $P$. Par abus de langage, et dans le cas o\`u  l'on s'int\'eresse \`a une 
	seule fibration pour un espace total $P$ donn\'e, on pourra dire que $P$ 
	lui m\^eme est un espace fibr\'e.
	\item  Intuitivement, dire que $P$ est un espace fibr\'e revient \`a dire 
	que $P$ est une collection de fibres de m\^eme type et ``coll\'ees'' 
	ensemble. L'ensemble des fibres (l'ensemble dont les diff\'erents \'el\'ements 
	sont les fibres) est, par d\'efinition, l'espace de base $M$ et 
	l'application $\pi$ est celle qui associe \`a tout point $z$ de $P$ la 
	fibre \`a laquelle il appartient. On dira \'egalement que $P$ est un espace 
	qui peut \^etre fibr\'e en espaces $F$ au dessus de $M$ (voir la 
	figure \ref{fig:cylindre}).
	
\end{itemize}

 \subsubsection{Fibr\'e trivial}
 
 Dire que $(P,M,\pi)$ est localement trivial consiste \`a affirmer, comme 
 nous l'avons vu, que {\it localement\/}, l'espace total $P$ ressemble \`a 
 un produit~: c'est la propri\'et\'e $\pi^{-1}(U) \simeq U\times F$.
 Cependant, le fait que $P \simeq M \times F$ n'est pas une propri\'et\'e 
 impos\'ee. Lorsque tel est le cas, on dit que $P$ est un fibr\'e trivial. 
 Bien qu'utilisable en toute g\'en\'eralit\'e, la th\'eorie des espaces fibr\'es 
 est surtout int\'eressante lorsque $P$ est seulement localement trivial, 
 mais n'est pas trivial. Nous rencontrerons une multitude de tels exemples 
 dans ce qui suit.
 
 \subsubsection{Trivialisations locales}
 
 Lorsque $U$ est un ouvert de $M$, dire que  $\pi^{-1}(U)$ est 
 diff\'eomorphe \`a $U \times F$ revient \`a dire qu'il existe un 
 diff\'eomorphisme $\psi_U$ entre les deux ensembles~: 
$$
 z \in  \pi^{-1}(U) \subset P \mapsto \psi_U(z) = (x,g) \in U \times F
{\hskip 0.5cm}\mbox{avec}{\hskip 0.5cm} x = \pi(z) 
$$
 L'application $\psi_U$ prend le nom de {\sl trivialisation locale 
 relative \`a l'ouvert $U$\/}. \index{trivialisations locales}
 Bien entendu, une telle trivialisation est loin d'\^etre unique puisque la 
 compos\'ee $(\id_M \times \phi) \circ \psi_U$ de $\psi_U$ avec $(\id_M 
 \times \phi)$, $\phi$ d\'esignant un diff\'eomorphisme quelconque de la 
 fibre type $F$, fournit une autre trivialisation.
 
 La trivialisation $\psi_U(z) = (x=\pi(z),g)$ est parfaitement 
 caract\'eris\'ee par l'application $g_U : \pi^{-1}(U) \rightarrow F$ d\'efinie 
 par $g_U(z)=g$; en d'autres termes, le point $z$ de $P$ est caract\'eris\'e 
 par le point $x=\pi(z)$ sur $M$ et le point $g=g_U(z)$ sur 
 $F$. Seul le point $x$ est canoniquement d\'efini par la fibration ; l'\'el\'ement  $g$ de $F$, 
 au contraire, r\'esulte du choix d'une trivialisation locale. Pour que ces 
 deux composantes (les deux points $(x,g)\in M\times F$) deviennent des coordonn\'ees (des 
 nombres), il suffit de choisir un rep\`ere local sur $M$ et sur $F$.
 On pourra sch\'ematiser la situation par la figure \ref{fig:trivialisation}.
\begin{figure}[htbp]
\epsfxsize=8cm
$$
    \epsfbox{trivialisation.eps}
$$

\caption{Trivialisation locale d'un espace fibr\'e}
\label{fig:trivialisation}
\end{figure}

\subsubsection{Sous-fibr\'es}

Soient $(P,M,\pi)$ et $(P',M',\pi')$ deux espaces fibr\'es. On dira que
le premier est un sous-fibr\'e du second si $P$ (respectivement $M$) est
une sous-vari\'et\'e de $P'$ (respectivement $M'$) et si $\pi$ co\"incide
avec la restriction correspondante de $\pi'$.


\section{Espaces fibr\'es principaux}
\subsection{La structure d'espace fibr\'e principal}\label{sec:principal}
Une fibration $(P,M,\pi)$ est un espace fibr\'e principal lorsque les trois 
conditions suivantes sont satisfaites~:
\begin{enumerate}
	\item  $(P,M,\pi)$ est un espace fibr\'e localement trivial.
	
	\item  Un groupe de Lie $G$ agit (\`a droite) sur $P$, et ce, de fa\c con 
	transitive dans chaque fibre.
	
	\item  Toutes les fibres sont hom\'eomorphes \`a $G$.
\end{enumerate}

Les trois conditions ci-dessus sont obligatoires pour qu'on puisse parler 
de fibr\'e principal car nous verrons un peu plus loin des exemples o\`u  $(1)$ et 
$(2)$ sont v\'erifi\'ees (mais pas $(3)$) et des exemples o\`u  $(1)$ et 
$(3)$ sont v\'erifi\'ees (mais pas $(2)$).

En g\'en\'eral, on consid\`ere des espaces fibr\'es principaux \`a droite, comme 
ci-dessus, mais il est bien \'evident qu'on peut \'egalement consid\'erer des 
espaces fibr\'es principaux \`a gauche.

Le groupe $G$ (la fibre type) est g\'en\'eralement d\'esign\'e sous le nom de 
{\sl groupe structural \/}\index{groupe structural} du fibr\'e consid\'er\'e.
Afin d'all\'eger les notations, nous noterons tr\`es simplement l'action de 
$G$ sur $P$~: Soient $z_1 \in P$ et $g \in G$, l'image $z_2$ de $z_1$ 
sous l'action de $g$ sera not\'ee $z_2 = z_1 g$, ce qui peut \^etre d\'ecrit, 
de fa\c con imag\'ee, par la figure~\ref{fig:fibre principal}.

\begin{figure}[htbp]
\epsfxsize=8cm
$$
    \epsfbox{action-fibre.eps}
$$

\caption{Action du groupe structural sur un espace fibr\'{e} principal}
\label{fig:fibre principal}
\end{figure}

Attention~: Parce que $G$ agit sur $P$, de nombreux physiciens d\'esignent ces 
transformations de $P$ dans
$P$ (du type $z\in P \mapsto {z'}=zg, g\in G$) sous le nom de {\sl 
transformations de jauge globales\/}\index{transformations de jauge globales}
 et d\'esignent \'egalement $G$ lui m\^eme sous le nom de 
{\sl groupe de jauge \/}\index{groupe de jauge};
 cependant nous r\'eserverons ce dernier vocable (groupe de jauge)  pour le 
{\sl groupe des transformations de jauge locales \/}\index{transformations de jauge locales}
\index{automorphismes verticaux}
 que nous d\'efinirons un 
peu plus loin.

La relation $z_2=z_1 g$ est formellement tr\`es semblable \`a la relation 
\'el\'ementaire 
$A_2=A_1 + \overrightarrow{V}$ o\`u  $A_1$ et $A_2$ d\'esignent deux points d'un espace 
affine et o\`u  $\overrightarrow{V}$ d\'esigne un vecteur de l'espace vectoriel 
sous-jacent. Les \'el\`eves de nos lyc\'ees savent bien qu'on peut 
``soustraire'' deux points en \'ecrivant $\overrightarrow{V}=A_2-A_1$ (on n'a pas le 
droit d'``additionner'' deux points!). De la m\^eme fa\c con, on pourra \'ecrire 
ici $g = z_1^{-1}z_2$, puisque
$z_1 g = z_2$ et que
 $g$ est bien d\'etermin\'e par la donn\'ee de 
$z_1$ et de $z_2$. Notons enfin que l'analogue de la c\'el\`ebre ``relation 
de Chasles'' s'\'ecrit $z_1^{-1}z_2 = (z_1^{-1}z_3)(z_3^{-1}z_2)$.

\subsection{Sections locales et trivialisations locales}
Dans le cas d'un fibr\'{e} principal, chaque fibre $G_x$ au dessus de $x$, 
\'el\'ement de $M$ est une ``copie'' du groupe $G$, mais il s'agit d'une 
copie au sens topologique (ou diff\'erentiable) du terme car l'origine 
du groupe $G$ (l'\'el\'ement neutre) est connue mais celle de la fibre $G_x$ ne 
l'est pas~! Afin de mieux faire sentir le sens de cette importante 
remarque, consid\'erons l'exemple suivant \ref{fig:origine}
\begin{figure}[htbp]
\epsfxsize=12cm
$$
    \epsfbox{origine.eps}
$$
\caption{L'origine de la fibre (copie du cercle $U(1)$) au dessus de $x$ est 
marqu\'ee par la section $\sigma$}
\label{fig:origine}
\end{figure}

Dans le cas pr\'esent, $P$ est un cylindre fini $P=M\times S^1$ o\`u  $M$ est 
un intervalle et $S^1$ d\'esigne le cercle de rayon $1$~; la fibre au dessus 
de $x$ est un cercle, et ce cercle, comme tous les cercles, est 
hom\'eomorphe au groupe $U(1)$. Sur ce cercle, tous les points ``se 
valent'' et on ne sait pas multiplier un point par un autre. Par contre, 
si le cercle est marqu\'e par une origine, il devient isomorphe au groupe 
$U(1)$ et on sait alors multiplier les points 
($e^{i\theta}e^{i\alpha}=e^{i(\theta+\alpha)}$). Le groupe $U(1)$ agit 
bien sur l'ensemble $P$ ci-dessus en faisant tourner un point quelconque 
$z\in P$ d'un angle $\theta$.

Revenons au cas g\'en\'eral d'un fibr\'{e} principal $(P,M,\pi)$ de groupe 
structural $G$. Le choix d'une section locale $x\in U\subset M \mapsto 
\sigma(x) \in P$ permet de ``marquer une origine'' sur chacune des 
fibres $G_x$ situ\'ees au dessus de l'ouvert $U$. En d'autres termes, le 
choix d'une section locale $\sigma$ permet d'identifier la fibre $G_x$ 
avec le groupe $G$ lui-m\^eme. La fa\c con la plus simple d'exprimer ceci de 
fa\c con alg\'ebrique consiste \`a montrer qu'\`a la section locale $\sigma$ on 
peut associer une trivialisation locale $\psi_U$ d\'efinie comme suit~: 
soit $z\in P$, alors $\psi_U(z)  =  (x;g_\sigma)$ o\`u  $x=\pi(z)$ et o\`u
$g_\sigma$ d\'esigne l'unique \'el\'ement de $G$ d\'efini par $z = \sigma(x) 
g_\sigma$. En effet, $z$ et $\sigma(x)$ \'etant dans la m\^eme fibre, il 
existe un et un seul \'el\'ement $g_\sigma$ de $G$ permettant de passer de 
$\sigma(x)$ \`a $z$~; l'existence et l'unicit\'e de cet \'el\'ement $g_\sigma$ 
r\'esulte des axiomes $(2)$ et $(3)$ de la structure de fibr\'{e} principal. 
Une section locale $\sigma$ d\'efinit donc \'egalement une application --- que 
nous noterons encore $g_\sigma$ --- de $P$ dans $G$~; en d'autres termes, 
les ``composantes'' de $z \in P$ sont $x = \pi(z) \in M$ et 
$g_\sigma=g_\sigma(z) \in G$. La composante $x$ est canoniquement d\'efinie 
par la structure fibr\'{e}e et la composante $g_\sigma$ r\'esulte du choix 
d'une section locale $\sigma$.

Il faut enfin noter que le choix d'une section locale permet de d\'efinir 
{\it localement\/} l'action {\it \`a gauche\/} du groupe $G$ sur $P$~; en 
effet, en plus de l'action \`a droite $z\in P, k \in G \rightarrow 
z \, k=(x;g_\sigma)k=(x;g_\sigma k) \in P$ qui ne d\'epend pas de $\sigma$ et qui est globalement d\'efinie
puisqu'elle r\'esulte de la structure d'espace fibr\'{e} principal, on 
peut d\'efinir localement une action \`a gauche
$z\in P, k \in G \rightarrow 
(k \, z)_\sigma = (x;k g_\sigma) \in P$, qui d\'epend de $\sigma$.

Supposons que nous ayons fait le choix d'une section locale $\sigma$ au 
dessus de l'ouvert $U$ et d'une section locale $\tau$ au dessus de 
l'ouvert $V$~; si on fait un choix de $z \in P$ tel que la projection 
$\pi(z)$ appartienne \`a l'intersection $U \cap V$, on peut \'ecrire aussi 
bien 
$z \mathrel{\oalign{$\, =$\cr$\sigma $}}(x;g_\sigma)$ que
$z \mathrel{\oalign{$\, =$\cr$\tau $}}(x;g_\tau)$.
 Il existe donc un \'el\'ement $g_{\sigma\tau}$ du 
groupe $G$ (et en fait une fonction $g_{\sigma\tau}(x)$ d\'efinie sur $U 
\cap V$) tel que $g_\sigma = g_{\sigma\tau} g_\tau$. Cette fonction 
porte le nom de {\sl fonction de transition\/} \index{fonctions de 
transition}. Ces fonctions de transition permettent en fait de reconstruire
le fibr\'{e} principal lui-m\^eme. On montre qu'\'etant
 donn\'es un atlas de $M$ et une famille de fonctions
de transition  ob\'eissant \`a une certaine propri\'et\'e (dite de cocycle)
 sur les triples intersections, il est possible de reconstruire
 l'espace fibr\'{e} dont on est parti.

\subsection{Exemple fondamental~: le fibr\'{e} des rep\`eres lin\'eaires}
\label{sec:exemple-fondamental}
L'exemple qui suit est fondamental, non seulement parce qu'il est 
math\'ematiquement important --- il est d'ailleurs \`a l'origine de toute la 
th\'eorie des espaces fibr\'{e}s --- mais aussi parce qu'il permet de fournir 
un support \`a notre intuition g\'eom\'etrique, en particulier dans le cas o\`u  
l'on s'int\'eresse \`a des fibr\'{e}s principaux $(P,M,\pi)$ quelconques. L'exemple 
fondamental \'etudi\'e ici nous permettra de d\'evelopper les analogies 
suivantes~:
\begin{itemize}
	\item  Consid\'erer l'espace total $P$ comme un ensemble de rep\`eres 
	g\'en\'eralis\'es sur la base $M$.
	
	\item  Consid\'erer le groupe structural  $G$ comme groupe de 
	transformations de rep\`eres.
	
	\item  Consid\'erer un \'el\'ement quelconque $z$ de $P$ comme un rep\`ere 
	(g\'en\'eralis\'e) situ\'e au point $x$ de $M$ (avec $x = \pi(z)$).
	
	\item  Consid\'erer toute section locale $\sigma(x)$ comme un rep\`ere 
	mobile (g\'en\'eralis\'e) dans l'ouvert $U$.
	
	\item  Consid\'erer les fonctions de transition $g_{\sigma \tau}$ comme 
	d\'ecrivant des changements de rep\`ere mobile.
	
	\item  \etc
\end{itemize}

Soit $M$ une vari\'et\'e diff\'erentiable de dimension $n$. En chaque point $x$ 
de $M$ nous avons un espace tangent $T(M,x)$ et nous pouvons consid\'erer 
l'ensemble $G_x$ de tous les rep\`eres en $x$. Un point $z$ de $G_x$ est 
donc un rep\`ere en $x$, c'est \`a dire la donn\'ee de $n$ vecteurs 
ind\'ependants de $T(M,x)$. Soit $P=\bigcup_{x\in M}^{}G_x$ 
l'ensemble de tous les rep\`eres de $M$. Notons $\pi$ l'application qui, 
\`a un rep\`ere centr\'e sur $x$, associe l'origine $x$ elle-m\^eme; il est facile 
de voir que $(P,M,\pi)$ est un espace fibr\'{e} principal de groupe 
structural $GL(n)$. Il est clair, en effet, que le groupe lin\'eaire 
$GL(n)$ agit transitivement sur chaque fibre de $P$~: la fibre $G_x$ au 
dessus de $x$ n'est autre que l'ensemble des rep\`eres en $x$ et il est 
bien \'evident qu'on peut toujours passer d'un rep\`ere $z={(z_i)}_{i\in 
\{1\ldots n\}}$ \`a un rep\`ere ${z'}=({z'}_j)$ au m\^eme point $x$ \`a 
l'aide d'un \'el\'ement $g=(g^i_j)$ de $GL(n)$~: $({z'}_j = z_i g^i_j$). 
Par ailleurs, le fait que l'ensemble $G_x$ des rep\`eres en $x$ soit 
hom\'eomorphe \`a $GL(n)$ peut se voir de la fa\c con suivante~: marquons 
(choisissons) un rep\`ere de r\'ef\'erence $\sigma  =  (\sigma)_i$ en $x$; alors, 
tout \'el\'ement $g$ de
$GL(n)$ d\'efinit un nouveau rep\`ere $z = \sigma g$ au m\^eme point, mais 
r\'eciproquement, tout nouveau rep\`ere $z$ d\'etermine un et un seul \'el\'ement $g$
de $GL(n)$ tel que $z = \sigma g$. On obtient donc une correspondance 
bi-univoque entre rep\`eres en $x$ et \'el\'ements de $GL(n)$; bien entendu, 
cette correspondance d\'epend du choix du rep\`ere de r\'ef\'erence $\sigma$. Il 
resterait \`a montrer que cette application est bel et bien continue et \`a 
v\'erifier les conditions de trivialit\'e locale. Le fibr\'{e} principal $P$ 
ainsi construit se note parfois $FM$ (pour ``{\it Frame bundle of M\/}'') et 
s'appelle le {\sl fibr\'{e} des rep\`eres lin\'eaires\/}\index{fibr\'{e} des rep\`eres lin\'eaires}
 sur $M$. Nous invitons 
le lecteur \`a relire la sous-section pr\'ec\'edente avec cet exemple en t\^ete~; 
il est alors clair qu'une section locale n'est autre qu'un rep\`ere mobile 
choisi dans le domaine d'un ouvert et qu'une fonction de transition n'est 
autre qu'un changement de rep\`ere mobile.

\subsection{Sous-espace des vecteurs verticaux en un point $z$ d'un 
espace fibr\'{e}}

\begin{itemize}
	\item  Soit $P$ un espace fibr\'{e} principal et $z$ un \'el\'ement de $P$.
$P$ est, en particulier, une vari\'et\'e, et
toutes les 
constructions \'etudi\'ees dans le contexte des vari\'et\'es diff\'erentiables 
peuvent \^etre effectu\'ees et on peut donc consid\'erer l'espace tangent \`a $P$ 
en $z$ que l'on notera $T(P,z)$. Cet espace vectoriel est \'evidemment de 
dimension $m+n$ lorsque $\mbox{dim}\, M = m$ et $\mbox{dim}\, G = n$, $M$ et 
$G$ d\'esignant respectivement la base et la fibre type de $P$.
	
	\item  Plut\^ot que de consid\'erer l'espace tangent \`a {\it tout\/} l'espace $P$ en 
$z$, nous pouvons consid\'erer l'espace tangent \`a la fibre $F_x$ de $P$ qui 
passe par $z$ (c'est \`a dire $x=\pi(z)$); cet espace vectoriel $V_z  =  
T(F_x,z)$ est naturellement un sous-espace vectoriel de $T(P,z)$. Sa 
dimension est \'egale \`a $n$ puisque toutes les fibres ont dimension $n$. Le 
sous-espace $V_z$ s'appelle {\sl espace tangent vertical\/} \index{espace 
tangent vertical} au point $z$.

\item Intuitivement, un vecteur est un petit 
d\'eplacement (une fl\`eche~!). Un vecteur de l'espace tangent $T(P,z)$ est 
donc un d\'eplacement infinit\'esimal d'un ``rep\`ere'' (nous pensons 
intuitivement \`a $z$ comme \'etant une sorte de rep\`ere g\'en\'eralis\'e).
Un d\'eplacement infinit\'esimal dans l'ensemble des rep\`eres peut s'analyser \`a 
l'aide de deux mouvements tr\`es diff\'erents: on peut faire tourner le 
rep\`ere (sans bouger l'origine) mais on peut \'egalement d\'eplacer l'origine 
du rep\`ere. Les d\'eplacements verticaux du point $z$ correspondent \`a des 
mouvements de $z$ dans sa fibre: on ne d\'eplace pas le point de base 
$x=\pi(z)$~; ainsi, les vecteurs de $V_z$ (les vecteurs verticaux en $z$) 
correspondent \`a des d\'eplacements infinit\'esimaux du rep\`ere $z$ qui 
n'entra\^inent aucun d\'eplacement de l'origine $x$~: on se contente de 
``faire tourner'' infinit\'esimalement (\`a l'aide d'une ``petite 
transformation'' du groupe $G$) le rep\`ere $z$ en $x$.
	
	\item  Le lecteur pourra s'\'etonner de ne trouver, dans ce chapitre, aucun 
paragraphe intitul\'e ``sous-espace horizontal'' \index{sous-espace 
horizontal}~; il y a, \`a cela, une excellente raison~: alors que la notion de 
sous-espace vertical peut se d\'efinir canoniquement, comme on vient de le 
voir, pour tout fibr\'{e} principal, il n'est par contre pas possible de 
d\'efinir canoniquement, tout au moins en g\'en\'eral, la notion de d\'eplacement 
horizontal~; une telle notion est tributaire d'un choix. L'\'etude des 
choix possibles ---pour un fibr\'{e} principal donn\'e--- d\'efinit, en quelque 
sorte, la th\'eorie des connexions et fait l'objet du chapitre suivant.
	
	\item  Le groupe $G$ agissant (\`a droite) sur l'espace $P$, on peut d\'efinir, 
comme d'habitude (voir le chapitre sur les groupes) des champs fondamentaux 
$\epsilon_\alpha$ associ\'es aux \'el\'ements $X_\alpha$ de l'alg\`ebre 
$Lie(G)$. Par construction, $\epsilon_\alpha(z)$ est un vecteur vertical au 
point $z$ et l'ensemble $\{\epsilon_\alpha(z)\}_{\alpha \in \{1\ldots 
n\}}$ constitue une base de l'espace vectoriel $V_z$.

Si $ {z'}=zg$ d\'esigne le rep\`ere issus de $z$ par une ``rotation'' 
finie $g$, on pourra \'ecrire $\epsilon_\alpha(z)=z X_\alpha$ et 
interpr\'eter $\epsilon_\alpha(z)$ comme une d\'eplacement infinit\'esimal du 
``rep\`ere'' $z$ \`a l'aide de la ``rotation infinit\'esimale'' $X_\alpha$. 

Pour ne pas alourdir le texte, nous supprimerons les guillemets autour 
des mots ``rep\`ere'' et ``rotation'' dans la suite du texte, mais le 
lecteur devra se souvenir que ces mots d\'esignent respectivement les 
\'el\'ements du fibr\'{e} principal consid\'er\'e (qui ne sont pas n\'ecessairement des rep\`eres au 
sens usuel du terme) et les \'el\'ements du groupe structural (qui n'est pas 
n\'ecessairement un groupe de rotations).
	
	\item  Les champs $z \rightarrow \epsilon_\alpha(z)$ sont des champs 
fondamentaux {\it \`a droite\/} puisque $P$ est ---comme d'habitude--- un 
fibr\'{e} principal muni d'une action \`a droite. A moins que $P$ ne soit 
trivial (voir section suivante) il n'existe pas d'action de $G$ \`a gauche 
de $P$, en tous cas, pas d'action qui soit canoniquement d\'efinie. Par 
contre, on peut toujours trivialiser $P$ localement en choisissant une 
section (locale) $x \in M \mapsto \sigma(x) \in P$~; on a vu 
 qu'une telle section permettait d'identifier la fibre $F_x$ avec 
$G$ lui-m\^eme en associant au point $z\in F_x$ l'\'el\'ement $g_\sigma$ de $G$ d\'efini 
par l'\'equation $z=\sigma(x)g_\sigma$.
On a alors non seulement une action de $G$ 
\`a droite mais \'egalement une action de $G$ \`a gauche d\'efinie par
$
k \in G , \, z = (x,g_\sigma) \in P \mapsto z'=(x,kg_\sigma) \in P.$
Cette action d\'epend de la section $\sigma$ et permet de d\'efinir 
localement des champs fondamentaux {\it \`a gauche\/} $e_\alpha(z)$; ces 
champs d\'ependent donc \'egalement du choix de la section $\sigma$ et, si la chose
est n\'ecessaire, on pourra les noter ${}^\sigma e_\alpha$.
	
	\item  Pour r\'esumer, on pourra dire que la fibre $F_x$ au dessus de $x$ appara\^it 
comme une copie du groupe $G$, le choix d'une section locale $\sigma(x)$ 
permet de marquer l'origine (l'identit\'e du groupe) sur la fibre $F_x$. On 
peut ainsi identifier $G$ avec $F_x$ et l'alg\`ebre de Lie de $G$ avec 
l'espace vertical $T(F_x,\sigma(x))$ au point $\sigma(x)$. Le groupe $G$ agit sur 
lui-m\^eme par multiplications \`a gauche et \`a droite, il agit canoniquement 
sur $F_x$, du c\^ot\'e droit, puisqu'on a affaire \`a un fibr\'{e} principal, mais on peut aussi le 
faire agir \`a gauche d\`es qu'on a identifi\'e $G$ avec $F_x$, c'est \`a dire 
d\`es qu'on a choisi une section $\sigma(x)$. Le choix de $\sigma$ 
permettant d'identifier $G$ avec $F_x$ permet donc \'egalement de d\'efinir 
une forme de Maurer-Cartan ${}^\sigma \theta(x)$ pour la fibre $F_x$. 
Cette forme ``ram\`ene'' donc \`a l'origine $\sigma(x)$ les vecteurs 
verticaux appartenant \`a $V_{\sigma(x)g}$.

\end{itemize}

\subsection{Fibr\'e principal trivial}\label{sec:trivial}
Un fibr\'{e} principal $(P,M,\pi)$ de groupe structural $G$ est trivial si, 
par d\'efinition, $P$ est hom\'eomorphe au produit cart\'esien $M\times G$
(la projection $\pi$ \'etant la projection sur le premier facteur).
dans ce cas, il existe plusieurs (en g\'en\'eral une infinit\'e de) sections 
globales puisque toute application diff\'erentiable de $M$ dans $G$ d\'efinit 
une section globale~: consid\'erer par exemple l'application constante qui, 
\`a tout point de $M$ associe l'identit\'e de $G$. R\'eciproquement, supposons 
qu'un fibr\'{e} principal poss\`ede une section globale $\sigma$, on peut alors 
consid\'erer l'application de $P$ dans $M\times G$ d\'efinie par $z 
\rightarrow (x,g)$ avec $x=\pi(z)$ et $g$ tel que $z=\sigma(x)g$; on 
fabrique ainsi un hom\'eomorphisme entre $P$ et $M\times G$.

En conclusion, un fibr\'{e} principal est trivial si et seulement s'il 
poss\`ede une section globale. Lorsque $P$ est trivial, son identification 
avec $M\times G$ r\'esulte, comme on vient de le voir, du choix de la section 
globale $\sigma$;  on \'ecrira simplement $P=M\times G$ si cela ne pr\^ete pas 
\`a confusion. Noter que, dans un tel cas, les champs fondamentaux \`a droite 
$\epsilon_\alpha$ et \`a gauche ${}^\sigma e_\alpha$ sont tous deux 
globalement d\'efinis. 

Attention : pour des fibr\'{e}s non principaux (voir plus loin), le fait de 
poss\'eder une section globale n'est pas suffisant pour assurer la 
trivialit\'e.

Nous n'aborderons pas le probl\`eme de la classification des espaces
fibr\'{e}s, le lecteur interess\'e devrait consulter \cite{Husemoller}.

\subsection{Formes basiques, invariantes et horizontales}
\begin{description}
\item[Formes basiques]
L'existence de l'application de projection $\pi : P \mapsto M$ permet, comme
nous le savons, de projeter les vecteurs de $TP$ sur les vecteurs de $TM$,
en utilisant l'application tangente $\pi_*$ ; l'application cotangente, $\pi^*$, permet, quant \`a elle, de faire voyager les formes dans l'autre sens.
L'image, par $\pi^*$ d'une forme diff\'erentielle sur $M$ est une forme
particuli\`ere sur $P$ qu'on appelle une {\sl forme basique\/}\index{formes basiques}.
 On
obtient un homomorphisme injectif d'alg\`ebres diff\'erentielles
$$
\pi^* : \Omega(M) \mapsto \Omega(P)
$$
On peut donc identifier l'alg\`ebre $\Omega(M)$ avec la sous-alg\`ebre
des formes basiques  $\pi^*(\Omega(M)) \subset \Omega(P)$.
\item[Formes horizontales]
Une forme sur $P$ est horizontale, par d\'efinition, 
 si elle s'annule sur les vecteurs verticaux. Etant donn\'e que l'espace
tangent vertical en un point $z$ de $P$ est engendr\'e par les champs
 de vecteurs  fondamentaux $X_\alpha(z)$, avec $X_\alpha \in Lie(G)$,
  il suffit de tester l'annulation sur les champs en question. En d'autres
termes, soit $X\in Lie(G)$ et d\'esignons $i_X$ le produit int\'erieur
d'une forme par le champ $X(z)$~; 
la forme $\omega \in \Omega(P)$ est donc {\sl une forme horizontale\/}
\index{formes horizontales}
 si et seulement si, pour tout $X$, $$i_X \omega = 0$$

\item[Formes invariantes]
Puisque le groupe $G$ agit sur $P$, on peut s'int\'eresser \`a son action
infinit\'esimale sur les  formes  d\'ecrite par la d\'eriv\'ee de Lie
 $L_X = d\, i_X + i_X\, d$.
On dit qu'une forme $\omega$ est une {\sl forme invariante\/}\index{formes invariantes} si et seulement
si, pour tout $X$,
$$
L_X \, \omega = 0
$$

\item[Formes basiques (bis)]
Le fait qu'une forme basique soit \`a la fois invariante et horizontale est
assez intuitif. Formellement cette propri\'et\'e d\'ecoule imm\'ediatement 
de l'invariance $\pi(zg) = \pi(z)$ lorsque $g\in G$.
Retenons : La forme $\omega$ est une {\sl forme basique\/}\index{formes basiques} si et seulement
si, pour tout $X$,
$$
L_X \omega  = 0 \; {\mbox et} \; i_X \omega = 0
$$
c'est \`a dire si et seulement si  $\omega$ et $d\omega$ sont horizontales.

\item[Remarque : Op\'eration de Cartan]
Nous  nous servirons assez peu de ces notions de formes basiques,
de formes invariantes ou de formes horizontales, dans la suite de cet ouvrage.
 Cela dit, il faut bien noter que les notions qui viennent
d'\^etre discut\'ees fournissent une formulation alg\'ebrique assez compacte
de la notion d'espace fibr\'{e} principal (nous n'avons rien utilis\'e d'autre~!)
On peut, de fait, utiliser ces propri\'et\'es pour d\'efinir la notion
d'op\'eration (de Cartan) d'une alg\`ebre de Lie,  ${\frak g}$,
 sur une alg\`ebre diff\'erentielle commutative gradu\'ee $\Omega$ (c'est
bien le cas de l'alg\`ebre des formes diff\'erentielles sur une vari\'et\'e).
 On dit
qu'on a une {\sl op\'eration de Cartan\/}\index{op\'erations de Cartan} 
lorsqu'\`a tout $X \in {\frak G}$,
on associe une anti-d\'erivation $i_X$ (de degr\'e $-1$) et une d\'erivation
$L_X = di_X + i_X d$ (de degr\'e $0$) telles que, $\forall X,Y \in {\frak G}$,
on ait $L_{[X,Y]}=L_XL_Y-L_YL_X$ et $i_{[X,Y]}=L_Xi_Y-i_YL_X$.
La donn\'ee d'un fibr\'{e} principal $P$ fournit automatiquement une
 op\'eration de Cartan de $Lie (G)$ sur $\Omega(P)$ mais il est certain
que la notion d'op\'eration de Cartan est plus g\'en\'erale.
Dans ce cadre plus g\'en\'eral, on d\'efinit encore les sous espaces
 ${\frak H}$,
${\frak I}$ et ${\frak B}= {\frak H} \cap {\frak I}$ des formes horizontales, invariantes et basiques, et
on montre ais\'ement que ces trois sous-espaces de ${\Omega}$ sont des
sous-alg\`ebres diff\'erentielles gradu\'ees de ${\Omega}$.

\item[Remarque : Champs de vecteurs projetables]
Nous rappelons ici la d\'efinition des champs de vecteurs 
projetables\index{Champs de vecteurs projetables}
par une application diff\'erentiable (dans ce cas, il 
s'agit de la projection 
$\pi: P \mapsto M$ du fibr\'e consid\'er\'e), 
notion g\'en\'erale d\'ej\`a introduite au chapitre 1.
Ici l'ensemble des ant\'ec\'edents de $x \in M$ par $\pi$ n'est autre que la 
fibre au dessus du point $x$.
Un champ de vecteurs $V \in \Gamma TP$ est donc dit projetable si et 
seulement si $\pi_{*} V_{z} = \pi_{*} V_{zg}$ pour tout $g\in G$.

Plusieurs propri\'et\'es des espaces fibr\'es (et des connexions) 
pourraient s'enoncer en utilisant cette notion, que nous
n'utiliserons pas explicitement dans la suite.

\end{description}

\subsection{Exemples}\label{sec:Fibres-Exemples}
\subsubsection{Le fibr\'{e} des rep\`eres lin\'eaires}
Nous avons d\'ej\`a \'etudi\'e cet exemple en d\'etail en \ref{sec:exemple-fondamental}
 et nous verrons un peu plus loin divers exemples analogues.

\subsubsection{Fibration d'un groupe $G$ en sous groupes $H$ au dessus de 
$G/H$}

\begin{itemize}
	\item  Notre premier exemple sera \`a la fois \'el\'ementaire et discret~: notre 
espace total $P = \ZZ$ est l'ensemble des entiers relatifs et la 
projection $\pi$ est celle qui, \`a un entier, associe sa classe modulo $p$ 
($p$ d\'esignant un entier quelconque choisi une fois pour toutes). Le 
groupe structural est alors le sous-groupe $p\ZZ$ de $\ZZ$. Illustrons 
ceci, dans le cas $p=3$ par la figure 
\ref{fig:Z3}
\begin{figure}[htbp]
\epsfxsize=8cm
$$
    \epsfbox{entiersmod3.eps}
$$
\caption{Les entiers modulo 3}
\label{fig:Z3}
\end{figure}

L'espace total $\ZZ$ s'\'ecrit donc ici comme r\'eunion de trois fibres. La 
fibre type est le groupe additif des multiples de $3$, not\'e $3\ZZ$ et 
l'espace des fibres ---la base--- poss\`ede trois points~: ${\ZZ /
3\ZZ} = \{\overline{0},\overline{1},\overline{2}\}$ et est un quotient de 
$\ZZ$. La notation adopt\'ee, pour l'action du groupe structural sur 
l'espace total $\ZZ$ est ici une notation additive et non pas une 
notation multiplicative (mais cela devrait \^etre assez clair~!), ainsi, 
l'\'el\'ement $11$ de la fibre $\overline{2}$ peut s'obtenir \`a partir de 
l'\'el\'ement $-1$ de la m\^eme fibre sous l'action de $3\ZZ$ en \'ecrivant 
$11=-1 + {3} \times 4$.
	
	\item  Apr\`es cet exemple discret et extr\^emement \'el\'ementaire (l'art de 
reconsid\'erer des choses bien connues avec un \'eclairage diff\'erent $\ldots$ 
!) passons \`a un autre exemple, presque aussi \'el\'ementaire, mais 
``continu''. L'espace total, comme dans l'exemple pr\'ec\'edent, est un 
groupe, ici le groupe $SU(2)$. Rappelons, que 
$SU(2)$ est topologiquement identifiable \`a la sph\`ere $S^3$. Nous 
choisissons un sous-groupe $U(1)$, c'est \`a dire un grand cercle passant 
par l'origine et effectuons une d\'ecomposition en classes de $SU(2)$ par 
rapport \`a cet $U(1)$~: soit $g$ un \'el\'ement de $SU(2)$ qui n'appartienne 
pas au $U(1)$ choisi, on construit alors l'ensemble $\overline{g}=gU(1) = 
\{gh \vert h \in U(1)\}$. Ensuite, on choisit un \'el\'ement $k$ qui 
n'appartienne ni au $U(1)$ choisi, ni \`a $\overline{g}$. On construit alors 
$\overline{k}=kU(1)$ et on continue $\ldots$ On \'ecrit ainsi le groupe 
$SU(2)$ comme une r\'eunion (infinie) de classes du type 
$\overline{g}=gU(1)$, l'ensemble de ces classes \'etant par d\'efinition, 
l'ensemble quotient $SU(2)/U(1)$ qu'on identifie (voir le chapitre sur les
groupes et espaces homog\`enes) avec la sph\`ere $S^2$. Le groupe $SU(2)$ ---c'est 
\`a dire la sph\`ere $S^3$--- peut donc \^etre consid\'er\'e comme r\'eunion d'une 
infinit\'e de cercles $S^1$ param\`etr\'ee par la sph\`ere $S^2$.

\begin{figure}[htbp]
\epsfxsize=14cm
$$
    \epsfbox{HopfS3.eps}
$$
\caption{La fibration de Hopf pour $S^3$}
\label{fig:HopfS3}
\end{figure}

Bien \'evidemment, le 
groupe $U(1)$ agit, par multiplication \`a droite, sur l'espace total 
$SU(2)$ (dans la construction pr\'ec\'edente on a choisi $U(1)$ comme 
sous-groupe de $SU(2)$). Cette fibration en cercles de $S^3$ est souvent 
utilis\'ee et porte le nom de ``fibration de Hopf'' (pour $S^3$).

Avant de g\'en\'eraliser cet exemple, notons que la sph\`ere $S^3$ n'est, en 
aucun cas hom\'eomorphe au produit cart\'esien $S^2\times S^1$, ce qui ne 
l'emp\^eche pas d'\^etre un fibr\'{e} en cercles au dessus de $S^2$. En d'autres 
termes, les deux espaces fibr\'{e}s principaux $S^3 \mapsto S^2$ et 
$S^2\times S^1 \mapsto S^2$ (avec projection canonique \'evidente) ont m\^eme 
structure locale ---ils sont tous deux fibr\'{e}s en cercles au dessus de 
$S^2$--- mais le second est trivial alors que le premier ne l'est pas.
	
	\item  Passons maintenant \`a la g\'en\'eralisation de l'exemple qui pr\'ec\`ede. 
	Soit $G$ un groupe de Lie et $H$ un sous-groupe de Lie, qu'on supposera 
	ferm\'e dans $G$ pour que la topologie du quotient soit s\'epar\'ee. On 
	consid\`ere la relation d\'efinissant les classes \`a gauche de $H$~: $g$ et 
	$k$ sont reli\'es si $k$ appartient \`a l'ensemble $gH$. 
	Il s'agit d'une relation d'\'equivalence et on peut \'ecrire $G$ comme 
	r\'eunion de ses classes $gH$ ; l'ensemble des classes \'etant, par 
	d\'efinition, l'ensemble quotient $G/H$. Il est \'evident que $H$ agit (\`a 
	droite) sur $G$ et que l'application $G \mapsto G/H$ qui, \`a tout \'el\'ement 
	associe sa classe $\overline{g}=gH$, d\'efinit une fibration principale. 
	Tout groupe $G$ est donc ainsi un espace fibr\'{e} principal au dessus de 
	$G/H$, le groupe structural \'etant $H$. Notons que l'espace quotient (la 
	base du fibr\'{e}) $G/H$ n'est g\'en\'eralement pas un groupe, \`a moins que $H$ 
	ne soit un sous groupe distingu\'e de $G$, c'est \`a dire \`a moins que les 
	classes \`a gauche et \`a droite ne co\"incident.
	La propri\'et\'e qui pr\'ec\`ede est illustr\'ee par la figure \ref{fig:fibration 
	de G} et est \`a l'origine d'une multitude d'exemples que le lecteur 
	pourra construire en utilisant les donn\'ees ``zoologiques'' concernant 
	les groupes de Lie et les espaces homog\`enes (voir chapitre pr\'ec\'edent).
\begin{figure}[htbp]
\epsfxsize=8cm
$$
    \epsfbox{fibration-homogene.eps}
$$
	\caption{Fibration principale d'un groupe au dessus d'un espace homog\`ene}
	\label{fig:fibration de G}
	\end{figure}
	
	\item  Indiquons simplement ci-dessous quelques familles de fibrations principales, 
	bas\'ees sur la construction pr\'ec\'edente. Nous utilisons la notation 
	$H \longrightarrow G \longrightarrow G/H$ pour  caract\'eriser une telle 
	fibration de $G$ au dessus de $G/H$. Le nom figurant en titre est celui 
	donn\'e \`a l'espace quotient.
	\begin{description}
		\item[ Vari\'et\'es de Stiefel r\'eelles] 
		$$SO(m) \longrightarrow SO(n) \longrightarrow V_{n-m}(\RR^n)=SO(n)/SO(m)=O(n)/O(m)$$
		
		\item[Vari\'et\'es de Stiefel complexes]  
        $$SU(m) \longrightarrow SU(n) \longrightarrow 
        V_{n-m}(\CC^n)=SU(n)/SU(m)=U(n)/U(m)$$ 
		
		\item[Vari\'et\'es de Stiefel quaternioniques]
		$$Sp(m) \longrightarrow Sp(n) \longrightarrow V_{n-m}(\HH^n)=Sp(n)/Sp(m)$$ 
	
	Si $x=x_0+i x_1 + j x_2 + k x_3 \in \HH$, alors 
	$\overline{x}=x_0-i x_1 - j x_2 - k x_3$, et $(x\vert y)  =  \sum x_i 
	{\overline y_i}$.
	Comme dans le chapitre pr\'ec\'edent, la notation $Sp(n)$ d\'esigne le groupe 
	de Lie compact simplement connexe correspondant \`a la forme 
	r\'eelle compacte de l'alg\`ebre de Lie complexe $C_n$. Avec d'autres 
	notations : $Sp(n)=U(n,\HH) = \{u \in GL(\HH) \vert (u(x) \vert u(y))=(x\vert y)$. 
	Le groupe symplectique (non compact) usuel 
	correspondant \`a la m\^eme alg\`ebre de Lie $C_n$ sera 
	g\'en\'eralement plut\^ot d\'esign\'e par la notation $Sp(2n,\RR)$.
	
	
	\end{description}

	Le cas $m=n-1$ m\'erite une attention particuli\`ere puisque nous obtenons les 
	sph\`eres de cette fa\c con.
	\begin{eqnarray*}
	    S^{n-1}	 & = & SO(n)/SO(n-1) = O(n)/O(n-1) \\
		S^{2n-1} & = & SU(n)/SU(n-1) = U(n)/U(n-1) \\
		S^{4n-1} & = & Sp(n)/Sp(n-1)
	\end{eqnarray*}

	Noter que la m\^eme sph\`ere peut \^etre obtenue comme base de plusieurs 
	fibrations diff\'erentes de groupes de Lie (trois possibilit\'es si elle est 
	de dimension $4n-1$, deux possibilit\'es si elle est de dimension $2n-1$ 
	et une seule possibilit\'e si elle est de dimension paire). Il existe encore quelques autres
possibilit\'es dites ``exceptionnelles'' et nous y reviendrons plus loin.
	
	Si nous divisons les groupes orthogonaux $O(n)$ ou $SO(n)$ --- ou leurs analogues complexes ou 
	quaternioniques --- par un sous groupe maximal quelconque, nous obtenons plus g\'en\'eralement les 
	vari\'et\'es de Grassmann et les fibrations principales correspondantes~:
	 	\begin{description}
		\item[Grassmaniennes r\'eelles orient\'ees] 
		$$SO(m) \times SO(n-m) \longrightarrow SO(n) \longrightarrow 
		SG_m(\RR^n)  =  SO(n)/{SO(m) \times SO(n-m)}$$
		
		\item[Grassmaniennes complexes orient\'ees]  
        $$SU(m) \times SU(n-m) \longrightarrow SU(n) \longrightarrow 
        SG_m(\CC^n)  =  SU(n)/{SU(m) \times SU(n-m)}$$ 
		
		\item[Grassmaniennes quaternioniques]
		$$Sp(m) \times Sp(n-m) \longrightarrow Sp(n) \longrightarrow 
		G_m(\HH^n)  =  Sp(n)/{Sp(m) \times Sp(n-m)}$$  
	\end{description}
    
    Les Grassmaniennes non orient\'ees r\'eelles et complexes sont
    $$G_m(\RR^n)  =  O(n)/{O(m) \times O(n-m)} \; {\mbox et} \; 
G_m(\CC^n)  =  
    U(n)/{U(m) \times U(n-m)}$$
    
    
    Le cas $m=n-1$ m\'erite \'egalement une mention particuli\`ere puisque nous 
    obtenons ainsi les espaces projectifs r\'eels ($\RR P^n$), complexes 
    ($\CC P^n$) et quaternioniques ($\HH  P^n$).
    	\begin{eqnarray*}
	   \RR P^n	 & = & SO(n)/{SO(n-1)\times \ZZ_2}  \\
	   \CC P^n & = & SU(n)/{S(U(n-1) \times U(1))}  \\
	   \HH  P^n & = & Sp(n)/{S(Sp(n-1) \times SU(2))}
	\end{eqnarray*}

	Le lecteur devrait \'egalement conna\^{i}tre l'existence des diff\'eomorphismes 
	exceptionnels suivants~: $\CC P^1 \sim S^2$ et $\HH  P^1 \sim S^4$.
	Une remarque sur les notations~: $H=S(U(n-1) \times U(1))$ d\'esigne un 
	sous groupe maximal de $SU(n)$; on \'ecrit quelquefois $SU(n-1) \times 
	U(1)$ pour d\'esigner ce m\^eme sous groupe $H$ mais une telle notation est 
	un peu abusive puisque $H$ est en fait en quotient du produit direct de 
	ces deux groupes par un groupe discret (il ne faut pas compter l'unit\'e 
	deux fois~!). Les deux objets ont bien \'evidemment la m\^eme alg\`ebre de 
	Lie. Une remarque analogue s'applique au cas symplectique (par ailleurs 
	on se rappelle que $Sp(1)$ et $SU(2)$ sont isomorphes, ce qui explique 
	l'apparition de ce dernier dans le tableau pr\'ec\'edent).
	\item  Nous venons de voir que $G$ peut \^etre consid\'er\'e comme espace 
	fibr\'{e} principal {\it \`a droite\/} au dessus de $G/H = \{gH \vert g\in G\}$ mais il 
	peut \^etre \'egalement consid\'er\'e comme fibr\'{e} principal {\it \`a gauche\/} au 
	dessus de $H\backslash G =  \{Hg \vert g\in G\}$. L'\'etude de ce ce cas est 
	\'evidemment tout \`a fait semblable \`a celle que l'on vient de mener.
	
	\item Voici un autre cas particulier de la construction pr\'ec\'edente.
	On part du groupe $G\times G$ (produit direct du groupe $G$ avec lui m\^eme).
	On consid\`ere le sous-groupe diagonal $G_\Delta = \{(g,g)\subset G\times G \vert g\in G\}$
	qui est d'ailleurs isomorphe \`a $G$ et on fabrique le quotient. En r\'esum\'e, on
	a un fibr\'{e} principal d'espace total $G\times G$, de fibre type $G_\Delta \sim G$ et de
	base ${G\times G \over G_\Delta} \sim G$. La base est elle-m\^eme, en tant que vari\'et\'e,
	diff\'eomorphe avec $G$, mais la structure de groupe ne passe pas au quotient puisque
	$G_\Delta$ n'a aucune raison d'\^etre distingu\'e dans $G\times G$.
	Posant $G_L = G\times 1$ et $G_R = 1 \times G$, on voit qu'on peut \'ecrire
	$G\sim G_L\, G_R /G_\Delta$. Cet exemple d'apparence
	innocent est assez subtil \`a analyser et pourra servir, dans la suite,
	pour discuter de connexions (ou de m\'etriques) particuli\`eres sur les groupes de Lie.
	On peut aussi consid\'erer les projections
	$G_L \, G_R \mapsto G_L \sim G$ et $G_L \, G_R \mapsto G_R \sim G$ qui d\'efinissent deux
	autres fibr\'{e}s principaux (cette fois-ci, la structure de groupe passe au quotient).


	\item  Finalement, consid\'erons le cas d'un sous groupe $G$ qui n'est pas 
	simplement connexe. On sait qu'il admet un rev\^etement universel 
	simplement connexe $\widehat{G}$ et que $G$ est isomorphe au groupe quotient 
	$\widehat{G}\vert H$ o\`u  $H$ est un sous groupe discret (distingu\'e) du centre 
	de $G$ isomorphe au groupe d'homotopie $\pi_1(G)$. On se retrouve donc 
	dans la situation consid\'er\'ee pr\'ec\'edemment d'une fibration de 
	$\widehat{G}$ au dessus de $G=\widehat{G} \vert H$ avec groupe structural (fibre 
	type) $H$, \`a ceci pr\`es que le groupe $H$ est ici un groupe discret 
	admettant  une interpr\'etation topologique particuli\`ere et que le quotient 
	$G$ est non seulement un espace homog\`ene, mais est lui-m\^eme un groupe. 
	Plus g\'en\'eralement d'ailleurs, tout sous groupe $K \subset H \sim 
	\pi_1(G)$ d\'efinit un rev\^etement $\widehat{G}/K$ qui est un fibr\'{e} principal 
	au dessus de $G=\widehat{G}/H$ avec fibres $H\vert K$ (c'est bien un groupe 
	puisque $H$ est ab\'elien); ce rev\^etement n'est pas universel puisque son 
	$\pi_1$ est \'egal \`a $K$. Tout ceci est presque intuitif si on se 
	repr\'esente ces fibrations par des figures telles que \ref{fig:revetements}.
\begin{figure}[htbp]
\epsfxsize=8cm
$$
    \epsfbox{revetement.eps}
$$
	\caption{Rev\^etements}
	\label{fig:revetements}
	\end{figure}
\end{itemize}


\subsubsection{Fibration d'un espace homog\`ene $G/H_1$ en groupes $H_2$ au 
dessus de $G/(H_1 \times H_2)$}

Soit $H$ un sous groupe de Lie d'un groupe de Lie $G$ et supposons que 
$H$ soit isomorphe au produit $H_1\times H_2$ de deux groupes de Lie. On 
peut alors consid\'erer $H_1$ (en fait $H_1 \times \mbox{Identit\'e}$) comme 
sous groupe de $G$ et on a une projection $G/H_1 \mapsto G/(H_1\times 
H_2)$ de fibre $H_2$. L'action de $H_2$ (\`a droite) sur $G/H_1$ est bien 
d\'efinie car $H_1$ et $H_2$ commutent, et donc $(gH_1)h_2 = (gh_2)H_1$ 
lorsque $h_2$ appartient \`a $H_2$. Vu la diversit\'e des cas \`a consid\'erer 
nous n'\'enoncerons aucun r\'esultat pr\'ecis dans ce cas. N\'eanmoins nous 
\'enoncerons les trois remarques suivantes~:
\begin{enumerate}
	\item  ``En g\'en\'eral'' la situation pr\'ec\'edente conduit \`a un fibr\'{e} 
	principal $H_2\longrightarrow G/H_1 \longrightarrow G/(H_1 \times H_2)$ 
	de groupe structural $H_2$.
	
	\item  Bien souvent, et en particulier lorsque $G$ est un groupe simple, 
	le sous groupe $H$ consid\'er\'e n'est pas isomorphe au produit $H_1 \times 
	H_2$ de deux groupes de Lie, mais au quotient d'un tel produit par un 
	groupe discret (on a donc $\mbox{Lie}H = \mbox{Lie}H_1 \oplus 
	\mbox{Lie}H_2$ au niveau des alg\`ebres de Lie). Dans ce cas, le r\'esultat 
	``g\'en\'eral'' pr\'ec\'edent est valable \`a condition de quotienter correctement 
	par le groupe discret appropri\'e.
	
	\item  Le lecteur pourrait \^etre \'egalement tent\'e de consid\'erer des 
	doubles classes $K\backslash G/H$ o\`u  $H$ et $K$ sont deux sous groupes de $G$. 
	Attention~: la projection $G/H \mapsto K\backslash G/H$ {\it ne d\'efinit en 
	g\'en\'eral pas\/} une fibration principale, ni m\^eme une fibration, car le 
	type topologique des fibres (ou m\^eme la cardinalit\'e) peut varier d'un point \`a 
	l'autre de la base.
\end{enumerate}

Afin de conclure cette sous section consacr\'ee aux exemples par un 
th\'eor\`eme pr\'ecis concernant les fibrations principales d'espaces 
homog\`enes, nous consid\'erons maintenant le cas suivant.

\subsubsection{Fibration principale de $G/H$ en groupes $N\vert H$ au 
dessus de $G/N$, $N$ \'etant le normalisateur de $H$ dans $G$}

Soit $H$ un sous groupe de Lie du groupe de Lie $G$ et soit $N$ son 
normalisateur dans $G$. On rappelle que $N = \{n\in G \vert nH=Hn \}$~; 
en d'autres termes, $N$ est le plus grand sous groupe de $G$ dans lequel 
$H$ est un sous groupe normal (on dit aussi sous groupe distingu\'e). $H$ 
\'etant normal dans $N$, il s'ensuit que les classes \`a gauche et \`a droite 
de $N$ par rapport \`a $H$ co\"incident (voir ci-dessus la d\'efinition de $N$) 
et que l'espace homog\`ene $N\vert H  =  N/H = H\backslash N$ poss\`ede une 
structure de groupe. Par ailleurs, $N$ agit \`a droite sur $G/H$~: soit $gH 
\in G/H$ et $n \in N$~; alors $gHn  =  gnH \in G/H$. Cette action 
n'est pas fid\`ele car les 
\'el\'ements de $H$ lui-m\^eme n'agissent pas~: si $h \in H$, alors $gHh=gH$. 
Le fait de quotienter $N$ par $H$ rend pr\'ecis\'ement cette action 
 fid\`ele. 
 On peut se repr\'esenter les actions de $G$ \`a gauche de $G/H$ et de $N\vert 
H$, \`a droite de $G/H$ par le sch\'ema:
$$
G \longrightarrow G/H \longleftarrow N\vert H
$$
En utilisant seulement l'action \`a droite, on obtient ainsi une fibration principale dont l'espace 
total est $G/H$ et le groupe structural est $N\vert H$, il est 
facile de voir que la base de la fibration est l'espace homog\`ene $G/N$ 
(voir figure \ref{fig:normalisateur}).
\begin{figure}[htbp]
\epsfxsize=8cm
$$
    \epsfbox{normalisateur.eps}
$$
\caption{Fibration principale associ\'ee \`a l'action du groupe $N\vert H$ 
sur l'espace homog\`ene $G/H$}
\label{fig:normalisateur}
\end{figure}

Ce type de fibration principale est \'egalement \`a l'origine d'une multitude 
d'exemples. Les fibrations de Hopf des sph\`eres au dessus des espaces 
projectifs r\'eels, complexes ou quaternioniques sont d'ailleurs de ce 
type. En effet, on a

\bigskip

\begin{tabular}{llll}
$ G=SO(n)$ & $ H = SO(n-1) $ & $ N=SO(n-1)\times \ZZ_2$ & $N\vert H=\ZZ_2 $\\
\end{tabular}


\begin{tabular}{ll}
{\hskip 1cm} $G/H=S^{n-1}$ & $G/N  = \RR P^n$\\
\end{tabular}

\smallskip

\begin{tabular}{llll}
$ G=SU(n)$  & $ H = SU(n-1) $  &  $ N=SU(n-1)\times U(1) $  &  $ N\vert H=U(1) $ \\
\end{tabular}


\begin{tabular}{ll}
{\hskip 1cm} $ G/H = S^{2n-1} $  &  $ G/N = \CC P^{n-1} $ \\
\end{tabular}

\smallskip

\begin{tabular}{llll}
$ G=Sp(n)  $  &  $ H = Sp(n-1) $  &  $ N=Sp(n-1)\times SU(2) $  &  $ N\vert H=SU(2)  $ \\
\end{tabular}


\begin{tabular}{ll}
{\hskip 1cm} $ G/H=S^{4n-1}$  & $G/N  = \HH P^n$\\
\end{tabular}
	
\bigskip
	et on peut illustrer les fibrations correspondantes par la figure 
	\ref{fig:spheres}
\begin{figure}[htbp]
\epsfxsize=12cm
$$
    \epsfbox{Hopf.eps}
$$
	\caption{Fibrations de Hopf des sph\`eres}
	\label{fig:spheres}
	\end{figure}

	On se souvient aussi que $Z_2 \equiv S^0$, $U(1) \equiv S^1$ et 
	$SU(2)\equiv S^3$~; ainsi les trois fibres types repr\'esent\'ees sur la 
	figure \ref{fig:spheres} sont non seulement des groupes, mais aussi des 
	sph\`eres.
	
	Le lecteur pourra fabriquer ais\'ement d'autres exemples de ce type en 
	choisissant, pour tout groupe $G$ donn\'e, un sous groupe $H$ qui ne soit 
	pas trop ``gros'' (de fa\c con \`a ce que $N\vert H$ ne soit pas trop 
	trivial). Voici un dernier exemple de ce type qui utilise les groupes de 
	Lie exceptionnels~: $G = E_8$, $H=E_6$ ,
  $N=(E_6\times SU(3))/\ZZ_3$, $N\vert H=SU(3)/\ZZ_3$.

\subsubsection{Fibrations exceptionnelles des sph\`eres et des espaces 
projectifs}

Il existe des fibrations exceptionnelles des sph\`eres et des espaces 
projectifs qui ne sont pas li\'ees aux
inclusions de groupes unitaires, orthogonaux ou symplectiques (forme compacte)
c'est \`a dire aux structures r\'eelles, complexes ou 
quaternioniques. Certaines de ces fibrations sont li\'ees \`a l'existence
de l'``alg\`ebre'' non-associative des octaves de Cayley ${\mathcal O}$
({\sl octonions\/}\index{octonions}). On sait que pour $n=1,2,4,8$ (et ce sont
les seules valeurs possibles),
 il existe une op\'eration bilin\'eaire $\RR^n\times \RR^n \mapsto \RR^n$ sans diviseurs
de z\'ero (c'est \`a dire que $a \times b = 0 \Rightarrow \; a = 0 \; \mbox{ou}\; b = 0$), conduisant \`a la d\'efinition des
 corps $\RR$, $\CC$, $\HH$ et des octaves~${\mathcal O}$.

 Certaines des fibrations mentionn\'ees ici ne sont pas des fibrations
 principales (en particulier la fibre type n'est pas un groupe)
mais elles y ressemblent beaucoup (on sait que la sph\`ere $S^7$, par 
exemple, est presque un groupe$\ldots$) Nous donnons ici une liste de 
fibrations qui sont \`a la fois int\'eressantes et c\'el\`ebres (la 
fibration de Hopf exceptionnelle de $S^{15}$~!) bien qu'elles ne 
s'inscrivent pas logiquement toutes dans cette section puisqu'il ne s'agit pas 
toujours de fibrations principales.
 Nous ne les utiliserons pas dans la suite et 
ne les mentionnons que pour des raisons culturelles, en esp\'erant que le 
lecteur pourra y retourner (soit-dit en passant, il reste \`a \'etudier de nombreux 
probl\`emes int\'eressants concernant ces objets).

\smallskip
$$\begin{matrix} 
\mbox{fibre}   &  \longrightarrow   &  \mbox{espace total}   &  \longrightarrow 
  &  \mbox{base} \\
{} & {} & {} & {} & {} \\
  S^7    &   \longrightarrow    &   S^{15}    &   \longrightarrow 
   &   S^8  \\
 Spin(7)    &   \longrightarrow    &   Spin(9)    &   \longrightarrow 
   &   S^{15}  \\
 SU(3)    &   \longrightarrow    &   G_2   &   \longrightarrow 
   &   S^6  \\
 \CC P^1=S^2    &   \longrightarrow    &   \CC P^{2n+1}    &   \longrightarrow 
   &   \HH P^n \\
 Spin(9)    &   \longrightarrow    &   F_4    &   \longrightarrow 
   &   {\mathcal O} P^2 
\end{matrix} 
$$

\section{Fibr\'es associ\'es}
\subsection{Introduction}
Comme nous l'avons vu pr\'ec\'edemment, \`a une vari\'et\'e diff\'erentiable donn\'ee, 
on peut attacher l'ensemble de tous les rep\`eres, et cet ensemble, qu'on 
d\'esigne sous le nom de {\sl fibr\'{e} des rep\`eres \/}\index{fibr\'{e} des rep\`eres}
 poss\`ede une structure 
d'espace fibr\'{e} principal. Il est d'autres ensembles qu'on peut attacher 
 \`a une vari\'et\'e donn\'ee, par exemple, l'ensemble de tous 
ses vecteurs tangents, ou l'ensemble de tous ses tenseurs de type donn\'e. 
Ces diff\'erents ensembles sont, d'une fa\c con que nous allons rendre 
pr\'ecise, ``associ\'es'' au fibr\'{e} des rep\`eres, en ce sens que le groupe 
structural --- le groupe lin\'eaire dans ce cas --- agit \'egalement sur les 
composantes des vecteurs, tenseurs \etc

Plus g\'en\'eralement, nous allons d\'efinir des fibr\'{e}s associ\'es en 
``rempla\c cant'' le groupe structural d'un fibr\'{e} principal par un ensemble 
sur lequel ce groupe op\`ere. D'un certain point de vue, on peut dire que 
les groupes eux-m\^emes n'ont un int\'er\^et que parce qu'ils agissent 
(op\`erent) sur des ensembles bien choisis et cette th\'eorie des actions de 
groupe --- que nous avons sommairement d\'ecrite dans la deuxi\`eme partie de 
cet ouvrage --- est particuli\`erement
riche lorsqu'il s'agit d'une action lin\'eaire sur un espace vectoriel 
(th\'eorie des repr\'esentations). Les groupes sont donc des ``machines \`a 
agir sur des espaces''. D'une fa\c con analogue, nous allons consid\'erer les
fibr\'{e}s principaux comme des ``machines \`a fabriquer des fibr\'{e}s associ\'es'' 
et la th\'eorie sera particuli\`erement riche lorsque ces fibr\'{e}s associ\'es 
seront fabriqu\'es \`a l'aide d'une repr\'esentation de groupe sur un espace 
vectoriel (th\'eorie des fibr\'{e}s vectoriels).

\subsection{Espaces fibr\'{e}s associ\'es g\'en\'eraux}\label{sec:associes}

Soit $P \stackrel{\pi}\mapsto M$ un espace fibr\'{e} principal (\`a droite), de 
groupe structural $G$, et soit $\rho$ une action (\`a gauche) de $G$ sur un 
ensemble $F$. On obtient alors une relation d'\'equivalence sur $P\times F$ 
en disant que $(z,f)\in P\times F$ est \'equivalent \`a $( {z'}, 
 {f'}) \in P\times F$ s'il existe un \'el\'ement $g$ de $G$ qui soit tel 
que $ {z'}=zg$ et $ {f'}=\rho(g^{-1}) f$. L'ensemble quotient $E 
= P \times_G F$ prend le nom de fibr\'{e} associ\'e \`a $P$ via l'action de $G$ 
sur $F$. En d'autres termes, on identifie $(z,f)$ avec 
$(zg,\rho(g^{-1}f))$. Cette d\'efinition un peu abstraite ne devrait pas 
rebuter le lecteur, en effet elle correspond \`a une situation bien connue 
: supposons l'action $\rho$ fix\'ee une fois pour toutes et notons 
$g^{-1}f$ l'objet que nous notions un peu plus haut $\rho(g^{-1}) f$~; 
par ailleurs, d\'esignons par $z.f$ la classe de $(z,f)$~; l'\'el\'ement 
$u=z.f$ de $E$ n'est donc rien d'autre que l'objet g\'eom\'etrique qui 
poss\`ede les ``composantes'' $f$ dans le ``rep\`ere'' $z$ et les ``composantes'' 
$g^{-1}.f$ dans le ``rep\`ere'' $zg$, en effet, $u = z.f =zg.g^{-1}f$. On voit donc 
ici que $u$ g\'en\'eralise la notion classique et \'el\'ementaire de ``vecteur''. 
Nous verrons un peu plus loin comment r\'ecup\'erer la notion d\'ej\`a introduite 
de vecteur tangent \`a une vari\'et\'e par cette construction.

L'espace $E$ est bien un espace fibr\'{e} et on a une projection, encore 
not\'ee $\pi$, de $E$ sur $M$, d\'efinie par $\pi(z.f)  =  \pi(z)$ o\`u  le 
$\pi$ du membre de droite se r\'ef\`ere \`a la projection dans le fibr\'{e} 
principal correspondant. Il est bien clair que cette d\'efinition ne d\'epend 
pas du choix du repr\'esentant choisi (puisque les diff\'erents $z$ possibles 
sont tous dans la m\^eme fibre~!) On se souvient, par ailleurs, qu'il est 
parfaitement l\'egitime et non ambigu de noter le point $x = \pi(z)$ de $M$ 
sous la forme $x=z G$ puisqu'il existe une correspondance 
bi-univoque entre points de $M$ et fibres de $P$. Par ailleurs,
la fibre de la nouvelle projection $\pi$ (dans $E$) \'etant, par construction, 
hom\'eomorphe \`a $F$, on a donc, de fait, ``remplac\'e'' $G$ par $F$, ce qui 
justifie de repr\'esenter cette construction, associant $E$ \`a $P$, par la figure suivante (fig. \ref{fig:fibre associe}):
\begin{figure}[htbp]
\epsfxsize=14cm
$$
    \epsfbox{fibre-associe.eps}
$$
\caption{Passage d'un fibr\'{e} principal \`a un fibr\'{e} associ\'e}
\label{fig:fibre associe}
\end{figure}

On dit que $G$ est le groupe structural du fibr\'{e} associ\'e $E$ (attention, 
dans le cas des fibr\'{e}s associ\'es, le groupe structural $G$ n'a aucune 
raison d'\^etre diff\'eomorphe \`a la fibre type $F$). Notons enfin que 
$\mbox{dim}\, E=\mbox{dim}\, M + \mbox{dim}\, F$.

Avant de donner quelques exemples de tels espaces, il importe de noter 
que, sauf exceptions, le groupe structural $G$ {\it n'agit pas\/} sur le 
fibr\'{e} associ\'e $E$ puisque $E$ est pr\'ecis\'ement obtenu via un quotient de 
l'action simultan\'ee de $G$ sur $P$ (c'est \`a dire sur les ``rep\`eres'') et 
sur $F$ (c'est \`a dire les ``composantes'').\par
 Une situation famili\`ere, bien 
connue du lecteur, nous est fournie par l'exemple des espaces vectoriels~:
 
Soit $E$ un espace vectoriel de dimension $n$~; les \'el\'ements de $E$ sont 
nos vecteurs familiers~; il faut bien voir que le groupe lin\'eaire 
$GL(n,\RR)$, d\'efini comme groupe de matrices, ne sait pas 
comment agir sur les vecteurs si aucune base n'a �t� choisie. Par contre, il sait agir sur les bases 
de $E$ (il fait passer d'une base \`a l'autre) et, une base \'etant choisie, il sait \'egalement agir 
sur les composantes des vecteurs de $E$. Il existe bien un groupe qui 
sait agir sur les vecteurs eux-m\^emes, c'est le groupe $Aut \, E$ des 
automorphismes de $E$, mais ce groupe ne peut s'identifier \`a $GL(n,,\RR)$ 
que moyennant le choix d'une base. Un espace vectoriel usuel n'est autre 
chose qu'un espace fibr\'{e} sur un point (la base est un point et la fibre 
s'identifie \`a l'espace vectoriel lui-m\^eme). Apr\`es quelques moments de 
r\'eflexion pass\'es \`a examiner ce cas assez trivial, mais instructif, le 
lecteur pourra sans doute se demander quel est l'objet g\'en\'eralisant $Aut 
\, E$ lorsqu'on passe de la situation bien connue \'evoqu\'ee ci-dessus au 
cas des espaces fibr\'{e}s plus g\'en\'eraux o\`u  la base est, en g\'en\'eral, une 
vari\'et\'e. Il se trouve que ce groupe $Aut E$ admet une g\'en\'eralisation, 
c'est \`a dire qu'il existe bien un groupe qui agit sur $E$~: c'est un 
objet d\'esign\'e sous le nom de  {\sl groupe de jauge\/}\index{groupe de jauge} et son \'etude fera 
l'objet de la
section \ref{sec:groupe-de-jauge}. Nous verrons qu'il est, en g\'en\'eral, 
de dimension infinie.

Une des  conclusions que nous voulons tirer de la pr\'esente discussion est 
la suivante~: le groupe structural $G$ d'un fibr\'{e} associ\'e n'agit pas sur 
l'espace fibr\'{e} associ\'e en question~; il y a bien un groupe $Aut \, E$ qui 
agit sur $E$, mais ce groupe ne co\"incide pas avec $G$.

\subsection{Espaces fibr\'{e}s en espaces homog\`enes, associ\'es \`a un fibr\'{e} 
principal de groupe structural $G$}\label{sec:fibres-homogenes}
Soit $P = P(M,G)$ un fibr\'{e} principal et $H$ un sous groupe de Lie du 
groupe structural $G$. On consid\`ere l'action \`a gauche de $G$ sur l'espace 
homog\`ene $F=G/H$ et on construit, en suivant la m\'ethode de construction 
g\'en\'erale des fibr\'{e}s associ\'es, l'ensemble $E = P \times_G G/H$. Les fibres 
de $E$ sont diff\'eomorphes \`a l'espace homog\`enes $G/H$ et la base est toujours $M$. La 
dimension de $E$ est donc \'egale \`a $dim \, M \, + dim \, G/H = dim \, M \, 
+ dim \, G - dim \, H$ et on peut repr\'esenter $E$ \`a l'aide de la figure suivante (fig. \ref{fig:Fibre en G/H})~:
\begin{figure}[htbp]
\epsfxsize=8cm
$$
    \epsfbox{fibre-homogeneII.eps}
$$
\caption{Fibr\'e associ\'e en espaces homog\`enes}
\label{fig:Fibre en G/H}
\end{figure}

On peut noter $E = P \, \mbox{mod} \, H$ ou simplement $E = P/H$.

A l'aide de cette m\'ethode g\'en\'erale et des exemples de fibr\'{e}s principaux donn\'es pr\'ec\'edemment,
on peut ainsi fabriquer une foule de nouveaux espaces. 
En voici quelques exemples~: 
\begin{figure}[htbp]
\epsfxsize=15cm
$$
    \epsfbox{exemples-homogenes.eps}
$$
\caption{Exemples de fibr\'{e}s associ\'es en espaces homog\`enes}
\end{figure}

\subsection{Fibration principale relative \`a un fibr\'{e} quotient}\label{sec:fibre-quotient}
La figure ci-dessous (\ref{fig:PsurE}),
le fait que $\mbox{$dim P = dim E +  dim H$}$ et le fait que $G$ soit 
lui-m\^eme un $H$-fibr\'{e} principal au dessus de $G/H$,
 sugg\`erent que l'espace total $P$ du fibr\'{e} 
principal $P(M,G)$ dont on est parti peut \'egalement \^etre consid\'er\'e comme fibr\'{e} 
principal $P(E,H)$ de fibre $H$ au-dessus du fibr\'{e} associ\'e $E = P\times_G 
G/H$. Il en est effectivement ainsi.

Soit $z \in P(M,G)$, on consid\`ere l'application $p : P \mapsto E = P 
\times_G G/H = P\, \mbox{mod}\, H$ d\'efinie par $p(z)=(z,eH)$, o\`u  $e$ 
d\'esigne l'\'el\'ement neutre du groupe $G$. La fibre passant par $z$ de cette 
application est simplement $zH$ puisque 
$$\forall h \in H, \, p(zh)=(zh,eH)=(z,heH)=(z,eH)=p(z)$$
On obtient donc ainsi un nouveau fibr\'{e} principal $Q(E,H)$ poss\'edant le 
m\^eme espace total que $P(M,G)$ mais cette fois-ci avec une base $E = P\, \mbox{mod}\, H$
et un groupe structural $H$.
Pour tout choix d'un sous groupe $H$ de $G$, on obtient 
ainsi une deuxi\`eme fibration principale de l'espace $P$ repr\'esent\'ee par 
la figure (\ref{fig:PsurE}).
\begin{figure}[htbp]
\epsfxsize=14cm
$$
    \epsfbox{double-fibration.eps}
$$
\caption{Deuxi\`eme fibration principale de $P$~: les espaces P(M,G) et P(E,H)}
\label{fig:PsurE}
\end{figure}

\subsection{Espaces fibr\'{e}s en $\ldots$ espaces fibr\'{e}s}
Voici une famille d'exemples assez surprenante~: on se donne $P_1 =  
P_1(M_1,G)$ et $P_2(M_2,G)$, deux espaces fibr\'{e}s principaux poss\'edant le 
m\^eme groupe structural. On supposera, de plus, que $P_1$ est un espace 
fibr\'{e} \`a droite ---comme d'habitude--- mais que $P_2$ est un espace fibr\'{e} \`a 
gauche, ce qui n'est pas vraiment une restriction puisqu'on peut toujours 
passer d'une action \`a droite \`a une action \`a gauche
(voir le chapitre sur les actions de groupes).
On va alors fabriquer un fibr\'{e} associ\'e en choisissant
 $P=P_1$, $F=P_2$ et 
en suivant la m\'ethode g\'en\'erale de construction des fibr\'{e}s associ\'es. On 
obtient ainsi un espace $E=P_1 \times_G P_2$ dont la base est $M_1$ et dont 
la fibre type est $P_2$.

Voici un exemple de cette construction. Soit $P_1 = G = P_1(G/H,H)$, un 
groupe de Lie fibr\'{e} en sous groupes de type $H$ au dessus de $G/H$ et $P_2 = K = P_2(H\backslash K,H)$, 
un autre groupe de Lie fibr\'{e} en sous groupe de type $H$ au dessus de 
$H\backslash K$~; on fabrique alors $E=G\times_H K$ qui a pour base $G/H$ et pour fibre type $K$. Une situation encore plus 
particuli\`ere correspond au choix $G=K$.

\subsection{Le fibr\'{e} adjoint $E=Ad \, P$}\label{sec:fibre-adjoint}
Soit $P=P(M,G)$ un fibr\'{e} principal. On peut faire agir $G$ sur lui-m\^eme 
via l'action adjointe $g\in G, \, Ad(g)k=gkg^{-1}$. On choisit alors 
$F=G$, $\rho=Ad$, et on construit $E=P\times_{Ad}G$, fibr\'{e} not\'e 
habituellement $Ad \, P$. Cet espace fibr\'{e} associ\'e a ceci de particulier que sa 
fibre type est un groupe de Lie ---c'est le groupe structural lui-m\^eme--- 
et donc, au niveau du ``dessin'', rien ne le distingue de $P$, puisqu'ils 
ont tous deux m\^eme base $M$ et m\^eme fibre type $G$. En revanche, $G$ 
op\`ere, comme il se doit, sur le fibr\'{e} principal $P$, alors qu'il ne sait pas agir sur $Ad \, 
P$. Cet exemple illustre bien la n\'ecessit\'e d'imposer la condition $2$ 
dans la d\'efinition des fibr\'{e}s principaux (voir section \ref{sec:principal}). 
A tout fibr\'{e} principal $P$, on peut donc associer un {\sl fibr\'{e} en groupes\/}
\index{fibr\'{e} en groupes} 
$Ad \, P$, dont l'importance s'av\'erera essentielle (nous verrons plus 
tard que les sections de $Ad \, P$ sont les {\sl transformations de 
jauge\/})\index{transformations de jauge locales}
\index{automorphismes verticaux}. Notons pour terminer que $G$ 
agit non seulement sur lui-m\^eme par l'action adjointe $Ad$ mais aussi sur 
$Lie(G)$ par l'action adjointe $ad$ d\'efinie par $ad(g)X=gXg^{-1}$, o\`u  
$X$ appartient \`a l'alg\`ebre de Lie de $G$. La construction g\'en\'erale 
peut encore \^etre effectu\'ee dans ce cas, et on fabrique ainsi le fibr\'{e} associ\'e $ad \,P = P 
\times_G Lie(G)$ qui est un fibr\'{e} en alg\`ebres de Lie, de base $M$.

\subsection{Le r\^ole du normalisateur}
\begin{itemize}
	\item  On vient de construire un nouvel espace fibr\'{e} en faisant
	 agir $G$ sur lui-m\^eme par l'action adjointe. On 
	pourrait se demander pourquoi ne pas faire tout simplement agir $G$ sur 
	lui-m\^eme par multiplications \`a gauche, et fabriquer le fibr\'{e} associ\'e 
	correspondant. Bien sur, on le peut, mais alors, on n'obtient ainsi rien 
	de neuf~! En effet, partons de $P=P(M,G)$, fibr\'{e} principal ( \`a droite) 
	et construisons $E = P\times_G G$ via l'action (multiplication) \`a gauche 
	de $G$ sur $G$. La prise du quotient identifie ces deux actions ---\`a 
	droite de $P$ et \`a gauche de $G$--- et ces deux actions s'annihilent donc 
	mutuellement (voir la remarque en fin de section \ref{sec:associes}). Par 
	contre, il existe encore une action de $G$ \`a droite de la fibre $F=G$, 
	de sorte que l'espace obtenu $E$ s'identifie canoniquement \`a $P$ 
	lui-m\^eme. La construction n'offre donc aucun int\'er\^et.
	
	\item  Dans le cas de fibrations en espaces homog\`enes du type 
	$E=P\times_G G/H$, nous avons vu que l'action de $G$ disparaissait, {\it 
	en g\'en\'eral\/}, au niveau de $E$. L'exemple qui pr\'ec\`ede (o\`u  $H$ se d\'eduit 
	\`a l'identit\'e) offre un bon contre-exemple, mais il s'agit l\`a d'une 
	situation un peu extr\^eme$\ldots$ On pourrait se demander s'il existe des 
	situations interm\'ediaires, c'est \`a dire des situations o\`u  il existe 
	encore une certaine action {\it \`a droite\/} au niveau du fibr\'{e} associ\'e 
	$E$. La r\'eponse est simple et a d\'ej\`a \'et\'e trouv\'ee
	dans notre \'etude succincte des espaces homog\`enes des 
	groupes de Lie~: l'espace $G/H$ est toujours muni d'une action de $G$ 
	\'evidente, du c\^ot\'e gauche, mais \'egalement d'une action \`a droite du {\it 
	groupe\/} $N\vert H$ o\`u  $N$ d\'esigne le normalisateur de $H$ dans $G$~; 
	en effet, on peut \'ecrire $(gH)n=(gn)H$ si $n$ est un \'el\'ement de $N$. 
	Sch\'ematiquement, on a $$G \rightarrow G/H \leftarrow N\vert H$$ Le 
	groupe $N\vert H$ agit donc toujours, \`a droite, sur l'espace fibr\'{e} 
	$E=P\times_G G/H$. Bien souvent, ce groupe $N\vert H$ est discret, mais 
	il peut ne pas l'\^etre. On a \'egalement le cas extr\^eme o\`u  $N$ et $G$ 
	co\"incident~; dans un tel cas, $H$ est sous groupe distingu\'e de $G$, 
	l'espace homog\`ene $G/H$ est un groupe, et $E$, muni de cette action \`a 
	droite, devient un fibr\'{e} principal. Dans le cas g\'en\'eral o\`u  $N$ et $G$ 
	sont distincts, et o\`u  $H$ n'est pas trivial, il faut se rappeler (voir 
        section \ref{sec:Fibres-Exemples}) que $G/H$ est lui m\^eme un $N\vert H$ fibr\'{e} 
	principal au dessus de $G/N$~; on fabrique ainsi une projection de $E$ 
	sur $M \times G/N$ et $E$ peut alors \^etre consid\'er\'e comme $N\vert H$ 
	fibr\'{e} principal au dessus de $M\times G/N$.
\end{itemize}

\subsection{Les espaces fibr\'{e}s vectoriels}

``A tout seigneur, tout honneur'', voici les espaces fibr\'{e}s vectoriels, 
espaces qui tiennent une place de choix dans la th\'eorie des espaces 
fibr\'{e}s, et dont l'\'etude peut se faire (et se fait souvent) de fa\c con 
ind\'ependante de la notion de fibr\'{e} principal. Dans notre approche, 
cependant, les fibr\'{e}s vectoriels sont des espaces fibr\'{e}s associ\'es 
comme les autres, \`a cette diff\'erence pr\`es que la fibre $F$ 
choisie est un espace vectoriel ($\RR^n$ ou $\CC^n$) et que l'action 
$\rho$ de $G$ sur $F$ est une repr\'esentation de $G$ sur cet espace 
vectoriel. Nous devons donc nous r\'ep\'eter~: soit $P=P(M,G)$ un fibr\'{e} 
vectoriel et $\rho$ une repr\'esentation de $G$ sur l'espace vectoriel $F$ 
; on construit le fibr\'{e} vectoriel $E = P \times_G F$. Les 
\'el\'ements $\overrightarrow{ v}$ de $E$ sont vraiment ici des 
``vecteurs'' (gardons la fl\`eche pour le moment) et on pourra 
sans danger ---et avec profit--- utiliser une notation ``avec 
des indices''. Comme on l'a vu en section \ref{sec:associes}, l'\'el\'ement 
$\overrightarrow{ v}$ de $E$ peut s'\'ecrire
${\overrightarrow{ v}} = (\epsilon).(v)=(\epsilon 
g).(\rho(g^{-1}v))$, avec $\epsilon \in P$ et $v \in F$, ce qui 
se lit ``${\overrightarrow{ v}}$ poss\`ede les composantes $(v)$ 
dans le rep\`ere $(\epsilon)$ et les composantes $(\rho(g^{-1}v))$ 
dans le rep\`ere $(\epsilon g)$''. Si on introduit des indices, on 
\'ecrira ${\overrightarrow{ v}} = \epsilon_i v^i$ o\`u  les $v^i$ sont 
des nombres r\'eels ou complexes et o\`u  $\{\epsilon_i \}$ d\'esigne 
un \'el\'ement de $P$, c'est \`a dire un rep\`ere g\'en\'eralis\'e au point $x$, rep\`ere 
qui peut, dans les cas simples (cas o\`u  $\rho$ d\'esigne une repr\'esentation 
fondamentale de $G$, par exemple) \^etre consid\'er\'e comme base de la fibre 
au point $x$.
 Sch\'ematiquement, on a la figure \ref{fig:fibre vectoriel}
\begin{figure}[htbp]
\epsfxsize=8cm
$$
    \epsfbox{fibre-vectoriel.eps}
$$
\caption{Espace fibr\'{e} vectoriel}
\label{fig:fibre vectoriel}
\end{figure}

Le fibr\'{e} vectoriel est dit r\'eel ou complexe suivant que $F=\RR^n$ 
ou $\CC^n$ et on pourra \'ecrire $E=E(M,F)$.  Le 
lecteur aura devin\'e que la notation utilis\'ee ici permet de nommer la 
base, la fibre et l'espace total correspondant.

L'exemple fondamental est celui fourni par le fibr\'{e} tangent \`a 
une vari\'et\'e $M$. Nous avons d\'ej\`a d\'efini cet espace de fa\c con 
\'el\'ementaire au premier chapitre. Il s'introduit ici de fa\c con 
parfaitement naturelle~: Soit $P=FM$ le fibr\'{e} principal des 
rep\`eres lin\'eaires sur $M$~; le groupe structural est 
$G=GL(m,\RR)$ avec $m={\mbox dim}\, M$. On consid\`ere la 
repr\'esentation fondamentale de $G$ sur $\RR^m$ et on construit 
le fibr\'{e} tangent $TM=FM \times_{GL(m,\RR)} \RR^m$ comme fibr\'{e} 
associ\'e \`a $FM$. Les \'el\'ements de $TM$ sont, par d\'efinition, des 
vecteurs tangents qu'on note $v=\epsilon_\mu .  v^\mu$, o\`u  
$\epsilon_\mu$ d\'esigne un \'el\'ement de $FM$, c'est \`a dire aussi 
une base de $T(M,x)$, l'espace tangent en $x$, c'est \`a dire la 
fibre de $TM$ au dessus de $x \in M$. On d\'ecide \'egalement de ne 
plus mettre de fl\`eche sur les vecteurs.
Noter que nous \'ecrivons les composantes $v^\mu$ de $v$ \`a droite 
de la base $\epsilon_\mu$ de fa\c con \`a rester compatible avec la 
notation g\'en\'erale que nous avons introduite pr\'ec\'edemment pour 
les espaces fibr\'{e}s associ\'es. Soit $\Lambda = 
(\Lambda_\mu^\nu)$ une matrice de $GL(m,\RR)$, on retrouve 
alors la propri\'et\'e bien connue 
$$
v = \epsilon_\mu v^\mu = (\epsilon_\nu 
\Lambda_\mu^\nu)({\Lambda^{-1}}_\rho^\mu v^\rho)
$$

Les tenseurs contravariants et covariants de tous ordres, qui ne sont autres que les 
\'el\'ements de $(TM)^{\otimes p} \bigotimes (T^*M)^{\otimes q}$ 
d\'ej\`a introduits au premier chapitre s'interpr\`etent ici comme 
des \'el\'ements des espaces fibr\'{e}s vectoriels $FM \times_G F$ o\`u  
$F$ d\'esigne la puissance tensorielle appropri\'ee de $\RR^m$ et 
o\`u  $GL(m,\RR)$ agit sur $F$ par la repr\'esentation tensorielle 
correspondante.

Les exemples qui pr\'ec\`edent sont d'une utilisation courante en 
physique de l'espace-temps (th\'eorie de la gravitation) mais il 
faut bien voir qu'il n'y a pas grande diff\'erence conceptuelle 
entre vecteurs de l'espace temps et $\ldots$ quarks! En effet, 
en th\'eorie des interactions fortes, par exemple, on consid\`ere un fibr\'{e} 
principal $P$ de groupe structural $SU(3)$ au dessus de 
l'espace-temps $M$, on choisit alors l'action de $SU(3)$ sur 
$\CC^3$ et on construit le fibr\'{e} vectoriel associ\'e $P 
\times_{SU(3)} \CC^3$~; un quark au point $x$ est alors d\'ecrit 
par un \'el\'ement de ce fibr\'{e} vectoriel. Nous reviendrons plus 
loin sur ces exemples utilis\'es en physique.

\subsection{Trivialit\'e des fibr\'{e}s vectoriels, vari\'et\'es parall\'elisables}

Revenons un peu sur la notion de trivialit\'e d\'ej\`a \'etudi\'ee, dans le 
cas des fibr\'{e}s principaux, en section \ref{sec:trivial}.  On se souvient qu'une 
condition n\'ecessaire et suffisante, pour assurer la trivialit\'e d'un fibr\'{e} principal $P$,
est l'existence 
d'une section globale. Contrairement au cas des fibr\'{e}s 
principaux, l'existence, pour un fibr\'{e} vectoriel, de sections
globales, est une propri\'et\'e \'evidente~: tout fibr\'{e} vectoriel, 
trivial ou non, poss\`ede des sections globales, par exemple la 
section nulle. Ce n'est donc pas ainsi qu'on d\'etecte la 
trivialit\'e. Par contre, nous avons vu que, d'une certaine 
fa\c con, on pouvait consid\'erer
un \'el\'ement du fibr\'{e} principal $P$ comme une base dans une certaine fibre du
fibr\'{e} associ\'e $E.$ L'existence pour $P$ d'une section globale \'equivaut donc, pour $E$, \`a 
l'existence de $n$ sections ind\'ependantes en tout point de $M$ 
 ($n$ d\'esignant ici la dimension de la fibre type).
On dit qu'une {\sl vari\'et\'e est parall\'elisable}\index{vari\'et\'e parall\'elisable}
  si son fibr\'{e} tangent est trivial. De fa\c con g\'en\'erale, les groupes de Lie sont des vari\'et\'es
parall\'elisables. En effet la donn\'ee d'une base dans l'alg\`ebre de Lie $\frak g$ du groupe $G$
d\'etermine $n=dim(G)$ champs de vecteurs ind\'ependants en tous points de $G$ (les champs invariants
\`a gauche associ\'es). On voit ainsi que le fibr\'{e} tangent $TG$ poss\`ede $n$ sections ind\'ependantes
(il est donc trivial), et que,
ce qui revient au m\^eme, le fibr\'{e} principal $FG$ (le fibr\'{e} principal des rep\`eres
sur la vari\'et\'e $G$, fibr\'{e} dont le groupe structural est $GL(n)$)
poss\`ede une section globale. Les groupes ne sont pas les seules vari\'et\'es
parall\'elisables ; l'exemple le plus c\'el\`ebre de vari\'et\'e
ne poss\'edant pas de structure de groupe mais \'etant n\'eanmoins parall\'elisable
est sans doute celui de la sph\`ere $S^7$
(seules les sph\`eres $S^0$, $S^1$ et $S^3$ poss\`edent une structure de groupe).
La d\'emonstration de cette propri\'et\'e repose sur l'utilisation du produit de Cayley
dans $R^8$ (l'``alg\`ebre'' des octonions). Les sph\`eres $S^n$ de 
dimension $n=0,1,3,7$ sont les seules sph\`eres \`a
\^etre parall\'elisables. Signalons sans d\'emonstration 
quelques autres exemples de vari\'et\'es parall\'elisables : 
les vari\'et\'es de Stiefel complexes $SU(n)/SU(k)$ (en excluant les sph\`eres, c'est \`a dire en
supposant $k \neq n-1$),
les espaces homog\`enes qui sont des quotients de $SU(n)$ par des
sous-groupes du type $SU(2)\times\ldots\times SU(2)$ (\`a condition d'exclure une seule exception,
la sph\`ere $S^5=SU(3)/SU(2)$), les quotients du type $Sp(n)/SU(2)$, l'espace homog\`ene
$SU(4)/H$ o\`u $H$ est le sous-groupe de $SU(4)$ isomorphe \`a $SU(2)$ constitu\'e des matrices du type
$\begin{pmatrix}  A & 0 \\ 0 & A\end{pmatrix}  $, avec $A \in SU(2)$.


\subsection{Sections de fibr\'{e}s associ\'es et champs}\label{sec:sections}

\begin{itemize}
	\item  Soit $P=P(M,G)$ un fibr\'{e} principal et $E=E(M,F)$ un fibr\'{e} associ\'e~.
 Nous savons ce qu'est une 
section $\sigma$ d'un espace fibr\'{e}, \`a savoir une application 
diff\'erentiable de $M$ dans $P$, ou dans $E$, telle que $\pi \, o \, 
\sigma$ soit l'identit\'e de $M$, $\pi$ d\'esignant la projection du fibr\'{e} 
correspondant. L'ensemble des sections globales est quelquefois vide (cas 
des fibr\'{e}s principaux non triviaux) mais on sait qu'un fibr\'{e} vectoriel 
admet de nombreuses sections globales. Soit $\Gamma E$ l'ensemble de ces 
sections. Dans le cas o\`u  $E$ est le fibr\'{e} tangent $TM$ d'une vari\'et\'e, il 
est \'evident qu'une section n'est autre qu'un champ de vecteurs~; 
de la m\^eme fa\c con,
les champs de tenseurs d'ordre sup\'erieur sont \'egalement des sections de 
fibr\'{e}s vectoriels appropri\'es.  
	
	\item  En physique, les champs de mati\`ere classiques sont toujours d\'ecrits par 
des sections de fibr\'{e}s associ\'es (le mot ``classique'' signifiant ici 
qu'on fait allusion aux th\'eories de champs classiques et non aux th\'eories 
de champs quantiques). C'est ainsi qu'un champ de quarks, par exemple, 
est une section d'un fibr\'{e} \`a fibres $\CC^3$, mentionn\'e au paragraphe 
pr\'ec\'edent, et que les champs de mati\`ere des ``mod\`eles $\sigma$ non 
lin\'eaires'' sont des sections de fibr\'{e}s en espaces homog\`enes. Pour cette 
raison, on dira quelquefois {\sl champ de mati\`ere \/}\index{champs de mati\`ere} au lieu de ``section de 
fibr\'{e} associ\'e''. L'ensemble $\Gamma E$ est donc l'espace des champs de 
mati\`ere caract\'eris\'e par le fibr\'{e} $E=E(M,F)$.
	
	\item  ll existe au moins quatre descriptions possibles d'un tel champ 
	de mati\`ere~; illustrons ces quatre descriptions dans le cas des champs 
	de vecteurs.
	\begin{enumerate}
		\item  On peut consid\'erer $x \in M \rightarrow v(x) \in E$ comme une 
		section de $E$ (cf. supra) et on \'ecrit
		 $\overrightarrow{v}(x)=\epsilon_\mu(x).v^\mu(x)$.
		
		\item  On peut consid\'erer $v$  (laissons tomber les fl\`eches !) 
		comme une application de $P$ dans $F$ (et 
		non plus de $M$ dans $E$) qui, au rep\`ere $\epsilon = (\epsilon_\mu)$, 
		un \'el\'ement de $P$, associe le $n$-uplet de composantes $v^\mu$ (un 
		\'el\'ement de $F$). Ce point de vue redonne (au deuxi\`eme degr\'e~!) un sens 
		intrins\`eque \`a la notation ``avec des indices''. Cette application de 
		$P$ dans $F$ n'est pas quelconque, elle est \'equivariante. En effet, si 
		le rep\`ere $(\epsilon_\mu)$ a pour image $v^\mu$, il faut que le rep\`ere 
		$(\epsilon_\mu)g$ ait pour image $\rho(g^{-1})v^\mu$. Ici $g$ et $\rho$ 
		d\'esignent respectivement un \'el\'ement du groupe structural et la 
		repr\'esentation d\'efinissant le fibr\'{e} associ\'e. {\it Il existe une 
		correspondance bijective entre l'ensemble des sections d'un fibr\'{e} 
		associ\'e $E=E(M,F)$
		et l'ensemble des applications \'equivariantes du fibr\'{e} principal 
		correspondant $P$ dans la fibre type $F$.}
		On peut donc identifier 
		$\Gamma E = \Omega^0(M,E)$ avec l'ensemble des applications de $P$ dans $F$ qui sont 
		\'equivariantes, c'est \`a dire $\Omega^0_\rho(P,F)$. Cette identification 
		se g\'en\'eralise au cas des $p$-formes sur $M$ \`a valeurs dans les sections 
		de $E$ : $\Omega^p (M,E) \sim \Omega^p_\rho(P,F).$
		
		\item  On peut \'evidemment fixer ---tout au moins localement--- une 
		section de $P$, c'est \`a dire choisir un ``rep\`ere mobile'' et 
		caract\'eriser $v(x)$ par ses composantes dans le rep\`ere choisi. C'est 
		cette m\'ethode qui est la plus utilis\'ee dans les calculs pratiques (on 
		ne regarde que $v^\mu(x))$ mais il faut alors se rappeler qu'on a 
		effectu\'e un choix et que ce choix est d'ordinaire local.
		
		\item  Les trois descriptions ci-dessus suffisent g\'en\'eralement \`a 
		discuter le cas de la ``g\'eom\'etrie commutative'' (la g\'eom\'etrie tout 
		court~?) ou de son pendant physique, la th\'eorie classique des champs. 
		Cela dit, dans le cas des espaces fibr\'{e}s vectoriels, il existe une 
		quatri\`eme description que nous n'aurons pas le loisir de discuter plus 
		avant mais qui se g\'en\'eralise parfaitement au cas de le g\'eom\'etrie non 
		commutative. La voici. On sait que l'ensemble des fonctions sur $M$ 
		constitue une alg\`ebre (commutative)~; soit $v$ un \'el\'ement de $\Gamma E$ 
		(un champ de mati\`ere) et soit $f$ une fonction sur la base, alors, il 
		est bien \'evident que $f v$ est encore un \'el\'ement de $\Gamma E$. En 
		d'autres termes, les champs de mati\`ere ---les sections de $E$---
		constituent un module sur l'alg\`ebre des fonctions sur $M$ (il s'agit 
		m\^eme d'un bimodule puisque l'alg\`ebre des fonctions sur $M$ est 
		commutative et d'un bimodule particulier puisque les deux actions \`a droite et \`a gauche
co\"incident).
		
		
	\end{enumerate}
	
\end{itemize}

\section[Elargissement et r\'eduction]{Changement de groupe structural dans les fibr\'{e}s principaux : 
\'elargissement et r\'eduction}
\subsection{D\'efinitions}
On consid\`ere la situation o\`u  nous avons deux fibr\'{e}s principaux $Q(M,H)$ 
et $P(M,G)$ avec $H \subset G$ et $f$, un diff\'eomorphisme de $Q$ sur 
$f(Q) \subset P$ tel que $\forall z \in Q , \forall s \in H , f(zs)=f(z)s$.
En pratique il s'agira souvent d'une inclusion $Q \subset P$, $f$ n'\'etant 
autre que l'identit\'e.

Dans une telle situation, on dit que $P$ est un {\sl \'elargissement\/}
\index{elargissement d'espace fibr\'{e}}
 (ou un {\sl prolongement\/}\index{prolongement d'espace fibr\'{e}})
de $Q$ et que $Q$ est une {\sl r\'eduction\/}\index{r\'eduction d'espace fibr\'{e}} de $P$.

Comme nous allons le voir, il est toujours possible d'\'elargir mais il 
n'est pas n\'ecessairement possible de r\'eduire.

\subsection{Elargissement (passage de $H$ \`a $G$ avec $H\subset G$)}

Soit $Q=Q(M,H)$ un fibr\'{e} principal. On veut ``agrandir'' le fibr\'{e} $Q$ 
sans modifier la base $M$ mais en agrandissant la fibre, c'est \`a dire en 
rempla\c cant le groupe de Lie $H$ par un groupe $G$ ``plus grand''. La 
construction est la suivante~: on se choisit un groupe de Lie $G$ tel que 
$H \subset G$ et on construit le fibr\'{e} $P$ associ\'e \`a $Q$ d\'efini par $P= Q
\times_H G$ o\`u  $H$ agit sur $G$ par multiplication \`a gauche.
Les actions de $H$ \`a droite de $Q$ et \`a gauche de $G$ s'annihilent, mais
il est \'evident que l'espace $P=P(M,G)$ est un $G$-fibr\'{e} principal puisque 
$G$ 
agit \`a droite de $P$ via $(z.k) {k'}=z.k {k'}$, avec $z \in Q$ et 
$k,  {k'} \in G$.
Par ailleurs, le diff\'eomorphisme de $Q$ sur $f(Q)$ recherch\'e est
d\'efini, pour $z \in Q$, par $f(z) = z.e$ ($e$ d\'esignant l'\'el\'ement neutre 
de $G$) et on notera simplement $f(z) = z$.

 En conclusion, le passage de $Q(M,H)$ \`a $P(M,G)$ avec 
$H\subset G$ est toujours possible. On dit que $P$ est obtenu \`a partir 
de $Q$ par \'elargissement du groupe structural et que $Q$ lui-m\^eme est une
r\'eduction de $P$. Notons que les 
repr\'esentations de $G$ sont toujours des repr\'esentations de $H$, mais que 
le contraire n'est pas n\'ecessairement vrai (ainsi, il faut, en g\'en\'eral, 
prendre la somme directe de plusieurs repr\'esentations de $H$ pour 
construire une repr\'esentation de $G$)~; les fibr\'{e}s associ\'es \`a $Q$ ne sont 
donc pas forc\'ement toujours des fibr\'{e}s associ\'es \`a un \'elargissement $P$.

\subsection{R\'eduction (passage de $G$ \`a $H$ avec $H\subset G$)}\label{sec:reduction}
\begin{description}
	\item[M\'ethode]
	
  Soit $P=P(M,G)$ un fibr\'{e} principal. On veut diminuer la 
	taille de $P$ sans modifier la base $M$ mais en ``raccourcissant'' la 
	fibre, c'est \`a dire en rempla\c cant $G$ par un groupe ``plus petit''. En 
	d'autres termes, si on consid\`ere les \'el\'ements de $P$ comme des rep\`eres 
	g\'en\'eralis\'es, on veut s'int\'eresser uniquement \`a une sous-classe 
	particuli\`ere de rep\`eres, sous-classe qui soit stable sous l'action d'un 
	sous-groupe $H$ de $G$. La m\'ethode du paragraphe pr\'ec\'edent ne s'applique 
	pas car le groupe $H (\subset G)$ n'est pas stable lorsqu'on le 
	multiplie \`a gauche par des \'el\'ements de $G$.
	
	Enon\c cons (et retenons) le r\'esultat suivant que nous d\'emontrerons un peu 
	plus bas~:
	
	{\it  Le choix d'une r\'eduction du fibr\'{e} 
	principal $P=P(M,G)$ \`a un sous-fibr\'{e} $Q=Q(M,H)$ de groupe structural $H$, lorsqu'il 
	existe, n'est pas en g\'en\'eral unique, et est caract\'eris\'e par le choix 
	d'une section globale dans un fibr\'{e} en espaces homog\`enes associ\'e \`a $P$,
	en l'occurrence le fibr\'{e} associ\'e $P\times_G G/H$\/}.
			 
	Ce th\'eor\`eme est d'une importance fondamentale car il permet, comme nous 
	allons le voir, de donner un sens pr\'ecis \`a l'id\'ee intuitive de ``choix 
	d'une g\'eom\'etrie'' pour la vari\'et\'e diff\'erentiable $M$.
	
	Preuve. Soit $\sigma$ une section globale de $E = P\; {\mbox mod}\; H$. Un
	th\'eor\`eme pr\'ec\'edemment discut\'e (voir les diverses mani\`eres de consid\'erer
	les sections de fibr\'{e} associ\'e) nous dit qu'on peut associer \`a $\sigma$ une 
	application $\overrightarrow{\sigma}$ du fibr\'{e} principal $P$ dans la fibre type $G/H$, 
	qui soit \'equivariante ($\overrightarrow{\sigma}(ys)=s^{-1}\overrightarrow{\sigma}(y)$).
	D\'efinissons $Q=\overrightarrow{\sigma}^{-1}(eH) \subset P$. La projection $\pi : Q 
	\mapsto M$ n'est autre, par d\'efinition, que la restriction \`a $Q$ de la 
	projection $\pi$ de $P$. Consid\'erons deux points $y_1$ et $y_2$ de la 
	m\^eme fibre de $Q$, c'est \`a dire $\pi(y_1)=\pi(y_2)$ ; nous savons qu'il 
	existe $s \in G \; \mbox{tel que}\; y_2=y_1 s$. Nous allons montrer qu'en 
	fait, cet \'el\'ement $s$ appartient au sous-groupe $H$. En effet,
	$$
	y_1,y_2 \in Q \Rightarrow \overrightarrow{\sigma}(y_1) =eH \, {\mbox et} \, \overrightarrow{\sigma}(y_2)=eH
	$$
	mais
	$
	\overrightarrow{\sigma}(y_2)=\overrightarrow{\sigma}(y_1 s) = s^{-1} \overrightarrow{\sigma}(y_1)
	$
	et donc, $ eH = s^{-1}(eH)$, ce qui montre que $s \in H$.
	Ainsi $Q \subset P$ est un fibr\'{e} principal de groupe structural $H$.
	
	R\'eciproquement, donnons nous $H \subset G$ et une r\'eduction 
	$Q=Q(M,H)\subset P=P(M,G)$. D\'efinissons $\overrightarrow{\sigma} : P \mapsto G/H$ par
	$\forall y \in Q \subset P, \overrightarrow{\sigma}(y)=eH \in G/H$. La fonction $\overrightarrow{\sigma}$ 
	est ainsi constante sur les fibres de $Q$. Soient maintenant deux points $y_0 \in Q$ 
	et $y \in P$ que nous prenons dans la m\^eme fibre de $P$ mais nous
	supposons que $y$ n'appartient pas n\'ecessairement \`a $Q$. Il existe donc 
	un \'el\'ement $g$ de $G$ tel que $y=y_0 g$, alors $\overrightarrow{\sigma}(y)=\overrightarrow{\sigma}(y_0 g) 
	= g^{-1} \overrightarrow{\sigma}(y_0)=g^{-1}eH=g^{-1}H \in G/H$. On a ainsi construit une 
	application $\overrightarrow{\sigma} : P \mapsto G/H$ \'equivariante sous l'action de $G$. 
	Cela d\'etermine, en vertu du th\'eor\`eme \'enonc\'e au $2$ du \ref{sec:sections}
        une section globale de 
	$E(M,G/H)=P\, {\mbox mod} \, H$.
			
		\item[R\'eduction de $GL(n,\RR)$ \`a $SO(n)$~: structures riemanniennes] 
	\ \\
 On peut rattacher canoniquement \`a une vari\'et\'e diff\'erentiable $M$ son 
	fibr\'{e} $FM$ des rep\`eres lin\'eaires. C'est un fibr\'{e} principal de groupe 
	structural $GL(n,\RR)$. {\it Choisissons\/} maintenant {\it une\/} 
	r\'eduction \`a un sous-fibr\'{e} de groupe structural $SO(n)$. Choisir une 
	telle r\'eduction consiste \`a s\'electionner une certaine classe de 
	rep\`eres, que nous appellerons {\em orthonorm\'es\/}, telle que l'un 
	quelconque d'entre eux puisse s'obtenir \`a partir de n'importe quel autre 
	\`a l'aide d'une matrice du groupe orthogonal $SO(n)$. Par d\'efinition, 
    une vari\'et\'e riemannienne ($M$ en l'occurrence) est une 
	vari\'et\'e diff\'erentiable de dimension $n$ pour laquelle on a {\it 
	choisi\/} une r\'eduction du fibr\'{e} $FM$ des rep\`eres lin\'eaires \`a un 
	sous-fibr\'{e} de groupe structural $SO(n)$. Le sous-fibr\'{e} en question se 
	note alors $OFM$ (``Orthogonal Frame Bundle'') et prend le nom de fibr\'{e} 
	des rep\`eres orthonorm\'es. Le lecteur peut se demander o\`u  est la m\'etrique 
	dans cette approche$\ldots$ La r\'eponse est la suivante~: le tenseur 
	m\'etrique s'identifie pr\'ecis\'ement avec la section globale du fibr\'{e} en 
	espaces homog\`enes $GL(n)/SO(n)$ qui d\'efinit la r\'eduction (nous oublions 
	momentan\'ement les probl\`emes li\'es \`a des exigences de non d\'eg\'en\'erescence, 
	de positivit\'e \etc). Noter que la dimension de cet espace homog\`ene est \'egale \`a
	$dim(GL(n)) - dim(SO(n)) = n^2 - n(n-1)/2 = n(n+1)/2$, et ses \'el\'ements 
	peuvent donc s'identifier, comme il se doit, \`a des tenseurs de rang deux compl\`etement 
	sym\'etriques. 
    Intuitivement, choisir une structure 
	riemannienne revient \`a conf\'erer une ``forme g\'eom\'etrique'' \`a une vari\'et\'e 
	diff\'erentiable~; c'est ainsi que c'est le choix de la m\'etrique qui fait 
	la diff\'erence entre un ballon de foot, un ballon de rugby et$\ldots$ une 
	bouteille de vin (bouch\'ee!) et 	la multiplicit\'e des r\'eductions possibles co\"incide avec 
	la multiplicit\'e des m\'etriques riemanniennes qu'on peut choisir, pour une 
	vari\'et\'e diff\'erentiable donn\'ee.
	Vu l'importance de cette notion, nous y reviendrons abondamment dans le chapitre suivant.
	
	\item[R\'eduction de $GL(2n,\RR)$ \`a $GL(n,\CC)$~: structures 
	presque-complexes]

 L'id\'ee est essentiellement la m\^eme que dans l'exemple 
	pr\'ec\'edent, \`a ceci pr\`es que $M$ est suppos\'e \^etre de dimension 
	paire et qu'on choisit maintenant une r\'eduction du fibr\'{e} des rep\`eres 
	lin\'eaires \`a un sous-fibr\'{e} dont le groupe orthogonal doit \^etre $SU(n)$. 
	Les vari\'et\'es pour lesquelles on a effectu\'e un tel choix se nomment 
	{\sl vari\'et\'es presque-complexes\/}\index{vari\'et\'es presque-complexes}
	 et l'analogue de la m\'etrique est ici la 
	donn\'ee, en chaque espace tangent $T(M,x)$ d'un endomorphisme $j$ de 
	carr\'e \'egal \`a $-1$. Cet op\'erateur peut encore s'identifier \`a une section 
	globale d'un fibr\'{e} en espaces homog\`enes $GL(2n,\RR)/GL(n,\CC)$. Le lecteur 
	peut sans doute se demander pourquoi on parle ici de vari\'et\'es presque-complexes 
	et non, tout simplement, de vari\'et\'es complexes. Il se trouve 
	que ces deux notions sont de nature assez diff\'erentes (et la terminologie 
	est d\'esormais consacr\'ee)~: la notion de structure presque-complexe est, 
	comme on vient de le voir, analogue \`a la notion de structure 
	riemannienne et est associ\'ee au choix d'une r\'eduction du fibr\'{e} des 
	rep\`eres pour une vari\'et\'e diff\'erentiable~; la notion de structure 
	complexe est, quant \`a elle, analogue \`a la notion de structure de vari\'et\'e 
	topologique, de vari\'et\'e lin\'eaire par morceaux ou analogue \`a la notion de 
	structure 
	diff\'erentiable elle-m\^eme (on choisit des cartes \`a valeur dans $\CC^n$ et 
	non plus dans $\RR^n$ et on impose aux fonctions de transitions d'\^etre 
	holomorphes). Nous n'aurons pas le loisir, dans cet ouvrage, 
	d'\'etudier la g\'eom\'etrie des vari\'et\'es complexes~; notons simplement que la 
	terminologie vient du fait qu'une vari\'et\'e complexe donn\'ee fournit 
    une vari\'et\'e diff\'erentiable qui se trouve automatiquement munie d'une 
	structure presque-complexe (l'endomorphisme $j$ de carr\'e \'egal \`a $-1$ 
	provenant tout simplement de la multiplication par le nombre complexe 
	$i$). Le passage inverse, celui d'une structure presque-complexe \`a une 
	structure complexe, n'est pas automatique car il n\'ecessite la v\'erification d'une certaine condition 
	d'int\'egrabilit\'e.
	
	On peut aussi parler de vari\'et\'es presque-hermitiennes lorsque la 
	r\'eduction du groupe structural va de $GL(2n,\RR)$ \`a $U(n)=O(2n)\cap 
	GL(n,\CC)$. Dans ce cas, il existe une m\'etrique $h$ compatible avec la 
	structure presque-complexe, en ce sens que $h(v_1,v_2)=h(j v_1,j v_2)$.
	On peut alors construire une forme hermitienne
 $H(v_1,v_2) = {1\over 
	2}(h(v_1,v_2)-i h(jv_1, v_2))$ et une forme presque-K\"ahlerienne 
	$\omega(v_1,v_2)=h(j v_1,v_2)$.
	
	\item[R\'eduction de $GL(4,n)$ \`a $Sp(n)$~: structures presque-quaternioniques]

	L'histoire est la m\^eme que dans le cas pr\'ec\'edent et les commentaires sont analogues. La section 
	globale du fibr\'{e} en espaces homog\`enes $GL(4n,\RR)/Sp(n)$ caract\'erisant 
	la r\'eduction peut s'identifier \`a la donn\'ee, en chaque espace tangent 
	$T(M,x)$ de trois op\'erateurs $j_1$, $j_2$, $j_3$, tous trois de carr\'e 
	\'egal \`a moins l'identit\'e, et satisfaisant aux relations $j_1 j_2=-j_2 j_1 
	= j_3$, $j_2 j_3=-j_3 j_2	= j_1$ et $j_3 j_1=-j_1 j_3 = j_2$.
	La raison du ``presque'' dans le presque-quaternionique est analogue celle
	donn\'ee dans le paragraphe pr\'ec\'edent \`a condition toutefois de remplacer 
	nombres complexes par quaternions. Ici $Sp(n)$ d\'esigne le groupe compact 
	des unitaires quaternioniques, quelquefois d\'esign\'e par $U(n,\HH)$.
	 Nous n'aurons pas le loisir de 
	revenir sur ce sujet dans le cadre de cet ouvrage. 
    
    \item[Remarques]

 Avant de quitter cette partie consacr\'ee aux 
    r\'eductions de fibr\'{e}s principaux, notons que les repr\'esentations d'un 
    groupe $G$ sont aussi des repr\'esentations de tout sous-groupe $H$ de 
    $G$. Ainsi donc, les fibr\'{e}s associ\'es \`a $P(M,G)$ sont aussi associ\'es \`a 
    tout sous-fibr\'{e} $Q(M,H)$ avec $H \subset G$. Il est rassurant de 
    savoir que le fibr\'{e} tangent $TM$ d\'efini \`a partir du fibr\'{e} des rep\`eres 
    lin\'eaires $FM$ co\"incide avec celui qu'on peut d\'efinir \`a partir du fibr\'{e} $OFM$ des rep\`eres 
    orthonorm\'es!

\end{description}

\section[Extension et quotient]{Changement de groupe structural dans les fibr\'{e}s principaux : 
extension et quotient}

  Les deux sous-sections pr\'ec\'edentes \'etaient, en 
	quelque sorte, compl\'ementaires, les deux qui suivent le seront 
	aussi. \index{extension d'espace fibr\'e}  \index{quotient d'espace fibr\'e} 
	
\subsection{Extension (passage de $G$ \`a $\widehat G$ avec $G \sim{\widehat 
G}\vert H$)}

\begin{description}
	\item[M\'ethode g\'en\'erale]
	
Le probl\`eme est le suivant~: on part d'un espace fibr\'{e} $P=P(M,G)$ et on 
veut remplacer le groupe structural $G$ par un groupe $\widehat G$ tel que 
$G$ soit isomorphe \`a $\widehat G/H$ o\`u  $H$ est un sous-groupe distingu\'e de 
$\widehat G$. Le cas le plus fr\'equent est celui o\`u  $G$ est un groupe qui n'est 
pas simplement connexe et o\`u  on veut le remplacer par son groupe de 
recouvrement universel ${\widehat G}$. $H$ est alors un sous-groupe discret 
du centre de ${\widehat G}$ et s'identifie au groupe d'homotopie $\pi_1(G)$ 
(voir le chapitre sur les groupes).
 Pour illustrer cette situation, voici un exemple 
dont l'importance physique est importante. La vari\'et\'e diff\'erentiable $M$ 
est un mod\`ele pour l'espace-temps et $P$ d\'esigne le fibr\'{e} des rep\`eres 
orthonorm\'es correspondant au choix d'une certaine m\'etrique sur $M$. 
Certains champs de mati\`ere vont \^etre repr\'esent\'es par des sections de 
fibr\'{e}s associ\'es \`a $P$. Ces fibr\'{e}s seront construits \`a partir de 
repr\'esentations du groupe $SO(n)$ (en physique quadri-dimensionelle, 
g\'en\'eralement $SO(3,1)$
ou $SO(4)$). Dans bien des cas, cependant, les champs de mati\`ere qui nous 
int\'eressent ne correspondent pas vraiment  \`a des repr\'esentations de $SO(n)$ mais \`a 
des repr\'esentations de son groupe de recouvrement universel $Spin(n)$ 
c'est \`a dire $Spin(3,1)=SL(2,\CC)$ si $G=SO(3,1)$ et $Spin(4)=SU(2)\times 
SU(2)$ si $G=SO(4)$. On se souvient en effet que les repr\'esentations de 
$G$ peuvent \'egalement \^etre consid\'er\'ees comme des repr\'esentations de ${\widehat 
G}$ mais que certaines repr\'esentations de ${\widehat G}$ ne correspondent \`a 
aucune repr\'esentation de $G$ (ainsi, les spins demi-entiers correspondent 
\`a des repr\'esentations de $SU(2)$ mais pas \`a des repr\'esentations de $SO(3)$).
Lorsque la topologie de $M$ et triviale, le fait de ``consid\'erer des 
spineurs'' n'offre aucune difficult\'e~; les choses changent lorsque $M$ 
cesse d'\^etre trivial~: en d'autre termes, il existe des espaces qui 
n'admettent pas de spineurs! A l'oppos\'e, il existe des espaces qui 
admettent plusieurs types de spineurs. Nous discuterons de nouveau de ces probl\`emes un 
peu plus loin.

Revenons \`a un cadre plus g\'en\'eral. On se donne $G={\hat G}/H$. On a donc un 
homomorphisme (surjectif) de groupe ${\widehat G}{\stackrel \lambda \mapsto} 
G$ de noyau $H=Ker \lambda$. On appellera ``extension de fibr\'{e} principal 
$P=P(M,G)$ \`a un fibr\'{e} de groupe structural ${\widehat G}$'' la donn\'ee d'un 
fibr\'{e} principal ${\widehat P} = {\widehat P}(M,{\widehat G})$ et d'un homomorphisme 
de fibr\'{e} ${\widehat \lambda} : {\widehat P} \mapsto P$ qui soit compatible avec 
les actions respectives de $G$ et ${\widehat G}$ et avec l'homomorphisme de 
groupe $\lambda$~; en d'autres termes, ${\widehat \lambda}$ doit pr\'eserver 
les fibres (c'est un homomorphisme de fibr\'{e}) et \^etre tel que
$$
{\widehat \lambda}({\widehat z}{\widehat k})=z \, k {\hskip 0.2cm}{\mbox {o\`u}} {\hskip 0.2cm}
{\widehat z}\in{\widehat P}, {\widehat k}\in{\widehat G}, z = {\widehat \lambda}(z) {\hskip 0.2cm}
{\mbox et} {\hskip 0.2cm} k=\lambda({\widehat k})
$$
L'existence d'une ou de plusieurs extensions correspondant \`a une 
situation donn\'ee (c'est \`a dire le fibr\'{e} $P$, le groupe ${G}$, \etc) 
d\'epend bien entendu de la situation consid\'er\'ee $\ldots$
	
	
Le probl\`eme de l'extension d'un fibr\'{e} principal (passage de $G$ \`a ${\widehat 
G}$ avec $G = {\widehat G}/H$) peut \^etre d\'ecrit de fa\c con imag\'ee par la figure 
suivante
(\ref{fig:extension}).

\begin{figure}[htbp]
\epsfxsize=8cm
$$
    \epsfbox{extension-fibre.eps}
$$
\caption{Extension d'un fibr\'{e} principal}
\label{fig:extension}
\end{figure}

On voit que ${\widehat P}$ est aussi un $H$-fibr\'{e} principal au dessus de $P$ (voir 
\'egalement la discussion men\'ee en section \ref{sec:fibre-quotient}) et que $P \sim {\widehat 
P}/H$.
	\item[Structures spinorielles]

  Dans le cas particulier o\`u  $P=OFM$ 
	d\'esigne le fibr\'{e} des rep\`eres orthonorm\'es d'une vari\'et\'e riemannienne $M$, 
	le groupe structural est $G=SO(n)$ et ${\widehat G = Spin(n)}$. On dit que 
	la vari\'et\'e $M$ est une vari\'et\'e spinorielle s'il existe une extension 
	${\widehat P} = {\widehat P}(M,Spin(n))$. {\sl Choisir une structure 
	spinorielle\/}\index{structure 
	spinorielle} pour une vari\'et\'e riemannienne donn\'ee $M$ consiste \`a choisir 
	une extension ${\widehat P}$ (s'il en existe une). Le fibr\'{e} ${\widehat P}$, s'il
	existe, est alors d\'esign\'e sous le nom de {\sl fibr\'{e} des rep\`eres 
	spinoriels\/}\index{fibr\'{e} des rep\`eres 
	spinoriels} ou, tout simplement {\sl fibr\'{e} de spin\/}\index{fibr\'{e} de spin}
	 et d\'enot\'e ${\widehat 
	OFM}$. Dans les bons cas (``bon'' signifiant qu'on peut ne pas se 
	soucier du probl\`eme!), ${\widehat OFM}$ existe et est unique, \`a isomorphisme 
	pr\`es. Notons encore que le choix d'une structure spinorielle est 
	tributaire du choix d'une structure riemannienne (on choisit d'abord 
	$P=P(M,SO(n))$, puis ${\widehat P}={\widehat P}(M, Spin(n))$, mais on doit se 
	souvenir que deux m\'etriques distinctes d\'efinissent des fibr\'{e}s 
	$P(M,SO(n))$ diff\'erents. On sait que la repr\'esentation fondamentale de 
	$SO(n)$ agissant sur $\RR^n$ permet de fabriquer le fibr\'{e} tangent $TM = 
	P \times_{SO(n)} \RR^n$ comme fibr\'{e} associ\'e \`a $P$ et de d\'efinir 
	l'ensemble des champs de vecteurs $\Gamma T M$ comme ensemble des 
	sections de $TM$. De la m\^eme fa\c con, la repr\'esentation fondamentale de 
	$Spin(n)$ agissant sur $\CC^s$ avec $s=2^{[n/2]}$, $[n/2]$ d\'esignant la 
	partie enti\`ere de $n/2$, permet de fabriquer le {\sl fibr\'{e} des spineurs\/}
	\index{fibr\'{e} des spineurs} $SM = 
	{\widehat P} \times_{Spin(n)} \CC^s$ comme fibr\'{e} associ\'e \`a ${\widehat P}$ et de 
	d\'efinir l'ensemble des champs de spineurs $\Gamma SM$ comme ensemble des 
	sections de $SM$.
	
	Pour ce qui est des rappels concernant la repr\'esentation fondamentale de 
	$Spin(n)$, les alg\`ebres de Clifford, \etc, voir la fin du chapitre pr\'ec\'edent.
	
	On montre que l'existence d'une structure spinorielle, pour une vari\'et\'e 
	donn\'ee, est li\'ee \`a l'annulation d'une certaine classe caract\'eristique 
	(la deuxi\`eme classe de Stiefel-Whitney). Mis \`a part une courte 
remarque du m\^eme type \`a la fin de cette section, ce ph\'enom\`ene
 ne sera pas 
	discut\'e dans le cadre de notre ouvrage.
	
	\item[Bosons et fermions]

  En physique th\'eorique, on montre que, dans le 
	cadre de la th\'eorie quantique des champs, et en dimension \'egale \`a $4$, 
	les particules ob\'eissant \`a une statistique de Fermi-Dirac (les fermions) 
	sont des particules de spin demi-entier, alors que celles ob\'eissant \`a 
	une statistique de Bose-Einstein (les bosons) sont des particules de 
	spin entier. Nous n'expliquerons pas ici la signification de ce r\'esultat 
	c\'el\`ebre (le {\sl th\'eor\`eme spin-statistique\/}\index{th\'eor\`eme spin-statistique})
	 puisque nous n'aborderons 
	pas la th\'eorie quantique des champs dans le cadre de cet ouvrage. 
	Cependant, le r\'esultat en question (qui n'est vraiment bien compris et 
	d\'emontr\'e qu'en dimension $4$) nous permet d'introduire la terminologie 
	suivante en dimension quelconque~: nous dirons qu'un champ est un champ 
	bosonique s'il s'agit d'une section d'un fibr\'{e} vectoriel associable au 
	fibr\'{e} des rep\`eres orthonorm\'es d'une vari\'et\'e riemannienne $M$~; nous 
	dirons que c'est un champ fermionique s'il s'agit d'une section d'un 
	fibr\'{e} vectoriel associable au fibr\'{e} des rep\`eres spinoriels d'une vari\'et\'e 
	riemannienne $M$, qui soit telle que la repr\'esentation correspondante (celle qui d\'efinit
	le fibr\'{e} associ\'e) soit une repr\'esentation de 
	$Spin(n)$ qui ne puisse pas \^etre consid\'er\'ee comme
	 une repr\'esentation du groupe $SO(n)$ mais seulement comme une repr\'esentation de $Spin(n)$.
	
	Notons que, bien que moins proche de notre intuition, les champs 
	spinoriels (sections de $SM$) sont plus ``fondamentaux'' que les champs 
	vectoriels (sections de $TM$). Ceci est d\'ej\`a \'evident au niveau de la 
	th\'eorie des repr\'esentations de $SU(2)$~: on peut construire n'importe 
	quelle repr\'esentation de ce groupe \`a partir de la fondamentale (qui est 
	{\sl spinorielle \/}). Cette derni\`ere est de dimension $2$ et correspond 
	physiquement \`a ce qu'on appelle un champ de spin $1/2$. Par ailleurs, il est facile de voir 
	qu'on peut construire un champ de vecteurs \`a partir 
	de (deux) champs de spineurs, mais pas le contraire$\ldots$
	
	\item[Structure quarkique]

  Nous venons de discuter le cas particulier 
	de $G=SO(n)= Spin(n)/\ZZ_2$  mais nous aurions pu \'egalement consid\'erer 
	le cas de fibr\'{e}s avec groupe structural $G=SU(3)/\ZZ_3$~: le fait qu'il 
	soit, ou non, possible, de d\'efinir des ``champs de quarks'' (associ\'es 
	\`a la repr\'esentation fondamentale de $SU(3)$, ou plus g\'en\'eralement des champs associ\'es \`a des
	repr\'esentations dont la trialit\'e est diff\'erente de z\'ero) pour une 
	vari\'et\'e $M$ consid\'er\'ee comme base d'un
 fibr\'{e} principal $P(M,SU(3)/\ZZ_3)$ n'est 
	pas quelque chose d'automatique$\ldots$. On pourrait alors parler de 
	``structure quarkique''~!

	\item[Structure encord\'ee]

  Il ne faudrait pas croire que le groupe $H$, 
	tel que $G = {\widehat G}/H$ dont il a \'et\'e question dans ce chapitre 
	consacr\'e aux extensions d'espaces fibr\'{e}s soit n\'ecessairement discret. 
	C'est ainsi qu'en th\'eorie des cordes, la vari\'et\'e $M$ est remplac\'ee par 
	$LM$ (le ``loopspace'' de $M$), c'est \`a dire l'ensemble des applications 
	de $S^1$ dans $M$, $G$ est, de la m\^eme fa\c con remplac\'e par $LG$ et $P$ 
	par $LP$. L'ensemble $LG$ est naturellement un groupe (de dimension 
	infinie) et $LP$ est fibr\'{e} en $LG$ au dessus de $LM$. Dans ce cas, 
	toutefois, ce ne sont pas tant les repr\'esentations de $LG$ qui nous 
	int\'eressent, mais celles d'une {\sl extension centrale\/}\index{extension centrale}
	 ${\widehat LG}$ de $LG$ 
	(on a $Lie({\widehat LG})=Lie(LG) \oplus \RR$). Dans ce cas, $H=U(1)$ et la
	question se pose de savoir si $LP$ peut \^etre \'etendu \`a un fibr\'{e} ${\widehat 
	LP}$ de groupe structural ${\widehat LG}$. L\`a encore, l'existence n'est pas 
	assur\'ee, et, en cas d'existence, l'unicit\'e non plus. Lorsque ${\widehat 
	LP}$ existe, on dit que {\sl le fibr\'{e} en boucles\/}\index{fibr\'{e} en boucles}
	 $LP$ poss\`ede une {\sl 
	structure encord\'ee\/}\index{structure encord\'ee}
	 (``a string structure for a loop bundle'' $\ldots$).

	\item[Interpr\'etation cohomologique]

  L'existence et l'unicit\'e (ou non) 
	des extensions de fibr\'{e}s peuvent se d\'ecrire de fa\c con cohomologique. 
	Cette interpr\'etation d\'epasse le cadre que nous nous sommes fix\'es. 
	Mentionnons seulement que l'existence de ${\widehat P}$ peut \^etre li\'e \`a 
	l'annulation d'une certaine classe de cohomologie appartenant \`a 
	$H^2(M,H)$. Dans le cas des structures spinorielles, $H=\ZZ_2$ et la 
	classe en question, dont l'annulation fournit une condition n\'ecessaire 
	et suffisante \`a l'existence de ${\widehat P}$, s'appelle {\sl deuxi\`eme classe 
	de Stiefel-Whitney \/}\index{classe de Stiefel-Whitney}. Dans le cas des 
	structures encord\'ees, il faut consid\'erer $H^2(LM,U(1))$ puisque $LM$ est 
	la base du fibr\'{e} consid\'er\'e, ce groupe de cohomologie (de $LM$)
   peut alors \^etre reli\'e \`a  $H^3(M,\ZZ)$ et $\ldots$ ceci est une autre histoire.
	
\end{description}


\subsection{Quotient (passage de $G$ \`a $K$ avec $K \sim G\vert H$, $H$ 
sous-groupe distingu\'e de $G$)}

On part de $P=P(M,G)$, on se choisit un sous-groupe distingu\'e $H$ de $G$ et 
on veut remplacer $P$ par $Q=Q(M,K)$, avec $K=G\vert H$. Cette op\'eration 
est, en un sens, inverse de celle pr\'ec\'edemment consid\'er\'ee. La m\'ethode est 
simple puisqu'il s'agit de ``diviser'' $P$ par $H$ en consid\'erant 
l'ensemble quotient $Q=P/H$ o\`u  l'action de $H$ sur $P$ est obtenue par 
restriction de celle de $G$. Il est \'evident que ce type de changement de 
groupe structural n'offre aucune difficult\'e, contrairement \`a la situation 
inverse d\'ecrite au paragraphe pr\'ec\'edent. Il faut \'evidemment veiller \`a ce 
que $H$ soit distingu\'e dans $G$, de fa\c con \`a ce que le quotient $G\vert H$ 
soit bien un groupe. Nous n'en dirons pas plus sur ce sujet puisque la 
discussion r\'esulte simplement des analyses d\'ej\`a effectu\'ees dans les 
sections pr\'ec\'edentes. En particulier, $P$ est un $H$-fibr\'{e} principal au 
dessus du fibr\'{e} quotient $Q$, lequel se trouve \^etre, \'egalement, dans ce 
cas particulier, un fibr\'{e} principal.

\section{Groupe des automorphismes. Groupe de jauge}

\subsection{Remarque terminologique}

Nous utiliserons souvent l'expression ``rep\`ere en $x$'' ---sans mettre les 
guillemets!--- pour parler d'un point $z$ de $P=P(M,G)$ se projetant au 
point $x$, m\^eme si le fibr\'{e} consid\'er\'e n'est pas un sous-fibr\'{e} de l'espace 
des rep\`eres  mais un fibr\'{e} principal quelconque au dessus de $M$, avec 
groupe de structure $G$. Le contexte devrait suffire \`a pr\'eciser s'il 
s'agit d'un rep\`ere de l'``espace interne'' ---comme disent les 
physiciens--- c'est \`a dire un \'el\'ement d'un fibr\'{e} principal quelconque non 
reli\'e au fibr\'{e} $FM$ des rep\`eres lin\'eaires, ou, au contraire, d'un rep\`ere 
de l'``espace externe'', c'est \`a dire un \'el\'ement de $FM$ (ou de $OFM$, ou 
d'un autre sous-fibr\'{e} de $FM$).

\subsection{Automorphismes verticaux d'un espace fibr\'{e} principal 
(d\'efinition)}\label{sec:groupe-de-jauge}
\begin{itemize}
	\item  L'action du groupe structural $G$ sur un fibr\'{e} principal $P = 
	P(M,G)$ est une notion qui nous est maintenant famili\`ere. Soient $z_1$ 
	et $z_2$ deux \'el\'ements de $P$ appartenant \`a des fibres diff\'erentes (en 
	d'autres termes, on consid\`ere un rep\`ere $z_1$ en $x_1$ et un rep\`ere 
	$z_2$ en $x_2$) et soit $g$ un \'el\'ement de $G$. L'action de $G$ 
	---d'ordinaire un groupe de Lie de dimension finie--- \'etant globalement 
	d\'efinie, nous savons ce que sont les points $z_1'=z_1 g$ et $z_2'=z_2 
	g$, mais il faut bien voir que ``c'est le m\^eme $g$'' qui agit en $z_1$ 
	et en $z_2$. Ce que nous voulons maintenant, c'est consid\'erer des 
	transformations plus g\'en\'erales (soit $\Phi$ une telle transformation 
	---une application de $P$ dans $P$---) qui, en un sens, autorise le 
	``petit $g$'' qui pr\'ec\`ede, \`a d\'ependre du point $x$ de la vari\'et\'e $M$~: 
	on veut remplacer $g$ par $g(x)$, c'est \`a dire remplacer les {\sl 
	transformations de jauge globales\/}\index{transformations de jauge globales}
	 par les {\sl transformations de jauge 
	locales\/}\index{transformations de jauge 
	locales}\index{transformations de jauge locales}.
	Passons maintenant \`a une d\'efinition pr\'ecise.
	
	\item  On dira que $\Phi$ est  un automorphisme vertical du fibr\'{e} $P$ si 
	et seulement si les trois conditions suivantes sont v\'erifi\'ees
	\begin{enumerate}
		\item  $\Phi$ est un diff\'eomorphisme de $P$.
		
		\item  $\Phi$ pr\'eserve les fibres, au sens suivant~: la fibre 
		$\pi^{-1}(x)$ au dessus de $x$ est stable par $\Phi$.
		
		\item  $\Phi$ commute avec l'action du groupe structural $G$, c'est \`a 
		dire que $$\forall z \in P, \forall g \in G, \Phi(zg)=\Phi(z)g$$.
	\end{enumerate}
	
	\item  Il est imm\'ediat de constater que l'ensemble ${\frak G}$ des 
	automorphismes verticaux est un groupe pour la loi de composition. Nous 
	d\'esignerons ${\frak G}$ sous le nom de {\sl groupe des automorphismes 
	verticaux\/}\index{automorphismes 
	verticaux}\index{transformations de jauge locales}
	du fibr\'{e} principal $P$, ou plus simplement sous le nom de 
	``groupe de jauge de $P$''. Nous noterons \'egalement ${\frak G}=Int P= 
	Aut_V P.$ Par l'expression ``transformation de jauge'' nous d\'esignerons 
	toujours un automorphisme vertical, c'est \`a dire une ``transformation de 
	jauge locale''.
	
		L'action de ${\frak G}$ commutant avec celle de $G$, il est commode 
	d'\'ecrire l'action de ${\frak G}$ \`a gauche et d'oublier les parenth\`eses, 
	on voit alors (la notation est faite pour cela~!) que
	$$\Phi \, z \, g = \Phi(z g) = \Phi(z) g$$
	
\end{itemize}

\subsection{Ecriture locale des transformations de jauge}

Soit $\Phi$ une transformation de jauge, $z$ un \'el\'ement de $P$ et 
$\sigma$ une section locale au voisinage de $x=\pi(z) \in M$. Le rep\`ere 
$\Phi(\sigma(x))$ \'etant dans la m\^eme fibre (au m\^eme point!) 
que le rep\`ere mobile $\sigma(x)$, il doit \^etre possible d'atteindre le 
premier \`a partir du second par l'action d'un \'el\'ement appropri\'e de $G$ que 
nous noterons $g(x)$, puisque $G$ est transitif sur les fibres. Cet 
\'el\'ement est donc d\'efini par l'\'equation $\Phi(\sigma(x))=\sigma(x) g(x)$ 
c'est \`a dire encore par $$g(x) = \sigma(x)^{-1} \Phi(\sigma(x)),$$ en 
utilisant la notation introduite en fin de section \ref{sec:principal}. Notons que $g(x)$ 
d\'epend non seulement de $\Phi$ mais aussi de la section $\sigma$ choisie 
; on pourrait utiliser la notation un peu lourde ${}^\sigma g(x)$ pour 
d\'esigner cet \'el\'ement.

Lorsque $P$ est trivial, on sait qu'il existe des sections globales. Soit 
$\sigma$ une telle section, alors, l'\'equation pr\'ec\'edente d\'efinit une 
application de $M$ dans $G$~; r\'eciproquement, la donn\'ee d'une application 
$g(x)$ de $M$ dans $G$ permet, lorsque le fibr\'{e} principal est trivial, de 
d\'efinir, via le choix d'une section globale $\sigma$, une transformation 
de jauge $\Phi$ par la m\^eme \'equation. Lorsque $P$ est trivial, on peut 
donc identifier le groupe de jauge ${\frak G}$ avec le groupe $\Omega^0(M,G)$ 
des applications diff\'erentiables de $M$ dans $G$. La correspondance n'est 
cependant pas canonique puisqu'elle d\'epend du choix d'une section globale 
$\sigma$. Cette identification ``explique'' pourquoi les automorphismes 
verticaux sont d\'esign\'es par les physiciens (des particules) sous le nom 
de ``transformations de jauge locales'', le mot ``local'' se r\'ef\'erant ici 
\`a la d\'ependance ``en $x$'' car la transformation $g(.)$ elle-m\^eme est 
globalement d\'efinie lorsque $P$ est trivial.

\subsection{Deux autres d\'efinitions des transformations de jauge}

\begin{itemize}
	\item  Soit $\Phi: P \mapsto P$ une transformation de jauge. On sait que 
	$\Phi(z)$ est dans la m\^eme fibre que $z$, on doit donc pouvoir obtenir 
	$\Phi(z)$ \`a partir de $z$ par action \`a droite d'un \'el\'ement de $G$ que 
	nous d\'esignerons par $\phi(z)$~:
	$$ \Phi(z) = z \, . \phi(z) $$
	Nous obtenons donc ainsi une application $\phi$ de $P$ dans $G$, mais 
	cette application n'est pas quelconque~; en effet, l'\'egalit\'e $\Phi(zg) = 
	\Phi(z)g$ peut s'\'ecrire \'egalement $zg \phi(zg) = z \phi(z) g$ et on obtient donc la 
	condition $$
	\phi(zg) = g^{-1} \phi(z) g
	$$
	Il est \'evident que $\Phi$ et $\phi$ se d\'eterminent l'un l'autre; on peut 
	donc identifier le groupe de jauge ${\frak G}$ avec l'ensemble 
	$\Omega^0_{Ad}(P,G)$ des applications de $P$ dans $G$ qui sont \'equivariantes 
	par l'action adjointe de $G$.
	
	\item La troisi\`eme d\'efinition de ${\frak G}$ r\'esulte en fait de la 
	pr\'ec\'edente et de la discussion men\'ee en section \ref{sec:sections} (description $2$). Soit, en 
	effet $Ad \, P = P \times_{Ad} G$ le fibr\'{e} adjoint, d\'efini comme fibr\'{e} 
	en groupes, associ\'e \`a $P$ gr\^ace \`a l'action adjointe de $G$ sur lui-m\^eme 
	(voir section \ref{sec:fibre-adjoint}),
	les \'el\'ements $\overrightarrow{\phi}$ de $Ad P$ sont des classes 
	d'\'equivalences $z . k = zg.g^{-1}kg$ et les sections $\overrightarrow{\phi}(x)$ de 
	$Ad \, P$ peuvent donc s'identifier aux applications de $P$ dans $G$ qui 
	sont $Ad$-\'equivariantes. Ainsi, nous 
	pouvons \'egalement d\'efinir le groupe de jauge ${\frak G}$ comme l'ensemble 
	des sections du fibr\'{e} adjoint $Ad \, P$.
\end{itemize}

\subsection{Automorphismes quelconques d'un espace fibr\'{e} principal}

Soit $P= P(M,G)$ un fibr\'{e} principal et ${\frak G}$ son groupe 
d'automorphismes verticaux (groupe de jauge). Nous allons \`a pr\'esent 
consid\'erer des automorphismes plus g\'en\'eraux que ceux consid\'er\'es dans les 
sous-sections pr\'ec\'edentes. Jusqu'\`a pr\'esent, nos automorphismes \'etaient 
verticaux, en ce sens que l'image $\Phi(z)$ de $z$ par $\Phi$ \'etait dans 
la m\^eme fibre que $z$. Nous allons garder les conditions 1 et 3 donn\'ees 
dans la d\'efinition du groupe de jauge (section \ref{sec:groupe-de-jauge})
 mais att\'enuer la deuxi\`eme condition~; les automorphismes devront 
pr\'eserver les fibres au sens suivant~: l'image d'une fibre devra \^etre une 
fibre, mais on n'imposera pas le fait qu'image et ant\'ec\'edent 
appartiennent \`a la {\it m\^eme} fibre! En d'autres termes, si $x$
d\'esigne un point de $M$, l'ensemble $\Phi(\pi^{-1}(x))$, image de la 
fibre au dessus de $x$ par $\Phi$ doit \^etre une fibre de $P$. Cette fibre 
image se projette en un certain point $y$ de $M$. L'action d'un tel 
automorphisme $\Phi$ d\'efinit donc \'egalement une application de $M$ dans 
$M$ (\`a $x$ on associe le point $y$ tel que $y=\pi(\Phi(\pi^{-1}(x)))$). 
Cette application est, par construction, un diff\'eomorphisme.
On obtient donc de cette fa\c con une projection du groupe $Aut \, P$ des 
automorphismes de $P$ (le groupe engendr\'e par les automorphismes que nous 
venons de d\'efinir) sur le groupe $Diff \, M$ des diff\'eomorphismes de $M$.
Deux automorphismes de $P$ se projetant sur le m\^eme diff\'eomorphisme de 
$M$ diff\`erent manifestement (au sens de la composition des morphismes) 
par un automorphisme ne changeant pas le point de base, c'est \`a dire par 
un automorphisme vertical. Localement, ces ``sym\'etries de jauge'' (nom 
qu'on donne quelquefois aux \'el\'ements de $Aut \, P$) sont donc cod\'ees \`a 
l'aide d'un diff\'eomorphisme de $M$ et d'une transformation de jauge.
 Plus pr\'ecis\'ement, on a la fibration principale ${\frak G}\mapsto Aut \, 
 P \mapsto Diff \, M$ (noter que ${\frak G}$ est un groupe de dimension 
 infinie) qu'on peut repr\'esenter par la figure \ref{fig:automorphisme-et-jauge}
 
\begin{figure}[htbp]
\epsfxsize=14cm
$$
    \epsfbox{automorphismes-fibre.eps}
$$
\caption{Automorphismes de fibr\'{e}s principaux}
\label{fig:automorphisme-et-jauge}
\end{figure}

%%%Discuter
%%%Produit semi-direct... Discuter en general les relations entre fibration et produit semi-direct
%%%


\subsection{Action des automorphismes sur les espaces fibr\'{e}s associ\'es}

\begin{itemize}
	\item  Soit $E = P \times_G F$ un espace fibr\'{e} associ\'e au fibr\'{e} principal 
	$P=P(M,G)$ obtenu \`a partir de l'action \`a gauche $\rho$ de $G$ sur $F$, 
	$z.f=zg.\rho(g^{-1})f \in E$. L'action \`a droite de $G$ sur $P$ n'existe 
	plus au niveau de $E$ puisqu'on a fabriqu\'e un espace quotient en 
	divisant par cette action. Par contre, on peut utiliser l'action \`a 
	gauche de ${\frak G}$ sur $P$ pour d\'efinir une action (\`a gauche) de ${\scr 
	G}$ sur n'importe quel fibr\'{e} associ\'e \`a $P$. Ainsi, 
	$$
	\Phi \in {\frak G}, z \in P, f \in F, z.f \in E {\stackrel \Phi 
	\longrightarrow} \Phi z.f \in E
	$$
	
	\item  ${\frak G}$ agit non seulement sur $E$ mais sur l'espace
        $\Gamma E$ de ses sections. Soit $ u \in \Gamma E$, $x \in M$, $u(x) \in E$~;on 
	d\'efinit
	$$
	[\Phi u](x)  =  \Phi(u(x))
	$$
\end{itemize}

\subsection{Le cas des espaces vectoriels (un cas trivial mais instructif!)}

Un fibr\'{e} vectoriel n'est autre, intuitivement, qu'une famille $E_x$ 
d'espaces vectoriels de m\^eme dimension, ``coll\'es'' ensemble, et 
param\'etris\'es par une vari\'et\'e ($x \in M$). Lorsque $M$ se r\'eduit \`a un seul 
point, on n'a qu'une seule fibre et donc un seul espace vectoriel. On 
peut donc consid\'erer un espace vectoriel $E$ comme un fibr\'{e} vectoriel au 
dessus d'un point! Ce fibr\'{e} vectoriel particuli\`erement trivial est 
associ\'e \`a un fibr\'{e} principal \'egalement constitu\'e d'une seule fibre, 
fibre qui n'est autre que l'ensemble $P$ des bases de l'espace vectoriel 
$E$. Si on suppose que $E$ est isomorphe \`a $\RR^n$, on voit que cette 
fibre est diff\'eomorphe au groupe $GL(n,\RR)$. Si on choisit une base 
$\sigma = \{\sigma_\mu\}$ de r\'ef\'erence (une section!), on obtient une 
correspondance bi-univoque entre bases (\'el\'ements de $P$) et matrices 
inversibles (\'el\'ements du groupe $GL(n,\RR)$). L'identification de $P$ avec 
$GL(n,\RR)$ d\'epend de la base $\sigma$ choisie. Le groupe matriciel 
$GL(n,\RR)$ agit sur $P$; cette action ne d\'epend pas du choix de $\sigma$.
En effet, si $e=(e_\mu)\in P$ et $\Lambda = (\Lambda^\mu_\nu) \in 
GL(n,\RR)$, on obtient $e'=e\, \Lambda \in P$ via $e_\nu'=e_\mu 
\Lambda^\mu_\nu$.
Les vecteurs $u$ de l'espace vectoriel $E$ sont des classes d'\'equivalence 
$u=e_\mu . u^\mu$ avec $e=\{e_\mu\}\in P$ et $u^\mu \in \RR^n$, la 
relation d'\'equivalence identifiant $e_\mu.u^\mu$ avec $e_\nu 
\Lambda^\nu_\mu .(\Lambda^{-1})^\mu_\rho u^\rho$. Le groupe $GL(n,\RR)$ 
n'agit donc pas sur $E$ (il n'agit que sur les composantes des vecteurs 
de $E$). Par contre, l'espace vectoriel $E$ poss\`ede un groupe 
d'automorphismes $Aut \, E$ (les applications lin\'eaires bijectives).
 Si $u\in E$ et $\Phi \in Aut \, E$ alors  $\Phi u \in E$~; les 
 automorphismes $\Phi$ agissent aussi sur les bases $e=\{e_\mu\}$, 
 l'action en question r\'esultant de l'action sur chacun des vecteurs de 
 base.
 Il faut bien voir que les groupes $Aut \, E$ et $GL(n,\RR)$ sont 
 diff\'erents!  {\it Cependant}, si on se choisit une base 
 $\sigma={\sigma_\mu}$ de r\'ef\'erence, on peut, de fa\c con \'el\'ementaire 
 ---voir cours de Terminale de nos 
 lyc\'ees--- associer, \`a tout \'el\'ement $\Phi$ de $Aut \, E$, une matrice 
 $\Lambda$ de $GL(n,\RR)$. Dans le cas d'un espace vectoriel, donc, le 
 groupe structural et le groupe des automorphismes (qui, dans ce cas, 
 sont n\'ecessairement verticaux), bien que conceptuellement distincts, 
 sont identifiables d\`es qu'on se choisit une base de r\'ef\'erence
(c'est \`a dire une section de ce fibr\'{e}~!). En particulier, lorsque $E$ est 
de dimension finie, ces deux groupes sont de dimension finie. D\`es qu'on 
passe au cas de fibr\'{e}s vectoriels au dessus d'une vari\'et\'e $M$ non r\'eduite 
\`a un point, l'identification n'est plus possible: $G=GL(n,\RR)$ reste ce 
qu'il \'etait mais ${\frak G}=Aut_V P$ devient un groupe de dimension 
infinie qu'on peut se repr\'esenter intuitivement comme une famille de 
groupe d'automorphismes d'espaces vectoriels (les fibres de $E$) 
param\'etris\'es par les points de la base $M$.
%%%
%%%
%%%\subsection{Sym\'etries des espaces fibr\'{e}s principaux}
%%% Otra vez ! (o nunca)
%%%
%%%%%%%%%%%%%%%%%%%%%%%%%%%%%%%%%%%%%%%
