
\chapter{Groupes de Lie et espaces homog\`enes}

\section{G\'en\'eralit\'es sur les groupes de Lie}

\subsection{G\'en\'eralit\'es et d\'efinitions \'el\'ementaires}

Les groupes de Lie et les espaces homog\`enes fournissent une multitude
d'exemples particuli\`erement simples de vari\'et\'es diff\'erentiables et 
c'est une des
raisons pour lesquelles nous leur consacrons une section de cet 
ouvrage. Une autre
raison importante est que les groupes de Lie vont \^etre utilis\'es comme 
``outils'' dans
les chapitres suivants.

Chacun est cens\'e \^etre d\'ej\`a familier avec la notion de structure de 
groupe.
L'introduction aux groupes et leur utilisation dans toutes les 
branches de la
physique est un th\`eme pr\'esent\'e et \'etudi\'e, suivant les ann\'ees et les 
r\'eformes de
l'enseignement secondaire, entre la classe de quatri\`eme et les ann\'ees 
de Licence...
Rappelons donc qu'un groupe est un ensemble (fini ou infini) muni 
d'une loi de
composition interne associative, poss\'edant un \'el\'ement neutre, et tel 
que tout \'el\'ement
poss\`ede un sym\'etrique pour la loi en question. Du point de vue du 
calcul, notons que,
dans un groupe, il est toujours possible de r\'esoudre une \'equation du 
premier degr\'e (du
type $a \, x = b,$ la solution \'etant $x = a^{-1} \, b$). Les exemples 
les plus simples
habituellement pr\'esent\'es aux \'el\`eves de nos lyc\'ees sont les suivants: 
Le groupe $(\ZZ,
+)$ des entiers relatifs, les groupes (additif et multiplicatif) de 
nombres rationnels
$(Q,+)$ et $(Q-\{0\},\times)$ ainsi que leurs g\'en\'eralisations r\'eelles 
et complexes,
les groupes de congruence $\ZZ_p=\ZZ/p\ZZ$, les groupes de sym\'etrie 
des solides
platoniques, les groupes de transformations lin\'eaires, affines ou 
projectives et les
groupes de substitutions. Les groupes ne sont pas n\'ecessairement 
commutatifs, comme
les derniers exemples le montrent clairement. Les groupes peuvent 
\^etre finis (comme
$\ZZ/p\ZZ$), infinis mais discrets (comme $\ZZ$) ou infinis et 
``continus'' (comme
$\RR$ ou comme le groupe $U(1)$ des rotations autour d'un axe). 
Regardons ce dernier
exemple d'un peu plus pr\`es. Toute rotation autour d'un axe est 
parfaitement
caract\'eris\'ee par un angle $\theta$ compris entre $0$ et $2 \pi$~; de 
surcro\^it, les
rotations d'angle $0$ et $2 \pi$ sont identiques. En d'autres termes, 
on peut
consid\'erer les rotations en question comme les diff\'erents points d'un 
cercle $S^1$
de rayon quelconque, l'\'el\'ement neutre (c'est \`a dire la rotation 
d'angle nul) \'etant
un point marqu\'e de ce cercle $S^1$. Ceci nous fournit un image 
``visuelle'' de ce
groupe $U(1)$, image qui peut nous faire oublier momentan\'ement la 
structure
alg\'ebrique proprement dite de cet ensemble (un groupe) mais qui 
attire notre
attention sur sa structure topologique ou m\^eme diff\'erentiable (un 
cercle).
La notion de groupe de Lie g\'en\'eralise ce dernier exemple en 
juxtaposant de fa\c con
axiomatique la structure de groupe et celles de vari\'et\'e.

Par d\'efinition, un groupe de Lie $G$ est donc une vari\'et\'e 
diff\'erentiable munie d'une
structure de groupe, de fa\c con \`a ce que les deux structures soient 
compatibles, c'est
\`a dire de fa\c con \`a ce que la multiplication \footnote{La loi de
composition interne
sera en effet, le plus souvent, not\'ee multiplicativement.} et le 
passage \`a l'inverse
soient des applications diff\'erentiables. Notons que la multiplication 
est une
application de $G\times G$ dans $G$ alors que le passage \`a l'inverse 
est une
application de $G$ dans $G$. Le lecteur pourra visuellement se 
repr\'esenter un
groupe de Lie comme un ``patato\"\i de'' avec multiplication (entre 
points) et origine
marqu\'ee (voir \ref{fig:groupe-Lie}).

\begin{figure}[htbp]
 %\def\epsfsize#1#2{0.5#1}
\epsfxsize=8cm
$$
    \epsfbox{groupe-Lie.eps}
$$
\caption{Un groupe de Lie}
\label{fig:groupe-Lie}
\end{figure}


La dimension d'un groupe de Lie est, par d\'efinition, sa dimension en 
tant que vari\'et\'e
(nous verrons de nombreux exemples un peu plus loin)~; notons d\`es \`a 
pr\'esent que le
groupe $U(1)$ pr\'esent\'e plus haut est de dimension $1$.

\subsection{Exemples \'el\'ementaires de groupes classiques}

On d\'esigne par $M(n,\CC)$ l'alg\`ebre (de dimension complexe $n^2$) des 
matrices carr\'ees
d'ordre $n$ \`a coefficients complexes et par $a^\dag$ l'adjointe d'une 
matrice $a$ de
$M(n,\CC)$ (si $a=(a_{ij})$, alors $a^\dag = (\overline{a_{ji}}) $). 
L'ensemble
pr\'ec\'edent n'est certes pas un groupe pour la loi de multiplication 
des matrices
puisqu'il contient de nombreux \'el\'ements non inversibles (toutes les 
matrices de
d\'eterminant nul) mais il contient plusieurs sous-ensembles 
int\'eressants qui, eux, sont
bien des groupes multiplicatifs, comme on pourra le v\'erifier ais\'ement.
\begin{itemize}
\item $GL(n,\CC) = \{a \in M(n,\CC) / \, \vert \, det \, a \neq 0 
\}$, le groupe lin\'eaire complexe.
\item $U(n) = \{a\in GL(n,\CC) / \, \vert \,  a^\dag = a^{-1}\}$, le 
groupe unitaire.
\item $SU(n) = \{a \in U(n,\CC) / \, \vert \, det \, a = 1\}$, le 
groupe unitaire unimodulaire, aussi
appel\'e groupe sp\'ecial unimodulaire.
\end{itemize}
Notons que les \'el\'ements de $U(n)$ ont automatiquement un d\'eterminant 
(un nombre
complexe) de module $1$, puisque $det  \, a^\dag = \overline{det  \, 
a} = 1 / det  \, a$,
mais pas n\'ecessairement \'egal \`a $1$.

Les groupes pr\'ec\'edents sont d\'efinis comme groupes de matrices~; les 
entr\'ees de ces
matrices (les ``\'el\'ements de matrice'') sont des nombres qui peuvent 
\^etre r\'eels mais
sont g\'en\'eralement complexes. Si on impose \`a ces \'el\'ements de matrice 
d'\^etre r\'eels, on
obtient de nouveaux groupes. Soit $M(n,\RR)$ l'alg\`ebre (de dimension 
r\'eelle $n^2$) des
matrices carr\'ees d'ordre $n$ \`a coefficients r\'eels. Cet ensemble, 
comme $M(n,\CC)$ est une
alg\`ebre associative mais n'est pas un groupe multiplicatif. On d\'efinit
\begin{itemize}
\item $GL(n,\RR) = GL(n,\CC) \cap M(n,\RR)$, le groupe lin\'eaire r\'eel.
\item $O(n,\RR) = U(n,\CC) \cap M(n,\RR)$, le groupe orthogonal, 
encore appel\'e groupe
des rotations.
\item $SO(n,\RR) = SU(n) \cap M(n,\RR)$, le groupe sp\'ecial orthogonal.
\end{itemize}
Les \'el\'ements du groupe unitaire ayant un d\'eterminant de module $1$, 
ceux de
$O(n,\RR)$ auront un d\'eterminant \'egal \`a $-1$ ou \`a $1$~; ceux pour 
lesquels il est
pr\'ecis\'ement \'egal \`a $1$ constituent le groupe $SO(n,\RR)$.
 On d\'esigne par ${\widetilde a}$ la
transpos\'ee d'une matrice $a$ de $M(n,\RR)$.

\section{G\'en\'eralit\'es sur les alg\`ebres de Lie}

\subsection{Application exponentielle et alg\`ebres de Lie}

\subsubsection{D\'efinition}

Une alg\`ebre de Lie ${\frak g}$ sur un corps commutatif ${\KK }$ est un ensemble 
qui est, d'une part
un espace vectoriel sur ${\KK }$ (sa loi de groupe ab\'elien est not\'ee 
$+$ et sa loi
externe sur ${\KK }$ est not\'ee multiplicativement), de dimension finie ou non, et qui, d'autre 
part, est muni d'une loi de
composition interne --non associative-- g\'en\'eralement not\'ee $[\, ,\,]$ 
v\'erifiant les
propri\'et\'es suivantes
\begin{description}
\item[\tt Anticommutativit\'e] {\hskip 0.1cm} $\forall X, Y \in {\frak 
g} {\hskip 1cm}  [X,Y]=-[Y,X] $
\item[\tt Identit\'e de Jacobi] {\hskip 0.1cm} $\forall X,Y,Z \in 
{\frak g}
 {\hskip 0.4cm} [X,[Y,Z]]+[Z,[X,Y]]+[Y,[Z,X]]=~0$ 
\end{description}
On suppose \'egalement v\'erifi\'ee la lin\'earit\'e par rapport aux scalaires, 
c'est \`a dire
$[\alpha X,Y]=[X,\alpha Y]=\alpha [X,Y]$ si $\alpha \in {\KK }$.
La loi $[\, ,\, ]$ est g\'en\'eralement d\'esign\'ee sous le nom de ``crochet 
de Lie''.
Dans toute la suite, le corps ${\KK }$ co\"incidera avec le corps $\CC$ 
des nombres complexes.

\subsubsection{Exemple fondamental}

Soit ${\cal A}$ une alg\`ebre associative~; on peut lui associer 
canoniquement une
alg\`ebre de Lie en d\'efinissant le crochet de Lie de la fa\c con suivante 
(auquel cas le
crochet de Lie peut \'egalement \^etre d\'esign\'e sous le nom de 
commutateur):
$$X,Y \in {\cal A} \rightarrow [X,Y]  =  X Y - Y X $$
Le crochet obtenu est g\'en\'eralement non nul, sauf \'evidemment si $X$ et 
$Y$ commutent.
Par ailleurs on v\'erifie ais\'ement que les propri\'et\'es 
d'anticommutativit\'e du crochet
ainsi que l'identit\'e de Jacobi sont automatiquement satisfaites. Les 
ensembles de
matrices $M(n,\CC)$ et $M(n,\RR)$ sont donc automatiquement des 
alg\`ebres de Lie.

\subsubsection{Constantes de structure d'une alg\`ebre de Lie ${\frak 
g}$}

Supposons que ${\frak g}$, en tant qu'espace vectoriel sur le corps 
des complexes $\CC$
soit de dimension finie $n$ et soit $\{X_\alpha\}_{\alpha \in \{1 
\ldots n\}}$ une base
de ${\frak g}$.
Le crochet de Lie $[X_\alpha,X_\beta]$ de deux vecteurs de base est 
{\it a priori} un
\'el\'ement de ${\frak g}$ et peut donc se d\'evelopper sur la base choisie:
$$ [X_\alpha,X_\beta] = C_{\alpha \beta}^\gamma X_\gamma $$
Les $n^3$ nombres $ C_{\alpha \beta}^\gamma$ sont les constantes de 
structure de
${\frak g}$ par rapport \`a la base choisie.

\subsubsection{Application exponentielle dans $M(n,\CC)$}

On d\'esigne par $\exp: \alpha \rightarrow 
\sum_{p=0}^{\infty}\alpha^p/p!$
l'application exponentielle d\'efinie sur $M(n,\CC)$.
Posons $g=e^A$. Il est facile de voir que $$det  \, g = e^{Tr \, A}$$
Cette relation est \'evidente si $A$ est diagonalisable (puisque 
$e^{\lambda_1} \ldots e^{\lambda_p}  =  
e^{\lambda_1+\ldots+\lambda_p}$). Si ce n'est pas le cas, on 
utilise pour d\'emontrer cette propri\'et\'e g\'en\'erale le fait que 
l'ensemble des matrices diagonalisables sur $\CC$ est 
dense.  Cette relation est \`a la 
base d'une quantit\'e de r\'esultats dont voici le premier: si $A \in 
M(n,\CC)$, alors $g = e^A \in GL(n,\CC)$~; en effet, $det  \, g$ 
n'est 
jamais nul puisque la fonction $z \rightarrow e^z$ ne s'annule pas.

$\triangle$ {\it ATTENTION\/}: On n'a pas dit que tout \'el\'ement de 
$GL(n,\CC)$ pouvait \^etre 
atteint par la fonction $\exp$ (c'est faux!).

\subsubsection{Cas des groupes de matrices: Correspondance entre 
groupes 
et alg\`ebres de Lie}

Soit $G$ un groupe de Lie d\'efini comme sous-ensemble de $M(n,\CC)$. 
On 
d\'efinit son alg\`ebre de Lie not\'ee ${\frak g}$ ou $Lie\, G$ comme 
suit,
$$ Lie\, G = \{A\in M_n(\CC) {\hskip 0.2cm}\vert \,  {\hskip 0.2cm} 
\forall t \in \RR,  {\hskip 0.2cm} e^{t A} \in G\}$$
 De fa\c con un peu imag\'ee, on peut dire que l'alg\`ebre de Lie d'un 
groupe 
$G$, c'est$\ldots$ son logarithme! De fait, l'utilisation de 
l'alg\`ebre 
de Lie de $G$ permet de lin\'eariser les propri\'et\'es des groupes, c'est 
\`a dire 
de transformer les multiplications en additions \etc .

La d\'efinition ci-dessus
 de l'alg\`ebre de Lie d'un groupe $G$ semble un peu 
restrictive en ce sens qu'elle semble ne pouvoir s'appliquer qu'aux 
groupes de matrices,
mais il 
existe une d\'efinition plus abstraite de la notion d'alg\`ebre de Lie 
d'un 
groupe de Lie, d\'efinition ne
 faisant pas l'hypoth\`ese d'une r\'ealisation 
matricielle ; nous y reviendrons plus loin.

%% En fait il est possible de d\'emontrer que tout groupe 
%%de Lie de dimension finie est isomorphe \`a un groupe de matrices 
%%$***$, 
%% donc l'hypoth\`ese qui pr\'ec\`ede n'est pas une restriction. Par ailleurs,

Soient $g$ et $h$ deux \'el\'ements de $G$ et supposons qu'on puisse 
\'ecrire $g = e^{tA}$ 
et $h=e^{tB}$ avec $A,B \in  {\frak g}$. Tout d'abord, notons que 
$g^{-1}=e^{-tA}$.
 On peut alors consid\'erer
 le commutateur de $g$ et $h$ au 
sens de la th\'eorie des groupes, c'est \`a dire l'\'el\'ement 
$c=ghg^{-1}h^{-1}$ 
de $G$. Au second ordre en $t$, il vient
\begin{eqnarray*}
c &  = & e^{tA}e^{tB}e^{-tA}e^{-tB} \\
  &  = &
(1 + tA + t^2 A^2/2! + \ldots)(1 + tB + t^2 B^2/2! + \ldots) \\
  & {} & {\hskip 1.25cm}
(1 - tA + t^2 A^2/2! + \ldots)(1 - tB + t^2 B^2/2! + \ldots) \\
   &  = &
1 + 0 \,  t + t^2 [A,B] + O(t^3)
\end{eqnarray*}
Il ne faudrait pas trop h\^ativement en d\'eduire que le commutateur dans 
$G$ est \'egal \`a l'exponentielle du commutateur dans $ Lie\, G$, 
mais c'est ``presque'' vrai, comme on vient de le voir
 ($c \, \mathrel{\oalign{$\, \sim$\cr${\scriptstyle t\rightarrow 0}$}} \,  1 + 
t^2[A,B]$). De plus, on peut d\'emontrer que 
$$e^{t^2 [A,B]} ={\lim_{n \rightarrow \infty}}
(e^{tA/n}e^{tB/n}e^{-tA/n}e^{-tB/n})^{n^2}$$
 C'est \`a l'aide de ces 
relations qu'on peut s'assurer que l'alg\`ebre de Lie 
d'un groupe de Lie est bien$\ldots$ une alg\`ebre de Lie (l'ensemble 
est 
bien stable par le commutateur).
%%%Remarques sur BCH$***$.

Soit $g \in G$ et supposons qu'on puisse \'ecrire $g=e^A$~; alors, en 
utilisant la structure d'espace vectoriel de $Lie G$, on voit qu'on 
peut 
d\'ecomposer $A$ sur une base $\{X_\alpha\}$~; ainsi, $A= \sum a^\alpha 
X_\alpha$. Les $n$ nombres $a^\alpha$ permettent donc de d\'efinir sur 
$G$ 
un syst\`eme de coordonn\'ees (une carte). Ceci montre \'egalement que la 
dimension de $G$, en tant que vari\'et\'e, est \'egale \`a celle de $ Lie\, 
G$, consid\'er\'e comme espace vectoriel.



\subsection{Correspondance entre groupes et alg\`ebres de Lie}

\subsubsection{Alg\`ebres de Lie des groupes classiques}

Notons d'abord que, pour les groupes unitaires,
$$ e^A \in U(n) \Longleftrightarrow e^A e^{A^\dag} = 1 
\Longleftrightarrow 
A + A^\dag = 0$$
ainsi la matrice $A$ est anti-hermitienne.

Nous avons d\'ej\`a rencontr\'e la relation $ det  \, e^A = e^{Tr A}$~; il 
s'ensuit que, si le d\'eterminant de $g=e^A$ est \'egal \`a $1$, la trace 
de 
$A$ est nulle. Ainsi,
$$
e^A \in SU(n) \Longleftrightarrow [A + A^\dag = 0 \mbox{ et } Tr A = 
0 ]
$$
Dans le cas des groupes orthogonaux, la d\'efinition implique 
imm\'ediatement
$$
e^A \in O(n) \Longleftrightarrow [e^A e^{A^t} = 1 \mbox{ et $A$ 
r\'eel}] \Longleftrightarrow  [A + A^t = 0 
\mbox{ et $A$ r\'eel}]
$$
Les matrices $A$ de l'alg\`ebre correspondante sont donc 
antisym\'etriques 
r\'eelles, ce qui, en particulier, implique la nullit\'e des \'el\'ements de 
matrice diagonaux et donc de la trace~; mais le seul fait que $tr 
A=0$ 
implique $det \, e^A=1$ et donc $e^A \in SO(n)$. Y aurait-il une 
contradiction? Comment donc obtenir une matrice orthogonale de 
d\'eterminant diff\'erent de $1$?
 Il est pourtant bien \'evident que la d\'efinition de $O(n)$ 
est diff\'erente de celle de $SO(n)$! La seule conclusion possible est 
la 
suivante~: les \'el\'ements de $O(n)$ qui ne sont pas dans $SO(n)$  ne 
sont 
pas atteints par la fonction $\exp$ (voir la remarque \`a la fin du 
pr\'esent 
paragraphe).

Pour calculer la dimension des groupes de Lie, le plus simple est en 
g\'en\'eral de calculer la dimension des alg\`ebres de Lie correspondantes. 
Voici un exemple que lecteur pourra g\'en\'eraliser sans peine: 
``Fabriquons'' une matrice carr\'ee antihermitienne. Une matrice $n 
\times 
n$ d\'epend, {\it a priori}, de $n^2$ param\`etres complexes~; nous 
enlevons 
d'abord la diagonale (donc il reste $n^2 - n$ param\`etres), puis nous 
fabriquons une matrice triangulaire inf\'erieure stricte (donc 
$(n^2-n)/2$ 
param\`etres)~; la partie triangulaire sup\'erieure est alors 
compl\`etement 
d\'etermin\'ee par la condition d'anti-hermiticit\'e~; finalement, cette 
m\^eme 
condition implique que les \'el\'ements diagonaux sont imaginaires purs: 
il 
nous faut donc rajouter $n$ param\`etres r\'eels. Au total, on a donc 
$2(n^2-n)/2+n=n^2$param\`etres r\'eels.
 Ainsi donc $dim_R \, U(n)= dim_R \,  Lie \, U(n)=n^2$.
 
Le lecteur pourra sans doute ainsi retrouver sans difficult\'e la 
dimension 
des alg\`ebres de Lie suivantes. Remarque : La notation $Sp(n)$ 
utilis\'ee 
ci-dessous d\'esigne le groupe unitaire-quaternionique (voir 
``remarques 
diverses'' en fin de section 2 concernant les groupes symplectiques) 
; les
matrices de l'alg\`ebre de Lie correspondante sont du type
$\begin{pmatrix} A & B \\ - B^\dag & -\widetilde A  \end{pmatrix}$ avec $A^\dag = -A$ et 
$\widetilde B = B$.

{\vskip 1.0 cm}
\begin{tabular}{lcr}
$G$ & $Lie \, G$ & $ dim_{\RR} \, G$  \\
{} & {} & {} \\
\hline
{} & {} & {} \\
$GL(n,\CC)$ & $M(n,\CC)$ & $2\, n^2$ \\
$GL(n,\RR)$ & $M(n,\RR)$ & $n^2$ \\
$U(n)$ & Matrices anti-hermitiennes & $n^2$ \\
$SU(n)$ & Matrices anti-hermitiennes de trace nulle & $n^2-1$ \\
$SO(n)$ & Matrices antisym\'etriques r\'eelles & $\frac{n(n-1)}{2}$ \\
$Sp(n)$ & Voir ci-dessus 
 & $\frac{2n(2n+1)}{2}$
\end{tabular}
{\vskip 1.0 cm}

{\bf Remarques}

\begin{itemize}

\item Si nous ne pr\'ecisons pas davantage, c'est que les alg\`ebres de 
Lie que nous 
consid\'erons sont des alg\`ebres de Lie {\em r\'eelles}. Il y a l\`a une 
petite 
subtilit\'e que nous allons illustrer en consid\'erant le cas de
 ${\frak u}(n) =  Lie\, U(n)$. Il s'agit d' un espace vectoriel 
sur 
$\RR$ de dimension $d=n^2$, ce qui signifie qu'une base de cet espace 
vectoriel r\'eel poss\`ede $d=n^2$ \'el\'ements (appelons les 
$\{X_\alpha\}_{\alpha 
= 1 \ldots d}$) et qu'un \'el\'ement quelconque $A$ de ${\frak u}(n)$ 
peut 
s'\'ecrire $A = \sum_{\alpha = 1}^{d} a^\alpha X_\alpha$, avec des 
composantes $a^\alpha$ qui sont des nombres r\'eels. Par contre, les 
\'el\'ements $\{X_\alpha\}$ sont, dans le cas pr\'esent des matrices 
antihermitiennes dont les \'el\'ements de matrice sont g\'en\'eralement 
complexes 
(comme ceux de $A$, d'ailleurs). Pour compliquer l\'eg\`erement les 
choses, 
les \'el\'ements $\{X_\alpha\}$ qu'on appelle traditionnellement {\sl  
g\'en\'erateurs \/}\index{g\'en\'erateurs infinit\'esimaux}
 de l'alg\`ebre de Lie ${\frak u}(n)$ ou encore {\sl g\'en\'erateurs 
 infinit\'esimaux\/}, sont souvent \'ecrits sous la forme $X_\alpha= i 
Y_\alpha$
  (dans le cas de ${\frak u}(n)$ les $Y_\alpha$ sont donc hermitiens) 
et 
  le d\'eveloppement de $	A$ sur la base $X_\alpha$ se re-\'ecrit
   $A = \sum_{\alpha = 1}^{d} a^\alpha i Y_\alpha$, de sorte que si 
on 
   pose $A = i B$ on obtient simplement $B=\sum_{\alpha = 1}^{d} 
a^\alpha 
   Y_\alpha$~; dans ce cas, il y a des facteurs $i$ au second membre 
des 
   relations de commutation des $Y_\alpha$ entre eux. Pour couronner 
   le tout les $Y_\alpha$ sont eux-aussi quelquefois d\'esign\'es sous le 
nom de 
   ``g\'en\'erateurs infinit\'esimaux'', bien qu'ils 
   n'appartiennent m\^eme plus \`a l'alg\`ebre de Lie si cette derni\`ere est r\'eelle !

\item L'application $\exp$ est continue. L'image continue d'un espace 
connexe est un espace connexe. Une alg\`ebre de Lie est un espace 
vectoriel 
et donc un espace connexe. L'ensemble $\exp {\frak g} = \{e^X \vert 
\,  X \in {\frak 
g}\}$ est donc connexe. Conclusion: si un groupe de Lie $G$ n'est pas 
connexe, les \'el\'ements qui n'appartiennent pas \`a la composante connexe 
de 
l'identit\'e ne peuvent pas \^etre atteints par la fonction $\exp$ (ils 
ne 
peuvent pas s'\'ecrire sous la forme $e^X$). Ceci montre que, dans bien 
des 
cas, l'application $\exp$ n'est pas surjective. Le calcul effectu\'e 
plus haut et concernant le groupe orthogonal $O(n)$ refl\`ete le fait 
que 
ce dernier n'est pas connexe. Par contre les 
groupes $U(n)$, $SU(n)$, $SO(n)$ et $Sp(n)$ sont connexes.

M\^eme si $G$ est connexe, l'application $\exp$ n'est pas 
n\'ecessairement 
surjective. Par contre, on d\'emontre que si $G$ est compact et 
connexe, 
cette application est surjective (c'est le cas de $U(n)$, $SU(n)$, 
$SO(n)$ et $Sp(n)$). Si $G$ est connexe mais non compact, on d\'emontre 
que $\exp$ est ``presque'' surjective, en ce sens que $$\forall g \in 
G,\, 
\exists A_1, A_2 \ldots A_p \in  {\frak g} {\hskip 0.3cm} p \mbox{ 
fini, tel que } g = e^{A_1} e^{A_2} \ldots e^{A_p}$$

\item Par d\'efinition, le {\sl rang}\index{rang} d'un groupe de Lie compact est 
\'egal \`a 
 \`a la dimension d'un sous groupe 
ab\'elien maximal contenu dans $G$ (on dit alors souvent ``tore maximal'' au 
lieu de
``sous groupe ab\'elien maximal''). \\
Cette d\'efinition sera suffisante pour nous, mais voici n\'eanmoins une d\'efinition valable dans un contexte plus g\'en\'eral: 
le rang d'un groupe de Lie est d\'efini comme \'etant celui de l'alg\`ebre de Lie correspondante, lui-m\^eme d\'efini comme la dimension de l'une quelconque de ses sous-alg\`ebres de Cartan (si le corps de base est celui des complexes et que l'alg\`ebre de Lie est de dimension finie, toutes ses sous-alg\`ebres de Cartan sont isomorphes);  dans ce cadre g\'en\'eral une sous-alg\`ebre de Cartan est une sous-alg\`ebre de Lie nilpotente qui coincide avec son propre normalisateur. Dans le cas semi-simple une sous-alg\`ebre de Cartan est simplement une sous-alg\`ebre de Lie ab\'elienne maximale. 



\end{itemize}

\subsubsection{Isomorphisme local: comparaison entre $SU(2)$ et 
$SO(3)$}

Nous avons d\'ej\`a vu (dans le cas du groupe orthogonal $O(n)$) que les 
\'el\'ements d'un groupe n'appartenant pas \`a la composante connexe de 
l'identit\'e ne pouvaient pas \^etre atteints par la fonction 
exponentielle. 
Pour cette raison, nous supposerons que tous les groupes de Lie 
consid\'er\'es dans la pr\'esente sous-section sont connexes (cas de 
$SO(n)$). 
Nous nous int\'eressons en effet ici \`a des ph\'enom\`enes plus fins que la 
connexit\'e.

\begin{itemize}

\item
Soient
 $$\sigma_1 =  \begin{pmatrix} 0&1\\ 1&0 \end{pmatrix}, \,
\sigma_2 = \begin{pmatrix}  0&-i\\ i&0\end{pmatrix}, \,
\sigma_3 = \begin{pmatrix}  1&0\\ 0&-1\end{pmatrix} $$ les trois {\sl matrices 
de Pauli\/}\index{matrices de Pauli}. 
$Lie(SU(2))$ est l'espace vectoriel engendr\'e par $X_1,X_2,X_3$ avec 
$X_j=i\sigma_j/2$ puisque $\{X_1,X_2,X_3\}$ constituent une base de 
l'alg\`ebre des matrices antihermitiennes de trace nulle. Notons que 
$$[X_i,X_j]=-\epsilon_{ijk}X_k$$ o\`u $\epsilon$ est compl\`etement 
antisym\'etrique et $\epsilon_{123}=1$. En d\'eveloppant la fonction 
exponentielle en  s\'erie et en utilisant les propri\'et\'es 
$\sigma_3^2=1$, 
$\sigma^{2p+1}=\sigma_3$, le lecteur montrera ais\'ement que 
$$\exp(\theta 
X_3)= \cos \frac{\theta}{2} + i \sigma_3 \sin \frac{\theta}{2} \in 
SU(2)$$
Notons que 
\begin{eqnarray*}
\exp[(\theta + 2 \pi)X_3] & = & - \exp[\theta X_3] \\
\exp[(\theta + 4 \pi)X_3] & = & + \exp[\theta X_3] 
\end{eqnarray*}

\item
Soient maintenant $$
X_1 = -\begin{pmatrix}  0&0&0\\ 0&0&1\\ 0&-1&0\end{pmatrix},
X_2 = -\begin{pmatrix}  0&0&1\\ 0&0&0\\ -1&0&0\end{pmatrix},
X_3 = -\begin{pmatrix}  0&1&0\\ -1&0&0\\ 0&0&0\end{pmatrix} $$
L'espace vectoriel engendr\'e par $X_1,X_2,X_3$ est constitu\'e par 
l'ensemble des matrices antisym\'etriques r\'eelles $3\times 3$~; il 
co\"incide 
donc avec l'alg\`ebre de Lie $Lie(SO(3))$. Comme dans le cas pr\'ec\`edent, 
on 
peut v\'erifier que que $$[X_i,X_j]=-\epsilon_{ijk}X_k$$ Les deux 
alg\`ebres
 $Lie(SO(3))$ et  $Lie(SU(2))$ sont donc isomorphes. Par ailleurs, en 
 d\'eveloppant la fonction $\exp$ en s\'erie et en utilisant les 
propri\'et\'es 
 $X_3^{2p}=diag\, ((-1)^p,(-1)^p,0)$, $X_3^{2p+1}=(-1)^{p}X_3$, il 
est facile de voir que 
$$\exp[\theta X_3] =  diag(\cos \theta,\cos \theta,1) + X_3 \sin 
\theta \in SO(3)$$
 Notons alors que
 \begin{eqnarray*}
 \exp[(\theta + 2 \pi)X_3] & = & + \exp[\theta X_3]
 \end{eqnarray*}
 
 \item
 Ainsi donc, lorsqu'``on fait un tour'' dans $SO(3)$, on revient \`a 
 l'identit\'e -- chose qu'on savait d\'ej\`a~! -- mais, dans $SU(2)$, pour 
 revenir \`a l'identit\'e, il faut faire~$\dots$ deux tours~{ !} 
 Cette diff\'erence de comportement entre les deux groupes peut sembler 
 assez surprenante \`a premi\`ere vue. Il est possible de l'illustrer de 
 fa\c con assez simple gr\^ace \`a une exp\'erience \'el\'ementaire. 
 
 {\em Exp\'erience utilisant \/} $SO(3)$:
 Prenez un objet quelconque, posez-le sur la table et faites-lui 
subir 
 une rotation de $360$ degr\'es autour d'un axe vertical~; la 
configuration 
 que vous obtenez est indiscernable de la configuration initiale.
 
 {\em Exp\'erience utilisant \/} $SU(2)$: 
 Prenez un objet quelconque, suspendez-le au milieu de la pi\`ece en 
 utilisant huit \'elastiques reli\'es aux huit coins (haut et bas) de la 
 pi\`ece (vous pouvez utiliser un moins grand nombre d'\'elastiques~!) et 
 faites subir \`a votre objet une rotation de $360$ degr\'es~; notez 
que les 
 \'elastiques sont emm\^el\'es~; essayez de d\'em\^eler les \'elastiques sans 
faire 
 tourner l'objet$\ldots$ vous n'y parvenez pas. Faites alors subir \`a 
 votre objet une seconde rotation de $360^\circ$ (depuis la 
configuration 
 initiale vous aurez ainsi effectu\'e une rotation de $4\pi=720^\circ 
$)~; les 
 \'elastiques semblent \^etre encore plus emm\^el\'es~; essayez de d\'em\^eler 
ces 
 \'elastiques (retrouver la configuration initiale) sans faire tourner 
 l'objet$\ldots$ A votre grande surprise (m\^eme si vous avez fait 
 cette exp\'erience plusieurs fois) vous y parvenez~! 
 
{\em Remarque}: Si vous avez vraiment des difficult\'es \`a d\'em\^eler les 
 \'elastiques, ouvrez l'ouvrage \cite{MTW} o\`u la suite des mouvements \`a 
 effectuer est d\'ecrite en d\'etails.
 
 Il existe une autre exp\'erience, encore plus simple, mais un peu plus 
 difficile \`a d\'ecrire ``avec des mots'', qui illustre la m\^eme 
diff\'erence 
 de comportement entre les deux groupes et qui illustre donc la fa\c con 
 dont $SU(2)$ d\'ecrit les ``rotations d'objets attach\'es \`a leur 
 environnement''. Prenez un verre (rempli de votre vin favori) et 
 essayez, par pivot du poignet, de lui faire subir une rotation de 
$360 
 ^\circ  \ldots$ \'echec~: \`a moins d'avoir des articulations tr\`es 
sp\'eciales, 
 vous vous retrouvez tout tordu. Essayez alors, \`a partir de cette 
 position (tordue) de faire subir \`a votre verre une seconde rotation, 
 dans le m\^eme sens, de $360 ^\circ $ (le coude doit normalement 
s'abaisser) 
 et \c ca marche~: Vous vous retrouvez dans l'\'etat initial~!
 
 Ce ph\'enom\`ene amusant est d'une importance physique capitale. C'est 
lui 
 qui, en d\'efinitive, explique la diff\'erence entre fermions et 
 bosons (rappelons que les \'electrons --- et plus g\'en\'eralement les 
 particules de spin demi-entier --- ob\'eissent \`a la statistique de 
 Fermi-Dirac alors que les photons (ou les noyaux d'H\'elium~!) --- et 
plus 
 g\'en\'eralement les particules de spin entier --- ob\'eissent \`a la 
statistique de 
 Bose-Einstein.
 
 \item
 Revenons aux math\'ematiques. Nous avons un homomorphisme de $SU(2)$ 
dans 
 $SO(3)$~: l'image de $\exp(\theta^i X_i) \in SU(2)$ est, par 
d\'efinition 
 $\exp(\theta^i X_i) \in SO(3)$ o\`u les $X_i$ sont, bien entendu, 
d\'efinis de deux fa\c cons 
 diff\'erentes, comme pr\'ec\'edemment. Ce morphisme surjectif n'est pas
injectif~; en 
 effet, les deux \'el\'ements distincts $\exp(2\pi X_3) = -1$ et 
$\exp(4\pi 
 X_3)=1$ de $SU(2)$ se projettent tous deux sur l'identit\'e de 
$SO(3)$. Le 
 noyau de cet homomorphisme est donc $\ZZ_2 = \{-1,1\}$ d'o\`u il 
s'ensuit 
 que $SO(3) = SU(2)/\ZZ_2$. En effet, un th\'eor\`eme tr\`es \'el\'ementaire de 
 th\'eorie des groupes nous apprend que si $\ell$ est un homomorphisme 
du 
 groupe $G$ dans le groupe $K$, alors l'image $\ell(G)$ est isomorphe 
au 
 quotient de $G$ par le noyau de $\ell$ (dans le cadre commutatif, ce 
 th\'eor\`eme g\'en\'eralise un r\'esultat bien connu et rencontr\'e, par 
exemple, 
 dans l'\'etude des espaces vectoriels).
 
 \item
 Deux groupes poss\'edant des alg\`ebres de Lie isomorphes sont dits {\sl 
 localement isomorphes \/}\index{groupes localement isomorphes}.
  Ainsi $SU(2)$ et $SO(3)$ sont localement 
 isomorphes. Ils ne sont cependant pas isomorphes.
 
 On admettra le r\'esultat suivant. 
 {\sf Deux groupes compacts connexes non 
 isomorphes peuvent admettre des alg\`ebres de Lie isomorphes
 (on dit qu'il s'agit de groupes 
 localement isomorphes). Les groupes de Lie qui admettent la m\^eme 
alg\`ebre 
 de Lie ${\frak g}$ sont tous de la forme $G_i=G/D_i$ o\`u $D_i$ est un 
 sous-groupe discret distingu\'e de $G$. Le sous-groupe $D_i$ est 
isomorphe au groupe 
 fondamental de $G_i$ (\ie au premier groupe d'homotopie 
$\pi_1(G_i)$) et le groupe $G$ est simplement connexe (ce qui 
signifie que son 
 sous-groupe fondamental est r\'eduit \`a l'identit\'e).
 $G$ et est appel\'e 
 rev\^etement universel de $G_i$. On note quelquefois $G=\widehat G_i$}
 
 \item
 Exemples de groupes de Lie localement isomorphes.


 \begin{tabular}{lcr}
 $SU(2)$ et $SO(3)=SU(2)/\ZZ_2$ & $\pi_1(SU(2))=1$ & 
$\pi_1(SO(3))=\ZZ_2$ \\
 $SU(3)$ et $SU(3)/\ZZ_3$  & $\pi_1(SU(3))=1$ & 
$\pi_1(SU(3)/\ZZ_3)=\ZZ_3$ \\
 $\RR$ et $U(1)=\RR/\ZZ$ & $\pi_1(\RR)=1$ & $\pi_1(U(1)) = \ZZ$
 \end{tabular}
 \item
 Les groupes $SO(n)$ ne sont jamais simplement connexes.
\begin{itemize}
\item
 Lorsque $n=2$, 
 $SO(2) = U(1) = S^1$ et on sait que $\pi_1(S^1)=\ZZ$~; le rev\^etement 
 universel de $U(1)$ est $\RR$, l'ensemble des r\'eels~: par 
 d\'efinition du cercle (p\'eriodicit\'e) on sait que $U(1)=\RR/\ZZ$. 
\item 
Lorsque $n=3$, on a vu que le rev\^etement universel de $SO(3)$ est 
$SU(2)$ 
et que $\pi_1(SO(3))=\ZZ_2$.
\item
Lorsque $n\geq 3$, on montre que $\pi_1(SO(n)) = \ZZ_2$. Le 
rev\^etement 
universel $\widehat{SO(n)}$ de $SO(n)$ se note $Spin(n)$. Le fait que 
$Spin(3)=SU(2)$ est une co\"incidence de basse dimension~; on montre 
que 
$Spin(4)=SU(2)\times SU(2)$, $Spin(5)=U(2,\HH)\equiv Sp(2) \equiv 
USp(4)$, $Spin(6)=SU(4)$. 
\item
Lorsque $n>6$, $Spin(n)$ n'est autre que$\ldots$ $Spin(n)$ et ne 
co\"incide pas avec un autre groupe classique. Pour construire 
explicitement $Spin(n)$, le plus simple est d'utiliser les alg\`ebres 
de 
Clifford (voir la discussion en fin de chapitre).
\end{itemize}
\end{itemize}

\subsection{Classification des groupes et alg\`ebres de Lie. 
G\'en\'eralit\'es.}

\subsubsection{Un peu de terminologie}
\begin{itemize}
\item Une alg\`ebre de Lie est {\sl ab\'elienne \/} si elle est$\ldots$ 
commutative.
\item Une alg\`ebre de Lie est {\sl simple \/}\index{alg\`ebre de Lie simple} si elle n'est pas 
ab\'elienne et 
si elle ne poss\`ede aucun id\'eal bilat\`ere non trivial.
\item Une alg\`ebre de Lie est {\sl semi-simple \/}\index{alg\`ebre de Lie 
semi-simple} si elle peut 
s'\'ecrire comme
(si elle est isomorphe \`a une) somme directe d'alg\`ebres simples.
\item Une alg\`ebre de Lie est {\sl non semi-simple\/} si elle n'est 
pas 
semi-simple. 
\end{itemize}
On a bien entendu une terminologie analogue au niveau des groupes.
\begin{itemize}
\item Un groupe de Lie est {\sl ab\'elien \/} s'il est$\ldots$ 
commutatif.
\item Un groupe de Lie  est {\sl simple \/} s'il n'est pas ab\'elien et 
s'il ne poss\`ede aucun sous groupe distingu\'e ({\sl invariant \/}) non 
trivial.
\item Un groupe de Lie  est {\sl semi-simple \/} s'il peut s'\'ecrire 
comme
(s'il est isomorphe \`a un) produit direct de groupes simples.
\item Un groupe de Lie  est {\sl non semi-simple\/} s'il n'est pas 
semi-simple.
\end{itemize}
\subsubsection{Id\'ees fondamentales de la classification}
\begin{itemize}
\item On tente d'abord de classifier les alg\`ebres de Lie. On en 
d\'eduit la 
classification des groupes de Lie. Nous supposerons toujours, dans 
cette
section, et sauf mention explicite du
contraire, que nous sommes en dimension finie.
\item On montre qu'une alg\`ebre de Lie quelconque peut toujours se 
d\'ecomposer en une somme directe d'une alg\`ebre de Lie semi-simple et 
d'une 
alg\`ebre de Lie non semi-simple particuli\`ere qu'on appelle son radical (d\'ecomposition de Levi). Pour d\'efinir le radical d'une alg\`ebre de Lie 
${\frak g}$, on proc\`ede comme suit: on commence par construire la ``s\'erie d\'eriv\'ee'' $({\frak g}^{(i)})$
 de $\frak g$ d\'efinie par 
${\frak g}^{(i+1)}  =  [{\frak g}^{(i)},{\frak g}^{(i)}]$. Chaque terme de cette suite est un id\'eal de
$\frak g$ contenant le terme suivant.
Notons que ${\frak g}$ est ab\'elienne lorsque le premier terme de cette suite 
(c'est \`a dire  ${\frak g}^{(1)}$) est nul. L'alg\`ebre de Lie $\frak g$ est dite {\sl r\'esoluble} lorsque ${\frak g}^{(k)} = 0$ pour une certaine valeur de $k$. Etre r\'esoluble est ainsi, pour une alg\`ebre de Lie, une notion un peu plus faible que celle d'\^etre ab\'elienne. Le {\sl radical} d'une alg\`ebre de Lie quelconque est alors, par d\'efinition le plus grand id\'eal r\'esoluble de cette alg\`ebre de Lie. Le radical d'une alg\`ebre de Lie semi-simple est, bien \'evidemment, nul.
L'existence de la d\'ecomposition de Levi montre qu'il faudrait classifier, pour bien faire, d'une part les alg\`ebres de Lie semi-simples et
et d'autre part les alg\`ebres de Lie non semi-simples.
\item La classification des alg\`ebres de Lie non semi-simples est 
difficile$\ldots$ (et probablement impossible).
\item La classification des alg\`ebres de Lie semi-simples (sur le 
corps 
$\CC$) a \'et\'e effectu\'ee par E. Cartan. Pour classer les 
alg\`ebres de Lie semi-simples, il suffit de classer les alg\`ebres de 
Lie 
simples.
\item On classifie d'abord les alg\`ebres de Lie simples complexes (\ie 
en 
tant qu'espace vectoriel, le corps des complexes est $\CC$). On 
d\'emontre 
qu'il existe quatre s\'eries infinies $A_n$, $B_n$, $C_n$ , $D_n$ 
d'alg\`ebres de Lie simples. Le symbole $n$ apparaissant en indice 
fournit le rang de l'alg\`ebre 
correspondante.
 Pour $n$ ``suffisamment petit'', il peut se 
faire que des individus appartenant \`a des s\'eries diff\'erentes 
co\"incident. 
Il peut se faire aussi, pour $n$ petit, que les alg\`ebres en question 
soient, non pas simples, mais semi-simples (en fait cela n'arrive 
qu'une 
seule fois). On y reviendra plus loin.
On d\'emontre aussi qu'il existe, en dehors des alg\`ebres de Lie 
classiques, 
qui sont, par d\'efinition, les membres 
des quatre s\'eries pr\'e-cit\'ees, un nombre fini (cinq) d'alg\`ebres de Lie 
simples. On les 
appelle ``exceptionnelles''; ce sont~: $G_2$,$F_4$,$E_6$,$E_7$ et 
$E_8$.
\item Pour une alg\`ebre de Lie complexe donn\'ee, on 
classifie les diff\'erentes alg\`ebres de Lie r\'eelles admettant la m\^eme 
extension complexe~; techniquement, ceci se fait en classifiant  les 
involutions. C'est ainsi que $D_n$, par exemple, admet les formes 
r\'eelles 
distinctes, not\'ees  ${\frak so}(p,q), p\geq 0, q \geq 0, p+q = n$,
et  ${\frak so}(2n)^*$.
\item
A chaque forme r\'eelle (c'est \`a dire, \`a chaque alg\`ebre de Lie r\'eelle 
correspondant \`a une 
alg\`ebre de Lie complexe donn\'ee) on associe un groupe de Lie connexe 
et 
simplement connexe, \`a l'aide de l'application exponentielle. On 
d\'emontre 
que, pour une alg\`ebre de Lie complexe donn\'ee (exemple $D_3$), une 
seule 
forme r\'eelle correspond \`a un groupe de Lie compact (dans notre 
exemple, 
il s'agit de $\widehat{SO(6)}=\exp({\frak so}(6)$). Les autres 
groupes de 
Lie ainsi obtenus, \`a savoir $\widehat{SO(5,1)}$, $\widehat{SO(4,2)}$, 
$\widehat{SO(3,3)}$ et $\widehat{SO(6)^*}$ sont non compacts. 
L'alg\`ebre 
de Lie r\'eelle unique dont l'exponentielle constitue un groupe de Lie 
compact 
s'appelle {\sl forme r\'eelle compacte \/} de l'alg\`ebre de Lie complexe 
donn\'ee 
(bien que, {\it stricto sensu \/} cette alg\`ebre poss\`ede \'evidemment 
une 
topologie non compacte puisqu'il s'agit d'un espace vectoriel~!).
\item
A chaque groupe de Lie connexe et simplement connexe $\widehat G$, on 
associe 
alors une famille de groupes de Lie $G_i$ connexes, mais non 
simplement 
connexes en quotientant $\widehat G$ par un sous-groupe distingu\'e 
discret 
$K_i$ (voir la sous-section pr\'ec\'edente): $G_i=\widehat G/K_i$. On a 
$\pi_1(G_i)=K_i$ et $\widehat G$ est le rev\^etement universel des 
$G_i$. Par 
exemple, on obtient ainsi $SO(6)=\widehat{SO(6)}/\ZZ_2$ (rappelons la 
notation consacr\'ee~: $Spin(n)  =  \widehat{SO(n)}$).
\item
Les groupes de Lie compacts correspondant \`a la forme r\'eelle compacte 
des
 alg\`ebres complexes $A_n$, $B_n$, $C_n$ et $D_n$ sont les groupes 
d\'ej\`a 
 rencontr\'es not\'es $SU(n+1)$, $Spin(2n+1)$, $Sp(n)$ et $Spin(2n)$ 
dont nous avons d\'ej\`a donn\'e les dimensions. Ceux 
 correspondants aux alg\`ebres de Lie exceptionnelles se notent 
g\'en\'eralement de la m\^eme 
 fa\c con que les alg\`ebres de Lie correspondantes. Les dimensions des 
cinq groupes exceptionnels $G_2, F_4, E_6, E_7, E_8$ sont 
respectivement $14, 52, 78, 133, 248$.
 \end{itemize}
 
 \subsubsection{Remarques diverses}
\begin{itemize}
 \item
 Tout le monde, ou presque, d\'esigne par $SO(n)$ le groupe 
$SO(n,\RR)$ et par $SU(n)$ le groupe $SU(n,\CC)$.
 Les groupes de Lie correspondant \`a la s\'erie $C_n$ se notent 
 malheureusement de fa\c cons tr\`es diverses suivant les auteurs. Nous 
avons 
 d\'ecid\'e de noter $Sp(n)$ le groupe compact correspondant et de 
r\'eserver 
 la notation $Sp(2n,\RR)$ pour  d\'esigner ``le'' groupe symplectique 
 (la forme r\'eelle non compacte de $C_n$ qui d\'efinit la g\'eom\'etrie de 
l'espace
  des phases en m\'ecanique). La notation $U(n,\HH)$ se r\'ef\'erant aux 
groupes unitaires 
  quaternioniques ($\HH$ est le corps non commutatif des quaternions) 
est 
  aussi assez en vogue pour d\'esigner le groupe compact $Sp(n)$. Le 
  m\^eme groupe est d\'esign\'e quelquefois par le symbole $USp(2n)$. La
  raison d'\^etre de cette derni\`ere notation est que ce groupe 
co\"incide avec 
  l'intersection des unitaires (les $U(n))$ et des symplectiques 
complexes
  (les $Sp(n,\CC)$) Pour cette raison on les appelle aussi ``les 
unitaires 
  symplectiques''. H\'elas, on peut \'egalement trouver des auteurs 
d\'esignant 
  ce m\^eme groupe $USp(2n)$ par $USp(n)\ldots$ Bref, c'est la 
  pagaille.
  \item
  Toutes les alg\`ebres de Lie, membres de s\'eries $A_n$, $B_n$, $C_n$ 
et 
  $D_n$ --- et tous les groupes correspondants --- sont simples, \`a 
  l'exception de $D_2 = A_1 \oplus A_1$. Au niveau des groupes, on 
peut 
  donc \'ecrire $Spin(4)=SU(2)\times SU(2)=Spin(3)\times Spin(3)$~; en 
  d'autres termes, $SO(4)$ et $SO(3)\times SO(3)$ ont m\^eme alg\`ebre de 
Lie.
  \item
  Comme annonc\'e plus haut, il existe des isomorphismes exceptionnels 
  entre membres de s\'eries diff\'erentes, lorsque $n$ est assez petit. 
Les 
  voici
  \begin{eqnarray*}
&&  A_1=D_1=C_1 {\hskip 3cm}  D_2 = A_1\oplus A_1 {\hskip 
0.7cm}\mbox{{\it (d\'ej\`a vu)}} \\
&&  A_3=D_3 {\hskip 4cm} C_2 = B_2
  \end{eqnarray*}

  Au niveau des groupes compacts correspondants, on obtient donc les 
isomorphismes
  
  \begin{eqnarray*}
&& SU(2) = Spin(3)= Sp(1) {\hskip 1cm}  {\hskip 1cm} Spin(4) = 
SU(2)\times SU(2)  \\
&& SU(4) = Spin(6) {\hskip 3.5cm}   Sp(2)=Spin(5)
  \end{eqnarray*}
  
  Citons enfin quelques isomorphismes concernant les groupes non 
compacts.
$Spin^{\uparrow}(p,q)$ d\'esigne ici la composante connexe de 
l'identit\'e dans
$Spin(p,q)$:
 $$Spin^{\uparrow}(2,1)=SL(2,\RR), Spin^{\uparrow}(3,1)=SL(2,\CC), 
Spin^{\uparrow}(4,1)=U(1,1,\HH)$$
$$
Spin^{\uparrow}(5,1)=SL(2,\HH), Spin^{\uparrow}(3,2)=Sp(4,\RR), 
Spin^{\uparrow}(4,2) = SU(2,2)$$
  
  \item
  La classification des alg\`ebres et groupes de Lie ainsi que l'\'etude 
des  probl\`emes qui s'y rattachent n\'ecessiterait de d\'ecupler la taille de 
ce chapitre. Nous ne pr\'etendons donc pas, dans ce paragraphe, expliquer 
 quoi que ce soit et nous nous contentons de faire un tour rapide du 
zoo$\ldots$
  Il est difficile de parler de la classification des groupes de Lie 
sans mentionner les diagrammes de Dynkin (dans un contexte diff\'erent on 
parle aussi de graphes de Coxeter). Mentionnons seulement que
  la classification de Cartan, pour les alg\`ebres de Lie simples, se 
  r\'eduit, en fin de compte, \`a un probl\`eme de combinatoire admettant 
une interpr\'etation graphique. A chaque alg\`ebre de Lie simple complexe, 
on  associe donc un petit diagramme (voir n'importe trait\'e de
  classification des groupes de Lie).  
Nous recommandons au lecteur de compl\'eter sa culture 
en allant consulter la litt\'erature appropri\'ee. Notons que ces 
 diagrammes apparaissent absolument partout, c'est \`a dire non 
seulement 
 dans un contexte li\'e \`a l'\'etude des alg\`ebres de Lie, mais encore dans 
bien 
 d'	autres domaines~: dans la th\'eorie des groupes engendr\'es par 
 r\'eflexions, en th\'eorie des singularit\'es, dans la th\'eorie des noeuds, 
 dans la classification des inclusions d'alg\`ebres d'op\'erateurs 
 (sous-facteurs), en arithm\'etique, dans la g\'eom\'etrie des solides 
 platoniques (en relation avec l'\'etude des sous-groupes finis de 
 $SO(3)$), dans la th\'eorie des carquois, dans celle des syst\`emes 
 int\'egrables (en m\'ecanique), dans les th\'eories conformes 
bi-dimensionelles, 
 en th\'eorie des cordes$\ldots$ Bref, partout.
 Nous esp\'erons donc que le lecteur, curieux, sera tent\'e de vouloir 
 comprendre pourquoi ces quelques petits dessins contiennent une 
telle 
 quantit\'e d'information.
 \item
 Un dernier mot sur ces diagrammes~: certains contiennent des lignes 
 doubles ou triples (exemple de $G_2$), et d'autres non. Ceux 
n'utilisant 
 que des lignes simples (ce sont ceux des s\'eries $A_n$, $D_n$ et 
$E_n$) 
 sont souvent consid\'er\'es, d'une certaine fa\c con, comme plus 
fondamentaux 
 que les autres~; les alg\`ebres de Lie correspondantes (les alg\`ebres 
 ``ADE'') sont \'egalement appel\'ees {\sl alg\`ebres simplement lac\'ees\/}.
 \end{itemize}

%%%%{Exemple: Le groupe de Lorentz. Le groupe de Poincar\'e.}
%%%%%%TO DO ONE DAY..?$***$ 
 
\subsection{Message}

 Un tout dernier mot~: passer en revue ``l'essentiel'' de la 
 th\'eorie des groupes de Lie en une seule section -- m\^eme en se 
limitant aux 
 g\'en\'eralit\'es et aux probl\`emes de classification -- est certainement 
une t\^ache 
 impossible. Un ouvrage entier serait d'ailleurs insuffisant. 
 Nous n'avons fait qu'aborder le sujet. Vouloir dresser la liste de 
ce qui n'a 
 pas \'et\'e effleur\'e serait \`a la fois inutile et$\ldots$ incomplet~!  
Voici 
 donc le message le plus important destin\'e \`a notre lecteur n\'eophyte~: 
La 
 section qui s'ach\`eve ici ne doit pas \^etre consid\'er\'ee comme un 
r\'esum\'e, 
 mais comme une invitation au voyage$\ldots$
 
\section{Actions de groupes et repr\'esentations}

\subsection{G\'en\'eralit\'es}

L'\'etude des groupes pour eux-m\^emes ne devrait pas nous faire oublier 
un 
fait essentiel~: un groupe sert surtout \`a agir sur ``quelque chose''. 
Historiquement, d'ailleurs, on d\'efinissait le plus souvent les 
groupes 
comme ``groupes de transformations'', pour s'apercevoir, apr\`es coup, 
du 
fait que deux groupes de transformations pouvant sembler tr\`es 
diff\'erents 
de prime abord, ne constituaient, en fait, qu'un seul et m\^eme groupe 
``abstrait'', 
agissant de deux fa\c cons diff\'erentes sur deux espaces diff\'erents.
Pour pr\'eciser cette notion d'action ainsi que pour d\'ecrire la fa\c con 
dont un groupe $G$ 
agit sur un ensemble $M$, il est utile d'introduire un vocabulaire 
appropri\'e.

\subsection{Groupe $G$ op\'erant \`a gauche sur un ensemble $E$}

A tout \'el\'ement $g$ de $G$ et \`a tout \'el\'ement $x$ (on dira ``point'') 
de 
$E$, on associe un point $y$ de $E$ qu'on appelera {\sl image de $x$ 
par 
la transformation $g$}. On \'ecrira 
$$ y = g \, x $$
On veut que $(g_1 g_2) x = g_1 (g_2 x)$ afin de pouvoir oublier les 
parenth\`eses. Plus pr\'ecis\'ement, une action (\`a gauche) de $G$ sur $E$ 
est 
la donn\'ee d'un homomorphisme $L$ du groupe $G$ dans le groupe des 
substitutions de $E$ (l'ensemble des bijections de $E$ dans $E$). 
L'image 
de $g \in G$ est not\'e $L_g$. L'application $L_g$ est donc une 
bijection 
de $E$ dans lui-m\^eme. Puisque $L$ est un homomorphisme, on a $L_{g_1 
g_2} = L_{g_1}L_{g_2}$. Par abus de langage, il est d'usage de noter 
$y = 
g \, x$ au lieu de $y = L_g(x)$. Le lecteur aura compris que le 
symbole 
$L$ vient de $\underline{L}eft$.

Pour d\'efinir une action quelconque, nous avons simplement suppos\'e que 
$L_g$ \'etait une bijection, mais on peut contraindre davantage la 
situation en imposant \`a $L_g$ d'\^etre un hom\'eomorphisme ($E$ \'etant 
alors 
suppos\'e muni d'une topologie), un diff\'eomorphisme ($E$ \'etant une 
vari\'et\'e 
diff\'erentiable), \etc. On parle alors d'action continue, 
diff\'erentiable, \etc.

\subsection{Action \`a droite (anti-action)}

On dit que $G$ agit \`a droite sur $E$ si on se donne un 
anti-homomorphisme 
$R$ de $G$ dans l'ensemble des substitutions de $E$. En d'autres 
termes, 
on remplace la condition $L_{g_1 g_2} = L_{g_1}L_{g_2}$ par la 
condition 
$R_{g_1 g_2} = R_{g_2}R_{g_1}$. Une action \`a droite n'est donc pas 
une 
action, au sens strict du terme, mais une anti-action. De fa\c con \`a 
pouvoir 
se d\'ebarrasser du symbole $R$, mis pour $\underline{R}ight$, on 
notera 
$y=x \, g$ au lieu de $y=R_g(x)$. L'\'ecriture de $g$, \`a droite de $x$ 
permet de composer correctement les transformations sans qu'il y ait 
besoin de parenth\`eses~: $R_{g_1 g_2}(x) = R_{g_2}R_{g_1}(x)$ implique 
en 
effet $x(g_1 g_2)=(x g_1)g_2$.

\subsection{Passage de la droite \`a la gauche (et inversement)}

Supposons donn\'ee une action \`a droite $R$ de $G$ sur $E$~; on peut 
canoniquement lui associer une action \`a gauche $L$ en d\'efinissant 
$L_g x  =  
R_{g^{-1}} x$ ; c'est \`a dire encore, avec des notations plus 
d\'epouill\'ees, $g x  =  x 
g^{-1}$. On peut ainsi toujours passer de la droite \`a la gauche et 
inversement. Cela dit, il est, quelquefois, dangereux d'effectuer ce 
passage sans notations protectrices$\ldots$ En effet, prenons par 
exemple 
$E=G$ lui-m\^eme~; on n'a alors certainement {\it pas} $g . k = k . 
g^{-1}$ dans le 
groupe $G$~! Une telle expression devrait donc s'\'ecrire $g \times k 
 =  k . 
g^{-1}$ et s'interpr\'eterait, non comme une \'egalit\'e dans $G$ mais 
comme une 
expression d\'efinissant, \`a partir de la multiplication ``$ . $'' une 
nouvelle
multiplication ``$\times$'' (qu'on appelle dailleurs la ``multiplication oppos\'ee'').

\subsection{Orbites, espace quotient}

\begin{itemize}
\item
Soit $G$ un groupe op\'erant \`a gauche sur $E$. L'orbite $Gx$ de $x\in 
E$ 
est l'ensemble $$Gx=\{y \in E {\hskip 0.4cm}\tq {\hskip 0.4cm}\exists 
g \in G ,\, y = g \, x\}$$
On peut ainsi passer d'un point \`a un autre de la m\^eme orbite en 
utilisant 
un \'el\'ement du groupe $G$.
\item
Le fait, pour deux points $x$ et $y$, d'appartenir \`a la m\^eme orbite 
est 
clairement une relation d'\'equivalence (utilisant l'existence d'un 
\'el\'ement 
neutre, l'existence, pour tout $g$, d'un inverse $g^{-1}$, et le fait 
que 
la loi de groupe soit interne). L'ensemble quotient n'est autre que 
l'ensemble des diff\'erentes orbites $\overline{x}=Gx$ et se note 
$G\backslash E$ 
pour une action \`a gauche. L'ensemble quotient pour une action \`a 
droite 
(les classes sont alors les orbites  $\overline{x}=xG$) se note $E/G$.
\end{itemize}

\subsection{Efficacit\'e}

\begin{itemize}
\item
L'action de $G$ sur $E$ est dite {\sl fid\`ele},  {\sl efficace\/},  ou {\sl effective} (``{\it effective or faithful
action\/}'')\index{action fid\`ele} \index{action effective} lorsque 
tous les \'el\'ements de $G$ (hormis l'\'el\'ement neutre) font effectivement 
quelque chose~! On consid\`ere le fait de ne rien faire comme une 
action particuli\`ere peu efficace$\ldots$
L'adjectif ``efficace'' est assez parlant, mais il semble que le mot ``fid\`ele'' soit maintenant g\'en\'eralement utilis\'e pour d\'esigner cette notion.
Pour une action donn\'ee du groupe $G$ sur l'ensemble 
$E$, 
on d\'efinit l'ensemble $I$ des \'el\'ements de $G$ qui n'agissent sur 
aucun 
des \'el\'ements de $E$, c'est \`a dire
$$ I = \{j \in G {\hskip 0.5cm}\tq  {\hskip 0.5cm} \forall x \in E, j 
\, x = x\} $$
Cet ensemble $I$ est manifestement un sous-groupe de $G$ (on pourrait 
l'appeler le sous-groupe des feignants~! ) et il caract\'erise 
l'efficacit\'e 
de l'action du groupe $G$. Plus il y a de feignants, moins l'action 
est efficace. 
Lorsque $I$ se r\'eduit \`a l'\'el\'ement neutre de $G$, on dit que 
l'action est fid\`ele. Lorsque $I$ co\"incide avec $G$, l'action est 
triviale.
\item
Manifestement, seules les actions  fid\`eles sont int\'eressantes. Pour 
cette raison, il est utile, lorsqu'on se donne une action 
non- fid\`ele de 
$G$ sur $E$, de fabriquer un nouveau groupe $G \vert I$ pour lequel 
l'action est  fid\`ele. Noter que $G \vert I$ est bien un groupe car 
$I$ 
est distingu\'e dans $G$ (en effet $g I (g^{-1} x) = g g^{-1} x = x$ 
donc 
$g I g^{-1} = I$).
\end{itemize}

\subsection{Libert\'e et stabilisateur}

\begin{itemize}
\item
On suppose donn\'ee une action fid\`ele du groupe $G$ sur l'ensemble 
$E$. 
Puisque l'action est fid\`ele, tous les \'el\'ements de $G$ -- sauf 
l'\'el\'ement 
neutre -- ``font quelque chose''. Cependant, il peut se faire que, 
pour 
un point particulier $x \in E$, il existe des \'el\'ements de $G$ 
laissant ce 
point invariant. On d\'efinit ainsi le {\sl stabilisateur\/} 
\index{stabilisateur}
$H_x$ de $x \in E$~: $$H_x = \{h \in G {\hskip 0.5cm}\tq{\hskip 
0.5cm} h \, x = x \}$$
Il est facile de voir que $H_x$ est un sous-groupe de $G$. Noter la 
diff\'erence entre la d\'efinition de $H_x$ et celle de $I$ donn\'ee dans 
le 
paragraphe pr\'ec\'edent~: la d\'efinition de $H_x$ d\'epend {\it a priori 
\/} de 
$x$~! Le stabilisateur de $x$ est quelquefois d\'enomm\'e 
(historiquement, 
dans le contexte de l'action du groupe de Lorentz sur l'espace de 
Minkowski de la th\'eorie 
de la Relativit\'e Restreinte) {\sl petit groupe de $x$ \/}. Le 
stabilisateur 
$H_x$ de $x$ est aussi appel\'e {\sl sous-groupe d'isotropie de $x \in 
E$}.\index{sous-groupe d'isotropie}
\item
Deux points appartenant \`a la m\^eme orbite ont des stabilisateurs 
conjugu\'es. En effet, soit $y = g\,x$, alors l'hypoth\`ese $H_x x=x$ 
implique 
$H_x g^{-1}y=g^{-1}y$. Ceci montre que $H_y = g H_x g^{-1}$. 
Notons que $H_x$ et $H_y$, bien qu'isomorphes, sont en g\'en\'eral 
distincts 
comme sous-groupes de $G$ ($H_x$ n'est g\'en\'eralement pas distingu\'e 
dans $G$).
\item
Il existe une bijection entre les points de l'orbite 
$\overline{x}=Gx$ de 
$x$ et les points de l'ensemble quotient $G/H_x$~: \`a $y=gx$ on 
associe 
l'\'el\'ement $gH_x$ de $G/H_x$ et r\'eciproquement. On assimile souvent 
l'orbite $Gx$ de $x$ \`a l'ensemble 
quotient $G/H$ o\`u $H$ d\'esigne le stabilisateur d'un point quelconque 
de 
l'orbite, mais il faut se rappeler que, pr\'ecis\'ement, cette 
identification n'est possible que si on a choisi un point. En 
d'autres 
termes, la bijection entre les 
deux ensembles n'est pas canonique puisqu'elle d\'epend du point $x$ 
choisi. Cette remarque (le fait qu'une telle bijection ne soit pas 
canonique) est 
\`a la base de l'id\'ee d'invariance de jauge, qui, elle-m\^eme, est \`a la 
base 
de pratiquement toutes nos th\'eories physiques. Nous y reviendrons 
avec force d\'etails 
dans le chapitre consacr\'e aux espaces fibr\'es, puis dans celui 
consacr\'e aux 
connexions. 
\item
Il peut se faire que, pour tout point $x$ de $E$, le stabilisateur 
$H_x$ 
se r\'eduise \`a l'identit\'e. Dans ce cas l'action 
\index{action libre} est dite {\sl libre\/}. 
Le r\'esultat pr\'ec\'edent montre alors que, dans un tel cas, 
chaque orbite est identifiable \`a $G$ lui-m\^eme. Cette situation est \`a 
la 
base de la th\'eorie des espaces fibr\'es principaux (chapitre suivant).
\item
Notons que libert\'e implique efficacit\'e \ldots
\end{itemize}

\subsection{Transitivit\'e}

L'action de $G$ sur $E$ est dite {\sl transitive\/} s'il n'existe 
\index{action transitive}
qu'une seule 
orbite, en d'autres termes, s'il est possible de passer de n'importe 
quel 
point de $E$ \`a n'importe quel autre point \`a l'aide d'un \'el\'ement de 
$G$.

\subsection{Action d'un sous-groupe $H$ sur un groupe $G$, 
normalisateur, 
centralisateur}
\begin{itemize}
\item
Le cas particulier o\`u $E=G$ et o\`u on consid\`ere donc l'action de $G$ 
sur 
lui-m\^eme par multiplication -- \`a gauche ou \`a droite -- m\'erite 
\'evidemment 
une mention sp\'eciale. Il s'agit alors d'une action fid\`ele, libre et 
transitive~; nous y reviendrons un peu plus loin car elle permet de 
donner 
une d\'efinition intrins\`eque de la notion d'alg\`ebre de Lie.
\item
Choisissons maintenant un sous-groupe $H$ de $G$. On peut alors 
d\'efinir 
une action \`a gauche de $H$ sur $G$ (les orbites sont les 
$\overline{g}=Hg$, c'est \`a dire les classes de $H\backslash G$) et 
une action \`a 
droite de $H$ sur $G$ (les orbites sont les $\overline{g}=gH$, c'est 
\`a 
dire les classes de $G/H$). En g\'en\'eral, les ensembles quotients $G/H$ 
et 
$H\backslash G$ ne sont pas des groupes, sauf dans le cas o\`u les 
classes \`a gauche 
et \`a droite co\"incident ($gH=Hg$), c'est \`a dire lorsque $H$ est {\sl 
distingu\'e\/} \index{sous-groupe distingu\'e} dans $G$ (on dit aussi dans ce cas que $H$ est un 
sous-groupe {\sl 
invariant\/} ou un sous-groupe {\sl normal\/}). En effet, on peut 
alors 
d\'efinir de fa\c con non ambigu� la multiplication des classes~: 
$\overline{g} \overline{k} = g\,H\,k\,H=g\,k\,H=\overline{gk}$.
\item
Soit $H\subset G$. On d\'efinit le {\sl normalisateur\/} $N$ de $H$ 
dans \index{normalisateur}
$G$ comme le plus grand sous-groupe de $G$ dans lequel $H$ est 
normal. $$
N = \{n \in G {\hskip 0.5cm}\tq{\hskip 0.5cm} nH=Hn \}$$
Par construction $H$ est distingu\'e dans $N$, donc $N\vert H$ est un 
groupe, et si $H$ est un sous-groupe distingu\'e de $G$, alors $N=G$. 
Notons 
que, dans un groupe ab\'elien, tout sous-groupe est distingu\'e.
\item
Il faut distinguer (pr\'ecis\'ement~!) les notions de normalisateur et de 
centralisateur. Le {\sl centralisateur\/}\index{centralisateur} ${\cal Z}_H$ de $H$ dans 
$G$ est 
l'ensemble des \'el\'ements de $G$ qui commutent (\'el\'ement par \'el\'ement) 
avec 
ceux de $H$~: $$
{\cal Z}_H=\{z \in G {\hskip 0.5cm}\tq{\hskip 0.5cm} \forall h \in H, 
h z = z h \}$$
Le centralisateur ${\cal Z}$  de $H$ dans $G$ (que nous notons 
\'egalement  ${\cal Z}_H$ pour
pr\'eciser) est bien \'evidemment un 
sous-groupe -- non n\'ecessairement ab\'elien -- de $G$. Il nous faut 
\'egalement 
rappeler la d\'efinition du {\sl centre\/}\index{centre} d'un groupe $G$ qui n'est 
autre 
que le centralisateur de $G$ dans lui-m\^eme.
 Bien entendu, le sous-groupe $H$ 
poss\`ede lui-m\^eme son propre centre $C_H$ et on a $C_H \subset 
\mbox{{\cal 
{\cal Z}}}_H$.
\end{itemize}

\subsection{Stratification}

Dans toute cette sous-section on consid\`ere un groupe $G$ agissant sur 
$E$ 
de fa\c con fid\`ele.
\begin{itemize}
\item
On sait que si deux points appartiennent \`a la m\^eme orbite, leurs 
stabilisateurs sont conjugu\'es, mais il peut se faire qu'ils 
co\"incident. 
Cela arrivera si $H_{y=gx}=H_x$ c'est \`a dire si $gH_xg^{-1}=H_x$, 
c'est \`a 
dire si $g$ appartient au normalisateur de $H_x$ dans $G$.
\item
Ce n'est pas parce que les stabilisateurs de $H_{x_1}$ et de 
$H_{x_2}$ 
sont conjugu\'es qu'ils appartiennent n\'ecessairement \`a la m\^eme orbite. 
Par 
contre, et par d\'efinition, on dit alors qu'ils appartiennent \`a la 
m\^eme 
{\sl strate\/}\index{strate}. Ainsi, une strate donn\'ee est caract\'eris\'ee par un 
certain 
sous-groupe $H$ de $G$ d\'efini \`a isomorphisme pr\`es. On dira que deux 
orbites sont du m\^eme type si les stabilisateurs des diff\'erents points 
sont 
isomorphes. Une strate est donc la r\'eunion de toutes les orbites d'un 
m\^eme type.
\item
On peut ainsi d\'ecomposer $E$ en une r\'eunion de strates $E_H$, 
chaque 
strate \'etant caract\'eris\'ee par un certain type de stabilisateur $H$. 
On 
peut \'egalement d\'ecomposer l'espace des orbites $G\backslash E$ en une 
r\'eunion 
d'ensembles $G\backslash E_H$. Lorsque $E$ est muni d'une topologie, 
on d\'emontre 
que l'une de ces strates (dite la {\sl strate 
g\'en\'erique\/}\index{strate g\'en\'erique}) est 
ouverte 
et dense dans $E$~; le groupe d'isotropie correspondant (le {\sl 
stabilisateur g\'en\'erique\/}) est le plus petit possible.

%%%\item 
%%%Choix d'une {\sl tranche\/} pour l'action d'un groupe $G$. $TO 
%%% DO?$***$ 

\end{itemize}

\subsection{Remarques}

Afin de se familiariser avec les concepts qui pr\'ec\`edent ainsi qu'avec 
la 
terminologie correspondante, nous sugg\'erons tr\`es fortement au lecteur 
de 
revoir toute la g\'eom\'etrie \'el\'ementaire (celle \'etudi\'ee dans les classes 
secondaires) en ces termes, c'est \`a dire en
utilisant l'action des groupes de translations, rotations, 
homoth\'eties, \etc.
Il pourra \^etre \'egalement extr\^emement utile de revoir la cin\'ematique 
classique (puis la cin\'ematique relativiste) sous cet angle, en 
\'etudiant 
l'action du groupe Euclidien, celle du groupe de Galil\'ee, du groupe 
de 
Lorentz \etc.

\section{Champs de vecteurs fondamentaux}\index{champs de vecteurs 
fondamentaux}

\subsection{Cas d'un groupe de Lie agissant sur une vari\'et\'e}

\begin{itemize}
\item
On se donne un groupe de Lie $G$ et une action \`a gauche (suppos\'ee 
diff\'erentiable) de $G$ sur une vari\'et\'e $M$. Il y a, au moins, trois 
fa\c cons de consid\'erer cette action:
\begin{enumerate}
\item Comme une application de $G\times M$ dans $M$~: $$(g,P)\in 
G\times M 
\stackrel{L}{\longmapsto} g P \in M$$
\item Comme la donn\'ee, pour tout point $P$ dans $M$, d'une 
application 
$$g\in G \stackrel{R_P}{\longmapsto}gP\in M$$
\item Comme la donn\'ee, pour tout \'el\'ement $g$ du groupe $G$, d'une 
application $$P\in M \stackrel{L_g}{\longmapsto}gP \in M$$
\end{enumerate}
Attention : Une action {\`a gauche} fournit une application not\'ee 
$L_g$ quand on g\`ele l'\'el\'ement $g$ du groupe 
mais fournit une application not\'ee $R_P$ quand on g\`ele le point 
$P$.
L'application $L_g$ n'est autre que celle qui nous a permis 
pr\'ec\'edemment de d\'efinir 
l'action d'un groupe sur un ensemble. Notons que 
$L_g=L(g,\cdot)$. C'est en fait surtout le point de vue $2$ qui nous 
int\'eresse 
ici et nous allons donc \'etudier l'application $R_P =  L(\cdot,P)$. 
L'application 
$R_P$ \'etant suppos\'ee diff\'erentiable, nous pouvons consid\'erer sa 
diff\'erentielle not\'ee suivant les auteurs, ${R_P}_*$, $TR_P$ ou 
simplement $dR_P$. Comme on le sait 
(voir la premi\`ere partie de cet ouvrage), $dR_P$ est une application 
lin\'eaire de l'espace tangent $T(G,g)$ dans l'espace tangent $T(M,gP)$ 
dont l'expression, relativement \`a un couple de rep\`eres mobiles dans 
$G$ 
et $M$ s'\'ecrit \`a l'aide de la matrice jacobienne. Si on choisit alors 
$g=e$ (l'\'el\'ement neutre de $G$), on obtient ainsi une application 
lin\'eaire $T(G,e) \longmapsto T(M,P)$ qu'on devrait noter 
${(dR_P)}_{g=e}$ 
mais que nous pr\'ef\'erons ne pas baptiser du tout. L'important est 
d'observer qu'on obtient ainsi, pour tout vecteur $X$ appartenant \`a 
$T(G,e)$ un vecteur not\'e $X^L(P)$ appartenant \`a $T(M,P)$. Puisque 
cette 
application existe pour tout $P$ de $M$, on obtient donc un champ de 
vecteurs $P\in M \longmapsto X^L(P) \in T(M,P)$. On dit que $X^L$ est 
le 
{\sl champ de vecteurs fondamental gauche\/} associ\'e \`a l'\'el\'ement $X$ 
de 
l'espace tangent \`a $G$ en l'identit\'e.

\item
Nous verrons un peu plus loin que l'alg\`ebre de Lie de $G$, que nous 
avons 
pr\'ec\'edemment
d\'efinie de fa\c con \'el\'ementaire \`a l'aide de la fonction exponentielle, 
peut s'identifier, {\it en tant qu'espace vectoriel\/} \`a 
l'espace tangent \`a $G$ en l'identit\'e~: $Lie(G)=T(G,e)$.
En anticipant l\'eg\`erement, nous voyons donc qu'\`a tout \'el\'ement $X$ de 
$Lie(G)$ on peut associer un champ de vecteurs $X^L$ sur $M$.
R\'esumons cette construction simple et fondamentale par le diagramme 
suivant~:

\begin{eqnarray*}
&\begin{pmatrix}  G & \rightarrow & M \\ g & \rightarrow & g \, 
P\end{pmatrix}\rightarrow 
\begin{pmatrix}  T(G,g) & \rightarrow & T(M,gP)\end{pmatrix}
\rightarrow
\begin{pmatrix}  Lie G = T(G,e) & \rightarrow & T(M,P) \\ X & \rightarrow & 
X^L(P)\end{pmatrix} & \\
& {\hskip 1.5cm}
\mbox{On diff\'erentie}
{\hskip 3.5cm}
\mbox{On particularise} 
{}&
\end{eqnarray*}

\item
Le champ fondamental gauche associ\'e \`a $X\in Lie G$ se note, soit 
$X^L(P)$ 
comme ci-dessus, soit, encore plus simplement $$X^L(P) = X\cdot P$$ 
Pour 
rendre cette notation naturelle, il suffit de d\'evelopper 
l'exponentielle dans l'\'ecriture 
$$ g(t) \cdot P = e^{tX} \cdot P \approx P + t 
X \cdot P + \ldots $$ et ne garder que les termes du premier ordre.
Ainsi $X^L(P)=X.P = {d\over dt}(g(t).P)\vert_{t=0}$.
\item
Tout ce que nous avons d\'ecrit depuis le d\'ebut de cette section 
consacr\'ee 
\`a l'\'etude des champs fondamentaux supposait donn\'ee une action {\em \`a 
gauche\/} de $G$ sur $M$. Nous obtenons des notions analogues en 
supposant 
que $G$ agit {\em \`a droite\/} sur $M$. En particulier, lorsque $M=G$, 
nous 
pouvons aussi bien consid\'erer l'action \`a gauche que l'action \`a droite 
du 
groupe sur lui-m\^eme, et donc, de la m\^eme fa\c con que nous avons 
construit 
des champs fondamentaux gauche, nous pouvons \'egalement construire, 
pour tout \'el\'ement 
$X$ de $Lie(G)=T(G,e)$, 
un champ de vecteurs fondamentaux droit (le {\sl champ 
fondamental droit\/} $$X^R(P)=P \cdot X$$ associ\'e \`a $X$). 

\item
Certains auteurs d\'esignent les champs fondamentaux sous le nom de 
{\sl champs de Killing\/}\index{champs de Killing}. Pour nous, les champs de Killing sont des 
champs fondamentaux 
particuliers, ceux associ\'es \`a l'action d'un groupe agissant par 
isom\'etries sur une vari\'et\'e riemannienne.

\end{itemize}

\subsection{Exemple~: le groupe euclidien agissant sur le plan 
$\RR^2$}

Le groupe euclidien $E(2)$ agit sur le plan affine $M=\RR^2$ par 
composition 
de translations et de rotations autour de l'origine (c'est un produit semi-direct 
du groupe des rotations $U(1)$ par le groupe des translations $\RR^2$).
Une carte (qui 
est d'ailleurs globale) de $\RR^2$ est d\'efinie par les coordonn\'ees 
$(x, y)$ relatives \`a un rep\`ere du plan. L'action du groupe 
euclidien s'\'ecrit $$(\theta,a,b) \cdot \begin{pmatrix}  x\\ y\end{pmatrix}  = \begin{pmatrix}  x' 
= x \cos \theta - y \sin \theta + a \\  y'= x \sin \theta + y \cos 
\theta + b\end{pmatrix} $$
Noter qu'un \'el\'ement $g$ du groupe euclidien peut s'\'ecrire \`a l'aide de 
la 
carte $g \rightarrow (\theta,a,b) \in \RR^3$~; $G$ est un groupe de 
Lie de 
dimension $3$. La diff\'erentielle de l'application
$$
g=(\theta,a,b) \stackrel{L_P}{\longmapsto} g P = {P'} = 
\begin{pmatrix}  x'\\  y'\end{pmatrix}$$
s'\'ecrit
$$
{[dL_P]}_{g=(\theta,a,b) }=\begin{pmatrix}  
\partial x'/ \partial \theta & \partial x'/ \partial a & 
\partial x'/ \partial b \\ 
\partial y'/ \partial \theta & \partial y'/ \partial a & 
\partial y'/ \partial b \end{pmatrix}
$$
En prenant $g=e=(0,0,0)$, il vient
${[dL_P]}_e=\begin{pmatrix}  y&1&0 \\  -x&0&1\end{pmatrix} $.
Gr\^ace \`a l'utilisation de quelques abus de notations \'evidents, nous 
voyons que
\begin{itemize}
\item
Le champ fondamental associ\'e \`a $X_\theta  =  \begin{pmatrix}  1\\ 0\\ 
0\end{pmatrix} $ est 

$X_\theta(P)=\begin{pmatrix}  y&1&0\\ -x&0&1\end{pmatrix} \begin{pmatrix}  1\\ 0\\ 
0\end{pmatrix} =\begin{pmatrix}  y\\ -x\end{pmatrix} $ et 
donc $X_\theta(P)= {\displaystyle {x{\partial \over \partial 
y}-y{\partial\over \partial x}}}$

\item
Le champ fondamental associ\'e \`a $X_a  =  \begin{pmatrix}  0\\ 1\\ 0\end{pmatrix} $ est 

$X_a(P)=\begin{pmatrix}  y&1&0\\ -x&0&1\end{pmatrix} \begin{pmatrix}  0\\ 1\\ 0\end{pmatrix} =\begin{pmatrix}  1\\ 
0\end{pmatrix} $ et 
donc $X_a(P)= {\displaystyle{{{\partial}\over{\partial x}}}}$
\item
Le champ fondamental associ\'e \`a $X_b  =  \begin{pmatrix}  0\\ 0\\ 1\end{pmatrix} $ est 

$X_b(P)=\begin{pmatrix}  y &1&0\\ -x&0&1\end{pmatrix} \begin{pmatrix}  0\\ 0\\ 1\end{pmatrix} =\begin{pmatrix}  0\\ 
1\end{pmatrix} $ et 
donc $X_a(P)= {\displaystyle{{{\partial}\over{\partial y}}}}$
\end{itemize}

\subsection{Un cas particulier fondamental~: le groupe $G$ agissant 
sur 
lui-m\^eme par translations \`a gauche et \`a droite}

\begin{itemize}
\item
Nous consid\'erons maintenant le cas o\`u $G$ op\`ere sur $M=G$ lui m\^eme 
($g,k\in 
G$ et $P \in G$). Comme nous le savons, il est possible de consid\'erer 
deux actions~: l'une \`a gauche $g \rightarrow g\cdot P$ et l'autre \`a 
droite
$k \rightarrow P\cdot k$. En 
cons\'equence, nous avons aussi des champs fondamentaux \`a gauche $X^L$ 
et 
des champs fondamentaux \`a droite $X^R$.  Soit $X \in Lie \, G$, alors
$$X^L(k) = X \, k {\hskip 0.2cm} {\mbox et}  {\hskip 0.2cm} X^R(k) = 
k \, X$$
Notons aussi que 
$$
X^R(k)=k \, X^L(k) \, k^{-1}
$$

\item
Les deux actions commutent~: $(gP)k=g(Pk)$.
Elles commutent donc aussi infinit\'esimalement,
$(X^L Y^R - Y^R X^L) (k) = X (k Y) - (X k) Y = 0$. D'o\`u
$$[X^L,Y^R]=0$$
\item
La propri\'et\'e $X^L(g)k=X^L(gk)$ caract\'erise l'invariance de $X^L$ 
(champ 
r\'esultant d'une action \`a gauche) lorsqu'on le multiplie \`a droite par 
$k$. 
On peut donc dire que le champ fondamental gauche $X^L$ est un {\sl 
champ 
invariant \`a droite} \index{champs invariants \`a droite}\index{champs 
invariants \`a gauche}.
Attention~: {\sl Un champ fondamental gauche est invariant \`a droite 
et un 
champ fondamental droit est invariant \`a gauche.\/} Attention, le 
champ 
invariant \`a {\it gauche\/} associ\'e \`a $X$ se note $X^R$.

\item
Soit $X \in Lie(G)$. Lorsque $t$ varie, l'\'el\'ement $g(t) = e^{tX}$ 
d\'ecrit une 
courbe dans le groupe $G$ et le vecteur tangent \`a l'origine de cette 
courbe est donn\'e par ${dg(t) \over dt}\vert_{t=0} = X$.
Inversement, \`a tout \'el\'ement $X$ de $T(G,e)$ on peut associer une 
courbe \`a 
un param\`etre $g(t)=e^{tX}$ (en fait il s'agit d'un groupe \`a un 
param\`etre 
puisque $g(t_1+t_2)=g(t_1)g(t_2)$).
On peut ainsi identifier $Lie(G)$, en tant qu'espace vectoriel,
 et d\'efini comme pr\'ec\'edemment \`a l'aide de 
la fonction exponentielle, avec l'espace tangent en l'identit\'e du 
groupe 
$G$ : 
$$ Lie(G) \equiv T(G,e) $$

\item
Un champ fondamental droit $X^R$ est parfaitement caract\'eris\'e --- que 
$M=G$ ou non --- par $X \in T(G,e)$ c'est \`a dire par un \'el\'ement de  
l'espace tangent \`a l'identit\'e du groupe $G$. Dans le cas o\`u $M=G$,
 cependant, la 
correspondance entre champs fondamentaux droits ({\sl champs 
invariants \`a 
gauche\/}) et \'el\'ements de $T(G,e)$ est bijective ($k X = k Y, \,  k 
\in G$ implique $X = Y$). 
Notons que si $dim \, G = n$, alors $dim \, T(G,e)=n$ et la dimension 
de 
l'espace des champs de vecteurs invariants \`a gauche est encore $n$, 
alors 
que la dimension de l'espace de {\em tous\/} les champs de vecteurs 
est 
infinie.

\item
Par ailleurs, on vient de voir que la correspondance entre $T(G,e)$ 
et 
l'ensemble des champs de vecteurs invariants \`a gauche (par exemple) 
\'etait 
bijective. En effet $X^R(g)$  est parfaitement caract\'eris\'e par $X 
 =  X^R(e)$
puisque $X^R(g) = g.X$.
Notons $\Gamma^G(TG)$ l'ensemble de ces champs de vecteurs. On 
peut donc identifier $Lie(G)$ avec $\Gamma^G(TG)$ :
$$Lie(G) \equiv \Gamma^G(TG)$$
 Une autre fa\c con de 
d\'efinir l'alg\`ebre de Lie d'un groupe de Lie $G$ est donc de la d\'efinir
comme espace des champs de vecteurs invariants \`a 
gauche sur un groupe de Lie. Le commutateur dans l'alg\`ebre de Lie (le 
crochet de Lie)
 est alors d\'efini simplement comme commutateur des champs de 
vecteurs~; il 
faut \'evidemment montrer que le commutateur de deux champs de vecteurs 
invariants \`a gauche est encore invariant \`a gauche :
$$[X^R,Y^R] = [X,Y]^R$$
Cette propri\'et\'e  r\'esulte de ce qui pr\'ec\`ede.

\item
On pourrait bien sur penser \`a utiliser les champs invariants \`a 
droite
pour d\'efinir l'alg\`ebre de
Lie, cependant (noter le signe), le commutateur des champs invariants 
\`a droite conduit
 \`a la relation (exercice!)
$$ [X^L,Y^L] = - [X,Y]^L $$

\end{itemize}

A titre d'exercice (ou d'illustration), v\'erifions ces 
propri\'et\'es g\'en\'erales dans le cadre de $SL(2,\CC)$.

Les g\'en\'erateurs (repr\'esentation fondamentale) sont donn\'es par 

\begin{tabular}{ccc}
$\underline{X_{+}}= \begin{pmatrix}  0 & 1 \\ 0 & 0 \end{pmatrix} $, & 
$\underline{X}_{-}= \begin{pmatrix}  0 & 0 \\ 1 & 0  \end{pmatrix} $, &
$\underline{X_{3}}= \begin{pmatrix}  1 & 0 \\ 0 & -1  \end{pmatrix} $ \\
\end{tabular}

les actions \`a droite et \`a gauche sont donn\'ees par:

\smallskip

\begin{tabular}{ccc}
$\underline{X}_{+} \begin{pmatrix}  a & b \\ c & d \end{pmatrix}  =  \begin{pmatrix}  c  & d \\ 0 & 0 \end{pmatrix} $ & , &
$ \begin{pmatrix}  a & b \\ c & d \end{pmatrix} 
\underline{X}_{+}  =   \begin{pmatrix}  0 & a \\ 0 & c  \end{pmatrix} $ \\
$\underline{X}_{-} \begin{pmatrix}  a & b \\ c & d  \end{pmatrix}   =   \begin{pmatrix}  0 & 0 \\ a & b  \end{pmatrix} $ & , &
$ \begin{pmatrix}  a & b \\ c & d  \end{pmatrix} 
\underline{X}_{-}   =   \begin{pmatrix}  b & 0 \\ d & 0  \end{pmatrix}$ \\
$\underline{X_{3}} \begin{pmatrix}  a & b \\ c & d \end{pmatrix}   =   \begin{pmatrix}  a & b \\ -c & -d  \end{pmatrix} $ & , &
$ \begin{pmatrix}  a & b \\ c & d  \end{pmatrix}
\underline{X_{3}}  =   \begin{pmatrix}  a & -b \\ c & -d \end{pmatrix}  $ \\
\end{tabular}

\smallskip 

\noindent

Notez que les g\'en\'erateurs $\underline{X}_{\pm}$ et 
$\underline{X}_{3}$ agissent par d\'erivations.
En effet, les actions classiques (droite et gauche) ci-dessus  peuvent 
aussi \^etre \'ecrites \`a l'aide des op\'erateurs diff\'erentiels 
suivants:

\smallskip

\begin{tabular}{ccc}
$\underline{X}_{+}^{L} = c {\partial \over \partial a} + d {\partial \over \partial b} $
&,&
$\underline{X}_{+}^{R} = a {\partial \over \partial b} + c {\partial \over \partial 
d}$ \\
$\underline{X}_{-}^{L} = a {\partial \over \partial c} + b {\partial \over \partial d}$
&,&
$\underline{X}_{-}^{R} = b {\partial \over \partial a} + d {\partial \over \partial 
c} $\\
$\underline{X}_{3}^{L} = a {\partial \over \partial a} + b {\partial \over \partial b}
- c {\partial \over \partial c} - d {\partial \over \partial d}$
&,&
$\underline{X}_{3}^{R} = a {\partial \over \partial a} - b {\partial \over \partial b}
+ c {\partial \over \partial c} - d {\partial \over \partial d}$
\end{tabular}

\smallskip 
\noindent

Il est alors facile de v\'erifier explicitement que, par exemple,

$$[\underline{X}_{3},\underline{X}_{+}] = + 2 \underline{X}_{+}, 
\hspace{0.25 cm}
[\underline{X}_{3}^{R},\underline{X}_{+}^{R}] = + 2 \underline{X}_{+}^{R}, \hspace{0.25 cm} 
[\underline{X}_{3}^{L},\underline{X}_{+}^{L}] = - 2 \underline{X}_{+}^{R} 
$$

\subsection{L'action adjointe de $G$}

Le groupe $G$ agit sur lui-m\^eme par multiplications \`a droite et \`a 
gauche, comme nous l'avons vu plus haut, mais \'egalement par 
l'application 
adjointe. Soit $g$ un \'el\'ement de $G$, on d\'efinit :
$$
Ad_g \, : {\hskip 0.5cm} k\in G \rightarrow Ad_g (k) = gkg^{-1} \in G
$$
Cette action n'est pas fid\`ele en g\'en\'eral car les \'el\'ements du centre 
$C$ 
n'agissent pas. Le groupe $G\vert C$ qu'on d\'esigne sous le nom de 
{\sl 
groupe adjoint} \index{groupe adjoint} ou {\sl groupe des automorphismes int\'erieurs} agit, 
bien 
sur, de fa\c con fid\`ele. L'application tangente \`a $Ad_g$, au point 
$k$, 
envoie $T(G,k)$ dans $T(G,gkg^{-1})$. Si on prend alors $k=e$ 
(l'\'el\'ement 
neutre), on voit que l'application tangente, not\'ee $ad_g  =  
(d(Ad_g))_{k=e}$ 
envoie $T(G,e)$ dans $T(G,gg^{-1}=e)$, c'est \`a dire $Lie(G)$ dans 
$Lie(G)$. Posant $k(t) = e^{tX}$, 
on voit que $ad_g(X)={d \over dt}(g e^{tX} g^{-1})_{\vert t = 0}$ et 
donc
$$ad_g(X) = g X g^{-1}$$


\subsection{Exemple : l'alg\`ebre de Lie du groupe euclidien}

Nous avons d\'ej\`a fait agir le groupe euclidien $G$ (\'el\'ements 
$g=(\theta,a,b)$) sur l'espace affine $\RR^2$. Nous 
allons maintenant faire agir $G$ sur lui-m\^eme, \`a droite.

Soit $P\in G$. On consid\`ere l'application
$$ 
\left(
\begin{array}{c}
G \stackrel{R_P}{\longmapsto}M=G \\
g \longmapsto Q=Pg
\end{array}
\right)
$$
ce qui, avec des coordonn\'ees, s'\'ecrit
$$
(\theta_Q,a_Q,b_Q) = (\theta_P,a_P,b_P)(\theta_g,a_g,b_g)
$$
soit, explicitement

$$
\begin{pmatrix}  \theta\\ a\\ b\\ 1\end{pmatrix} _Q=\begin{pmatrix}  1&0&0&\theta \\ 0&\cos\theta 
&\sin\theta&a\\ 
0&-\sin\theta&\cos\theta&b\\ 0&0&0&1\end{pmatrix}_P \begin{pmatrix}  \theta\\ a\\ b\\ 
1\end{pmatrix}_g
$$
La diff\'erentielle de $R_P$, \cad l'application tangente est \'egale \`a
$$
dR_P=\begin{pmatrix}  
\frac{\partial \theta_Q}{\partial \theta^g} &
\frac{\partial \theta_Q}{\partial a^g} &
\frac{\partial \theta_Q}{\partial b^g} \\ 
\frac{\partial      a_Q}{\partial \theta^g} &
\frac{\partial      a_Q}{\partial a^g} &
\frac{\partial      a_Q}{\partial b^g} \\ 
\frac{\partial      b_Q}{\partial \theta^g} &
\frac{\partial      b_Q}{\partial a^g} &
\frac{\partial      b_Q}{\partial b^g} \end{pmatrix}
$$
On choisit, comme base de $T(G,e)$ la base 
$X_\theta(e)=\frac{\partial}{\partial \theta}$, 
$X_a(e)=\frac{\partial}{\partial a}$, 
$X_b(e)=\frac{\partial}{\partial b}$.
\par
On calcule $dR_P \begin{pmatrix}  1\\ 0\\ 0\end{pmatrix} =\begin{pmatrix}  1\\ 0\\ 0\end{pmatrix} $, $dR_P 
\begin{pmatrix}  0\\ 1\\ 0\end{pmatrix} =\begin{pmatrix}  0\\  \cos \theta \\  -\sin \theta\end{pmatrix} $, 
$dR_P \begin{pmatrix}  0\\ 0\\ 1\end{pmatrix} =\begin{pmatrix}  0\\  \sin \theta\\  \cos 
\theta\end{pmatrix} $. 
\par
La base correspondante de $Lie G \equiv \Gamma^G(TG)$ est donc
$$X_\theta(P)={{\partial}\over{\partial \theta}}, X_a(P)=\cos \theta 
{{\partial}\over{\partial a}} - \sin \theta {{\partial}\over{\partial 
b}} 
{\mbox et}
X_b(P)=\sin \theta 
{{\partial}\over{\partial a}} + \cos \theta {{\partial}\over{\partial 
b}}$$

Nous laissons au lecteur le soin de v\'erifier les relations de 
commutation

$$[X_\theta,X_a]=-X_b, {\hskip 1cm} [X_\theta,X_b]=+X_a
 {\hskip 1cm} {\mbox et}  {\hskip 1cm} [X_a,X_b]=0$$


%%%{Autre exemple : $SO(3,1)$ agissant sur $\RR^4$} et sur lui-m\^eme
%%%$***TO DO ***$

\subsection{Exemple : champs invariants sur $SU(2)$}
Le groupe $SU(2)$ est diff\'eomorphe \`a  la sph\`ere $S^3$. Pour le 
voir, il
suffit d'\'ecrire un \'el\'ement $g$ de $SU(2)$ comme une matrice
$\begin{pmatrix}  \alpha & \beta \\ -\beta^* & \alpha^*\end{pmatrix} $, ob\'eissant \`a la
condition $g^\dag = g^{-1}$. Alors, $det \, g^\dag g = 1$, c'est \`a 
dire
$$Re^2(\alpha)+Im^2(\alpha)+Re^2(\beta)+Im^2(\beta)=1$$
On obtient ainsi l'\'equation cart\'esienne d'une $3$-sph\`ere.
On peut donc se repr\'esenter visuellement $SU(2)$ comme une sph\`ere 
dot\'ee d'une structure multiplicative (non commutative d'ailleurs). 
Attention, il ne faudrait pas se laisser abuser par cet exemple : 
seules les sph\`eres $S^0 = \ZZ_2$, $S^1=U(1)$ et $S^3=SU(2)$ sont 
des groupes (et $S^7$ est ``presque'' un groupe).
Ces particularit\'es des dimensions $0,1,3,7$ sont li\'ees \`a 
l'existence
des alg\`ebres de division suivantes :
les corps $\RR$ (les r\'eels), $\CC$ (les complexes), $\HH$ (les 
quaternions)
et les octaves de Cayley (octonions) ${\cal O}$.

Revenons \`a la sph\`ere $S^3$ qu'on peut donc identifier avec le 
groupe de Lie
$SU(2)$.  Posons $X_i = i/2 \sigma_i$, o\`u les $\sigma_i$ sont les 
matrices de Pauli (section 2.2.2).
On peut param\'etriser un point quelconque $g$ par trois angles
d'Euler $\psi,\theta,\phi$ en \'ecrivant
$$ g = R_3(\psi)R_1(\theta)R_3(\phi), \quad
0<\psi\leq 4\pi, 0<\theta\leq \pi, 0\leq \phi\leq 2\pi
$$
 $R_i(x)= exp(t X_i)$ est une rotation d'angle $x$ autour de l'axe 
$i$. On consid\`ere, dans $SU(2)$ les courbes obtenues par 
translation \`a droite,
$D_i(t) =  g\, R_i(t)$ et nous  notons $X_i^R(g)$ les champs 
fondamentaux
 \`a droite correspondants (les champs invariants \`a gauche).
En terme du rep\`ere naturel associ\'e aux coordonn\'ees d'Euler, on 
obtient le rep\`ere
mobile:
$$
\begin{pmatrix}  X_1^R \\ X_2^R \\ X_3^R\end{pmatrix} = N
 \begin{pmatrix}  
{\partial \over \partial \theta} \\
{\partial \over \partial \psi} \\
{\partial \over \partial \phi}\end{pmatrix}
$$
avec
$$
N = \begin{pmatrix}  
\cos \phi & {\sin \phi \over \sin \theta} & {-\sin \phi \cot \theta} 
\\ 
\sin \phi & {-\cos \phi \over \sin \theta} & {\cos \phi \cot \theta} 
\\ 
0 & 0 & 1 
\end{pmatrix}
$$
Les relations de commutation s'\'ecrivent $$[X_1^R,X_2^R] = - X_3^R 
\quad {\hbox \etc}$$
Le corep\`ere mobile correspondant $\{X^{iR}\}$ (le dual du rep\`ere
mobile $\{X_i^R\}$) est donn\'e par
$$(X^{1R},X^{2R},X^{3R}) = (d\theta, d\psi, d\phi) N^{-1}$$

On peut aussi consid\'erer les courbes $G_i(t)=R_i(t)\, g$ obtenues 
par translation \`a gauche.
L'expression des champs de vecteurs invariants \`a droite $X_i^L$ (et 
des formes correspondantes
$X^{iL}$) s'exprime \`a l'aide des formules pr\'ec\'edentes
en interchangeant simplement partout les coordonn\'ees $\phi$ et 
$\psi$.
Les relations de commutation s'\'ecrivent alors  $$[X_1^L,X_2^L] = + 
X_3^L \quad {\hbox \etc}$$
et on v\'erifie que $$[X_i^R,X_j^L] = 0$$

\subsection{Une remarque sur les constantes de structure}

Soit $G$ un groupe de Lie et choisissons une base $X_\alpha$ dans son 
alg\`ebre de Lie, ensemble que nous identifions, en tant qu'espace 
vectoriel, avec l'espace tangent $T(G,e)$. Les vecteurs $X_\alpha$ 
d\'eterminent, comme nous l'avons vu, des champs de vecteurs invariants 
\`a 
gauche $X_\alpha(\cdot)$. L'espace de ces champs de vecteurs
 \'etant, comme on le sait, de dimension finie et \'etant lui-m\^eme 
identifiable \`a l'alg\`ebre de Lie de $G$, on peut \'ecrire, en tout point 
$P$ 
de $G$,  $$[X_\alpha(\cdot), 
X_\beta(\cdot)](P)=f_{\alpha\beta}^{\gamma} (P) 
X_\gamma(P)$$ On voit qu'on a ainsi obtenu un rep\`ere mobile {\em 
global\/} (les 
$\{X_\alpha (P)\}$) pour lequel les fonctions de structure sont les 
$f_{\alpha\beta}^{\gamma} (P)$. En fait, ces 
$f_{\alpha\beta}^{\gamma} 
(P)$ sont des constantes~: elles ne d\'ependent pas de $P \in G$. Ceci
r\'esulte du fait que le commutateur de deux champs invariants \`a gauche 
est lui-m\^eme un champ invariant 
\`a gauche.

Rappelons que, pour une vari\'et\'e 
diff\'erentiable quelconque, les fonctions de structure 
d'un rep\`ere mobile d\'ependent g\'en\'eralement du point o\`u elles sont 
\'evalu\'ees~; 
par contre, on voit ici que, lorsque cette vari\'et\'e est un groupe de 
Lie et que le 
rep\`ere mobile choisi est un champ de vecteurs invariant \`a gauche, ces 
fonctions de structure $f_{\alpha\beta}^\gamma$ sont des {\sl 
constantes de structure\/}~: \index{constantes de structure} elles 
ne d\'ependent que de la base choisie dans $T(G,e)$ et non du point $P$ 
o\`u elles sont calcul\'ees.

En utilisant des champs invariants \`a droite, on pourrait mener une 
discussion analogue, \cad, en particulier, associer \`a toute base 
$\{X_\alpha\}$ de $T(G,e)$ un rep\`ere mobile 
global constitu\'e de champs invariants \`a droite 
$X^L(g)=Xg$ et obtenir des constantes de structure
 $g_{\alpha\beta}^{\gamma} = - f_{\alpha\beta}^{\gamma}$.

\subsection{La forme de Maurer-Cartan}

\begin{itemize}
	\item  Il existe en fait {\it deux\/} formes de Maurer-Cartan~: 
l'une 
	est ``gauche'' et l'autre est ``droite''. Tout le monde utilisant 
des champs
invariants \`a gauche pour d\'efinir l'alg\`ebre de Lie,
	on parle alors de ``la'' forme de Maurer-Cartan.
	
	\item  La forme de Maurer-Cartan $\theta$ est une forme au sens 
	g\'en\'eralis\'e du mot. En effet, elle est \`a valeurs, non pas dans le 
corps 
	des r\'eels (ou des complexes) mais dans une alg\`ebre de Lie. Son r\^ole 
est 
	de ramener les champs invariants (\`a gauche) \`a l`origine~: soit 
	$X_\alpha(g)$ un champ invariant \`a gauche, on d\'efinit $\theta_g$ par
	$$
	\mbox{\fbox{$
	\theta_g(X_\alpha(g))  =  X_\alpha(e) 
	$}}
	$$
	En notant $\{\theta^\alpha(g)\}$ la base duale, au point $g$ de $G$ 
de 
	la base $\{X_\alpha(g)\}$ et en notant simplement $X_\alpha  =  
	X_\alpha(e)$, on voit que
	$$
	\theta_g  =  \theta^\alpha(g) \otimes  X_\alpha  \in 
T^*(G,g)\otimes T(G,e)
	$$
	en effet, 
	
$$\theta_g(X_\beta(g))=\theta^\alpha(g)(X_\beta(g))X_\alpha=\delta^\alpha_\beta 	
X_\alpha=X_\beta$$
	L'application $\theta_g$ va de $T(G,g)$ dans $T(G,e)$. La forme de 
	Maurer-Cartan elle-m\^eme $\theta = \theta^\alpha(\cdot)\otimes 
X_\alpha$, qu'on 
	peut simplement noter $\theta^\alpha X_\alpha$, va de $TG$ dans $Lie 
G = 
	T(G,e)$. En r\'esum\'e, $\theta \in \Omega^1(G, Lie \, G)$.
	
	Si $u \in TG$, c'est \`a dire que $u$ est un vecteur en un certain 
point 
	$g$, on peut, {\it a priori\/} d\'ecomposer $u$ sur une base de champs 
	invariants \`a gauche au point $g$ : $u=u^\alpha X_\alpha(g)$. On sait 
que 
	$\theta(u)$ est alors l'\'el\'ement de l'alg\`ebre de Lie (identifi\'ee ici 
avec 
	$T(G,e)$) \'egal \`a $\theta(u) = u^\alpha X_\alpha(e)=u^\alpha 
X_\alpha$.
	Puisque $\theta = \theta^\alpha X_\alpha$, on d\'efinit $d\theta 
 =  d 
	\theta^\alpha X_\alpha$ (rappelons que $X_\alpha \equiv 
X_\alpha(e)$), 
	mais on sait que, pour un rep\`ere mobile quelconque (voir chapitre 
	pr\'ec\'edent), 
	on a $d\theta^\alpha + \frac{1}{2} 
f^\alpha_{\beta\gamma}\theta^\beta 
	\theta^\gamma = 0$ o\`u les $f^\alpha_{\beta\gamma}$ sont les 
fonctions de 
	structure du rep\`ere mobile~; ici les ``fonctions de structure'' sont 
les 
	{\sl constantes de structure. \/} Pour deux formes $\omega$ et 
$\sigma$ \`a 
	valeurs dans une alg\`ebre de Lie ($\omega = \omega^\alpha X_\alpha$ 
et 
	$\sigma = \sigma^\alpha X_\alpha$) on d\'efinit le crochet
	$$
	[\omega \land \sigma]  = 
 [\omega^\alpha X_\alpha \land 	\sigma^\beta X_\beta] =
 \omega^\alpha \land \omega^\beta [X_\alpha,X_\beta] =
  \omega^\alpha \land \sigma^\beta \, f^\gamma_{\alpha \beta} \, 
X_\gamma
	$$
	Ainsi donc l'\'equation de structure de Maurer-Cartan s'\'ecrit
\index{equation de Maurer-Cartan}
	$$
   	\mbox{\fbox{$
	d\theta + \frac{1}{2}[\theta \land \theta] = 0
	$}}
	$$
	
		\item {\em Attention aux facteurs $1/2$ et aux notations\/}~: la 
	pr\'esence du $[\, ,\,]$ autour du symbole $\land$ est indispensable 
dans la 
	d\'efinition de $[\omega \land \sigma]$ et on voit que le crochet, en 
ce 
	sens, d'une $p$-forme \`a valeurs dans $Lie G$ avec une $q$-forme du 
m\^eme 
	type est une $(p+q)$-forme \`a valeurs dans $Lie G$. 
	Prenons de nouveau 
	$\omega$ et $\sigma$ dans $\Omega^1(G,Lie \, G)$ et \'evaluons-les sur 
des 
	vecteurs $u$ et $v$~: $\omega(u)=\omega^\alpha(u) X_\alpha$, 
	$\sigma(v)=\sigma^\beta(v)X_\beta$. On peut aussi d\'efinir
	$$[\omega,\sigma](u,v)  =  [\omega(u),\sigma(v)]$$
	Alors $[\omega^\alpha(u)X_\alpha, 	
\sigma^\beta(v)X_\beta]=\omega^\alpha(u)\sigma^\beta(v)[X_\alpha,X_\beta]=	
\omega^\alpha(u)\sigma^\beta(v)	f^\gamma_{\alpha \beta} X_\gamma$, 
mais
par ailleurs,
	$\omega^\alpha\land	\omega^\beta f^\gamma_{\alpha \beta}  
X_\gamma(u,v)=
	\left( 
(\omega^\alpha\otimes\omega^\beta-\omega^\beta\otimes\omega^\alpha)f^\gamma_{\alpha 	
\beta}X_\gamma\right)(u,v)=2 \omega^\alpha(u) \omega^\beta(v) 
f^\gamma_{\alpha 
	\beta}X_\gamma$
	Ainsi $[\omega\land\omega](u,v)=2[\omega,\omega](u,v)$ et l'\'equation 
de 
	Maurer-Cartan peut s'\'ecrire \'egalement sous la forme
	$$
	d\theta+[\theta,\theta]=0
	$$
	On peut utiliser indiff\'eremment le crochet $[\, \land  \,]$ ou le 
crochet
   $[\, ,  \,]$ mais ils diff\`erent par des facteurs num\'eriques.
   Par ailleurs, de nombreux auteurs d\'esignent $[\, \land \,]$ par [\, , \,] !
	
\item La forme de Maurer-Cartan ci-dessus, d\'efinie \`a l'aide de champs 
	fondamentaux \`a droite (c'est \`a dire \`a l'aide de champs invariants \`a 
	gauche) est celle qui est le plus utilis\'ee. Il ne faut pas oublier 
	qu'``A travers le miroir'' existe une forme analogue, d\'efinie \`a 
	partir des champs invariants \`a droite. Notons $\omega$ la forme de 
	Maurer-Cartan ``\`a droite''.
	Par une m\'ethode analogue \`a celle qui pr\'ec\`ede, on montre que $\omega$ 
	satisfait \`a l'\'equation de structure
	$$
		d\omega - \frac{1}{2}[\omega \land \omega] = 0
	$$ 
	
%%% TODO?$***LIen avec$	$g^{-1}dg$

\end{itemize}



\section[Repr\'esentations]{Action d'un groupe sur un espace 
vectoriel~: la th\'eorie des 
repr\'esentations}

Une {\sl repr\'esentation\/}\index{repr\'esentation} $L$ d'un groupe $G$ dans un espace 
vectoriel $E$ (sur 
le corps $K$) est un 
cas particulier de la notion d'action. L'espace 
$E$ n'\'etant pas quelconque mais dot\'e d'une structure d'espace 
vectoriel, 
on impose \`a l'action $L_g$ d'\^etre lin\'eaire. En d'autres termes, \`a 
tout 
\'el\'ement $g$ de $G$, on associe un automorphisme $L_g$ de $E$ (une 
transformation lin\'eaire bijective de $E$ sur lui-m\^eme). Si $E$ est 
de 
dimension finie $p$, moyennant un choix de bases, on peut \'ecrire 
l'automorphisme $L_g$ \`a l'aide d'une matrice inversible $p\times p$ 
encore d\'esign\'ee 
par $L_g$. On peut donc d\'efinir une repr\'esentation $L$ comme un 
homomorphisme du groupe $G$ dans le groupe $GL(p,K)$.
On dit qu'une repr\'esentation est {\sl fid\`ele}\index{repr\'esentation 
fid\`ele} lorsque l'homomorphisme $L$
ci-dessus est injectif.

La th\'eorie des repr\'esentations est un chapitre essentiel de la 
th\'eorie 
des groupes et est \'egalement d'une importance capitale dans 
pratiquement toutes 
les branches de la physique. Les diff\'erents aspects de la th\'eorie des 
repr\'esentations ne seront pas \'etudi\'es dans cet ouvrage. 
 

\section{Espaces homog\`enes}

Soit $G$ un groupe et $H$ un sous-groupe. On d\'efinit la relation 
d'\'equivalence $g_1 \sim g_2$ si et seulement si $g_1 \in g_2 H$. 
L'ensemble des classes d'\'equivalence, c'est \`a dire l'ensemble quotient $G/\sim$ se note 
$G/H$. On dit que cet ensemble
est un espace homog\`ene pour le groupe $G$. Le vocable 
``homog\`ene'' vient du fait que les
propri\'et\'es alg\'ebriques de $G/H$ sont les m\^emes en tous ses 
points puisqu'on peut passer
de l'un \`a l'autre par action de $G$.

On d\'emontre, lorsque $G$ est topologique, que $H$ doit \^etre 
ferm\'e pour que le quotient ait une
topologie s\'epar\'ee (propri\'et\'e de Haussdorf). C'est toujours ce 
que nous supposerons.

Lorsque $G$ est un groupe de Lie et $H$ un sous groupe de Lie, $G/H$ 
est une vari\'et\'e diff\'erentiable. Les espaces homog\`enes 
fournissent donc une quantit\'e d'exemples int\'eressants de 
vari\'et\'es. Ce sont les vari\'et\'es les plus ``simples'' qui 
soient (les groupes de Lie eux-m\^emes \'etant des cas particuliers 
d'espaces homog\`enes).
Nous aurons de nombreuses fois l'occasion d'y revenir lors de notre 
\'etude des espaces
fibr\'es. Attention, une vari\'et\'e donn\'ee peut parfois s'\'ecrire 
de diverses fa\c cons comme espace homog\`ene de groupes de Lie. En 
d'autres termes, deux quotients $G_1/H_1$ et $G_2/H_2$ peuvent
tr\`es bien \^etre diff\'eomorphes, m\^eme si $G_1 \neq G_2$ (par 
exemple $SU(3)/SU(2)$ et $SO(6)/SO(5)$ sont tous deux diff\'eomorphes 
\`a la sph\`ere $S^5$). Ainsi, deux groupes
diff\'erents peuvent agir transitivement sur le m\^eme espace.

Les r\'esultats concernant la th\'eorie des espaces homog\`enes (en 
particulier
tout ce qui concerne les espaces sym\'etriques) sont
d'un usage constant dans de nombreuses branches des math\'ematiques 
et de la physique th\'eorique. L\`a encore, comme pour la th\'eorie 
des
repr\'esentations, que nous n'avons fait que mentionner$\ldots$
nous conseillons vivement au lecteur de se cultiver sur le sujet en 
consultant
les ouvrages appropri\'es.

\section{Alg\`ebres de Clifford et groupes Spin}

\subsection{D\'efinitions g\'en\'erales} 

Bien conna\^\i tre la structure  des alg\`ebres de Clifford est une 
chose essentielle,
 aussi bien pour les g\'eom\`etres que pour les physiciens des 
particules, ou plus g\'en\'eralement
pour les physiciens th\'eoriciens. Cette section est bien trop courte 
pour couvrir tous leurs
aspects. Nous nous contenterons de donner leur
d\'efinition, de discuter leur structure g\'en\'erale, et de montrer 
comment se  servir de ces alg\`ebres
 pour obtenir une description explicite des groupes $Spin$.

L'{\sl alg\`ebre de Clifford\/} \index{alg\`ebre de Clifford}
r\'eelle $C(p,q)$ est l'alg\`ebre 
associative unitaire engendr\'ee sur $\RR$
 par $n=p+q$ symboles $\gamma^\mu$ soumis aux relations 
$(\gamma^\mu)^2=1$ pour $\mu \in \{1,2,\ldots,p\}$,  
$(\gamma^\nu)^2=-1$ pour $\nu \in \{p+1,p+2,\ldots,p+q\}$, et 
$$\gamma^\mu \gamma^\nu + \gamma^\nu \gamma^\mu = 0$$ quand $\mu \neq 
\nu$.

Il est utile d'introduire une matrice diagonale 
$\eta  =   diag(1\ldots 1, -1, \ldots  -1)$ 
et d'\'ecrire les relations pr\'ec\'edentes \`a l'aide d'un anticommutateur
 ($\{,\}$) sous la forme $\{ \gamma^\mu,\gamma^\nu, \}=2 \eta^{\mu\nu}$

Soit $E$ un espace vectoriel de dimension $n$ sur $\RR$ muni d'un 
produit scalaire non d\'eg\'en\'er\'ee $g$ (la m\'etrique), de 
signature $(p,q)$. Soit $\{e_\mu\}$ une base orthonorm\'ee et  
$\{e^\mu\}$ la base duale. On a encore une m\'etrique de composantes 
$g_{\mu\nu}$ sur le dual.
 On peut associer, \`a tout vecteur $v=v_\mu e^\mu$ du dual, un 
\'el\'ement $Cliff(v) = v_\mu \gamma^\mu$ de l'alg\`ebre de Clifford 
$C(p,q)$. Abus de notations: Nous noterons $v = Cliff(v)$. Les 
physiciens
des particules utilisent en g\'en\'eral la notation ``slash'' de 
Feynmann.
De cette fa\c con nous obtenons la relation $uv + vu = 2 g(u,v)$ pour 
tout couple de vecteurs du dual.

Une base vectorielle de $C(p,q)$ peut \^etre choisie comme suit:
 $$\{1,\gamma^\mu,\gamma^\mu \gamma^\nu, \gamma^\mu \gamma^\nu 
\gamma^\rho, \ldots \}$$ avec
$\mu<\nu<\rho<\ldots$. La dimension (r\'eelle) de $C(p,q)$ est donc 
$$1+(^n_1)+(^n_2)+\ldots = 2^n$$

L'adjectif {\it r\'eel\/} est important : Par exemple, en dimension 
$(3,1)$, l'\'el\'ement $\gamma^5 = i\gamma^0 \gamma^1 \gamma^2 
\gamma^3$ n'est pas un \'el\'ement de $C(p,q)$ mais un \'el\'ement de 
l'alg\`ebre complexifi\'ee $C^{{\scriptstyle \CC}} = C(p,q)\otimes \CC$.

L'alg\`ebre $C(p,q)$ n'est g\'en\'eralement pas isomorphe \`a 
$C(q,p)$. Il faut se rappeler que
$(p,q)=(p_+,q_-)$.

Puisque l'alg\`ebre de Clifford $C=C(p,q)$ et l'alg\`ebre 
ext\'erieure $\Lambda (E)$
ont m\^eme dimension,
ils sont isomorphes en tant qu'espaces vectoriels. La correspondance 
entre les deux lois d'alg\`ebre
est la suivante : pour $u,v\in E^*$, on peut directement d\'efinir le 
produit de Clifford $uv = u\land v + g(u,v)$.

Soit $C_0$ la partie paire de $C$, c'est \`a dire la sous-alg\`ebre 
lin\'eairement 
engendr\'ee par les produits
d'un nombre pair de g\'en\'erateurs $\gamma$. 

Soit $$
\mbox{\fbox{$
\epsilon = \gamma^0  \gamma^1 \ldots  \gamma^n
$}}
$$ 
l'{\sl op\'erateur d'orientation\/}\index{op\'erateur d'orientation}.
On doit choisir un ordre sur les g\'en\'erateurs.  La
d\'efinition de $\epsilon$ n'est pas ambigu� si on a choisi une 
orientation de $E$
 (sinon, il n'est d\'efini qu'au signe pr\`es).


Soit $Z$ le centre de $C$ et $Z_0$ le centre de $C_0$.


\subsection{Le groupe $Spin$}

Le groupe des rotations $O(n)$ poss\`ede, comme nous l'avons vu, deux 
composantes connexes, mais lorsqu'on
autorise une signature pseudo-euclidienne $(p,q)$, le groupe des 
rotations correspondant, not\'e
 $L =  O(p,q)$ en  poss\`ede en g\'en\'eral quatre.
 $$L = L_+^\uparrow \cup L_+^\downarrow \cup L_-^\uparrow \cup 
L_-^\downarrow $$
Le signe $\pm$ fait r\'ef\'erence au signe du d\'eterminant
 (transformations qui changent, ou non, l'orientation
totale) et le signe $\uparrow$ ou $\downarrow$ fait r\'ef\'erence
aux transformations qui changent, ou non, l'orientation temporelle.
On dit g\'en\'eralement, pour une signature $(p,q)$, que la 
m\'etrique poss\`ede
 $min(p,q)$ dimensions de temps et $max(p,q)$ dimensions d'espace.
On consid\`ere les groupes et sous-groupes suivants
 $$ L = O(p,q), \quad SO(p,q) = L_+^\uparrow \cup L_-^\downarrow ,
\quad SO^\uparrow(p,q) = L_+^\uparrow
$$

Aucune des composantes connexes de $L$ n'est simplement connexe. Il 
est donc naturel de consid\'erer, pour chacun des groupes mentionn\'e 
leur
groupe de recouvrement universel. On note $Pin$, $Spin$ et 
$Spin^\uparrow$ les groupes simplement connexes
correspondant \`a $O$, $SO$ et $SO^\uparrow$.

Le {\sl groupe de Clifford\/} $\Gamma$ est d\'efini comme l'ensemble
\index{groupe de Clifford} 
de tous les \'el\'ements $s$ de l'alg\`ebre de Clifford $C=C(p,q)$  
qui sont  inversibles et satisfont \`a la propri\'et\'e : $$\forall x 
\in E, sxs^{-1} \in E$$
$E$ d\'esigne ici l'espace vectoriel (pseudo) euclidien, de signature 
$(p,q)$ auquel l'alg\`ebre $C$ est
associ\'ee.  Attention, l'alg\`ebre de Clifford, comme toute alg\`ebre
associative, est stable par le crochet de Lie d\'efini comme le 
commutateur,
$C$ est donc aussi une alg\`ebre de Lie {\it mais} cette alg\`ebre de 
Lie
n'est pas (n'est jamais) l'alg\`ebre de Lie du groupe de Clifford.
Il est \'evident que tout \'el\'ement inversible de $E$ appartient au 
groupe de Clifford (on plonge
$E$ dans $C$ en associant \`a $v=v_\mu e^\mu$ l'\'el\'ement $v=v_\mu 
\gamma^\mu$). En effet
$$
vxv^{-1} = {1\over v^2}vxv = {1\over v^2}(-vvx + 2 g(v,x))=-x+2g(v,x) 
v /v^2
$$
La transformation $x \mapsto vxv^{-1}$ est donc l'oppos\'ee de la 
sym\'etrie par rapport \`a l'hyperplan
conjugu\'e \`a $v$. Plus g\'en\'eralement, pour $s \in \Gamma$, la 
transformation
$$\chi(s) : x\in E \rightarrow s x s^{-1}$$
appartient au groupe orthogonal $O(p,q)$ en vertu d'une propri\'et\'e 
\'el\'ementaire connue
disant que les groupes
de rotations peuvent \^etre engendr\'es par produit de  sym\'etries 
par rapport \`a des hyperplans.

L'application $\chi$ d\'efinit manifestement une
repr\'esentation du groupe de Clifford, mais cette repr\'esentation
n'est pas fid\`ele puisque $s$ et $3s$, par exemple, d\'eterminent la 
m\^eme rotation. On va obtenir le
groupe $Spin$ \`a partir du groupe de Clifford
 en introduisant une condition de normalisation. Notons tout d'abord 
avec une
``barre'', plac\'ee au dessus d'un symbole, l'involution principale 
de $C$,
d\'efinie par ${\overline \gamma^\mu} = \gamma^\mu$, c'est \`a dire
$${\overline {\gamma^1 \gamma^2 \ldots  \gamma^p}}= \gamma^p  \ldots  
\gamma^2 
 \gamma^1$$
On voit que $$s\in \Gamma \rightarrow N(s) = {\overline s}s \in \RR$$
On d\'efinit alors 
$$Pin  =  \{s\in \Gamma \, \hbox {tels que} \, \vert N(s)\vert = 
1\}$$
$$Spin   =  Pin \cap C_0 $$
$$Spin^\uparrow   =  \{s \in Spin \, \hbox{tels que} \, N(s) = 
1\}$$
La repr\'esentation $\chi$ du groupe de Clifford est encore une 
repr\'esentation des groupes ci-dessus. Le lecteur pourra se 
convaincre du fait que
$$\chi(Spin) = L_+ \quad \hbox{et}\quad \chi(Spin^\uparrow) = 
L_+^\uparrow $$
Plus pr\'ecis\'ement, tout \'el\'ement de $Pin$ peut s'\'ecrire comme 
un produit $u_1u_2\ldots u_k$ o\`u
les $u_i$ sont des (co)-vecteurs de $E$ :
\begin{eqnarray*}
k \, \hbox {pair et} \, {\overline {s}s} > 0 &\Longleftrightarrow & 
\chi(s) \in L_+^\uparrow \\
k \, \hbox {pair et} \, {\overline {s}s} < 0 &\Longleftrightarrow & 
\chi(s) \in L_+^\downarrow \\
k \, \hbox {impair et} \, {\overline {s}s} > 0 &\Longleftrightarrow & 
\chi(s) \in L_-^\uparrow \\
k \, \hbox {impair et} \, {\overline {s}s} < 0 &\Longleftrightarrow & 
\chi(s) \in L_-^\downarrow 
\end{eqnarray*}
Par ailleurs, si $s\in Spin$, alors $-s\in Spin$, et $\chi(s) = 
\chi(-s)$. En fait $Ker(\chi)=\{-1,+1\}$
ce qui montre que $Spin^\uparrow$ recouvre bien $L_+^\uparrow$ avec 
un noyau $\ZZ_2$.

On peut non seulement d\'ecrire le groupe $Spin(p,q)$ comme un sous 
ensemble
 de l'alg\`ebre de Clifford $C(p,q)$ mais aussi l'alg\`ebre de Lie 
correspondante. Soient $x,y,z \in E$, alors $[xy,z] \in E$. En effet,
$[xy,z]=xyz - zxy =2( x g(y,z) - g(x,z) y)$. En d\'eveloppant 
l'exponentielle, on obtient ainsi
$exp(txy)\, z \, exp(-txy) \in E$,  donc $exp(txy)$ appartient au  
groupe de Clifford et $xy$ appartient
 \`a l'alg\`ebre de Lie du groupe de Clifford.
On peut aussi, en d\'eveloppant l'exponentielle, montrer que 
${\overline {exp(s)}}=exp({\overline {s}})$.

Ces remarques montrent que l'alg\`ebre de Lie du groupe de Clifford 
est engendr\'ee par $1$ et les
produits $xy$ lorsque $x,y \in E$. Cette alg\`ebre de Lie est un peut 
trop ``grosse'', celle de $Spin^\uparrow$ est plus int\'eressante.
 En effet, soient $x$ et $y$ deux (co)-vecteurs, alors
\begin{eqnarray*}
& exp(xy) \in Spin^\uparrow & \Leftrightarrow N(exp(xy))=1 \\
&{}& \Leftrightarrow exp({\overline {xy}})exp(xy) = 1 \\
&{}& \Leftrightarrow
 exp(yx+xy) = 1  \Leftrightarrow xy + yx = 0 \\
&{}& \Leftrightarrow g(x,y) = 0 
\end{eqnarray*}
Ainsi, $z \in C(p,q)$ appartient \`a $Lie(Spin^\uparrow)$ si et 
seulement si il
peut \^etre \'ecrit comme une combinaison lin\'eaire de produits $xy$ 
o\`u
$x$ et $y$ sont orthogonaux, c'est \`a dire par les produits 
$\gamma^\mu\gamma^\nu$. On retrouve bien le fait que 
$dim(Lie(Spin(p,q)))_{\vert{n=p+q}}=n(n-1)/2$.

 Exemple. Prenons $(p=3,q=1)$. Le groupe des rotations correspondant
est le groupe de Lorentz de la physique relativiste. Soit $\beta$ un
r\'eel quelconque. Alors ${\beta \over 2} \gamma^0 \gamma^1 \in 
Lie(Spin^\uparrow)$ et $s  =  exp({\beta \over 2} \gamma^0 
\gamma^1) \in Spin^\uparrow$.
Un calcul facile (d\'evelopper l'exponentielle) montre alors que
$$s = \cosh {\beta \over 2} + \gamma^0 \gamma^1 \sinh  {\beta \over 
2}$$
et que
\begin{eqnarray*}
&s\gamma^0s^{-1} &= \gamma^0 \cosh \beta + \gamma^1 \sinh \beta \\
&s\gamma^1s^{-1} &=  \gamma^0 \sinh \beta + \gamma^1 \cosh \beta \\
&s\gamma^is^{-1} &= \gamma^i \, \hbox{pour} \,  i = 2,3
\end{eqnarray*}
Notons que $s$ peut s'\'ecrire aussi
$$
s = \gamma^0(-\gamma^0  \cosh {\beta \over 2} + \gamma^1 \sinh{\beta 
\over 2})$$

Ceci montre donc que la transformation de Lorentz $\chi(s)$ est un 
``boost'' 
(rotation hyperbolique) le long de l'axe des $x$. Plus 
g\'en\'eralement, si
$$
\mbox{\fbox{$
s=exp({\beta \over 2} \gamma^0 ({\vr v}.{\vr \gamma}))
$}}
$$ $\chi(s)$ 
d\'esigne
un boost de param\`etre $\beta$ le long de la direction 
$\overrightarrow v$ et si
$$
\mbox{\fbox{$
s= exp({\theta \over 2}  {\vr n}.{\vr \gamma})
$}}
$$ $\chi(s)$ 
d\'esigne une
rotation d'angle $\theta$ autour de la direction ${\vr n}$ avec 
($\vert n \vert^2 = 1$).

Notons pour finir que $Spin(p,q) \subset C_0(p,q)=C_0(q,p)$, mais que 
$C(p,q) \neq C(q,p)$ en g\'en\'eral.
L'inclusion est \'evidente au vu de la d\'efinition du groupe $Spin$. 
Les \'egalit\'es (ou in\'egalit\'es)
entre ces ensembles r\'esultent de l'\'etude g\'en\'erale qui suit.

\subsection{Structure des alg\`ebres de Clifford r\'eelles}

Pour voir si $C$ ou $C_0$ sont des alg\`ebres simples, il faut voir 
si on peut, ou non, fabriquer
un projecteur non trivial $P$  ($P^2=P$) qui commute avec $C$ (ou 
$C_0$). Un tel projecteur peut,
dans certains cas, se
fabriquer \`a l'aide de $\epsilon$. 

La discussion est tr\`es diff\'erente suivant que la dimension $n$ 
est paire
ou impaire ($\epsilon$ appartiendra, suivant les cas, au centre de 
$C$, ou seulement au centre de $C_0$).

\subsubsection{Cas $n=p+q$ pair}

Dans ce cas, l'op\'erateur d'orientation $\epsilon$ commute avec les 
\'el\'ements pairs de $C$ mais
anticommute avec les \'el\'ements impairs (d\'emonstration 
imm\'ediate en utilisant les relations
de commutation des g\'en\'erateurs $\gamma^\mu$). Le centre de $C_0$ 
est en fait engendr\'e, dans ce cas
 par $1$ et par $\epsilon$.

La discussion d\'epend alors du carr\'e de $\epsilon$. Si  
$\epsilon^2 = 1$ on peut fabriquer deux
projecteurs $P_R  =  {1 + \epsilon \over 2}$ et 
 $P_L  =  {1 -  \epsilon \over 2}$ permettant de ``couper'' la 
sous-alg\`ebre $C_0$ en deux composantes
simples (on voit imm\'ediatement que $P_L^2 = P_L$ et $P_R^2 = P_R$).

La discussion peut se r\'esumer comme suit:
{\vskip 0.5cm}
$C$ est une alg\`ebre simple pour $n$ pair.
\begin{eqnarray*}
& C_0\, \hbox{ est simple} & \Leftrightarrow \epsilon^2 = -1 
\Leftrightarrow p-q = 2 \,\hbox{ mod}\, 4 \Leftrightarrow Z_0 \sim 
\CC \\
& C_0 \, \hbox{n'est pas simple} &  \Leftrightarrow \epsilon^2 = +1 
\Leftrightarrow p-q = 0 \,\hbox{ mod}\, 4 \Leftrightarrow Z_0 \sim 
\RR \oplus \RR \\
&{}& {\hskip 1cm} \hbox{ dans ce dernier cas} \quad  C_0 = P_L C_0 
\oplus P_R C_0
\end{eqnarray*}

Remarque : dans le cas de l'alg\`ebre de Dirac (nom donn\'e \`a 
l'alg\`ebre de Clifford dans le cas
$(p=3, q=1)$), $\epsilon^2=-1$. La sous alg\`ebre paire $C_0$ est 
simple. Pour pouvoir la casser en deux,
il faut introduire le nombre complexe $i$ et fabriquer un projecteur 
$(1 \pm \gamma_5)/2$ \`a l'aide de
$\gamma_5 = i \, \epsilon$, mais$\ldots$ cela ne peut \'evidemment se 
faire qu'en autorisant des
coefficients complexes, c'est \`a dire en  complexifiant l'alg\`ebre 
$C$.

\subsubsection{Cas $n=p+q$ impair}


Dans ce cas, l'op\'erateur d'orientation $\epsilon$ commute avec les 
\'el\'ements impairs de $C$ {\it et}
commute aussi avec les \'el\'ements impairs (d\'emonstration 
imm\'ediate en utilisant les relations
de commutation des g\'en\'erateurs $\gamma^\mu$). Le centre de $C$ 
est en fait engendr\'e, dans ce cas
 par $1$ et par $\epsilon$.

La discussion d\'epend encore alors, comme pr\'ec\'edemment du 
carr\'e de $\epsilon$. Si  $\epsilon^2 = 1$ on peut fabriquer deux
projecteurs ${1 + \epsilon \over 2}$ et 
 ${1 -  \epsilon \over 2}$ permettant de ``couper'' l'alg\`ebre de 
Clifford $C$ en deux composantes
simples.

La discussion peut se r\'esumer comme suit:
{\vskip 0.5cm}
$C_0$ est une alg\`ebre simple.
\begin{eqnarray*}
& C \, \hbox{est simple} & \Leftrightarrow \epsilon^2 = -1 
\Leftrightarrow p-q = 3 \,\hbox{ mod}\, 4 \Leftrightarrow Z \sim \CC 
\\
& C \, \hbox{n'est pas simple} &  \Leftrightarrow \epsilon^2 = +1 
\Leftrightarrow p-q = 1 \,\hbox{ mod}\, 4 \Leftrightarrow Z \sim \RR 
\oplus \RR \\
&{}&  {\hskip 1cm} \hbox{ dans ce dernier  cas} \quad  C = {1 -  
\epsilon \over 2} C \oplus {1 -  \epsilon \over 2} C
\end{eqnarray*}

De plus, $C = C_0 \oplus C_0 \epsilon$.

\subsubsection{Structure des alg\`ebres de Clifford r\'eelles et 
p\'eriodicit\'e modulo $8$}

La discussion qui pr\'ec\`ede pourrait laisser croire que la 
structure des alg\`ebres de Clifford sur $\RR$
d\'epend de $(p-q)$ modulo $4$. En fait une analyse plus fine montre 
qu'elle d\'epend
 de $(p-q)$ modulo $8$. L'id\'ee est la suivante : on commence par 
\'etudier explicitement la structure
de quelques alg\`ebres de Clifford de basse dimension, puis on 
construit les autres par produits tensoriels.
\begin{description}
\item[\tt Un th\'eor\`eme de r\'eduction dimensionelle] Soit $F$ un 
espace vectoriel muni d'un produit scalaire $g$ et $E \subset F$ un 
sous espace vectoriel de dimension paire tel que la restriction de $g$
\`a $E$ ne soit pas singuli\`ere. Soit $E^\perp$ le compl\'ement 
orthogonal de $E$ dans $F$. Alors 
$$C(F,g) = C(E,g_E) \otimes C(E^\perp,\epsilon_E^2 \,   g_{E^\perp})$$
o\`u $\epsilon_E$ est l'op\'erateur d'orientation de $E$
La d\'emonstration est imm\'ediate et s'appuie sur les relations 
suivantes. Soient 
$\gamma^\mu$ les g\'en\'erateurs associ\'es \`a $(E,g_E)$ et 
$\gamma^\alpha$ les g\'en\'erateurs associ\'es \`a $E^\perp$. On 
fabrique alors les g\'en\'erateurs $\Gamma^\alpha = \epsilon \otimes 
\gamma^\alpha$ et
 $\Gamma^\mu = \gamma^\mu \otimes 1$. On v\'erifie que les 
g\'en\'erateurs
$\Gamma$ v\'erifient bien les relations de d\'efinitions pour 
$C(F,g)$.

Ce th\'eor\`eme permet d'obtenir imm\'ediatement les r\'esultats 
suivants : Soit $E$ un espace vectoriel
de dimension $2$. Soit $\epsilon = \gamma^1 \gamma^2$ son op\'erateur 
d'orientation.
\begin{itemize}
\item Si la signature est $(2,0)$ alors $\epsilon^2 = -1$ et 
$C(2,0)\otimes C(p,q) = C(q+2,p)$
\item Si la signature est $(0,2)$ alors $\epsilon^2 = -1$ et 
$C(0,2)\otimes C(p,q) = C(q,p+2)$
\item Si la signature est $(1,1)$ alors $\epsilon^2 = +1$ et 
$C(1,1)\otimes C(p,q) = C(p+1,p+1)$
\end{itemize}


\item[\tt Alg\`ebres de Clifford de basse dimension]
En regardant les relations de commutation des g\'en\'erateurs 
$\gamma^\mu$
en basse dimension, on voit facilement que
$$C(1,0)=\CC, C(0,1)=\RR \oplus \RR, C(2,0) = M(2,\RR)$$ 
$$C(1,1) = M(2,\RR) \,\hbox{et} \, C(0,2) = \HH$$
Dans ce dernier cas, on rappelle que les \'el\'ements du corps non
commutatif des quaternions peuvent s'\'ecrire
$\begin{pmatrix}  a-id & ib + c \\ ib - c & a+ id\end{pmatrix} $
\item[\tt Produit tensoriel des alg\`ebres] $\RR$, $\CC$ et $\HH$
On utilise ensuite les isomorphismes suivants (seuls ceux mettant en 
jeu
les quaternions ne sont pas \'evidents et on peut s'assurer du 
r\'esultat
en repr\'esentant les quaternions par des matrices $2\times 2$ 
complexes du type
pr\'ec\'edent et en effectuant des produits tensoriels de matrices).
\begin{eqnarray*}
M(n,\RR) \otimes M(m,\RR) = M(nm,\RR)& \quad &\CC \otimes \CC = \CC 
\oplus \CC \\
M(n,\RR) \otimes \CC = M(n,\CC)& \quad &\HH \otimes \CC = M(2,\CC) \\
M(n,\RR) \otimes \HH = M(n,\HH)& \quad &\HH \otimes \HH = M(4,\RR)
\end{eqnarray*}
\item[\tt Le th\'eor\`eme de p\'eriodicit\'e]
Les r\'esultats pr\'ec\'edents impliquent imm\'ediatement :
\begin{eqnarray*}
C(n+8,0)& = & C(2,0) \otimes C(0,n+6) \\
{} & = & C(2,0) \otimes C(0,2) \otimes C(n+4,4) \\
{} & = & C(2,0) \otimes C(0,2) \otimes C(2,0) \otimes C(0,n+2) \\
{} & = & C(2,0) \otimes C(0,2) \otimes C(2,0) \otimes C(0,2) \otimes 
C(n,0)  \\
{} & = & M(2,\RR) \otimes \HH \otimes M(2,\RR) \otimes \HH \otimes 
C(n,0) \\
{} & = & M(4,\RR) \otimes M(4,\RR) \otimes C(n,0) \\
{} & = & M(16,\RR) \otimes C(n,0)
\end{eqnarray*}
Le r\'esultat final montre que l'alg\`ebre $C(n+8,0)$ est de m\^eme 
type que $C(n,0)$, les dimensions
sont, bien s\^ur, diff\'erentes (il faut tensorialiser par les 
matrices r\'eelles $16\times 16$) mais la
structure est la m\^eme.
De la m\^eme fa\c con, nous obtenons (supposons $q < p$) :
\begin{eqnarray*}
C(p,q) & = & C(1,1) \otimes C(p-1,q-1) \\
{} & = & C(1,1) \otimes C(1,1)\otimes \ldots \otimes C(p-q,0) \\
{} & = & M(2^q,\RR)\otimes C(p-q,0)
\end{eqnarray*}
Les deux r\'esultats $$C(n+8,0) =  M(16,\RR) \otimes C(n,0) $$ et 
$$C(p,q)=  M(2^q,\RR)\otimes C(p-q,0)$$ montre qu'il suffit 
d'\'etudier le cas purement euclidien et que
la classification ne d\'epend que de $(p-q) \, \hbox{modulo}\,  8$.

\item[\tt La classification des $C(p,q)$.]
Il suffit d'\'etudier la structure des huit premiers $C(n,0)$. Cela 
se fait sans difficult\'e en utilisant
les r\'esultats rappel\'es plus haut sur les produits tensoriels de 
matrices dans les corps $\RR$, $\CC$ et $\HH$. Par exemple, 
\begin{eqnarray*}
C(5,0)& = & C(2,0) \otimes C(0,3) = C(2,0) \otimes C(0,2) \otimes 
C(1,0) \\
{} & = & M(2,\RR) \otimes \HH \otimes (\RR \oplus \RR) = M(2,\HH) 
\oplus M(2,\HH)
\end{eqnarray*}
\end{description}
Les th\'eor\`emes de p\'eriodicit\'e montrent alors que
 $$C(p,q) = M(2^{{n-1\over 2}-1}) \oplus  M(2^{{n-1\over 2}-1}) \, 
\hbox{lorsque}\, p-q = 5 \, \hbox{mod} \, 8
$$
On peut r\'esumer tous les r\'esultats dans la table suivante ($n = 
p+q$).


\begin{center}
\begin{tabular}{|l||c|c|c|c|}
\hline
$p-q \,\hbox {mod}\, 8 $ & $ 0 $ & $ 1 $ & $ 2 $ & $ 3 $\\
\hline
$C(p,q)  $ & $ M(2^{n/2},\RR) $ & $M(2^{{n-1 \over 2}},\RR)\oplus 
M(2^{{n-1 \over 2}},\RR)  $ & $ M(2^{n/2},\RR) $ & $ M(2^{{n-1 \over 
2}},\CC) $ \\
\hline
\hline
$p-q \, \hbox {mod} \, 8$  & $ 4 $ & $ 5 $ & $ 6 $ & $ 7 $ \\
\hline
$C(p,q) $ & $ M(2^{{n\over 2}-1},\HH) $ &$ M(2^{{n-1 \over 2}},\HH) 
\oplus M(2^{{n-1 \over 2}},\HH) $ & $ M(2^{{n\over 2}-1},\HH) $ & 
$M(2^{{n-1 \over 2}},\CC)$ \\
\hline
\end{tabular}
\end{center}

Le lecteur pourra faire usage des deux tables suivantes (qui se 
d\'eduisent
 de la pr\'ec\'edente)  o\`u on \'etudie explicitement
le cas particulier d'une signature euclidenne $(n,0)$ ou $(0,n)$ et 
d'une
signature  hyperbolique $(n-1,1)$ ou $(1,n-1)$, pour les huit cas
 $n=4 \ldots 11$.
{\vskip 0.3cm}
Cas euclidien (de $n=4$ \`a $n=11$). \\
Pour r\'eduire la taille de la  table, on a not\'e $M(d,K)$ sous la forme $(d,K)$.
{\vskip 0.3cm}
\begin{center}
\begin{tabular}{|c||c|c|c|c|c|}
\hline
$n$ & $C^{{\scriptstyle \CC}}$ & $C(n,0)$ & $C(0,n)$ & $C_0(n,0)$ & $\theta$ \\
\hline
$4$ & ${}(4,\CC)$      &  ${}(2,\HH)$                 & 
${}(2,\HH)$ &  ${}\HH \oplus \HH$ & $ \epsilon $\\
$5$ & ${}(4,\CC) \oplus {}(4,\CC)$  & ${}(2,\HH) \oplus {}(2,\HH)$ &  
${}(4,\CC)$&  ${}(2,\HH)$ & {}  \\
$6$ & ${}(8,\CC)$                   &   ${}(4,\HH)$                & 
${}(8,\RR)$&   ${}(4,\CC)$ & $i \epsilon $\\
$7$ & ${}(8,\CC)\oplus {}(8,\CC)$   &   ${}(8,\CC)$                & 
${}(8,\RR)\oplus {}(8,\RR)$&   ${}(8,\RR)$ & {} \\
$8$ & ${}(16,\CC)$                  &   ${}(16,\RR)$               &  
${}(16,\RR)$&  ${}(8,\RR)\oplus {}(8,\RR)$ & $  \epsilon $ \\
$9$ & ${}(16,\CC)\oplus {}(16,\CC)$ &  ${}(16,\RR)\oplus {}(16,\RR)$ 
& ${}(16,\CC)$&   ${}(16,\RR)$ & {} \\
$10$ & ${}(32,\CC)$                 &  ${}(32,\RR)$                & 
${}(16,\HH)$&   ${}(16,\CC)$ & $ i \epsilon$ \\
$11$ & ${}(32,\CC)\oplus {}(32,\CC)$ & ${}(32,\CC)$   & 
${}(16,\HH)\oplus (16,\HH)$ & ${}(16,\HH)$ & {} \\
\hline
\end{tabular}
\end{center}
{\vskip 0.3cm}
Cas hyperbolique (de $n=4$ \`a $n=11$).\\
Pour r\'eduire la taille de  la table, on a not\'e $M(d,K)$ sous la forme $(d,K)$.
{\vskip 0.3cm}
\begin{center}
\begin{tabular}{|c||c|c|c|c|c|}
\hline
$n$ & $C^{{\scriptstyle \CC}}$ & $C(n-1,1)$ & $C(1,n-1)$ & $C_0(n-1,1)$ & 
$\theta$ \\
\hline
$4$ & ${}(4,\CC)$ &  ${}(4,\RR)$ &  ${}(2,\HH)$ &   ${}(2,\CC)$ & $ i 
\epsilon $\\
$5$ & ${}(4,\CC) \oplus {}(4,\CC)$ &  ${}(4,\CC)$ &  ${}(2,\HH) 
\oplus {}(2,\HH)$ &   ${}(2,\HH)$ & {}  \\
$6$ & ${}(8,\CC)$ &  ${}(4,\HH)$ &  ${}(4,\HH)$ &   ${}(2,\HH)\oplus 
{}(2,\HH)$ & $ \epsilon $\\
$7$ & ${}(8,\CC)\oplus {}(8,\CC)$ &  ${}(4,\HH)\oplus {}(4,\HH)$ &  
${}(8,\CC)$ &   ${}(4,\HH)$ & {} \\
$8$ & ${}(16,\CC)$ &  ${}(8,\HH)$ &  ${}(16,\RR)$ &   ${}(8,\CC)$ & $ 
i \epsilon $ \\
$9$ &  ${}(16,\CC)\oplus {}(16,\CC)$ &  ${}(16,\CC)$ &  
${}(16,\RR)\oplus {}(16,\RR)$ &   ${}(16,\RR)$ & {} \\
$10$ & ${}(32,\CC)$ &  ${}(32,\RR)$ &  ${}(32,\RR)$ &   ${}(16,\RR) 
\oplus {}(16,\RR)$ & $  \epsilon$ \\
$11$ &  ${}(32,\CC)\oplus {}(32,\CC)$ &  ${}(32,\RR) \oplus 
{}(32,\RR)$ &  ${}(32,\CC)$ &   ${}(32,\RR)$ & {} \\
\hline
\end{tabular}
\end{center}
{\vskip 0.3cm}
Remarque : en observant ces tables, on voit que
 $$C(0,n) \neq C(n-1,1) {\hskip 0.2cm}\hbox{mais}{\hskip 0.2cm} 
C(n,0)=C(1,n-1)$$
On voit aussi que $$C_0(n,0)=C_0(0,n)=C(0,n-1)$$ et que 
$$C_0(n-1,1)=C_0(1,n-1)=C(1,n-2)$$

Ces derni\`eres \'egalit\'es constituent un cas particulier de la 
relation g\'en\'erale (cons\'equence directe des
relations de p\'eriodicit\'e) : $$C_0(p,q) = 
C(p,q-1)=C(q,p-1)=C_0(q,p)$$

On sait que le groupe $Spin(p,q)$ est inclus dans la sous-alg\`ebre 
de Clifford paire $C_0(p,q)$.
Les tables ci-dessus permettent donc aussi de trouver les inclusions 
de $Spin(p,q)$ dans les groupes $SL(n,K)$ o\`u
$K=\RR, \CC, \HH$. Dans les cas de plus basse dimension, les 
inclusions peuvent \^etre des \'egalit\'es.
Par exemple $Spin(5)=SL(2,\HH)$, $Spin(10)\subset SL(16,\CC)$, 
$Spin(6,1)\subset SL(4,\HH)$ etc.

\subsection{La structure des alg\`ebres de Clifford complexes}

Si on complexifie l'alg\`ebre de Clifford r\'eelle $C$, c'est \`a dire si on autorise les scalaires  
de l'alg\`ebre \`a appartenir au corps des
nombres complexes, on peut alors toujours fabriquer, \`a l'aide de 
$\epsilon$, un op\'erateur de carr\'e un.
Cet op\'erateur est appel\'e {\sl op\'erateur de chiralit\'e\/} et
\index{op\'erateur de chiralit\'e} 
not\'e traditionnellement $\gamma_5$.
Si $\epsilon$ est d\'ej\`a de carr\'e \'egal \`a $1$, il n'y a plus 
rien \`a faire et 
on pose simplement $\gamma_5 = \epsilon$.
 Sinon, on le multiplie par $i$ et on pose $\gamma_5 = i \epsilon$. A 
l'aide
de cet op\'erateur de carr\'e $1$ on peut alors toujours fabriquer 
$2$ projecteurs ${1 \pm \gamma_5 \over 2}$ \`a l'aide desquels on 
pourra d\'ecomposer $C_0^{{\scriptstyle \CC}}$ en deux composantes simples 
lorsque $n$ est pair et
$C^{{\scriptstyle \CC}}$ en deux composantes simples lorsque $n$ est 
impair. La discussion se r\'esume par :
\begin{eqnarray*}
&\hbox{ n pair} & \quad \hbox{$ C^{{\scriptstyle \CC}}$ est simple} \\
&{}& \quad C_0^{{\scriptstyle \CC}} = {1 - \gamma_5 \over 2}C_0^{{\scriptstyle 
\CC}} \oplus  {1 + \gamma_5 \over 2}\, \hbox{$ C_0^{{\scriptstyle \CC}}$ 
n'est pas simple} \\
&\hbox{ n impair} & \quad \hbox{$ C_0^{{\scriptstyle \CC}}$ est simple} \\
&{}& \quad C^{{\scriptstyle \CC}} = {1 - \gamma_5 \over 2}C^{{\scriptstyle \CC}} 
\oplus  {1 + \gamma_5 \over 2}\, \hbox{$ C^{{\scriptstyle \CC}}$ n'est pas 
simple} \end{eqnarray*}

Remarque: Lorsque $n$ est impair, on dit parfois que $\gamma_5$ 
n'existe pas$\ldots$ Il faut pr\'eciser ce
qu'on entend par l\`a : l'op\'erateur que nous venons de d\'efinir et 
de noter $\gamma_5$ existe dans tous
les cas. Cela dit, dans le cas impair, il ne peut pas servir \`{a} 
d\'ecomposer $C_0^{{\scriptstyle \CC}}$ en deux, 
pr\'ecis\'ement parce que, dans ce cas $C_0^{{\scriptstyle \CC}}$ est simple 
! En d'autre termes, lorsque $n$ est impair,
on ne peut pas \'ecrire un spineur de Dirac comme somme de deux 
spineurs de Weyl (terminologie pr\'ecis\'ee
un peu plus bas).
 
\subsection{Type des repr\'esentations}
On dit qu'une repr\'esentation d'une $K$-alg\`ebre associative $A$ 
sur le $K$-espace vectoriel $V$ est 
de {\sl type \/} $F$ si l'ensemble des automorphismes de $V$ qui 
commutent avec l'action de $A$ est \'egal \`a $F$.
On d\'emontre que $F=GL_A(V)$ est un corps et que $K$ est un 
sous-corps de $F$. Lorsque la repr\'esentation est
irr\'eductible et $K=\RR$, les seules possibilit\'es pour $F$ sont 
$\RR$, $\CC$ ou $\HH$. Les tables qui 
pr\'ec\`edent illustrent ce ph\'enom\`ene. En effet, lorsqu'on 
\'ecrit, par exemple $C(1,3)=M(2,\HH)$, c'est qu'il 
existe, entre ces deux objets, un isomorphisme d'alg\`ebre 
associative r\'eelle. En d'autres termes, il existe une
repr\'esentation fid\`ele $\rho$ de la $\RR$-alg\`ebre $C(1,3)$ sur 
l'espace vectoriel $\HH^2$ consid\'er\'e
comme espace vectoriel r\'eel de dimension $8$. Les g\'en\'erateurs 
$\gamma^\mu$ sont alors repr\'esent\'es par des
matrices $2\times 2$ not\'ees $\rho(\gamma^\mu)$ \`a coefficients 
dans $\HH$. Il est \'evident que si $\psi \in \HH^2$ et si
$k \in \HH$, alors $(\rho(\gamma^\mu)\psi)k=\rho(\gamma^\mu)(\psi 
k)$, ce qui montre que le commutant de la repr\'esentation $\rho$ est 
bien \'egal \`a $\HH$ : $\rho$ est de type $\HH$. Bien entendu, si on 
se donne explicitement la
repr\'esentation $\rho$ par la donn\'ee de matrices 
$\rho(\gamma^\mu)$, on peut fabriquer une infinit\'e de
repr\'esentations \'equivalentes en changeant simplement de base.
Attention, l'alg\`ebre de Clifford $C(p,q)$ est d\'efinie par des 
g\'en\'erateurs $\gamma^\mu$ particuliers 
ob\'eissant aux relations de Clifford, alors que ce n'est pas le cas 
{\it a priori} pour une alg\`ebre de matrices
donn\'ee. Il est bon de garder \`a l'esprit la distinction entre ces 
alg\`ebres de Clifford ``abstraites'' 
et les alg\`ebres de matrices qui les repr\'esentent fid\`element, 
alg\`ebres qui sont 
donn\'ees par les tables pr\'ec\'edentes.
On parlera cependant abusivement du type de l'alg\`ebre $C(p,q)$.


Dans le cas $n$ pair, on voit, en consultant la table, que le type 
est 
toujours, soit r\'eel, soit quaternionique.

Dans le cas $n$ impair, les trois types peuvent exister mais 
on peut noter que si $C(p,q)$ est de type 
complexe, alors $C(q,p)$ est de type r\'eel ou de type quaternionique 
(on 
n'a jamais deux types complexes en m\^eme temps pour des signatures 
oppos\'ees).
{\vskip 0.3cm}
On peut aussi changer de point de vue et partir d'une 
repr\'esentation complexe de l'alg\`ebre de 
Clifford $C(n)^{{\scriptstyle \CC}}$ avec $n=p+q$. On veut alors retrouver 
les repr\'esentations des formes r\'eelles
$C(p,q)$ \`a partir de celle(s) de $C^{{\scriptstyle \CC}}$
(cette derni\`ere est d'ailleurs unique, \`a \'equivalence pr\`es, 
dans le cas pair, puisque l'alg\`ebre est simple). De fa\c con 
g\'en\'erale, \`a une forme r\'eelle correspond
une involution (une ``\'etoile'') $*$ qui doit satisfaire aux 
relations $*(ab)=*(b)*(a)$ et $**=1$. Les \'el\'ements
``r\'eels'' sont ceux qui sont hermitiens pour cette involution, 
c'est \`a dire, qui sont tels que $*a=a$. Ainsi, 
par exemple $M(2,\HH)$ est l'ensemble des \'el\'ements hermitiens de 
$M(4,\CC)$ pour une \'etoile particuli\`ere.

Au niveau de l'espace vectoriel support d'une repr\'esentation de 
l'alg\`ebre complexifi\'ee,
l'existence des trois types de repr\'esentations est li\'ee \`a 
l'existence, ou non, d'un op\'erateur $c$, appel\'e (en physique 
surtout) un {\sl op\'erateur de conjugaison de charge\/}, qui est 
d\'efini comme \index{conjugaison de charge}
un op\'erateur antilin\'eaire de carr\'e $\pm 1$, commutant avec les 
g\'en\'erateurs $\gamma^\mu$ de $C^{{\scriptstyle \CC}}$. En d'autres 
termes $c$ doit 
\^etre un $C^{{\scriptstyle \CC}}$-isomorphisme de l'espace vectoriel 
complexe $E_{Dirac}$ 
sur l'espace vectoriel conjugu\'e ${\overline E_{Dirac}}$ (le m\^eme 
espace vectoriel muni de la loi externe conjugu\'ee). Il existe trois 
cas :
Premier cas : $c$ existe et $c^2=-1$. Dans ce cas le type est 
quaternionique. L'op\'erateur $c$ (qui n'est pas unique)
correspond \`a la conjugaison complexe dans le commutant, suivie de 
la multiplication par un des trois
g\'en\'erateurs quaternioniques.
Deuxi\`eme cas :
 il n'existe pas de conjugaison de carr\'e $-1$ mais il existe une 
conjugaison $c$ de carr\'e $1$.
Dans ce cas, le type est r\'eel et $c$ correspond \`a la conjugaison 
complexe du commutant.
Troisi\`eme cas : lorsqu'il n'existe pas de conjugaison, le type est 
complexe.


\subsection{Spineurs}
\begin{description}
\item[Spineurs de Dirac]
Les spineurs de Dirac sont les \'el\'ements d'un espace vectoriel 
complexe $E_{Dirac}$ sur lequel est repr\'esent\'e, de fa\c con 
irr\'eductible, l'alg\`ebre de Clifford complexifi\'ee $C^{{\scriptstyle 
\CC}}=C(p,q)^{{\scriptstyle \CC}}$.


Si $n=p+q$ est pair, il n'existe qu'une seule repr\'esentation de 
Dirac, 
puisque $C^{{\scriptstyle \CC}}$ est simple.

Si $n$ est impair, nous en avons deux, puisque $C^{{\scriptstyle \CC}}$ est 
somme de deux 
alg\`ebres simples, mais ces deux repr\'esentations ne sont pas 
fid\`eles 
(la somme des deux l'est).

Dans tous les cas,  $E_{Dirac} = \CC^f$ o\`u $f=2^{[n/2]}$, $[n/2]$ 
d\'esignant la partie enti\`ere de $n/2$. Ceci r\'esulte 
imm\'ediatement 
du fait que $dim \, C = 2^n$.

\item[Spineurs de Weyl]
Si $n$ est pair, la restriction de la repr\'esentation de Dirac \`a 
la 
sous alg\`ebre paire $C_0^{{\scriptstyle \CC}}$ devient r\'eductible :  
$E_{Dirac} =  
E_{Weyl}^L \oplus  E_{Weyl}^R$. Les spineurs de Weyl gauche et droit 
sont 
d\'efinis par 
$$ E_{Weyl}^L = \{\psi \in E_{Dirac} \, \vert\, ({1 - \gamma_5 \over 
2}\psi=\psi
\Leftrightarrow \gamma_5 \psi = -\psi \}$$
$$ E_{Weyl}^R = \{\psi \in E_{Dirac} \, \vert \, ({1 + \gamma_5 \over 
2}\psi=\psi
\Leftrightarrow \gamma_5 \psi = \psi \}$$
On peut donc toujours d\'ecomposer $\psi \in E_{Dirac} : \psi = 
\psi_L + 
\psi_R$ avec $\psi_L = ({1 - \gamma_5 \over 2})\psi$ et $\psi_R =
({1 + \gamma_5 \over 2})\psi$.

Si $n$ est impair, la restriction de la repr\'esentation de Dirac \`a 
la 
sous alg\`ebre $C_0^{{\scriptstyle \CC}}$ reste irr\'eductible puisque 
$C_0^{{\scriptstyle \CC}}$ est 
simple. L'op\'erateur $\gamma_5$ ne permet pas de d\'ecomposer un 
spineur 
de Dirac (on a d\'ej\`a $\gamma_5 \psi = \pm 1$).

Rappelons que $Pin(p,q)$ et $Spin(p,q)$ (le 
recouvrement de $SO(p,q)$) sont respectivement obtenus comme 
sous-ensembles des 
alg\`ebres $C(p,q)$ et $C_0(p,q)$. Les spineurs de Dirac et de Weyl 
fournissent \'egalement des repr\'esentations irr\'eductibles des 
groupes 
correspondants.

\item[Spineurs de Majorana]

Etant donn\'e une dimension $n$ et un couple d'entiers $(p,q)$ avec 
$n=p+q$,
on dit qu'il existe des {\sl spineurs de Majorana\/}
\index{spineurs de Majorana}
 si l'{\it une des deux\/} alg\`ebres $C(p,q)$ ou $C(q,p)$ est de
 type r\'eel. 


L'existence de spineurs de Majorana se lit sur les tables 
pr\'ec\'edentes :

Dans le cas $n$ pair, on a des spineurs de Majorana
lorsque la signature $(p,q)$ est telle que
 $p-q = 0, 2 \, \hbox {ou}\,  6 \, \hbox{modulo} \, 8$.
Exemple: En dimension $4$, lorsque la signature est proprement 
euclidienne, on n'a pas de spineurs de Majorana.
Par contre, toujours avec 
$n=4$, on a des spineurs de Majorana, aussi bien lorsque la signature 
est 
hyperbolique (un seul temps) que dans le cas neutre (cas $(2,2)$).

Dans le cas impair, on a des spineurs de Majorana lorsque $p-q = 1 \, 
\hbox{ou} \, 7 \, 
\hbox{modulo} \, 8$. Ces repr\'esentations (cas impair) ne sont pas 
fid\`eles.

Soit $E_{Dirac}$ l'espace des 
spineurs de Dirac. Dans le cas $n$ pair, il n'y a pas 
d'ambigu\"\i t\'e. Dans 
le cas impair, on peut supposer $\epsilon^2=1$ (si ce n'est pas le 
cas, 
on remplace $C(p,q)$ par $C(q,p)$). Avec cette derni\`ere hypoth\`ese,
on peut toujours trouver un op\'erateur antilin\'eaire $c$.
Lorsque $c^2=1$, on 
d\'efinit l'espace vectoriel des spineurs de Majorana par
$$E_{Majorana} = \{\psi \in E_{Dirac}\, \vert \quad c\psi = \psi \}$$
Il est bien \'evident qu'un op\'erateur pour lequel $c^2=-1$ ne peut 
pas 
avoir de vecteurs propres non nuls ($c\psi=\psi$ implique 
$c^2\psi=-\psi=\psi$).



\item[Spineurs de Weyl-Majorana]
Dans le cas o\`u $n$ est pair et o\`u on peut donc parler de spineurs 
de 
Weyl, on peut se demander si les deux conditions de Weyl et de 
Majorana 
peuvent \^etre compatibles : peut-on avoir des spineurs qui sont \`a 
la fois
r\'eels et (par exemple) de chiralit\'e gauche ? Pour cela il faut 
que 
$\gamma_5$
et $c$ commutent. Sachant que $c$ est un op\'erateur antilin\'eaire, 
il 
est il est \'evident que $c$ et $\gamma_5$ commutent lorsque 
$\gamma_5 = 
\epsilon$ mais anti-commutent lorsque $\gamma_5 = i \epsilon$.

Lorsque $(p-q) = 0 \,\hbox{modulo} \, 8$, $\gamma_5 \, c = c \, 
\gamma_5$, 
 on peut donc d\'efinir les {\sl spineurs de Weyl-Majorana\/} de 
chiralit\'es \index{spineurs de Weyl-Majorana}
gauche et droite :
$$ E_{Majorana}^L = \{\psi \in E_{Weyl}^L   \vert \quad c\, 
\psi=\psi\}$$
$$ E_{Majorana}^R = \{\psi \in E_{Weyl}^R \vert \quad c \, 
\psi=\psi\}$$
On peut alors d\'ecomposer un spineur de Majorana quelconque en une 
partie
droite et une partie gauche : $ E_{Majorana} =  E_{Majorana}^L \oplus 
E_{Majorana}^R$.
Cela se produit donc, avec une signature proprement euclidienne,
 lorsque $n=8, 16, 24\ldots$ et,
avec une signature hyperbolique, lorsque $n=2, 10, 18 \ldots$

Lorsque cette d\'ecomposition est impossible, c'est \`a dire lorsque
$\gamma_5 \, c = - c \, \gamma_5$, il faut noter que si $\psi \in 
E_{Weyl}^L$, alors $c(\psi) \in E_{Weyl}^R$


\item[Remarques]

Il existe de nombreuses fa\c cons d'\'ecrire les 
g\'en\'erateurs $\gamma^\mu$ de $C(p,q)$ sous forme matricielle.
En g\'en\'eral il n'est pas indispensable d'effectuer un choix 
quelconque 
pour faire des calculs : la manipulation formelle des g\'en\'erateurs 
suffit. Si on tient absolument \`a utiliser des matrices, il faut 
noter que
l'expression de l'op\'erateur $c$, lorsque ce dernier
 existe, d\'epend elle aussi de la 
repr\'esentation choisie. Cet op\'erateur, \'etant antilin\'eaire, 
s'\'ecrira toujours comme la compos\'ee de la conjugaison complexe, 
dont 
la d\'efinition d\'epend de la repr\'esentation, et d'une matrice 
g\'en\'eralement not\'ee {\it C}, dont l'expression d\'epend aussi de 
la
 repr\'esentation choisie.

 

 En physique des particules, les spineurs d\'ecrivent les particules 
 \'el\'ementaires fermioniques les plus ``fondamentales'' : celles de 
 spin $1/2$. Les spineurs de Dirac d\'ecrivent les particules 
charg\'ees 
 (type \'electron). L'op\'erateur de conjugaison de charge associe, 
\`a toute 
 particule d\'ecrite par le spineur $\psi$, une autre particule 
 (l'anti-particule de la premi\`ere) d\'ecrite par le spineur 
$c(\psi)$.
 Dans le cadre du mod\`ele standard des interactions \'electrofaibles, 
les 
 neutrinos sont d\'ecrits par des spineurs de Weyl de chiralit\'e 
gauche (et les 
 anti-neutrinos, bien \'evidemment, par des spineurs de Weyl de 
chiralit\'e  
 droite). La nature n'est pas invariante par parit\'e : les neutrinos 
 droits ne semblent pas exister (et, m\^eme s'ils existent, ils ont 
des 
 propri\'et\'es tr\`es diff\'erentes de leurs homologues gauches).
 Rappelons qu'il n'existe pas de spineurs de Weyl-Majorana en 
dimension $4$ lorsque la
 signature est euclidienne ou hyperbolique. Il faudrait utiliser une 
signature $(2,2)\ldots$ !
 Par ailleurs, bien que les spineurs de Majorana soient 
math\'ematiquement 
 disponibles en dimension $(3,1)$,
 ils n'entrent pas dans la 
 formulation du mod\`ele standard d\'ecrivant les particules 
fondamentales connues.
 
 \end{description}

