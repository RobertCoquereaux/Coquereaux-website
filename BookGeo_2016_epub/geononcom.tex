
\chapter{Calcul diff\'erentiel pour alg\`ebres non commutatives}
\chaptermark{Calculs diff\'erentiels et GNC}

\section{Remarques philosophico-math\'ematiques sur les espaces non 
commutatifs}

Lorsqu'on se donne un ``espace'' $M$ (techniquement, un ensemble dont 
les \'el\'ements sont surnomm\'es ``points''), on sait construire l'alg\`ebre 
des fonctions sur $M$ \`a valeurs r\'eelles ou complexes. Cette alg\`ebre 
est commutative puisque multiplication et addition sont d\'efinies 
ponctuellement comme suit: $(fg)[x] =  f[x]g[x]=g[x]f[x]=(gf)[x]$ 
et $(f+g)[x] =  f[x]+g[x]=g[x]+f[x]=(g+f)[x]$. Lorsque notre 
espace $M$ est \'equip\'e d'une structure topologique, on sait construire 
l'alg\`ebre $C^{0}(M)$ des fonctions continues et lorsque $M$ est \'equip\'e d'une
 structure diff\'erentiable, on sait construire  l'alg\`ebre
$C^{\infty}(M)$ des fonctions diff\'erentiables.

Il est possible de compl\`etement inverser cette d\'emarche: en d'autres termes, il 
est possible de partir d'une alg\`ebre ${\mathcal A}$ commutative, abstraitement 
d\'efinie, et de fabriquer une espace $M$, tel que  ${\mathcal A}$ 
s'identifie avec l'alg\`ebre des fonctions sur $M$.
Nous allons maintenant pr\'eciser cette construction.

On se donne ${\mathcal A}$ une alg\`ebre de Banach, c'est \`a dire 
une alg\`ebre associative sur $\CC$, munie d'une norme $\vert \, 
\vert$ qui soit telle que $\vert \vert f g \vert\vert \leq \vert\vert f \vert\vert \, \vert\vert g 
\vert\vert$ et telle que l'espace vectoriel sous-jacent soit un espace de 
Banach (un espace vectoriel norm\'e complet).

On appelle {\sl caract\`ere\/}\index{caract\`ere} de ${\mathcal A}$ tout homomorphisme 
non nul de ${\mathcal A}$ vers le corps $\CC$ des complexes. 
L'ensemble des caract\`eres $M$ s'appelle le {\sl 
spectre\/} de ${\mathcal A}$. 

On suppose maintenant l'alg\`ebre ${\mathcal A}$ {\sl commutative\/}.
On appelle {\sl transformation de Gelfand \/} \index{transformation de 
Gelfand} l'application
${\mathcal  F}$ de ${\mathcal A}$ dans l'alg\`ebre commutative ${C^{0}(M)}$ 
qui \`a $f \in {\mathcal A}$ associe $\hat f \in {C^{0}(M)}$, 
d\'efini, pour tout caract\`ere $x \in M$ de ${\mathcal A}$ par
$$
\mbox{\fbox{$
\hat f(x) = x(f)
$}}
$$

R\'esultat (sans d\'emonstration) : ${\mathcal F}$ est un homomorphisme 
d'alg\`ebre de Banach commutative.

Encore quelques d\'efinitions : 

Une alg\`ebre de Banach involutive \index{alg\`ebre de Banach involutive} est une alg\`ebre de Banach  munie d'une 
{\sl \'etoile\/} c'est \`a dire une 
involution ($f^{**} = f$), anti-lin\'eaire $(\lambda f)^{*} = 
\overline{\lambda} f^{*}$ (pour $\lambda \in \CC$) et  
anti-multiplicative ($(fg)^{*}=g^{*}f^{*}$), telle que l'\'etoile 
soit isom\'etrique $\vert \vert f^{*} \vert \vert = \vert \vert f \vert \vert$.

Une {\sl $C$ - \'etoile alg\`ebre\/}, \index{$C$ - \'etoile alg\`ebre} est une alg\`ebre de Banach involutive 
telle que, $\forall f \in {\mathcal A}$ 
$\vert \vert f^{*}f \vert \vert = \vert \vert f^{*} \vert \vert \, \vert 
\vert f \vert \vert = \vert \vert f \vert \vert^{2}
$

Th\'eor\`eme de Gelfand  (sans d\'emonstration) : 
Lorsque ${\mathcal A}$ est une $C$-\'etoile alg\`ebre \underline{commutative}, la 
transformation de Gelfand entre ${\mathcal A}$ et l'alg\`ebre ${C^{0}(M)}$ des 
fonctions continues sur le spectre de ${\mathcal A}$ est un isomorphisme.

En fait, on peut pr\'eciser davantage: lorsque ${\mathcal A}$ est une 
$C$-\'etoile alg\`ebre {commutative} unitale (c'est \`a dire 
avec unit\'e), l'espace $M$ est compact. D'une certaine fa\c con, rajouter 
une unit\'e \`a une alg\`ebre qui n'en a pas revient \`a compactifier 
(via Alexandrov) son spectre.

Ce qui ressort de cette discussion, c'est le fait que s'int\'eresser \`a un espace (un ensemble de points) 
ou s'int\'eresser \`a une alg\`ebre  commutative sont deux activit\'es {\it grosso modo \/}
essentiellement \'equivalentes. En langage savant, on dit que la 
transformation de Gelfand ${\mathcal F}$ permet de d\'efinir un
foncteur r\'ealisant une \'equivalence entre la cat\'egorie des espaces topologiques 
(compacts) et celle des $\CC^{*}$ alg\`ebres commutatives (unitales).

Une alg\`ebre non commutative ne peut pas \^etre consid\'er\'ee comme une alg\`ebre de 
fonctions (\`a valeurs r\'eelles ou complexes) sur un espace, puisque 
l'alg\`ebre serait alors commutative. 
La ``g\'eom\'etrie non commutative'', au sens le plus large du terme, 
consiste souvent \`a re-\'ecrire les diverses propri\'et\'es g\'eom\'etriques des 
``espaces'' dans le langage des alg\`ebres commutatives, c'est \`a dire 
sans utiliser la notion de point, puis \`a effacer, partout o\`u cela est 
possible, le mot ```commutatif''. Ce faisant, on invente alors une 
nouvelle g\'eom\'etrie, celle des alg\`ebres non commutatives. Les espaces 
non commutatifs n'existent donc pas, mais les alg\`ebres qui les 
d\'efinissent, elles, existent bel et bien.

Du point de vue de la physique, il est pratique (et d'usage courant!) 
de d\'ecrire notre environnement \`a l'aide de points (pensez \`a la t\^ete du 
voyageur \`a qui on dirait ``Voyez-vous ce caract\`ere de la 
$C$-\'etoile alg\`ebre $C^{0}(S^{2})$ ?'' au lieu de ``Voyez-vous ce 
point \`a la surface de la Terre ?''). Cela dit, les deux points de vue 
sont \'equivalents, et on passe d'un point de vue \`a l'autre \`a l'aide de la 
superbe formule $\hat f(x) = x(f)$ sur laquelle il est bon 
de m\'editer\ldots
A ce sujet, nous invitons le lecteur \`a relire le paragraphe de 
l'Introduction intitul\'e ``Du classique au quantique''.

\section{Calculs diff\'erentiels} 
\label{sec: calcdiffnoncom}
\subsection{Remarques}
Dans le cadre commutatif, \'etant donn\'e une vari\'et\'e $M$, nous avons 
d\'ecrit, de fa\c con d\'etaill\'ee, {\sl une\/} alg\`ebre diff\'erentielle gradu\'ee, en l'occurence, 
celle, not\'ee $\Lambda(M)$ des formes diff\'erentielles: le ``complexe de 
De Rham''. Cette alg\`ebre est diff\'erentielle (puisque munie d'une 
diff\'erentielle $d$) et diff\'erentielle gradu\'ee puisque $d$ envoie 
$\Lambda^{p}(M)$ dans $\Lambda^{p+1}(M)$. 
De plus, elle est telle que $\Lambda^{0}(M) = C^\infty(M)$.
Comme nous le verrons un peu plus bas, le lecteur devrait se garder de croire 
qu'il s'agit l\`a de la seule possibilit\'e.


Dans le cadre non commutatif, nous supposons donn\'ee une alg\`ebre associative ${\mathcal A}$, 
unitale pour simplifier, mais non n\'ecessairement commutative. ${\mathcal 
A}$ va remplacer, dans la construction, l'alg\`ebre commutative 
$C^\infty(M)$, c'est \`a dire, ``philosophiquement'', l'espace $M$ 
lui-m\^eme. On veut pouvoir associer \`a ${\mathcal A}$ une alg\`ebre 
diff\'erentielle gradu\'ee $\Omega$, qui co\"incide avec ${\mathcal A}$ en 
degr\'e z\'ero. Les \'el\'ements de $\Omega$ vont remplacer les formes 
diff\'erentielles usuelles. On pourrait dire que ce sont des {\sl formes 
diff\'erentielles quantiques \/}\index{formes 
diff\'erentielles quantiques }.

Nous cherchons \`a fabriquer une alg\`ebre diff\'erentielle gradu\'ee qui, en 
degr\'e z\'ero, co\"incide avec ${\mathcal A}$. En fait, il existe de 
nombreuses possibilit\'es, chaque possibilit\'e d\'efinit ce qu'on  
appelle un {\sl calcul diff\'erentiel sur l'alg\`ebre\/} ${\mathcal A}$.  
Cependant, une de ces possibilit\'es est plus 
g\'en\'erale que les autres, en un sens que nous allons pr\'eciser.  
C'est celle qu'on d\'esigne sous le nom d'alg\`ebre $\Omega{\mathcal A}$ des  {\sl formes universelles}. 

\subsection{L'alg\`ebre diff\'erentielle des formes universelles  $\Omega{\mathcal A}$}
\index{formes diff\'erentielles universelles}


	\subsubsection{Universalit\'e}  Soit $\mathcal A$ une alg\`ebre associative unitale. 
	On veut construire une alg\`ebre diff\'erentielle $\ZZ$-gradu\'ee $(\Omega{\mathcal 
	A},\delta)$ qui soit ``la plus g\'en\'erale possible'', et qui soit telle 
	que $\Omega{\mathcal A}^{0} = {\mathcal A}$. Etre ``la plus g\'en\'erale 
	possible'' signifie que tout autre alg\`ebre du m\^eme type pourra 
	s'obtenir \`a partir de celle-ci en imposant des relations 
	suppl\'ementaires. Techniquement, 
	cela revient \`a dire que si $(\Xi, d)$ est une autre alg\`ebre 
	diff\'erentielle $\ZZ$-gradu\'ee, avec $\Xi^{0} = {\mathcal A}$, alors, 
	c'est qu'il existe un morphisme $\alpha$ (morphisme d'alg\`ebre diff\'erentielle 
	gradu\'ee) de $\Omega{\mathcal A}$ {\sl sur\/} $\Xi$ tel que
    l'alg\`ebre $(\Xi, d)$ apparaisse comme un quotient de l'alg\`ebre 
    des formes universelles
	$(\Omega{\mathcal A},\delta)$ : $$\Xi = \Omega{\mathcal A} / K$$ 
	Ici, le noyau
	$K$ de $\alpha$ est un id\'eal bilat\`ere gradu\'e diff\'erentiel de $\Omega{\mathcal 
	A}$ (id\'eal bilat\`ere car $\Omega{\mathcal A}.K \subset K$, $K.\Omega{\mathcal 
	A} \subset K$ et diff\'erentiel car $\delta K \subset K$).
	En d'autres termes, $(\Omega{\mathcal A},\delta)$  est un objet 
	universel dans la cat\'egorie des alg\`ebres diff\'erentielles 
	$\ZZ$-gradu\'ees et on pourrait \'ecrire tout ceci \`a l'aide de 
	diagrammes commutatifs\ldots
	

	\subsubsection{Construction de $\Omega{\mathcal A}$ par g\'en\'erateurs 
	et relations}  \label{sec: Omega}
	On part de ${\mathcal A}$. D\'esignons les \'el\'ements de ${\mathcal A}$ 
	par des symboles $a_{p}$. On introduit alors de {\it nouveaux} 
	symboles qu'on va d\'esigner par $\delta a_{p}$. Attention, pour 
	l'instant, $\delta$ n'est pas (encore) un op\'erateur : le symbole $\delta 
	a_{p}$ doit \^etre pris comme un tout : c'est une copie de l'\'el\'ement 
	$a_{p}$. L'espace vectoriel engendr\'e par les symboles $\delta a_{p}$ est 
	simplement une copie de l'espace vectoriel ${\mathcal A}$. Ensuite, on 
	fabrique des {\sl mots}, en concat\'enant librement des \'el\'ements de ${\mathcal 
	A}$ (donc des $a_{p}$)  et des \'el\'ements du type $\delta a_{q}$. 
	Ainsi $a_{0} \, \delta a_{1} \, a_{2} \, a_{3} \, \delta a_{4}\, 
	\delta a_{5} \, a_{6}$ est un mot. On d\'ecide alors d'additionner et 
	de multiplier librement ces mots de fa\c con \`a ce que la structure 
	obtenue soit une alg\`ebre. Jusque l\`a, on n'obtient rien de tr\`es 
	palpitant : juste une alg\`ebre ``libre'' engendr\'ee par des symboles.
	Pour finir, on va imposer des relations : celles de ${\mathcal A}$, 
	tout d'abord, mais surtout, les deux suivantes (pour tout $a,b$ 
	dans ${\mathcal A}$) :
	$$
	\mbox{\fbox{$
	\delta ab = (\delta a) \,  b + a \, (\delta b)
	$}}
	$$
	$$  
	\mbox{\fbox{$
	 \one \delta a = \delta a \, {\mbox et } 
	\, \delta \one = 0
	$}}
	$$
	La premi\`ere relation identifie deux \'el\'ements, jusque l\`a diff\'erents, de 
	l'alg\`ebre libre. L'ensemble obtenu est, par construction, une alg\`ebre, qu'on note 
	$\Omega{\mathcal A}$. La derni\`ere chose \`a faire consiste \`a introduire
	l'op\'erateur not\'e $\delta$,  d\'efini pour tout \'el\'ement $a$ de ${\mathcal 
	A}$ par $\delta(a) = \delta a$ et $\delta \delta a = 0$. L'alg\`ebre 
	obtenue devient ainsi une alg\`ebre diff\'erentielle.
	
	On pourrait, bien entendu, formaliser la construction ci-dessus, en 
	terme d'id\'eaux et de relations, mais, le r\'esultat est, somme toute, 
	tr\`es simple : on part des \'el\'ements $a$ de ${\mathcal A}$ et on 
	introduit des diff\'erentielles $\delta a$ (attention, ce ne sont pas 
	des \'el\'ements de ${\mathcal A}$) de fa\c con \`a ce que la r\`egle de 
	Leibniz (la r\`egle de d\'erivation d'un produit) soit v\'erifi\'ee.
	
	Les r\`egles ci-dessus permettent de re-\'ecrire n'importe quel \'el\'ement 
	de $\Omega{\mathcal A}$ sous la forme d'une combinaison lin\'eaire de 
	termes du type $a_{0} \, \delta a_{1} \,  \delta a_{2} \, \ldots \,  
	\delta a_{p}$ o\`u tous les $a_{i}$ sont des \'el\'ements de ${\mathcal A}$ 
	et o\`u le seul \'el\'ement qui n'est pas diff\'erenti\'e ($a_{0}$) se 
	situe {\it \`a gauche}. En effet, par exemple 
	\begin{eqnarray*}
	a_{0} \, \delta a_{1} \, \delta a_{2} \, a_{3} & = &
	a_{0} \, \delta a_{1} \delta (a_{2}a_{3}) - a_{0}\, \delta a_{1}\, 
	a_{2} \, \delta a_{3} \\
	{} & = & 
		a_{0} \, \delta a_{1} \delta (a_{2}a_{3}) - a_{0}\, \delta 
		(a_{1}a_{2}) \, \delta	a_{3} +
		  a_{0} \, \delta a_{1} \, \delta a_{2} \, \delta a_{3}
    \end{eqnarray*}
    Cette remarque montre que $\Omega{\mathcal A} = 
    \bigoplus_{p=0}^{\infty} \Omega^p{\mathcal A}$, o\`u $\Omega^p{\mathcal A}$ est 
    l'espace vectoriel engendr\'e par les termes du type 
    $a_{0} \, \delta a_{1} \,  \delta a_{2} \, \ldots \, 	\delta 
    a_{p}$, avec $a_{i}\in {\mathcal A}$. Ainsi, $\Omega{\mathcal A}$ est 
    donc bien $\ZZ$-gradu\'ee. Il est facile de v\'erifier, en utilisant 
    les r\`egles pr\'ec\'edentes que, pour $\sigma \in \Omega^p{\mathcal 
    A}$ et $\tau \in \Omega{\mathcal A}$
    $$
    \mbox{\fbox{$
     \delta (\sigma \tau) = \delta(\sigma) \tau + (-1)^p \sigma 
    \delta(\tau)
    $}}
    $$
    
    Le fait que l'alg\`ebre diff\'erentielle $\Omega{\mathcal A}$ soit 
    universelle vient du fait que, dans sa construction, nous n'avons 
    rien impos\'e d'autre que la r\`egle de Leibniz ainsi que les relations 
    alg\'ebriques d\'ej\`a pr\'esentes dans $\mathcal A$. Tout autre alg\`ebre 
    diff\'erentielle construite sur $\mathcal A$ contiendra donc 
    automatiquement des relations suppl\'ementaires. Soit $(\Xi, d)$ 
    une autre alg\`ebre diff\'erentielle, \'egalement associ\'ee \`a ${\mathcal 
    A}$, on sait qu'il doit alors exister un morphisme $\alpha$ de 
    $\Omega{\mathcal A}$ dans $\Xi$, ce morphisme est simplement 
    d\'efini sur les \'el\'ements de base, par
     $\alpha( a_{0} \, \delta a_{1} \,  \delta a_{2} \, \ldots \,  
	\delta a_{p}) = a_{0} \, d a_{1} \,  d a_{2} \, \ldots \,  
	d a_{p} $ et \'etendu par lin\'earit\'e sur toute l'alg\`ebre $\Xi$.
	
	\subsubsection{Construction explicite de $\Omega{\mathcal A}$ par produit 
	tensoriel}  La construction pr\'ec\'edente est simple et, en principe 
	suffisante. Cela dit, il est agr\'eable de pouvoir consid\'erer $\delta 
	a$ comme un objet construit concr\`etement  ``\`a partir'' de $a$ et non comme un 
	symbole abstrait. Voici donc une seconde construction de l'alg\`ebre 
	des formes universelles qui r\'epond \`a ce souci.
	
	Soit $m : {\mathcal A} \otimes {\mathcal A} \mapsto {\mathcal A}$, 
	l'op\'erateur de multiplication $m (a\otimes b) = ab$.
	
	Posons $\Omega^0{\mathcal A}  =  {\mathcal A}$
	On d\'ecide de noter (prenons $a$ et $b$ dans ${\mathcal A}$):
	$$
	\mbox{\fbox{$
	\delta b  =  \one \otimes b - b \otimes \one
	$}}
	$$
	Ainsi, $\delta b$ apparait comme une sorte de diff\'erence discr\`ete 
	(nous verrons un peu plus loin comment, dans l'exemple  o\`u ${\mathcal A}$ d\'esigne une 
	alg\`ebre de fonctions sur une vari\'et\'e comment ceci est explicitement 
	r\'ealis\'e). Plus g\'en\'eralement, nous poserons :
	$$
	\mbox{\fbox{$
	a\delta b  =  a \otimes b - ab \otimes \one
	$}}
	$$
	
	Soit $\Omega^{1}{\mathcal A}$ l'espace vectoriel engendr\'e par les 
	\'el\'ements de ${\mathcal A} \mapsto {\mathcal A}$ du type $a\delta b$.
	Notons que $a \delta b$ appartient au noyau de l'op\'erateur de 
	multiplication $m(a \delta b) = ab - ab \one = 0$. Plus 
	g\'en\'eralement, il est \'evident que les \'el\'ements de $Ker(m)$ sont des 
	combinaisons lin\'eaires d'\'el\'ements de ce type. En d'autres termes, on a
	$$ \Omega^{1}{\mathcal A}  =  Ker(m)$$
	
	On pose alors
	$$ \Omega^{2}{\mathcal A}  = 
	 \Omega^{1}{\mathcal A} \otimes_{{\mathcal A}} \Omega^{1}{\mathcal A}$$
	 et plus g\'en\'eralement
	$$ \Omega^{p}{\mathcal A}  = 
	 \Omega^{1}{\mathcal A} \otimes_{{\mathcal A}} \otimes_{{\mathcal A}} \ldots 
	 \otimes_{{\mathcal A}} \Omega^{1}{\mathcal A}$$
	 Notons que $ \Omega^{p}{\mathcal A}$ est inclus dans la $(p+1)$-i\`eme puissance 
	 tensorielle de $\mathcal A$ (bien noter cette translation d'une unit\'e !).
	 Attention : le produit tensoriel est pris {\it au dessus de }${\mathcal A}$ et non pas au dessus du corps 
	 des scalaires~! Cela signifie, en clair, la chose suivante:
	 Consid\'erons le produit de l'\'el\'ement 
	 $(a_{0} \delta a_{1}) \in \Omega^{1}{\mathcal A} \subset {\mathcal 
	 A} \otimes {\mathcal A} $ par 
	 l'\'el\'ement $(\delta a_{2}) \in \Omega^{1}{\mathcal A} \subset {\mathcal 
	 A} \otimes {\mathcal A}$.  Ce produit, pris dans $\Omega^{1}{\mathcal A} 
	 \otimes \Omega^{1}{\mathcal A}$ est l'\'el\'ement 
	 $$
	 (a_{0}\otimes a_{1} - a_{0} a_{1} \otimes \one)\otimes
	 (\one\otimes a_{2} - a_{2} \otimes 1)
	 $$
	 de  ${\mathcal A}\otimes {\mathcal A}\otimes {\mathcal A}\otimes{\mathcal A}\otimes{\mathcal A}$
	 tandis que le produit dans $\Omega^{1}{\mathcal A} \otimes_{{{\mathcal 
	 A}}} \Omega^{1}{\mathcal A}$ est un \'el\'ement de ${\mathcal A}\otimes 
	 {\mathcal A}\otimes {\mathcal A}\otimes{\mathcal A}$, en 
	 l'occurence, il s'agit de 
	 \begin{eqnarray*}
	 a_{0} \delta a_{1} \delta a_{2} & = & 
	 	 (a_{0}\otimes a_{1} - a_{0} a_{1} \otimes \one)
	 (\one\otimes a_{2} - a_{2} \otimes \one) \\
	  {} & = & a_{0} \otimes a_{1} \otimes a_{2} - a_{0}a_{1}\otimes \one 
	  \otimes a_{2} - a_{0} \otimes a_{1} a_{2} \otimes \one + a_{0} 
	  a_{1}\otimes a_{2} \otimes \one
	 \end{eqnarray*}
      L'\'ecriture explicite de $ a_{0} \delta a_{1} \delta a_{2}$ en 
      termes de produit tensoriels contient donc un unique terme $a_{0} 
      \otimes a_{1} \otimes a_{2}$ et une somme altern\'ee d'autres 
      termes, chacun d'entre eux contenant l'unit\'e de l'alg\`ebre ainsi 
      qu'un unique produit du type  $a_{p}a_{p+1}$. On pourrait 
      \'egalement  partir de cette derni\`ere \'ecriture 
      explicite pour d\'efinir l'alg\`ebre $\Omega {\mathcal A}$. Noter 
      que la multiplication, lorsqu'on \'ecrit 
	  explicitement les \'el\'ements de cette alg\`ebre en termes de produits
	  tensoriels, s'\'ecrit explicitement en concat\'enant les diff\'erents 
	  termes et en effectuant la multiplication dans ${\mathcal A}$.
	 
	

	\subsubsection{Formes universelles dans le cadre commutatif : le calcul diff\'erentiel 
non local}
\index{calcul diff\'erentiel non local}
  Soit $M$ une vari\'et\'e diff\'erentiable, ou m\^eme, un ensemble 
absolument quelconque. On peut alors construire l'alg\`ebre 
{\it commutative} des fonctions sur $M$ (bien entendu, lorsque $M$ est un 
espace topologique, ou une
vari\'et\'e diff\'erentielle, on peut choisir les fonctions continues, les fonctions 
diff\'erentiables etc). Notons 
encore ${\mathcal A}$ cette alg\`ebre, sans pr\'eciser davantage.
La construction de $\Omega{\mathcal A}$ reste valable, puisque nous 
n'avons rien eu a supposer d'autre que l'associativit\'e de l'alg\`ebre ${\mathcal A}$.
Consid\'erons l'\'el\'ement $$a \delta b =a\otimes b - ab \otimes \one \in 
\Omega^{1}{\mathcal A} \subset {\mathcal A} \otimes {\mathcal A}$$
Puisque les \'el\'ements de ${\mathcal A}$ sont des fonctions sur $M$ (des 
fonctions d'une variable $x \in M$) les 
\'el\'ements de ${\mathcal A} \otimes {\mathcal A}$ sont des fonctions 
de {\it deux} variables: 
$$
\mbox{\fbox{$
[a \delta b] (x,y) = a(x) b(y) - a(x) b(x) = a(x) (b(y) - b(x))
$}}
$$
Cette fonction, comme, il se doit, s'annulle lorsqu'on pose $x=y$, 
puisque l'op\'erateur de multiplication $m (a\otimes b) = ab$, dans le cas pr\'esent, 
peut s'\'ecrire sous la forme $m(a(x)b(y))=a(x)b(x)$. Ainsi, 
$\Omega^{1}{\mathcal A}$ est constitu\'e de l'ensemble des fonctions de 
deux variables sur l'espace $M$, qui s'annulent sur la diagonale.

Remarque : lorsque $M$ est discret, il est d'usage d'identifier, 
comme nous venons de le faire,
l'alg\`ebre des fonctions $Fun(M \times M \ldots \times M)$ du produit cart\'esien  
de l'espace $M$ par lui-m\^eme avec l'alg\`ebre produit tensoriel $Fun(M) \otimes 
Fun(M) \otimes \ldots Fun(M)$. Lorsque $M$ est un espace topologique 
(en particulier une vari\'et\'e), on n'a, en g\'en\'eral, qu'une inclusion stricte
 de $Fun(M) \otimes Fun(M) \otimes \ldots Fun(M)$ dans $Fun(M \times M \ldots 
\times M)$, et il faudrait tenir compte de la topologie utilis\'ee pour 
pouvoir pr\'eciser davantage. Nous ne tiendrons pas compte de cette 
subtilit\'e topologique dans ce qui suit.


Consid\'erons maintenant un \'el\'ement de $\Omega^{2}{\mathcal A}$:

\begin{eqnarray*}
a \delta b \delta c 
& = & a \otimes b \otimes c - ab\otimes \one \otimes c - a \otimes b c \otimes \one + a  b\otimes c \otimes \one \cr	  
[a \delta b \delta c] \, (x,y,z) & = & a(x) b(y) c(z) - 
a(x) b(x) c(z) - a(x) b(y) c(y) + a(x)  b(x) c(y) \cr 
& = & a(x)[b(y) - b(x)][c(z) - c(y)]
\end{eqnarray*}

Cet \'el\'ement peut donc s'interpr\'eter comme une fonction de trois 
variables, qui s'annule lorsque $x = y$ ou lorsque $y = z$ (mais pas 
lorsque $x = z$).

Plus g\'en\'eralement, les \'elements de $\Omega^{p}{\mathcal A}$ peuvent 
\^etre consid\'er\'es comme des fonctions de $p+1$ variables qui s'annulent lorsque deux arguments 
successifs sont \'egaux.  
 
On voit que $\delta b$  d\'esigne bien ici la diff\'erence 
discr\`ete $b(y) - b(x)$. Lorsque $M$ est une vari\'et\'e diff\'erentiable, on 
peut faire tendre $y$ vers $x$ et obtenir ainsi la forme 
diff\'erentielle usuelle $db(x) = {\partial b \over \partial x^\mu} dx^\mu$.
La th\'eorie g\'en\'erale s'applique \'evidemment dans ce cas particulier: 
$\Omega^{p}{C^\infty (M)}$ est une alg\`ebre diff\'erentielle universelle 
mais il existe par ailleurs une alg\`ebre de formes diff\'erentielles 
$(\Lambda M,d)$ que nous connaissons bien (le complexe de De Rham),  
il existe donc un morphisme $\alpha$ de la premi\`ere alg\`ebre sur la 
seconde. Ce morphisme envoie $a_{0} \delta a_{1}\delta a_{2}\ldots $ 
(dans le cas pr\'esent $a_{0}(x) (a_{1}(y)-a_{1}(x))( a_{2}(z)-a_{2}(y) )
\ldots $) sur la forme diff\'erentielle $a_{0} da_{1}\wedge da_{2}\ldots $.

Notons que le noyau de ce morphisme est tr\`es gros. D'une part, on 
sait que lorsque $p > dim (M)$, $\Lambda^p M = 0$, alors que 
$\Omega^{p}{C^\infty (M)}$ n'est jamais nul (quel que soit $p$). Par 
ailleurs, m\^eme si $p \leq dim (M)$ il est facile de trouver des 
\'el\'ements de $\Omega{C^\infty (M)}$ qui s'envoient sur z\'ero: par 
exemple, l'\'el\'ement $a \, \delta (bc) - ab \, \delta c - ca \, \delta b$ n'est 
certainement pas nul dans $\Omega^{1}{C^\infty (M)}$, alors que
$a \,  d (bc) - ab \, d c - ca d b$ est nul dans 
$\Lambda^{1}M$.


	\subsubsection{L'exemple de $\CC \oplus \CC$}
	
	L'exemple qui nous venons de consid\'erer montre bien que cette alg\`ebre 
	diff\'erentielle de De Rham, dont nous avons l'habitude, est loin 
	d'\^etre la seule possible, m\^eme dans le cadre commutatif, lorsqu'on 
	veut d\'efinir un calcul diff\'erentiel. L'inconv\'enient de l'alg\`ebre des 
	formes universelles, c'est qu'elle est g\'en\'eralement tr\`es (trop) 
	``grosse''  et peu maniable.
	Cependant, il est des cas, m\^eme commutatifs, o\`u 
	l'alg\`ebre de De Rham n'est pas utilisable --- par exemple lorsque $M$ 
	n'est pas diff\'erentiable --- et il est bien pratique de pouvoir faire 
	appel \`a la derni\`ere construction. Un autre cas interessant est celui d'une 
	vari\'et\'e $M$ qui n'est pas connexe~: on peut alors, bien s\^ur, faire du 
	calcul diff\'erentiel ``\`a la De Rham'' sur chaque composante connexe, 
	mais, ce faisant, on perd de l'information, car les formes 
	universelles non nulles du type $a \delta b [x,y]$ o\`u $x$ et $y$ 
	appartiennent \`a deux composantes connexes distinctes n'ont aucune 
	correspondance dans l'alg\`ebre de De Rham. Pour illustrer ce 
	ph\'enom\`ene, qui se trouve poss\'eder une interpr\'etation physique aussi bien 
	inattendue que capitale, nous allons choisir l'exemple d'un espace 
	non connexe extr\^emement simple : celui fourni par la donn\'ee de deux 
	points. Dans ce cas, les $1$-formes usuelles (celles de De Rham) 
	n'existent pas. Par contre, on va pouvoir construire et utiliser 
	l'alg\`ebre des formes universelles $\Omega  =  \Omega(\CC \oplus 
	\CC)$.
	
	
	Consid\'erons donc un ensemble discret $\{L,R\}$ constitu\'e de deux \'el\'ements
	que nous d\'esignons par les lettres $L$ et $R$ (penser a {\it Left} 
	et {\it Right}). Soit $x$ la fonction coordonn\'ee $x(L)  =  1,
x(R)  =  0$ et  $y$ la fonction coordonn\'ee $y(L)  =  0, y(R)
 =  1$. Remarque:  $xy=yx=0$, $x^2=x, y^2=y$ and $x + y = \one$
o\`u $\one$ est la fonction unit\'e $\one(L)=1, \one(R)=1$. Un \'el\'ement quelconque 
de cette alg\`ebre associative (et commutative)  ${\cal A}$ engendr\'ee 
par
$x$ et $y$ peut s'\'ecrire $\lambda x + \mu y$ (o\`u $\lambda$ et
$\mu$ sont deux nombres complexes) et peut \^etre repr\'esent\'e par une 
matrice diagonale $\begin{pmatrix} \lambda & 0 \\ 0 & \mu  \end{pmatrix}$. 
On peut \'ecrire ${\cal A} = \CC x \oplus \CC y$. L'alg\`ebre est donc isomorphe \`a $\CC \oplus \CC$. 
Nous introduisons maintenant deux symboles $\delta x , \delta y$, 
ainsi qu'une diff\'erentielle $\delta$ qui 
satisfait \`a $\delta^2=0$, qui doit satisfaire \`a
$\delta \one = 0$ et \`a la r\`egle habituelle de d\'erivation d'un 
produit (r\`egle de Leibniz).
Il est \'evident que $\Omega^1$, l'espace des 
diff\'erentielles de degr\'e $1$ est engendr\'e par les deux quantit\'es 
ind\'ependantes $x\delta x$ and $y\delta y$. En effet, la relation $x+y=\one$
implique $\delta x + \delta y = 0$; de plus, les relations $x^2=x$ and $y^2=y$
impliquent $(\delta x)x+x(\delta x) = (\delta x)$, donc $(\delta x) x
= (\one-x) \delta x$ and $(\delta y) y = (\one-y) \delta y$. Ceci implique 
\'egalement, par exemple, $\delta x = \one \delta x = x \delta x + y \delta x$,
$x \delta x = - x \delta y$, $y \delta x = (\one-x) \delta x$, $(\delta
x)x = y\delta x = - y \delta y$ \etc. Plus g\'en\'eralement, d\'esignons par
$\Omega^p$, l'espace des diff\'erentielles de degr\'e $p$; les relations 
ci-dessus montrent qu'une base de cet espace vectoriel est fourni par 
les \'el\'ements
$\{x\delta x\delta x\ldots\delta x,y\delta y\delta y\ldots\delta y 
\}$. Posons $\Omega^0 = {\cal A} $ et $\Omega = \bigoplus_p \Omega^p
$. L'espace $\Omega$ est une alg\`ebre : on peut multiplier les formes 
librement, mais il faut tenir compte de la r\`egle de Leibniz, par 
exemple $x(\delta x) x(\delta x) = x(\one-x)(\delta x)^2$. 
Attention: l'alg\`ebre $\Omega$ est de dimension infinie, comme il se 
doit puisque $p$ parcourt toutes les valeurs de $0$ \`a l'infini.
Bien entendu, la diff\'erentielle $\delta$ ob\'eit \`a la r\`egle de 
Leibniz lorsqu'elle agit sur les \'el\'ements de ${\cal A}$ mais elle ob\'eit \`a la r\`egle de 
Leibniz gradu\'ee lorsqu'elle agit sur les \'el\'ements de $\Omega$, en 
l'occurence
$\delta(\omega_1 \omega_2) = \delta(\omega_1) \omega_2 +
(-1)^{\partial \omega_1} \omega_1 \delta(\omega_2)$ o\`u ${\partial
\omega_1}$ d\'esigne $0$ ou $1$ suivant que $\omega_1$ est pair ou impair.\par

Dans le cas particulier de la g\'eom\'etrie d'un ensemble \`a deux points, 
 $\{L,R\}$ nous retrouvons le fait qu'un \'el\'ement $A$ de $\Omega^1$ 
 consid\'er\'e comme fonction de deux variables doit ob\'eir aux contraintes 
 $A(L,L)=A(R,R)= 0$ et peut donc \^etre \'ecrit comme une matrice $2 \times 
 2$ index\'ee par $L$ et $R$ dont les \'el\'ements non diagonaux sont 
 nuls (``matrice hors diagonale''). Un \'element $F$ de $\Omega^2$ peut \^etre 
 consid\'er\'e comme fonction de trois variables ob\'eissant aux contraintes
$F(L,L,R)=F(R,R,L)=F(L,R,R)=F(R,L,L)=F(R,R,R)=F(L,L,L)=0$.
 Les deux 
seules composantes non nulles sont donc $F(L,R,L)$ and $F(R,L,R)$. Le 
fait que
$dim(\Omega^p) = 2$ pour tout $p$ sugg\`ere la possibilit\'e d'utiliser 
des matrices de taille fixe (en l'occurence des matrices
$2\times 2$) pour toutes valeurs de $p$.
 Ceci ne serait pas le cas pour 
une g\'eom\'etrie \`a plus de deux points. En effet,
on peut ais\'ement g\'en\'eraliser la construction pr\'ec\'edente, par exemple 
en partant de trois points au lieu de deux. Mais dans ce cas, $\Omega^1$
est de dimension $6$ et
$\Omega^2$ de dimension $12$. Avec $q$ points, la dimension de
$\Omega^p$ est $q(q-1)^p$. Ce dernier r\'esultat vient du fait que 
$dim({\mathcal A}\otimes{\mathcal A}) - dim (Ker(m)) = dim({\mathcal 
A})$. On a donc $dim(\Omega^{1}) = q^{2} - q$.

Pour revenir au cas de la g\'eom\'etrie \`a deux points, nous voyons qu'il est possible de
 repr\'esenter $\lambda x
(\delta x)^{2p} + \mu y (\delta y)^{2p}$ comme une matrice diagonale
$\begin{pmatrix} \lambda & 0 \\ 0 & \mu  \end{pmatrix} $ et l'\'el\'ement $\alpha x
(\delta x)^{2p+1} + \beta y (\delta y)^{2p+1}$ comme la matrice 
``hors'' diagonale $\begin{pmatrix} 0 & i\alpha \\ i\beta & 0  \end{pmatrix}$  
Autrement dit nous pouvons repr\'esenter les formes paires par des 
matrices paires (i.e. diagonales) et les formes impaires
par des matrices impaires (i.e. ``hors'' diagonales); 
ceci est non seulement naturel mais obligatoire
si on veut que la multiplication des matrices soit compatible
 avec la multiplication dans $\Omega$. En effet, les relations $$
\begin{array}{*2l} x (\delta x)^{2p} x &= x (\delta x)^{2p} \\ x
(\delta x)^{2p} y &= 0 \\ x (\delta x)^{2p+1} x &= 0 \\ x (\delta
x)^{2p+1} y &=x (\delta x)^{2p+1} \end{array} $$ montrent que cette
repr\'esentation utilisant des matrices $2 \times 2$ 
est effectivement un homomorphisme d'alg\`ebres  $\ZZ_2-$gradu\'ees, de $\Omega$
 (gradu\'e par la parit\'e de $p$) dans l'alg\`ebre des marices complexes $2 \times 2$
(avec graduation $\ZZ_2-$ associ\'ee avec la d\'ecomposition d'une matrice 
en une partie diagonale et hors diagonale).
La pr\'esence du facteur $i$ dans les matrices hors diagonales
repr\'esentant les \'el\'ements impairs est n\'ecessaire pour que les deux 
types de produits soient compatibles. L'alg\`ebre $\Omega$ 
s'obtient en effectuant la {\it somme directe \/} des espaces 
vectoriels $\Omega^p$. Comme on l'a dit, 
l'alg\`ebre $\Omega$ est donc de dimension infinie 
mais si nous repr\'esentons toute l'alg\`ebre
$\Omega$ \`a l'aide de matrices $2\times 2$ agissant sur un espace 
vectoriel fix\'e de dimension $2$, la $p$-graduation est perdue et seule 
la graduation $\ZZ_{2}$ est conserv\'ee.

Nous verrons un peu plus loin qu'il est possible, en g\'eom\'etrie 
non commutative, de donner un sens \`a la notion de connexion.
Dans le cas le plus simple, la forme de connexion $A$ n'est autre qu'une 
forme de degr\'e $1$ appartenant \`a une alg\`ebre diff\'erentielle 
$(\Xi, \delta)$ associ\'ee \`a l'alg\`ebre associative $\mathcal A$ choisie.
On verra que la courbure $F$, dans ce cas, peut \'egalement 
s'\'ecrire comme $F = \delta A + A^{2}$.

Dans le cas pr\'esent, $\Xi = \Omega$. Une forme de degr\'e $1$ est un \'el\'ement de
$\Omega^1$. Prenons $A  =  ( \varphi \, x \delta x + \bar  \varphi \,
y \delta y)$. La repr\'esentation matricielle de $A$ se lit donc $$ A
=  \begin{pmatrix} 0 & i  \varphi \\ i \bar  \varphi & 0 \end{pmatrix} $$ 
La courbure correspondante est alors $F  =  \delta A + A^2$, mais $A^2 =
- \varphi \bar  \varphi x^2 \, \delta x \delta x - \bar  \varphi
 \varphi \, y^2 \delta y \delta y =  - \varphi \bar  \varphi x \,
\delta x \delta x - \bar  \varphi
 \varphi \, y \delta y \delta y$ et  $\delta A =  \varphi \delta x
\delta x +  \bar  \varphi  \delta y \delta y = (\bar  \varphi + 
\varphi)(x \delta x \delta x +  y \delta y \delta y)$. $F$ peut 
donc s'\'ecrire aussi
 $$ F =  ( \varphi + \bar  \varphi -  \varphi \bar  \varphi )(x
\delta x \delta x + y \delta y \delta y)
 = \begin{pmatrix}  \varphi + \bar  \varphi -  \varphi \bar  \varphi & 0
\\ 0 &  \varphi + \bar  \varphi -  \varphi \bar  \varphi  \end{pmatrix} $$ 
Nous pouvons choisir un produit hermitien sur $\Omega$ en d\'ecidant que
la base $x(\delta
x)^p,y(\delta y)^q$ est orthonormale. Alors $\vert F \vert^2 = \bar F
F = ( \varphi + \bar
 \varphi -  \varphi \bar  \varphi)^2$. Le lecteur familier des 
 th\'eories de jauge avec brisure de symm\'etrie reconnaitra ici un 
 potentiel de Higgs \index{Higgs} translat\'e $V[\phi]  =  \vert F \vert^2$.
 (voir figure \ref{fig:Higgs}).
\begin{figure}[htbp]
\epsfxsize=8cm
$$
    \epsfbox{higgs.eps}
$$
\caption{Potentiel de Higgs $V[Re(\varphi),Im(\varphi)]$}
\label{fig:Higgs}
\end{figure}
 
Notre calcul diff\'erentiel, dans le cas pr\'esent, est commutatif, puisque
l'alg\`ebre des fonctions sur un espace \`a deux points est simplement 
l'alg\`ebre des matrices diagonales $2\times 2$ avec des coefficients 
complexes (ou r\'eels) mais notre calcul diff\'erentiel est, en un sens, 
``non local'' puisque la ``distance'' entre les deux points \'etiquet\'es 
par $L$ et $R$ ne peut pas tendre vers z\'ero\ldots
Le lecteur aura sans doute remarqu\'e que ces r\'esultats peuvent 
s'interpr\'eter en termes de champs de Higgs. Nous y reviendrons 
(exemple poursuivi en 6.2.4).

\subsection{L'alg\`ebre diff\'erentielle $\Omega_{Der}{\mathcal A}$}
\index{formes diff\'erentielles associ\'ees aux d\'erivations}

\subsubsection{Rappel sur les d\'erivations d'alg\`ebre}


 Rappelons  (relire 1.10.1)  que
 
 \begin{itemize}
 	\item $Der({\mathcal A})$ est une alg\`ebre de Lie  
 
 	\item  $Der({\mathcal A})$ n'est pas un module sur ${\mathcal A}$
 
 	\item  $Der({\mathcal A})$ est un module sur le centre de ${\mathcal A}$
 
 	\item  
$Int({\mathcal A}) = \{ad_{a} / a\in {\mathcal A} \}$ avec 
$ad_{a}(b) = [a,b]$ qu'on appelle {\sl ensemble des d\'erivations 
int\'erieures\/} est un id\'eal d'alg\`ebre de Lie de $Der {\mathcal 
A}$ (en effet soit $X_{a}  =  ad_{a} \in Int {\mathcal A}$ et $Y 
\in Der {\mathcal A}$, alors $[Y,X_{a}] = ad_{Y(a)} \in {\mathcal A}$).
 
 	\item  On note $Out {\mathcal A} = Der {\mathcal A} / Int {\mathcal A}$
 \end{itemize}
 
\subsubsection{Formes diff\'erentielles}

Classiquement, une forme diff\'erentielle $\omega$ est une $n$-forme 
sur l'alg\`ebre de Lie des champs de vecteurs, antisym\'etrique, lin\'eaire par rapport aux 
scalaires, bien s\^ur, mais aussi lin\'eaire par rapport aux fonctions, 
et \`a valeur dans les fonctions.

On va d\'efinir ici les formes diff\'erentielles comme des objets qui 
soient des $n$-formes
sur l'alg\`ebre de Lie des d\'erivations de ${\mathcal A}$, antisym\'etrique, lin\'eaire par rapport au 
scalaires, bien sur, mais aussi lin\'eaire par rapport au centre ${\mathcal 
Z(A)}$  de ${\mathcal A}$ et \`a valeurs dans l'alg\`ebre ${\mathcal A}$.

En d'autres termes, on pose
$${\underline{{\Omega}^{n}}_{Der}}({\mathcal A}) = {\mathcal C}^n_{{\mathcal Z(A)}}(Der {\mathcal 
A}, {\mathcal A}) \,   \hbox{  avec  } \,  \underline{{\Omega}^0}_{Der}({\mathcal A}) = {\mathcal A}$$

Cette d\'efinition est due \`a \cite{MDV-cras}

C'est une alg\`ebre diff\'erentielle gradu\'ee avec un produit d\'efini par

$$
(\alpha \beta)(v_1, v_2, \dots, v_{m + n}) = \sum_{\sigma \in 
{\mathcal S}_{m+n}}{ (-1)^{\vert \sigma \vert} \over m! n!} 
\alpha(v_{\sigma_{1}}, v_{\sigma_{2}}, \dots, v_{\sigma_{m}})
\beta(v_{\sigma_{m+1}}, v_{\sigma_{m+2}}, \dots, v_{\sigma_{m+n}})
$$
o\`u $\alpha \in {\underline{\Omega}_{Der}}^{m}({\mathcal A})$ et $\beta \in 
{\underline{\Omega}_{Der}}^{n}({\mathcal A})$,
et o\`u $d$ est une diff\'erentielle d\'efinie comme suit:  
la forme diff\'erentielle 
$d \omega$ peut se d\'efinir directement par son action sur tout
$(k+1)$-uplet
$\{ v_1, v_2, \dots, v_{k + 1} \}$ de d\'erivations, en posant 
\begin{eqnarray*}
 d \omega (v_1, v_2, \dots, v_{k + 1}) 
& = & \sum_{i = 1}^{k + 1} (- 1)^{i + 1} v_i
\Bigl[
\omega (v_1, \dots, \widehat v_i, \dots, v_{k + 1})
\Bigr]                                               \nonumber \\
& & + 
\sum_{i \le i \le j \le k+1 } (- 1)^{i + j} \omega
\Bigl(
[v_i, v_j], v_1, \dots, 
\widehat v_i, \dots, 
\widehat v_j, \dots, v_{k + 1}
\Bigr) 
\end{eqnarray*}
o\`u le symbole $\, \widehat{} \,$ d\'esigne l'omission de l'argument correspondant.
 
 En particulier, pour une $1$-forme $da$ (agissant sur la d\'erivation 
 $v$) on a simplement
 $$
 \mbox{\fbox{$ 
 da(v)=v(a)
 $}}
 $$
 
 On peut imm\'ediatement v\'erifier que $d$ est alors une d\'erivation 
 gradu\'ee de degr\'e $1$ sur l'alg\`ebre ${\underline{\Omega}_{Der}}({\mathcal A})$ et 
 que $d^{2}=0.$

Les d\'efinitions qui pr\'ec\`edent sont tout \`a fait naturelles puisque ce 
sont exactement les m\^emes que pour les diff\'erentielles habituelles 
(rappelons encore une fois que, dans le cas usuel de la g\'eom\'etrie 
"commutative", les d\'erivations d'alg\`ebres $v_{k}$ de l'alg\`ebre 
${\mathcal A} = C^\infty(M)$ ne sont autres que 
les champs de vecteurs).

\subsubsection{Distinction entre $\underline{\Omega}_{Der}$ et 
$\Omega_{Der}$}

En utilisant la r\`egle de Leibniz, on voit qu'un produit quelconque 
d'\'el\'ements de $a$ et de diff\'erentielles (du type $da$) peut se 
r\'eordonner sous la forme d'une somme de termes du type 
$a_{0}da_{1}\ldots da_{n}$. Cela dit, il y a une petite subtilit\'e : 
avec la d\'efinition que nous avons adopt\'ee, il n'est pas clair que 
tout \'el\'ement de $\underline{\Omega}_{Der}{\mathcal A}$ puisse s'\'ecrire comme une 
somme {\it finie \/} d'\'el\'ements de ce type. Ceci conduit \`a 
introduire la d\'efinition suivante : on pose $\Omega_{Der} {\mathcal A}= 
\bigoplus_{n=0}^{\infty} {\Omega_{Der}}^n {\mathcal A}$ o\`u 
${\Omega_{Der}}^n {\mathcal A}$ est le sous-espace vectoriel de 
$\underline{{\Omega_{Der}}}^n {\mathcal A}$ constitu\'e des sommes 
finies du type $a_{0}da_{1}\ldots da_{n}$. On d\'emontre alors 
\cite{MDV-cras} que $\Omega_{Der} {\mathcal A}$ est la plus petite 
sous alg\`ebre diff\'erentielle gradu\'ee de $\underline{{\Omega_{Der}}}{\mathcal 
A}$ contenant ${\mathcal A}$.


En g\'en\'eral, on peut oublier cette distinction entre  $\underline{\Omega}_{Der}$ et 
$\Omega_{Der}$. Dans le cas de la g\'eom\'etrie des vari\'et\'es (vari\'et\'es connexes ou r\'eunion d\'enombrables de 
vari\'et\'es connexes), on peut d\'emontrer
que les deux notions co\"incident lorsque la vari\'et\'e $M$ est 
paracompacte. Cela qui revient \`a dire que la vari\'et\'e admet une base 
topogique d\'enombrable\ldots (et dans ce cas elle admet \'egalement un atlas 
comprenant au plus une infinit\'e d\'enombrable de cartes). Pour des 
vari\'et\'es paracompactes, donc (hypoth\`ese qu'on fait presque toujours!),
les deux notions co\"incident et co\"incident \'evidemment avec l'alg\`ebre 
des formes diff\'erentielles usuelles (ceci d\'ecoulant imm\'ediatement de 
l'identit\'e entre les d\'efinitions ci-dessus et celles qu'on peut 
trouver en \ref{sec:diffext}). Lorsque l'alg\`ebre ${\mathcal A}$ n'est pas 
commutative, on aura  \'egalement les deux possibilites: 
$\underline{\Omega}_{Der}{\mathcal A}$ et  
$\Omega_{Der}{\mathcal A}$ qui peuvent co\"incider ou non. Le cas le moins 
``sauvage'' est \'evidemment celui o\`u les deux notions co\"incident 
(analogue non commutatif du cas paracompact). Dans la suite de cette 
section, on supposera que c'est le cas.

\subsubsection{Exemples}

\begin{itemize}
	\item  Cas des vari\'et\'es. On sait d\'eja que ${\mathcal A}=C^\infty(M)$, 
	que $Der{\mathcal A} = \Gamma TM$ (champs de vecteurs) et que
	${\Omega}_{Der}{\mathcal A} = \Lambda M = \Gamma T^{*}M$ (formes 
	diff\'erentielles usuelles).

	\item  Cas des matrices. ${\mathcal A}=M(n,\CC)$. Il est bien connu 
	que toutes les d\'erivations de cette alg\`ebre sont int\'erieures, c'est 
	\`a dire qu'elles sont obtenues en calculant des commutateurs avec une 
	matrice donn\'ee. Cela dit, prendre un commutateur avec une matrice $X$ 
	ou prendre un commutateur avec une matrice $X + k \one$ donne le 
	m\^eme r\'esultat. On peut donc d\'ecider de normaliser en fixant la trace de 
	$X$ \`a z\'ero. L'ensemble des matrices $n\times n$ de trace nulle 
	co\"incide, on le sait (cf chapitre 2) avec l'alg\`ebre de Lie de 
	$SL(n,\CC)$. 
	On voit que $Der{\mathcal A} = Lie(SL(n,\CC))$. Les \'el\'ements de 
	${\Omega}_{Der}^n{\mathcal A}$ sont donc les $n$ formes 
	(antisym\'etriques) sur $Lie(SL(n,\CC)$ \`a valeurs dans l'alg\`ebre ${\mathcal 
	A}$ elle-m\^eme, c'est \`a dire dans $M(n,\CC)$. On peut finalement \'ecrire
	$${\Omega}_{Der}M(n,\CC) = M(n,\CC) \otimes \land (Lie(SL(n,\CC)))^{*}$$

	\item  Cas des matrices \`a \'el\'ements fonctions sur une vari\'et\'e. On 
	prend  ${\mathcal A}=M(n,\CC) \otimes C^\infty(M)$. Nous ne 
	d\'etaillerons pas cet exemple (voir \cite{MDV-cras}). Le r\'esultat est assez 
	intuitif  et s'obtient \`a partir des deux cas pr\'ec\'edents :
	 $Der{\mathcal A} = Der(C^\infty(M)) \otimes \one \oplus 
	 C^\infty(M) \otimes Der(M(n,\CC))$ et 
	 $\Omega_{Der}(M(n,\CC) \otimes C^\infty(M)) = \Lambda(M) \otimes 
	 \Omega_{Der}M(n,\CC)$. Cet exemple a \'et\'e utilis\'e pour la construction 
	 de certains mod\`eles physiques g\'en\'eralisant les th\'eories de jauges usuelles 
	 (voir \cite{MDVKernerMadore})
	 
\end{itemize}

Remarque: il existe des alg\`ebres tr\`es simples qui n'admettent pas de 
d\'erivations\ldots par exemple l'alg\`ebre des nombres complexes!
Dans ce cas, la construction qu'on vient d'exposer ne donne rien 
(bien que l'alg\`ebre des formes universelles soit n\'eanmoins non triviale).



\subsection{Alg\`ebres diff\'erentielles pour espaces non connexes}
\index{formes diff\'erentielles pour espaces non connexes}
Soient ${\mathcal A}$ et ${\mathcal B}$ deux alg\`ebres associatives 
(commutatives ou non). Il est certain que l'alg\`ebre des formes 
universelles pour l'alg\`ebre ${\mathcal A} \otimes {\mathcal B}$  n'est pas 
isomorphe au produit tensoriel gradu\'e des alg\`ebres universelles de  
${\mathcal A}$ et ${\mathcal B}$ s\'epar\'ement. 
$$
\Omega({\mathcal A} \otimes {\mathcal B}) \neq \Omega{\mathcal A} \otimes 
\Omega{\mathcal B}
$$
En effet, par exemple,  $\Omega^{1}({\mathcal A} \otimes {\mathcal B}) 
\subset {\mathcal A} \otimes {\mathcal B} \otimes {\mathcal A} \otimes 
{\mathcal B}$ alors que 
$$
(\Omega{\mathcal A}\otimes\Omega{\mathcal B})^{1} = 
\Omega^{0}{\mathcal A}\otimes \Omega^{1}{\mathcal B} \oplus
\Omega^{1}{\mathcal A}\otimes \Omega^{0}{\mathcal B}
\subset
{\mathcal A} \otimes {\mathcal B}\otimes {\mathcal B} \oplus
{\mathcal A} \otimes {\mathcal A}\otimes {\mathcal B}
$$
Cependant, $\Omega({\mathcal A} \otimes {\mathcal B})$ et $\Omega{\mathcal A} \otimes 
\Omega{\mathcal B}$ sont toutes deux des alg\`ebres diff\'erentielles 
$\ZZ$-gradu\'ees dont le terme de degr\'e z\'ero co\"incide avec ${\mathcal A} \otimes {\mathcal B}$.
La premi\`ere \'etant universelle, il existe donc un morphisme 
surjectif de la premi\`ere sur la seconde.

Supposons maintenant que qu'on s'int\'eresse \`a une vari\'et\'e non connexe 
obtenue comme r\'eunion (disjointe) de plusieurs copies (deux pour 
simplifier) d'une m\^eme vari\'et\'e connexe $M$. On se retrouve donc dans la 
situation pr\'ec\'edente avec ${\mathcal A} = C^\infty(M)$ et 
${\mathcal B} = \CC \oplus \CC$. En effet 
${\mathcal A} \otimes {\mathcal B} = C^\infty(M) \oplus C^\infty(M).$
L'alg\`ebre diff\'erentielle $\Omega(C^\infty(M)) \otimes \Omega(\CC 
\oplus \CC)$ est encore peu commode \`a utiliser (on se souvient que 
les \'el\'ements de $\Omega(C^\infty(M))$ sont des fonctions de plusieurs 
variables qui s'annulent lorsque deux arguments successifs sont 
\'egaux). Par contre, rien ne nous interdit de  remplacer cette 
derni\`ere par l'alg\`ebres des formes diff\'erentielles usuelles $\Lambda M$. 
On obtient ainsi le diagramme suivant, o\`u chaque fl\`eche d\'esigne 
un morphisme surjectif d'alg\`ebres diff\'erentielles gradu\'ees :
$$
\Omega(C^\infty(M) \oplus C^\infty(M)) \mapsto \Omega(C^\infty(M)) \otimes \Omega(\CC 
\oplus \CC) \mapsto \Lambda M \otimes \Omega (\CC \oplus \CC)
$$

L'alg\`ebre $$\Xi  =  \Lambda M \otimes \Omega (\CC \oplus \CC)$$
produit tensoriel gradu\'e du complexe de De Rham usuel par l'alg\`ebre 
des formes universelles sur l'espace \`a deux points $\CC \oplus \CC$, 
constitue une alg\`ebre diff\'erentielle int\'eressante \`a plus d'un titre et tr\`es 
facile \`a utiliser. Elle a \'et\'e \'etudi\'ee dans \cite{CHS} et utilis\'ee auparavant dans \cite{CEV}.
Sa structure s'obtient imm\'ediatement \`a partir de notre \'etude de 
$\Omega (\CC \oplus \CC)$. On se souvient que 
$\Omega^p (\CC \oplus \CC)$ est toujours de dimension $2$ et 
repr\'esentable, soit \`a l'aide de matrices $2\times 2$ diagonales 
(lorsque $p$ est pair) soit \`a l'aide de matrices $2\times 2$ hors 
diagonales (lorsque $p$ est impair). Les \'el\'ements de 
$$\Xi^n = \sum_{p=0}^n \Lambda^p M \otimes \Omega^{n-p}(\CC \oplus \CC)$$
peuvent donc s'\'ecrire \`a l'aide de matrices $2 \times 2$ dont les 
\'el\'ements sont des formes diff\'erentielles usuelles sur la vari\'et\'e $M$ (de degr\'e $p$ 
variant de $0$ \`a $n$) et positionn\'ees soit sur la diagonale (quand 
$p$ est pair) soit en dehors de la diagonale (lorsque $p$ est impair). 
Le produit dans $\Xi$ s'obtient imm\'ediatement \`a partir du produit 
ext\'erieur dans $\Lambda M$ et du produit d\'ej\`a \'etudi\'e dans $\Omega 
(\CC \oplus \CC)$.
$$ (\rho \otimes \alpha)(\sigma \otimes \beta)  = 
(-1)^{\vert \alpha \vert \, \vert \sigma \vert}(\rho \land \sigma)\otimes 
(\alpha \beta) $$
On peut finalement encore g\'en\'eraliser la construction pr\'ec\'edente en 
rempla\c cant l'alg\`ebre des 
formes diff\'erentielles sur la vari\'et\'e $M$ par l'alg\`ebre des 
formes diff\'erentielles sur $M$ \`a valeurs dans l'alg\`ebre (associative) 
des matrices $n \times n$ complexes.

On pourrait ici continuer notre exemple des connexions sur $\CC 
\oplus \CC$, en choisissant cette fois-ci pour forme de connexion un 
\'element quelconque de $\Xi^{1}$. La norme carr\'e de la courbure 
s'interpr\`ete alors physiquement comme le lagrangien d'un mod\`ele de 
jauge $U(1) \times U(1)$, avec potentiel de Higgs \index{Higgs} et symm\'etrie 
bris\'ee. Un des deux champs de jauge devient massif (le boson 
$Z_{0}$) et l'autre reste sans masse (le photon).

\subsection{L'alg\`ebre diff\'erentielle $\Omega_{D}{\mathcal A}$}
\index{formes diff\'erentielles associ\'ees au choix d'un op\'erateur 
de Dirac}
La construction qui suit est un peu plus \'elabor\'ee que les pr\'ec\'edentes, 
en ce sens qu'elle utilise un plus grand nombre d'ingr\'edients. On 
a vu que la construction de l'alg\`ebre des formes diff\'erentielle universelle
$\Omega({\mathcal A})$ \'etait 
possible, pour une alg\`ebre associative quelconque ${\mathcal A}$. L'alg\`ebre 
diff\'erentielle $\Omega_{Der}({\mathcal A})$, quant \`a elle, fait jouer un r\^ole 
particulier aux d\'erivations de ${\mathcal A}$ (pour autand qu'elles 
existent). L'alg\`ebre diff\'erentielle que nous allons pr\'esenter 
maintenant, et dont la construction est due \`a A. Connes, repose sur la 
donn\'ee d'un ``triplet spectral'', donn\'ee qui englobe, non seulement 
l'alg\`ebre associative ${\mathcal A}$ elle-m\^eme, mais \'egalement 
d'autres donn\'ees qui peuvent \^etre consid\'er\'ees comme le codage d'une 
structure riemannienne non commutative. Certains rappels et/ou 
constructions annexes sont n\'ecessaires.

Dans l'approche traditionnelle de la g\'eom\'etrie diff\'erentielle, on commence 
par se donner un espace $M$ (on peut alors parler de l'alg\`ebre des 
fonctions sur $M$), on le munit tout d'abord d'une 
topologie (on peut alors parler de l'alg\`ebre $C^{0}(M)$ des fonctions continues 
sur $M$), puis d'une structure diff\'erentiable (ce qui revient \`a 
choisir une sous-alg\`ebre particuli\`ere  $C^{\infty}(M)$  incluse dans 
dans $C^{0}(M)$),
puis d'une structure riemannienne (choix d'une m\'etrique), puis d'une 
structure spinorielle (si la vari\'et\'e le permet), on construit alors
le fibr\'e des spineurs, puis l'op\'erateur de Dirac relatif \`a la m\'etrique 
choisie et agissant sur les champs de spineurs (sections du fibr\'e 
des spineurs). Dans le cas d'une vari\'et\'e compacte et d'une m\'etrique 
proprement riemannienne, on peut alors fabriquer un produit scalaire 
global et un espace de spineurs (l'espace $L^{2}$ des champs de 
spineurs de carr\'e int\'egrable). Dans le cas o\`u la vari\'et\'e est de 
dimension paire, on peut \'egalement d\'ecomposer cet espace de Hilbert en 
deux sous-espaces suppl\'ementaires correspondant \`a des demi-spineurs de 
chiralit\'es oppos\'ees, l'op\'erateur de Dirac allant d'un sous-espace \`a 
l'autre (on rappelle que cet op\'erateur anti-commute avec l'op\'erateur 
de chiralit\'e).

Tout ceci est maintenant bien connu du lecteur (voir chapitres 
pr\'ec\'edents). L'approche ``\`a la A. Connes'' \cite{ACbook} de la g\'eom\'etrie non commutative consiste \`a 
``renverser la vapeur'' en \'ecrivant tout ceci \`a l'envers, et sous forme alg\'ebrique 
(en utilisant des alg\`ebres commutatives), 
puis de promouvoir l'essentiel de ces transcriptions au rang de 
d\'efinitions, en effa\c cant l'adjectif ``commutatif''. 

La th\'eorie se 
divise alors en deux : il existe un cas dit ``pair''  et un cas dit 
``impair''. Nous allons simplement \'ebaucher la discussion du cas 
pair, cas qui g\'en\'eralise au cas non commutatif la g\'eom\'etrie associ\'ee \`a 
la donn\'ee d'un op\'erateur de Dirac sur une vari\'et\'e de dimension paire.
On se donne un triplet $({\mathcal A}, {\mathcal H}, D)$ 
poss\'edant les propri\'et\'es suivantes : ${\mathcal H}$ est un espace de 
Hilbert $\ZZ_{2}$ gradu\'e (l'op\'erateur de graduation est alors appel\'e 
op\'erateur de chiralit\'e),  ${\mathcal A}$ est une alg\`ebre associative munie d'une involution ($*$) et 
repr\'esent\'ee fid\`element dans ${\mathcal H}$ \`a l'aide d'op\'erateurs born\'es 
pairs, et $D$ est un op\'erateur auto-adjoint tel que les commutateurs 
$[D,a], a \in {\mathcal A}$ soient born\'es; on impose \'egalement \`a la r\'esolvente  $(D+i)^{-1}$  d'\^etre un 
op\'erateur compact. 


Un tel triplet est appel\'e {\sl triplet spectral} \index{triplet 
spectral}
mais on pourrait peut-\^etre, de fa\c con plus imag\'ee, le d\'esigner sous le 
nom d'{\sl espace riemannien quantique\/}\index{espace riemannien quantique}. Dans le cas de la g\'eom\'etrie 
commutative, ${\mathcal A}$ co\"inciderait avec la
complexifi\'ee de l'alg\`ebre 
$C^\infty(M)$, ${\mathcal H}$ avec l'espace de Hilbert $L^{2}(\mathcal 
S)$ des champs de spineurs de carr\'e int\'egrable, et $D$ avec 
l'op\'erateur de Dirac lui-m\^eme.

Dans le cas classique (commutatif), si on n'impose pas de propri\'et\'e 
de compacit\'e pour la r\'esolvente de $D$, l'alg\`ebre ${\mathcal A}$ (qui est telle que les commutateurs de ses \'el\'ements avec $D$ 
soient born\'es) n'est autre que l'alg\`ebre des fonctions  
Lipschitziennes sur $M$, c'est \`a dire celle dont les \'el\'ements sont 
tels que  $\vert f(x) - f(y) \vert 
\leq c \, d(x,y), \, \forall x,y \in M$.

Dans ce cadre commutatif, il se trouve 
qu'il est en fait possible de retrouver la distance riemannienne 
$d(x,y)$ entre deux points quelconques $x$ et $y$ de $M$ \`a partir de ces 
donn\'ees. En effet, on montre que
$$
d(x,y) = Sup\{\vert f(x) - f(y) \vert, f\in {\mathcal A}, 
\vert[D,f]\vert \leq 1 \}
$$
Le concept de distance, qu'on relie d'habitude \`a un proc\'ed\'e de 
minimisation entre diff\'erents points est alors obtenu gr\^ace \`a un 
proc\'ed\'e de maximisation pour les fonctions d\'efinies sur ces points.

Nous avons maintenant tout ce qu'il nous faut pour construire l'alg\`ebre 
diff\'erentielle $\Omega_{D}({\mathcal A})$.  Nous savons d\'ej\`a 
construire l'alg\`ebre des formes universelles $\Omega({\mathcal A})$. 
Soit $\omega = a_{0}\delta a_{1}\delta a_{2}\ldots \delta a_{n}$, une $n$-forme 
universelle (un \'el\'ement de $\Omega^n({\mathcal A})$). Nous lui 
associons l'op\'erateur born\'e  $$
\mbox{\fbox{$
\pi[\omega] = a_{0}[D,a_{1}][D,a_{2}]\ldots [D,a_{n}]
$}}
$$
Il est facile de v\'erifier que cette application est une repr\'esentation 
de l'alg\`ebre $\Omega({\mathcal A})$ dans l'alg\`ebre des op\'erateurs 
born\'es sur l'espace de Hilbert ${\mathcal H}$ (on se souvient que ${\mathcal 
A}$ est, par hypoth\`ese, repr\'esent\'e dans ${\mathcal H}$). Ceci vient 
du fait que la d\'erivation d'alg\'ebre $d$ est repr\'esent\'ee par l'op\'eration 
$[D,\,.\,]$ qui est elle-m\^eme une d\'erivation. 
La premi\`ere est de carr\'e nul, mais ce n'est malheureusement pas le cas 
de la seconde. En d'autres termes, la repr\'esentation $\pi$ n'est pas 
une repr\'esentation d'alg\`ebre {\it diff\'erentielle\/}. Il est cependant 
facile de rem\'edier \`a cela. Soit $K$ le noyau de $\pi$; c'est un id\'eal 
de $\Omega({\mathcal A})$, puisque $\pi$ est une repr\'esentation 
d'alg\`ebre. Mais $K$ n'est pas en g\'en\'eral un id\'eal diff\'erentiel : $\delta K$ 
n'est pas dans $K$. On pose alors $J  =  K \oplus \delta K$. Par 
construction $J$ est alors un id\'eal diff\'erentiel. On pose alors
$$ 
\Omega_{D}({\mathcal A})   =  \Omega({\mathcal A}) / J
$$
Par construction, l'alg\`ebre obtenue $\Omega_{D}({\mathcal A})$ est bien une alg\`ebre 
diff\'erentielle. On peut finalement la regraduer en consid\'erant les 
intersections de $Ker(\pi)$ avec l'alg\`ebre universelle.
La construction est donc achev\'ee et on d\'emontre que, dans le cas classique 
(o\`u ${\mathcal A} = C^\infty(M)$), l'alg\`ebre diff\'erentielle 
$\ZZ$-gradu\'ee $\Omega_{D}({\mathcal A}) $ obtenue est isomorphe au complexe de De Rham $\Lambda(M)$, c'est \`a dire 
\`a l'alg\`ebre des formes diff\'erentielles usuelles.

Nous n'irons pas plus avant dans cette direction. Le lecteur interess\'e 
pourra consulter une lit\'erature plus sp\'ecialis\'ee. Cela dit, il est 
peut-\^etre important de signaler ici que les constructions math\'ematiques 
pr\'esent\'ees dans cette section --- et m\^eme dans le pr\'esent chapitre --- 
sont souvent r\'ecentes, ce qui signifie que les d\'efinitions et 
constructions propos\'ees n'ont peut \^etre pas encore suffisemment b\'en\'efici\'e 
du m\^urissement n\'ecessaire. 
Cela ne signifie pas qu'elles sont erron\'ees mais elles n'ont peut \^etre
pas atteint le m\^eme degr\'e de stabilit\'e temporelle que les autres concepts 
pr\'esent\'es auparavant dans cet ouvrage.

\section{Excursion au pays des math\'ematiques non commutatives}
\sectionmark{Excursion en GNC}
\subsection{Remarques et pr\'esentation g\'en\'erale}
En vertu de la dualit\'e existant entre un espace $M$ et l'alg\`ebre 
commutative $C(M)$ des fonctions sur cet espace (la correspondance 
pr\'ecise a \'et\'e donn\'ee plus haut), on peut essayer de re-\'ecrire toutes 
les math\'ematiques traitant des propri\'et\'es des ``espaces'' dans le 
langage purement alg\'ebrique de la th\'eorie des alg\`ebres commutatives. 
On peut essayer, \'egalement, de re-exprimer tous ces concepts d'une 
fa\c con qui ne fasse pas explicitement appel \`a la commutativit\'e de 
l'alg\`ebre. Bien entendu, ce n'est pas toujours possible, mais, lorsque 
c'est le cas, on peut alors effacer l'adjectif ``commutatif'' et 
promouvoir le concept en question au niveau (par exemple) d'une d\'efinition, valable 
pour les alg\`ebres non commutatives, en g\'en\'eral. 
D'une certaine fa\c con, on pourrait voir les {\sl 
math\'ematiques non commutatives\/} simplement comme  une \'etude des 
alg\`ebres associatives non commutatives. Un tel point de vue ne 
correspondrait cependant pas \`a la d\'emarche psychologique adopt\'ee : c'est 
en effet la g\'eom\'etrie ordinaire --- plus pr\'ecisemment la notion de 
point ---  qui est souvent choisie comme support de 
notre intuition;  les th\`emes qui int\'eressent la {\sl g\'eom\'etrie non 
commutative\/} sont pr\'ecisemment les propri\'et\'es des alg\`ebres non 
commutatives qui g\'en\'eralisent les propri\'et\'es des espaces 
``ordinaires'', m\^eme si les points n'existent plus. De cette fa\c con, on peut 
alors construire une th\'eorie de la mesure non commutative, une 
topologie non commutative, un calcul diff\'erentiel pour les alg\`ebres 
non commutatives (voir {\it supra\/}), une th\'eorie des connexions, des 
espaces fibr\'es (non 
commutatifs) et m\^eme une g\'en\'eralisation de la th\'eorie des groupes (la 
th\'eorie des {\sl groupes quantiques\/}). Notre propos n'est pas ici de 
d\'etailler et d'\'etudier toutes ces th\'eories, mais simplement 
d'illustrer les consid\'erations qui pr\'ec\`edent et d'effectuer un tour rapide de ce 
zoo non commutatif, en esp\'erant que le lecteur aura plaisir \`a y 
retourner en consultant la litt\'erature sp\'ecialis\'ee. L'ouvrage pr\'esent \'etant essentiellement d\'edi\'e \`a l'\'etude de 
certains aspects de la g\'eom\'etrie diff\'erentielle, nous avons d\'ecid\'e de consacrer 
n\'eanmoins la section pr\'ec\'edente \`a une \'etude un peu plus d\'etaill\'ee des notions relatives aux 
calculs diff\'erentiels non commutatifs. Pour le reste, notre \'etude ne 
sera gu\`ere plus qu'une \'ebauche.


\subsection{Topologie non commutative et th\'eorie de la mesure non commutative}

Nous avons d\'ej\`a parl\'e de la transformation de Gelfand \'etablissant une 
correspondance entre espaces topologiques compacts et $C^{*}$-alg\`ebres 
commutatives unitales (l'existence d'une unit\'e est li\'ee \`a l'hypoth\`ese de compacit\'e).
 On voit donc, en enlevant l'adjectif ``commutatif''  que la topologie non commutative n'est autre que l'\'etude des 
$C^{*}$-alg\`ebres non commutatives.

Passons \`a la th\'eorie de la mesure. Classiquement, au lieu de d\'emarrer 
avec un espace topologique $M$, on peut partir de l'alg\`ebre $C(M)$ des 
fonctions continues sur $M$ et d\'efinir les mesures \index{mesures} (positives) comme 
les formes lin\'eaires continues (positives) sur l'alg\`ebre $C(M)$, c'est 
\`a dire comme des fonctionnelles $\mu$ telles que $\mu[\overline{f}f] 
\geq 0, \forall f \in C(M)$. La correspondance avec la notion 
\'el\'ementaire de mesure se fait gr\^ace au th\'eor\`eme de Riesz, c'est \`a 
dire en \'ecrivant $\mu[f] = \int_{X} f \, d\mu$. A partir de 
$C(M)$, nous d\'efinissons les mesures; pour une mesure $\mu$ donn\'eee, 
nous pouvons fabriquer l'espace de Hilbert ${\mathcal H} = 
L^{2}(M,\mu)$ des fonctions de carr\'e int\'egrable pour cette mesure.
$C(M)$ agit dans cet espace de Hilbert ${\mathcal H}$ par 
multiplication: nous avons une repr\'esentation $\pi$ d\'efinie par $\pi(f)  
g = fg$, avec $f\in C(M)$ et $g \in {\mathcal H}$.
A partir de ${\mathcal H}$, nous pouvons fabriquer l'alg\`ebre 
$L^\infty(M,\mu)$ des fonctions mesurables essentiellement born\'ees 
sur $M$. Soit ${\mathcal L(H)}$ l'alg\`ebre des op\'erateurs 
born\'es sur $\mathcal H$. Rappelons que l'alg\`ebre $L^\infty(M,\mu)$ peut \^etre construite comme le 
commutant de l'action $\pi$ de $C(M)$ dans ${\mathcal L(H)}.$
$$L^\infty(M,\mu) = \{ T \in {\mathcal L(H)} \,  \mbox{t.q.} \,  T \pi(f) = 
\pi(f) T, \,  \forall f \in C(M) \}$$
La mesure $\mu$ peut alors \^etre \'etendue \`a l'alg\`ebre $L^\infty(M,\mu)$ 
tout enti\`ere.
Cette derni\`ere alg\`ebre  poss\`ede la propri\'et\'e 
remarquable d'\^etre \'egale \`a son propre commutant dans ${\mathcal 
L(H)}$ (cette propri\'et\'e  caract\'erise pr\'ecisemment un type de 
sous-alg\`ebres de ${\mathcal L(H)}$ qu'on appelle {\sl alg\`ebres de Von 
Neumann \/}\index{alg\`ebres de Von Neumann}). 

Tout ce qu'on vient de rappeler figure --- peut \^etre dans un ordre 
diff\'erent --- dans un cours standard de 
th\'eorie de la mesure. Le trait essentiel, dans la pr\'esentation qui 
pr\'ec\`ede est de ne pas faire intervenir les points de l'espace $M$.
En recopiant tout ceci, mais en effa\c cant l'adjectif 
``commutatif'', on peut alors {\it inventer\/} une version non 
commutative de la th\'eorie de la 
mesure\ldots Soit dit en passant, les physiciens 
th\'eoriciens ont invent\'e la plupart de ces diff\'erents concepts, dans le cadre de la m\'ecanique 
statistique quantique, bien avant qu'ils aient \'et\'e formalis\'es par 
des  math\'ematiciens!
Reprenons donc rapidement ce qui pr\'ec\`ede, en partant d'une $C^{*}$-alg\`ebre non 
commutative ${\mathcal A}$, rempla\c cant la donn\'ee de $C(M)$.
On d\'efinit les {\sl \'etats \/} \index{\'etats} (ce sont pr\'ecisemment des mesures non 
commutatives) \index{mesures non commutatives}comme ci-dessus : un \'etat $\mu$ est une forme lin\'eaire 
positive continue sur ${\mathcal A}$, c'est \`a dire $\mu \in {\mathcal 
A}^{*}$ et $\mu[\overline{f}f] \geq 0, \forall f \in {\mathcal A}$. On 
peut supposer $\mu$ norm\'e : $\mu[\one]=1$. On construit alors un 
espace de Hilbert ${\mathcal H}$ en d\'efinissant tout d'abord le 
produit scalaire $(f,g)  =  \mu[f^{*}g]$ sur l'espace ${\mathcal A}$ 
lui-m\^eme (on n'a alors qu'une structure pre-Hilbertienne) puis en 
fabriquant l'espace de Hilbert correspondant (compl\'et\'e et s\'epar\'e). 
Cette construction bien connue porte le nom --- en math\'ematiques non 
commutatives --- de {\sl construction GNS\/} 
(Gelfand-Naimark-Segal)\index{construction GNS}. 
Comme dans le cas commutatif, ${\mathcal A}$ agit dans ${\mathcal H}$ 
par multiplication, ce qui fournit une repr\'esentation $\pi$ de ${\mathcal 
A}$  dans l'espace des op\'erateurs born\'es ${\mathcal L(H)}$. On 
consid\`ere alors ${\mathcal M}$,  le {\it bi \/}-commutant de ${\mathcal 
A}$ dans ${\mathcal L(H)}$. Ce bi-commutant est une alg\`ebre de Von 
Neumann (il est \'egal \`a son propre bi-commutant); c'est donc l'analogue non 
commutatif de $L^\infty(M,\mu)$. \footnote{Remarque: dans le cas non commutatif, 
il faut effectivement construire ${\mathcal M}$ comme le bi-commutant ${\mathcal A}''$
de ${\mathcal A}$ (qui, dans le cas non commutatif, diff\`ere du 
commutant ${\mathcal A}'.$)} Rappel: lorsque ${\mathcal A}$ est une alg\`ebre 
d'op\'erateurs, ${\mathcal A}$, ${\mathcal A}'$ et ${\mathcal A}''$ 
sont d'ordinaire diff\'erents, mais ${\mathcal A}' = {\mathcal A}'''$.
La derni\`ere \'etape consiste \`a \'etendre la d\'efinition de l'\'etat $\mu$ \`a 
l'alg\`ebre de Von Neumann ${\mathcal M}$ tout enti\`ere (on a \'evidemment ${\mathcal 
A} \subset {\mathcal M}$). 

La th\'eorie que l'on vient d'\'ebaucher est \`a 
la base de tr\`es nombreux d\'eveloppements, aussi bien en math\'ematiques (th\'eorie 
des facteurs), qu'en physique (m\'ecanique quantique statistique des 
syst\`emes avec nombre fini ou infini de degr\'es de libert\'e). Notre but, 
comme nous l'avions anonc\'e plus haut, n'\'etait que d'attirer 
l'attention du lecteur sur le parall\`ele \'evident existant entre ces 
deux th\'eories : th\'eorie de la mesure (en fait mesures de Radon) et 
th\'eorie des alg\`ebres de Von Neumann; l'un \'etant en quelque sorte la g\'en\'eralisation non 
commutative de l'autre. 

\subsection{Calcul diff\'erentiel non commutatif}

Comme on l'a vu en \ref{sec: calcdiffnoncom}, \'etant donn\'e une alg\`ebre 
associative  ${\mathcal A}$, on peut toujours fabriquer une alg\`ebre 
diff\'erentielle $\ZZ$-gradu\'ee qui co\"incide avec ${\mathcal A}$ en 
degr\'e $0$. Le choix d'une telle alg\`ebre diff\'erentielle n'est pas, en 
g\'en\'eral, unique: on dit qu'on fait alors le choix d'un calcul 
diff\'erentiel pour l'alg\`ebre ${\mathcal A}$. On peut faire un choix 
qui soit plus ``g\'en\'eral'' que les autres (formes diff\'erentielles 
universelles). Les diff\'erentes alg\`ebres diff\'erentielles possibles (les autres 
calculs diff\'erentiels associables \`a une alg\`ebre associative donn\'ee) sont des quotients de l'alg\`ebre 
des formes universelles. Nous renvoyons le lecteur \`a la section 
pr\'ec\'edente pour une analyse plus d\'etaill\'ee de ces diff\'erents choix.

\subsection{Espaces fibr\'es non commutatifs et modules projectifs}

En g\'eom\'etrie diff\'erentielle ordinaire, un espace fibr\'e 
principal peut \^etre consid\'er\'e comme un  outil servant \`a 
la fabrication de fibr\'es associ\'es, de la m\^eme fa\c con que les 
groupes eux-m\^emes servent \`a fabriquer des
repr\'esentations. En g\'eom\'etrie non commutative, on
pourrait, bien sur,  tenter de g\'en\'eraliser dans un premier
temps la structure de groupe  elle-m\^eme (c'est la th\'eorie
des {\sl groupes quantiques\/}),  puis la structure de fibr\'e
principal, et enfin celle de  fibr\'e associ\'e. Ces
g\'en\'eralisations existent. Cependant la  d\'efinition et
l'\'etude des groupes quantiques (ou {\sl alg\`ebres de
Hopf\/}) nous  entrainerait trop loin. Nous pr\'ef\'erons donc
suivre ici une approche plus  directe, qui n'utilise pas cette
notion.

Nous partons de la constatation suivante : en g\'eom\'etrie diff\'erentielle 
ordinaire, l'ensemble $\Gamma E$ des sections 
d'un fibr\'e associ\'e $E$ (les {\it champs de mati\`ere\/} de la 
physique) constitue un module sur l'alg\`ebre $C^\infty(M)$ des fonctions sur la base.
Par exemple, si $x \in M \mapsto V(x) \in \Gamma E$ est un champ de tenseurs (ou 
de spineurs \ldots), et si $x \in M \mapsto f(x) \in \RR $ (ou $\CC$) est 
une fonction, alors $ [fV](x) = f(x) \, V(x)$ est aussi un champ de 
tenseurs (ou de spineurs etc\ldots).

Ce n'est pas la notion d'espace fibr\'e vectoriel associ\'e que nous allons 
g\'en\'eraliser, mais celle de l'ensemble de ses sections.
Etant donn\'e une alg\`ebre associative ${\mathcal A}$, possiblement 
non commutative, nous allons donc consid\'erer tout module $\Gamma$ sur ${\mathcal 
A}$ comme l'analogue non commutatif d'un fibr\'e vectoriel associ\'e.
En fait, dans le cas commutatif, les modules obtenus par 
construction de fibr\'e associ\'e sont d'un type un peu particulier. On dit 
qu'ils sont projectifs de type fini (th\'eor\`eme de Serre-Swann). 
Sans rentrer dans les d\'etails, cela signifie la chose suivante.
L'ensemble des sections d'un fibr\'e vectoriel trivial dont la fibre 
type est de dimension $n$ est 
manifestement isomorphe au module ${(C^\infty(M))}^n$. Lorsque le fibr\'e 
n'est pas trivial, il suffit de se placer dans un espace un peu plus grand 
(c'est \`a dire de rajouter un certain nombre de dimensions \`a la fibre) pour le trivialiser.
Le fibr\'e de d\'epart est alors obtenu comme $p {(C^\infty(M))}^n$, 
$p$ d\'esignant un projecteur 
($p^{2}=p$) de l'alg\`ebre des matrices $n \times n$ sur $C^\infty(M)$.

Dans le cadre non commutatif, on remplacera donc la notion d'``espace 
des sections d'un fibr\'e vectoriel''  (physiquement l'espace des 
champs de mati\`ere d'un certain type) par la notion de module 
projectif fini sur une alg\`ebre associative ${\mathcal A}$. L'espace 
vectoriel $p {\mathcal A}^n$, $p$ d\'esignant un projecteur, est manifestement un module (\`a droite) sur ${\mathcal 
A}$.

 Si ${\mathcal A}$ n'est pas commutative, il faut \'evidemment faire la distinction 
entre les modules \`a droite et les modules \`a gauche.

Notons, pour finir, qu'un cas int\'eressant de module sur ${\mathcal A}$ 
est celui o\`u on choisit un module particulier \'egal \`a l'alg\`ebre 
elle-m\^eme op\'erant sur elle-m\^eme par multiplication (c'est l'analogue non commutatif d'un fibr\'e en droites). 

 
\subsection{Connections g\'en\'eralis\'ees en  geometrie non commutative}

Soit $\Xi$ un calcul diff\'erentiel sur une alg\`ebre $\mathcal A$, 
c'est \`a dire une alg\`ebre diff\'erentielle $\ZZ$-gradu\'ee, avec 
$\Xi^{0}  = \mathcal A$. Soit $\mathcal M$ un module \`a droite sur 
$\mathcal A$.
Une diff\'erentielle covariante $\nabla$ sur $\mathcal M$ est une 
application
 ${\mathcal M} \otimes_{\mathcal A} \Xi^{p} \mapsto {\mathcal M} 
\otimes 
 _{\mathcal A} \Xi^{p+1}$ telle que
 $$ 
 \mbox{\fbox{$
 \nabla( \psi \lambda) = (\nabla \psi) \lambda + (-1)^s \psi \, d \lambda 
$}}
$$
lorsque $\psi \in {\mathcal M} \otimes_{\mathcal A} \Xi^{s}$ et
 $\lambda \in \Xi^{t}$. L'op\'erateur
 $\nabla$ n'est certainement pas lin\'eaire par rapport \`a l'alg\`ebre $\mathcal 
A$ mais il est facile de constater que la courbure $\nabla^{2}$ est 
un op\'erateur lin\'eaire par rapport \`a  $\mathcal A$. 

Dans le cas particulier o\`u l'on choisit le  module  $\mathcal M$ 
comme l'alg\`ebre  $\mathcal A$ elle-m\^eme, toute $1$-forme $\Xi$ 
(tout \'el\'ement de 
$\Xi^{1}$) permet de d\'efinir une diff\'erentielle covariante: on pose 
simplement
$$
\mbox{\fbox{$
\nabla \one = \omega
$}}
$$
 o\`u $\one$ est l'unit\'e de l'alg\`ebre
$\mathcal A$.
Lorsque $f \in \mathcal A$, on obtient $$\nabla f = \nabla (\one f) = 
(\nabla \one) f + \one df = df + \omega f$$
De plus, $\nabla^{2} f = \nabla (df + \omega f) = d^{2}f + \omega 
df + (\nabla \omega) f - \omega df = (\nabla  \omega) f$. La 
courbure, dans ce cas, est \'egale \`a
$$\rho  =  \nabla \omega = \nabla \one \omega = (\nabla \one) 
\omega + 
\one d \omega = d \omega + \omega ^{2}.$$

Choisissons $u$, un \'el\'ement inversible de $\mathcal A$ et agissons 
avec $d$ sur l'equation $u^{-1} u = \one$. On obtient (utilisant le fait que 
$d\one = 0$) l'equation 
$$du^{-1} = - u^{-1} du u^{-1}.$$

D\'efinissons \'egalement $\omega' = u^{-1} 
\omega u + u^{-1} du$ et calculons la nouvelle courbure $\rho' = d\omega' 
+  
{\omega'}^2$. On obtient imm\'ediatement $\rho' = u^{-1} ( d \omega + 
\omega^2) u = u^{-1} \rho u$ o\`u $$
\mbox{\fbox{$ \rho = d \omega +
\omega^2.
$}}$$ Ceci montre que les formules usuelles sont valables,
sans qu'il soit  besoin de supposer la commutativit\'e de
l'alg\`ebre  $\mathcal A$.

Remarque : Ici nous avons choisi un module $\mathcal M$ (un ingr\'edient 
n\'ecessaire pour construire n'importe quelle th\'eorie de jauge)
\'egal \`a l'alg\`ebre ${\mathcal A}$ elle-m\^eme. Plus g\'en\'eralement, nous 
aurions pu choisir un module libre ${\mathcal A}^n$, ou m\^eme,  un 
module projectif $p{\mathcal A}^n$ sur $\mathcal A$. 
Dans ce dernier cas, le formalisme pr\'ec\'edent doit \^etre 
l\'eg\'erement modifi\'e. En effet, le projecteur $p$ va intervenir dans 
le calcul de la courbure (c'est un peu comme si nous faisions de la 
g\'eom\'etrie diff\'erentielle classique de fa\c con extrins\`eque, 
en plongeant notre espace dans un espace ``plus grand''). Comme 
toujours, la courbure est $\rho  =  \nabla \nabla$. La 
diff\'erentielle covariante est $$ \nabla = p \, d + \omega$$ o\`u 
$\omega$ est un \'el\'ement de $\Xi^{1}$ tel que $p \, \omega \, p = \omega$. En effet, si $X \in 
{\mathcal M}$, $\nabla X = p \, dX + \omega \, X$ et il est facile de 
v\'erifier que cela d\'efinit bien une connexion : prenons $f \in 
{\mathcal A}$, alors
\begin{eqnarray*}
\nabla(Xf) &=& p \, d(Xf) + \omega \, X \, f \cr
{} &=& p \, (dX) \, f + \omega \, X \, f + p \, X \, df \cr
{} &=& \nabla(X) \, f + X \, df
\end{eqnarray*}
Nous avons utilis\'e le fait que $p \, X = X$.
La courbure $\rho = \nabla^{2}$ se calcule alors comme suit:
\begin{eqnarray*}
\nabla^{2}(X) &=& pd(pdX + \omega X) + \omega(p dX + \omega X) \cr
{} &=& pd(pdX + \omega X) + \omega  (dX + \omega X) \cr
{} &=& p(dp) \, dX + pd^{2}X + (d\omega ) X - \omega (dX) + \omega  dX + \omega ^{2} X \cr
{} &=& p(dp\,  dX + (d\omega ) X) + \omega ^{2} X = [p dp dp + d\omega  + \omega ^{2}] X
\end{eqnarray*}
Nous avons utilis\'e les propri\'et\'es $\omega p=p$, $dX=d(pX)=(dp)X+pdX$ 
ainsi que $p(dp)p = pd(p^{2)}-pdp=0$, ce qui entraine 
$pdpdX=pdpdpX+p(dp)pX=pdpdpX.$ En conclusion, la courbure, dans le cas 
o\`u le projecteur ne se r\'eduit pas \`a l'identit\'e est \'egale \`a
$$
\mbox{\fbox{$
\rho = pdpdp + d\omega  + \omega ^{2}
$}}
$$
Il faut remarquer le fait que la courbure s'obtient \`a partir de 
$\nabla^{2}$ (ce qui en fait bien un op\'erateur lin\'eaire par 
rapport aux \'el\'ements de $\mathcal A$) et non pas en recopiant servilement la
formule classique $d\omega + \omega^{2}$, ce qui serait faux!

\subsection{Cohomologie des espaces non commutatifs}

\subsubsection{La cohomologie de Hochschild} \index{cohomologie de Hochschild}

\begin{itemize}


	\item  Nous savons que la diff\'erentielle  $\delta $   sur  $\Omega {\mathcal A} $   
est presque triviale, du point de vue cohomologique. Le ``presque'' 
vient du fait que  $\delta \one = 0 $ et on rappelle qu'il est 
n\'ecessaire de supposer que  ${\mathcal A}$ poss\`ede une 
unit\'e pour construire $\Omega {\mathcal A}$ (on
verra un peu plus  loin comment faire si ce n'est
pas le cas). Il est donc  raisonnable de chercher
\`a d\'efinir une autre th\'eorie  cohomologique
ou homologique qui se restreigne, dans le cas 
classique, \`a celle d\'ej\`a connue (celle de De
Rham). En fait,  nous allons voir que ceci se
fait en deux temps : on d\'efinit tout  d'abord
la {\sl cohomologie de Hochschild\/}, puis la
{\sl cohomologie  cyclique\/}, \index{cohomologie cyclique} et c'est cette
derni\`ere qui va nous fournir un  analogue non
commutatif de la cohomologie de De Rham.  


	\item   Nous avons d\'ej\`a mentionn\'e le fait que, dans le cas classique (le cas 
 de la g\'eom\'etrie ``commutative'' usuelle o\`u ${\mathcal A} = 
 C^\infty(X)$), l'alg\`ebre $  \Omega {\mathcal A} $  des formes universelles \'etait 
 ``bien plus grande'' que celle des formes diff\'erentielles usuelles $\Lambda(X) $.
 Nous allons voir que, dans le cas classique, il existe une fa\c con purement (co)homologique de 
r\'ecup\'erer l'alg\`ebre des formes diff\'erentielles usuelles, ou plut\^ot, en 
travaillant de fa\c con duale, l'espace des courants de De Rham.
 Le but de cette section est donc d'\'ebaucher une construction qui 
conduise aux courants de De Rham dans le cas usuel, mais qui soit, bien 
entendu, valable pour une alg\`ebre associative quelconque.


\item 
On d\'efinit  l'op\'erateur co-bord de Hochschild $b$ (aussi appel\'e {\sl 
diff\'erentielle de Hochschild\/}) comme suit 
\index{diff\'erentielle de Hochschild}: \par
Soit $\phi$ une $(n+1)$-forme lin\'eaire $  \phi(a_0,a_1,\ldots,a_n) $ 
sur l'alg\`ebre ${\mathcal A}$.  Alors
\begin{eqnarray*}
     [b\phi](a_0,a_1,\ldots,a_{n+1})&=&
\sum_{j=0}^n \phi(a_0,\ldots,a_ja_{j+1},\ldots,a_n) + \cr
{} & {} & (-1)^{n+1} \phi(a_{n+1}a_0,a_1,\ldots,a_n)  
\end{eqnarray*}
Par exemple, 
 \begin{eqnarray*}
[b\phi](a_0,a_1,a_2,a_3]&=&\phi(a_0a_1,a_2,a_3)-\phi(a_0,a_1a_2,a_3)+\phi(a_0,a_1,a_2a_3)\cr
{} & {} & -\phi(a_3a_0,a_1,a_2)
\end{eqnarray*}
L'\'etape suivante consiste \`a montrer que  $  b^2=0 ,$   ce qui est \`a la 
fois imm\'ediat, et p\'enible$\ldots $  
\par  

Puisque nous avons un op\'erateur cobord (il est de carr\'e nul et envoie bien les $n$ formes dans 
les $n+1$ formes), nous pouvons d\'efinir  l'espace des cocycles de 
Hochschild  $  Z^n=\{\phi \in C^n / b\phi = 0\},$ l'espace des 
cobords de
Hochschild $  B^n=\{\phi \in C^n / \phi = b\psi \, \rm{for} \, \psi \in C^{n-1} \}$  
et les groupes de cohomologie (de Hochschild) correspondants  $  H^n 
 =  Z^n/B^n .$  
Ci-dessus,  la notation $  C^n $, l'espace des cochaines de 
Hochschild,  d\'esigne l'espace des formes
 $ n+1 $   multilin\'eaires sur  $  {\mathcal A} $ (attention \`a la translation 
 d'une unit\'e). \par 
 
Remarque terminologique : un lecteur curieux, qui chercherait la d\'efinition 
de la cohomologie de Hochschild dans un ouvrage d'alg\`ebre 
homologique pourrait \^etre surpris car celle-ci fait 
d'ordinaire r\'ef\'erence au choix d'un certain bimodule. Ici, le bimodule 
en question n'est autre que le
dual de  $ {\mathcal A} $. Nous n'avions pas besoin de mentionner ceci plus 
haut mais il est bon de savoir que c'est pr\'ecisemment ce choix 
particulier de bimodule (ainsi que l'existence d'un accouplement naturel entre ${\mathcal A}$ et son dual 
${\mathcal A}^{*}$) qui est \`a l'origine de la d\'efinition pr\'ec\'edente de $b$.


	\item  Dans le cas classique (celui de la g\'eom\'etrie diff\'erentielle ``commutative'' 
 habituelle), nous savons que les courants de De Rham (voir 
 \ref{sec: courants de De Rham}) sont d\'efinis comme distributions sur les formes 
 diff\'erentielles de De Rham. En d'autres termes,  si  $  C $   is
un  $  p $-courant et  $  \omega $   est une  $  p $-forme, alors$  \langle 
C,\omega \rangle $ est un nombre.
Nous allons montrer qu'il existe une correspondance entre courants 
arbitraires et cocycles de Hochschild, dans le cas particulier des 
$2$-formes, en laissant au lecteur le soin de g\'en\'eraliser cette 
propri\'et\'e au cas $p>2.$
 \par 

Des courants de De Rham aux cocycles de Hochschild : \'etant 
donn\'e  $  C $, nous construisons  $  \phi(f,g,h)  =  \langle C,f dg \wedge dh \rangle .$   
on peut alors v\'erifier que  $  b\phi = 0.$   \par
 
Des cocycles de Hochschild aux courants de De Rham :   \'etant donn\'e $  \phi ,$
nous construisons  $  \langle C, f dg \wedge dh \rangle   =  \phi(f,g,h)-\phi(f,h,g). $  \par

 Les deux formules ci-dessus sont diff\'erentes car il n'y a aucune 
 raison de supposer qu'un cocycle de Hochschild donn\'e  $\phi $  soit antisym\'etrique.
 \par
 Si $  \phi $   est un cobord de Hochschild, il reste \`a v\'erifier que le 
 courant de De Rham correspondant s'annule. Ceci est une cons\'equence 
 imm\'ediate de la d\'efinition de $b$ et de l'antisymm\'etrie du produit 
 ext\'erieur.\par 
 
De fa\c con g\'en\'erale, le $p$-i\`eme groupe de cohomologie de Hochschild co\"incide 
avec l'espace des courants de De Rham en degr\'e  $ p .$ 
On peut en particulier v\'erifier que la dimensionalit\'e de l'espace $  
H^p $ est triviale d\`es que  $  p $   est plus grand que la dimension de la 
vari\'et\'e $X$ elle - m\^eme.



\item
Il peut \^etre int\'eressant de comparer l'expression  de $[b\phi](a_0,a_1,a_2,a_3]$ avec le 
calcul de $  a_0\delta (a_1) \delta (a_2)  a_3 $  effectu\'e dans la 
section consacr\'ee \`a la d\'efinition de $\Omega {\mathcal A}$ 
(\ref{sec: Omega}).
On voit que le fait de calculer $b\phi $  revient, dans cet exemple, 
\`a \'evaluer $\phi$ sur un type particulier de 
commutateurs (dans $\Omega {\mathcal A} $), en l'occurence $  \phi([a_0\delta a_1\delta a_2,a_3]).$  

Cette remarque peut \^etre g\'en\'eralis\'ee, en ce sens qu'on peut 
\^etre tent\'e de consid\'erer les $p$ formes sur ${\mathcal A}$ comme des 
formes lin\'eaires sur l'alg\`ebre $ \Omega{\mathcal A}$ et de 
d\'efinir $b$ non pas sur les formes $p$-lin\'eaires sur ${\mathcal A}$ 
mais sur les formes lin\'eaires sur $ \Omega{\mathcal A}$.  En fait, on 
se heurte alors \`a un probl\`eme un peu subtil li\'e au r\^ole 
particulier jou\'e par l'unit\'e dans la construction de l'alg\`ebre 
des formes universelles.

Notons $\tilde {\mathcal A}$ l'alg\`ebre obtenue en rajoutant une 
unit\'e $\one$ \`a ${\mathcal A}$, que celle-ci en poss\`ede d\'ej\`a une ou non. Les \'el\'ements de cette 
augmentation sont, par d\'efinition, des paire $(a,c)$, avec $a \in {\mathcal A}$ et $c 
\in \CC$. La nouvelle unit\'e est $\one  =  (0,1)$. 
On identifie $a \in {\mathcal A}$ avec $(a,0) \in \tilde{\mathcal 
A}$. Les \'el\'ements $(a,c)$ de l'alg\`ebre augment\'ee sont not\'es 
simplement
$a + c \one$. La multiplication est telle que $ (a_{1} + c_{1} 
\one)(a_{2} + c_{2} \one) = a_{1}a_{2} + c_{1}a_{2} + a_{1}c_{2} + 
c_{1} c_{2}\one $; elle doit donc \^etre formellement d\'efinie par
$$(a_{1},c_{1})(a_{2},c_{2})  =  (a_{1}a_{2} + c_{1}a_{2} + 
a_{1}c_{2}, c_{1} c_{2})$$
Si ${\mathcal A}$ ne poss\'ede pas  d'unit\'e, il n'y a pas de confusion possible. Si ${\mathcal A}$ en 
poss\'ede d\'ej\`a une, nous la d\'esignons par $e  =  (e,0)$ et il est certain que 
$e$ n'est plus l'unit\'e de $\tilde {\mathcal A}$, mais seulement un 
projecteur ($e^{2}=e$). 
Notons que, avec $a \in {\mathcal A}$ et $c \in \CC$,  $\delta (a + c \one) = 
\delta a$ dans $ \Omega \tilde {\mathcal A}$.
On peut donc identifier les formes multilin\'eaires sur ${\mathcal A}$ 
avec certaines formes lin\'eaires sur  $ \Omega \tilde {\mathcal A}$, en 
l'occurence avec les formes $\phi$ qui sont telles que
$\phi(\one \delta a_{1} \delta a_{2} \ldots \delta a_{n} ) = 0$ en 
posant, pour $a_{i} \in  {\mathcal A}$ 
$$ \phi(a_{0},a_{1},\ldots,a_{n})  =  \phi(a_{0} \, \delta a_{1} \ldots \delta a_{n}) $$

Gr\^ace \`a cette identification, on peut effectuer toutes les 
constructions de nature cohomologique en utilisant comme cochaines ce type 
particulier de formes lin\'eaires sur  $\Omega \tilde {\mathcal A}$ plut\^ot 
que de faire appel \`a des formes multilin\'eaires sur ${\mathcal A}$.
Nous n'irons cependant pas plus loin dans cette direction.

\par 


\end{itemize}


\subsubsection{La cohomologie cyclique : une cohomologie de De Rham non 
commutative} \index{cohomologie cyclique}

\begin{itemize}
	\item  Dans le cas non commutatif, nous n'avons pas encore pr\'esent\'e 
	de construction qui g\'en\'eralise la cohomologie de De Rham 
	(section \ref{sec: cohomologie de De Rham}). 
	 En fait, puisque nous travaillons maintenant sur les alg\`ebres 
	elle-m\^emes (dans le cas commutatif on consid\`ere l'alg\`ebre 
	commutative $C^\infty(X)$ des fonctions et non pas les points de $X$ 
	eux-m\^emes), 
	c'est d'un analogue de l'op\'erateur d'homologie $  \partial $ sur les courants dont nous avons besoin.
	 Rappelons que, classiquement, cet op\'erateur agit sur les courants de De 
	  Rham de la fa\c con suivante (th\'eor\`eme de Stokes): 
	  $$  \langle \partial C, \omega \rangle  = \langle C, d\omega \rangle  ,$$
	  Ici $  \omega$ est une forme diff\'erentielle quelconque sur $X$.
	  
	  \item 
La d\'efinition la plus simple  (mais c'est peut-\^etre une question de go\^ut) est 
celle qui suit.
On d\'efinit tout d'abord la notion de cyclicit\'e pour une forme 
multilin\'eaire  $$ 
\mbox{\fbox{$  
\phi \rm{\ est \ cyclique} \Leftrightarrow \phi(a_0,a_1,\ldots,a_n)=(-1)^n
\phi(a_n,a_0,a_1,\ldots,a_{n-1}).
$}} $$    
On fait alors la remarque suivante \cite{ConnesIHES} :
Si  $  \phi $  est cyclique, alors  $  b \phi $ l'est aussi. \par 
Il devient alors naturel de consid\'erer le sous complexe cyclique du 
complexe de Hochschild, c'est \`a dire de restreindre l'op\'erateur $b$ 
(le m\^eme que pr\'ec\'edemment) aux cochaines de Hochschild cycliques.
On d\'efinit alors les espaces $  Z_\lambda^n ,$  $  B_\lambda^n $  des 
cocycles et cobords cycliques, ainsi que leurs quotients, les groupes 
de cohomologie cyclique $ H_\lambda^n .$ \par 

\item
Dans le cadre classique, \ie  avec  $ {\mathcal
A}=C^\infty(X) ,$ on montre \cite{ConnesIHES}
que  $$H_\lambda^k= Ker \partial \oplus H_{k-2}
\oplus H_{k-4} \ldots  $$     o\`u
 $ Ker \partial $   est le noyau de l'op\'erateur $  \partial $  
 agissant dans l'espace des courants de De Rham de degr\'e $k$ et o\`u  $  
 H_p $   d\'esigne le groupe d' \underbar {homo}logie
de degr\'e $  p $   (pour les courants). 

Ainsi, nous n'obtenons pas une correspondance bi-univoque entre les 
groupes de cohomologie cyclique et les groupes d'homologie de De 
Rham; n\'eanmoins, l'information contenue est la m\^eme, puisque, 
en choisissant $k$ assez grand, les groupes de cohomologie cycliques 
pairs ou impairs seront respectivement \'egaux \`a la somme directe des 
groupes d'homologie de De Rham (pairs ou impairs). \par 
 
  Ce r\'esultat sugg\`ere qu'il existe une fa\c con canonique d'envoyer
 $  H_\lambda^p $  dans  $  H_\lambda^{p+2},$ et c'est effectivement le 
 cas (pour une alg\`ebre ${\mathcal A}$ quelconque, d'ailleurs).
 En fait, on peut d\'emontrer un r\'esultat encore plus fort: pour toute 
 alg\`ebre, on peut d\'efinir un op\'erateur  $  S ,$ souvent d\'esign\'e sous 
 le nom de ``op\'erateur de p\'eriodicit\'e de Connes'', qui envoie  $  
 C_\lambda^p $   dans
 $  C_\lambda^{p+2} $   -- le symbole $  C_\lambda^*$ se r\'eferrant aux 
 cochaines cycliques. 
 
 \item Cette fa\c con de d\'efinir la cohomologie cyclique (comme sous 
 complexe de celle de Hochschid), se fait donc sans qu'il soit besoin 
 d'introduire une 
 g\'en\'eralisation non commutative de l'op\'erateur $\partial $. Cela dit, 
 un tel op\'erateur existe (il est not\'e $B_{0}$, ou $B$ --- voir 
 ci-dessous ---) et on peut aussi d\'efinir la cohomologie cyclique 
 gr\^ace \`a lui. La d\'efinition de cette cohomologie, en utilisant 
 les op\'erateurs en question, est un peu plus subtile, et nous nous 
 contenterons de donner la d\'efinition des op\'erateurs $B_{0}$ et 
 $B$.
 Le lecteur interess\'e pourra consulter \cite{ConnesIHES}, \cite{ACbook} et les 
 r\'ef\'erences indiqu\'ees dans cet ouvrage.
 
 \item
 
Outre l'op\'erateur de p\'eriodicit\'e d\'ej\`a mentionn\'e,  $  S : C_{\lambda}^p
\rightarrow C_\lambda^{p+2} $,  on  consid\`ere aussi les 
op\'erateurs suivants :\par
 \begin{itemize}
 	\item  L'op\'erateur d'antisym\'etrisation cyclique.
 	 \begin{eqnarray*}
   [A \phi](a_0,a_1,\ldots, a_n)  &=& \, \phi(a_0,a_1,\ldots,a_n) + (-1)^n
\phi(a_n,a_0,\ldots, a_{n-1}) + \cr 
{} & + &\phi(a_{n-1},a_n,a_0,\ldots) + (-1)^n\phi(a_{n-2},a_{n-1},a_n ,a_0,a_1,\ldots) + \ldots 
\end{eqnarray*}
 
 	\item  L'op\'erateur bord non antisym\'etris\'e $B_0$ d\'efini comme
 	 $$     [B_0 \phi](a_0,a_1,\ldots,a_n) = \phi(e,a_0,\ldots,a_n) - 
\phi(a_0,\ldots,a_n,e) $$   Ici,  $e$  d\'esigne l'unit\'e de l'alg\`ebre ${\mathcal A}$ (et il faut 
effectivement supposer que l'alg\`ebre est unitale).
 
 	\item  L'op\'erateur bord cyclique $ B  =  A B_0.$ (ou {\sl 
 	diff\'erentielle de Connes\/}). \index{diff\'erentielle de Connes}
 	
 \end{itemize}
 
On montre alors que  $B$ envoie $ C^n $ sur $ C^{n-1}  $, que
$   B^2 = 0$ et que   $bB+Bb=0.$  En utilisant ces deux derni\`eres 
propri\'et\'es, ainsi que  $b^2=0 $,    on peut construire un bi-complexe 
(puisque $  b $   and  $  B $  agissent dans des directions oppos\'ees) 
\`a partir duquel on peut \'egalement d\'efinir la cohomologie cyclique.

\item 

 En utilisant ce dernier bi-complexe on d\'efinit aussi la  ``cohomologie 
 cyclique enti\`ere'' de la fa\c con suivante. Les cocycles entiers sont des 
 \underbar{suites}  $  (\phi_{2n}) $  
ou $  (\phi_{2n+1}) $   de fonctionnelles paires ou impaires  $  \phi $   
qui doivent satisfaire \`a la contrainte suivante (nous ne l'\'ecrivons 
que pour le cas impair) :   $$      b \phi_{2n-1} +
B\phi_{2n+1} = 0  .$$    A l'aide de tels cocycles (techniquement, il 
faut aussi supposer qu'une certaine condition de croissance est 
satisfaite), on peut d\'efinir des fonctions enti\`eres sur l'alg\`ebre $  
{\mathcal A} $,  $$     F_\phi(x) =
\sum_{n=0}^\infty {(-1)^n \over n!} \phi_{2n}(x,x,\ldots,x) .$$    

La cohomologie cyclique enti\`ere fournit un formalisme 
appropri\'e pour l'\'etude de certaines alg\`ebres non commutatives de 
dimension infinie apparaissant en th\'eorie quantique des champs.

\end{itemize}

\subsection{ Remarque finale}
Comme nous l'avons signal\'e plus haut, 
notre propos, dans cette derni\`ere section \'etait simplement d'effectuer un tour rapide dans certains secteurs du 
zoo non commutatif, en esp\'erant que le lecteur aura plaisir \`a y 
retourner en consultant la litt\'erature sp\'ecialis\'ee. 
Le pr\'esent ouvrage est en effet essentiellement d\'edi\'e \`a l'\'etude de 
plusieurs aspects de la g\'eom\'etrie diff\'erentielle;  en 
l'occurence, la th\'eorie des connexions et des espaces fibr\'es.
Cependant, la physique du vingti\`eme si\`ecle n'est  (n'\'etait !)  pas seulement courbe : elle est (\'etait) aussi quantique.
Il e\^ut donc \'et\'e  dommage de passer sous silence ces quelques 
d\'eveloppements r\'ecents --- et passionnants --- des 
math\'ematiques, qui g\'en\'eralisent les notions habituelles et quasi intuitives de la g\'eom\'etrie 
``ordinaire'' (celle des espaces) au monde, encore un peu 
myst\'erieux, des espaces non commutatifs.

